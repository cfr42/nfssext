% \iffalse meta-comment
%%%%%%%%%%%%%%%%%%%%%%%%%%%%%%%%%%%%%%%%%%%%%%%%%
% fontscripts.dtx
% Additions and changes Copyright (C) 2010-2025 Clea F. Rees.
% Code from skeleton.dtx Copyright (C) 2015-2024 Scott Pakin (see below).
%
% This work may be distributed and/or modified under the
% conditions of the LaTeX Project Public License, either version 1.3c
% of this license or (at your option) any later version.
% The latest version of this license is in
%   https://www.latex-project.org/lppl.txt
% and version 1.3c or later is part of all distributions of LaTeX
% version 2008-05-04 or later.
%
% This work has the LPPL maintenance status 'muaintained'.
%
% The Current Maintainer of this work is Clea F. Rees.
%
% This work consists of all files listed in manifest.txt.
%
% The file fontscripts.dtx is a derived work under the terms of the
% LPPL. It is based on version 2.4 of skeleton.dtx which is part of
% dtxtut by Scott Pakin. A copy of dtxtut, including the
% unmodified version of skeleton.dtx is available from
% https://www.ctan.org/pkg/dtxtut and released under the LPPL.
%%%%%%%%%%%%%%%%%%%%%%%%%%%%%%%%%%%%%%%%%%%%%%%%%
% \fi
%
% \iffalse
%<*driver>
\RequirePackage{svn-prov}
\ProvidesFileSVN{$Id: fontscripts.dtx 11047 2025-06-25 08:20:08Z cfrees $}[v0.3 \revinfo][\filebase DTX: l3build scripts for font installation]
\DefineFileInfoSVN[fontscripts]
\documentclass[10pt,british,lm-default=false]{l3doc}
% l3doc loads fancyvrb
% fancyvrb overwrites svn-prov's macros without warning
% restore \fileversion \filerev in case we're using l3doc
\GetFileInfoSVN{fontscripts}
\EnableCrossrefs
\CodelineIndex
\RecordChanges
^^A \OnlyDescription
\DoNotIndex{\verb,\ProvidesPackageSVN,\NeedsTeXFormat,\ProcessKeyOptions}
\usepackage{babel}
\usepackage{fancyhdr}
\pdfmapfile{-clm.map}
\pdfmapfile{+clm.map}
\usepackage[rm={proportional,lining},sf={proportional,lining},tt={monowidth,lining,tabular}]{cfr-lm}
\usepackage{cleveref}
\usepackage{array,tabularx}
\usepackage{xcolor}
\usepackage{xurl}
\urlstyle{sf}
\usepackage{microtype}
% addaswyd o chronos.tex
\MakeAutoQuote{‘}{’}
\MakeAutoQuote*{“}{”}
\usepackage{caption}
\DeclareCaptionFont{lf}{\lstyle}
\captionsetup[table]{labelfont=lf}
% sicrhau hyperindex=false: llwytho CYN bookmark
\usepackage{hypdoc}% ateb Ulrike Fischer: https://tex.stackexchange.com/a/695555/
\usepackage{bookmark}
\hypersetup{%
  colorlinks=true,
  citecolor={moss},
  extension=pdf,
  linkcolor={strawberry},
  linktocpage=true,
  pdfcreator={TeX},
  pdfproducer={pdfeTeX},
  urlcolor={blueberry}%
}
\NewDocElement[%
  idxtype=etx,
  idxgroup=font encodings,
  printtype=\textit{etx},
]{Etx}{encoding}
\NewDocElement[%
  idxtype=mtx,
  idxgroup=font metrics,
  printtype=\textit{mtx},
]{Mtx}{metrics}
\NewDocElement[%
  idxtype=lua,
  idxgroup=lua script fragments,
  printtype=\textit{lua},
]{Lua}{luafrag}
\NewDocElement[%
  idxtype=tpt,
  idxgroup=templates,
  printtype=\textit{tpt},
]{Tpt}{template}
\NewDocElement[%
  idxtype=targ,
  idxgroup=targets,
  printtype=\textit{targ},
]{Targ}{target}
\NewDocElement[%
  idxtype=var,
  idxgroup=variables,
  printtype=\textit{var},
]{lVar}{luavariable}
\NewDocElement[%
  idxtype=fn,
  idxgroup=functions,
  printtype=\textit{fn},
]{lFn}{luafunction}
\ExplSyntaxOn
\NewDocumentCommand \ivals { +m }
{
  {
    \clist_if_empty:nF { #1 }
    {
      \normalfont
      \itshape
      < \clist_use:nn { #1 } { >\texttt{,} ~ < } >
    }
  }
}
\keys_define:nn { fnt / doc }
{
  unknown .code:n = {
    \cs_if_free:cT { \l_keys_key_str }
    {
      \tl_gset:cn { \l_keys_key_str } { #1 }
    }
  },
}
\NewDocumentCommand \fntdocset { +m }
{
  \keys_set:nn { fnt / doc } { #1 }
}
\ExplSyntaxOff
\fntdocset{%
  bug={\href{https://codeberg.org/cfr/nfssext/issues}{\textsc{bugtracker}}},
  codeberg={\href{https://codeberg.org/cfr/nfssext}{\textsc{codeberg}}},
  github={\href{https://github.com/cfr42/nfssext}{\textsc{github}}},
  ctan={\href{https://ctan.org/}{\textsc{ctan}}},
}
\newcommand*{\lpack}[1]{\textsf{#1}}
\title{\filebase}
\author{Clea F. Rees\thanks{%
    Bug tracker:
    \href{https://codeberg.org/cfr/nfssext/issues}{\url{codeberg.org/cfr/nfssext/issues}}
    \textbar{} Code:
    \href{https://codeberg.org/cfr/nfssext}{\url{codeberg.org/cfr/nfssext}}
    \textbar{} Mirror:
    \href{https://github.com/cfr42/nfssext}{\url{github.com/cfr42/nfssext}}%
}}
\date{\fileversion~\filedate}
\pagestyle{fancy}
\fancyhf{}
\fancyhf[lh]{\itshape\filebase}
\fancyhf[rh]{\itshape\fileversion}
\fancyhf[cf]{\itshape--- \thepage~/~\lastpage{} ---}
\ExplSyntaxOn
\hook_gput_code:nnn {shipout/lastpage} {.}
{
  \property_record:nn {t:lastpage}{abspage,page,pagenum}
}
\cs_new_protected_nopar:Npn \lastpage 
{
  \property_ref:nn {t:lastpage}{page}
}
\NewDocumentCommand \plarg {+m} {{\ttfamily (\ivals{#1})}}
\ExplSyntaxOff
\makeatletter
%^^A sylwad david carlisle: https://tex.stackexchange.com/questions/329949/ignore-par-after-the-end-of-a-macro-then-insert-noindent#comment808392_329949
\NewDocumentCommand \istable{m}{\texttt{= \{}\ivals{#1}\texttt{\}}\@afterindentfalse\@afterheading\par} 
\NewDocumentCommand \isstring{m}{\texttt{= ''}\meta{#1}\texttt{''}\@afterindentfalse\@afterheading\par} 
\NewDocumentCommand \isbool{}{\texttt{= }\meta{true\textbar false}\@afterindentfalse\@afterheading\par} 
\newcommand*\afm{\textsc{afm}}
\newcommand*\enc{\textsc{enc}}
\newcommand*\etx{\textsc{etx}}
\newcommand*\fd{\textsc{fd}}
\newcommand*\map{\textsc{map}}
\newcommand*\mtx{\textsc{mtx}}
\newcommand*\pfb{\textsc{pfb}}
\newcommand*\pl{\textsc{pl}}
\newcommand*\tfm{\textsc{tfm}}
\newcommand*\vf{\textsc{vf}}
\newcommand*\vpl{\textsc{vpl}}
\definecolor{strawberry}{rgb}{1.000,0.000,0.502}
\definecolor{blueberry}{rgb}{0.000,0.000,1.000}
\definecolor{moss}{rgb}{0.000,0.502,0.251}
\def\@xobeysp{\leavevmode\penalty100\ }
\makeatother
\newcolumntype{T}{>{\ttfamily\arraybackslash}l}
\fvset{gobble=0}
\begin{document}
\DocInput{\filename}
\end{document}
%</driver>
% \fi
%
% \changes{v0.1}{2024-09-24}{First public release.}
% \changes{v0.2}{2025-01-28}{Use \texttt{fnt} prefix in Lua scripts.}
% \changes{v0.4}{2025-06-25}{Minor documentation corrections.}
% 
% \maketitle\thispagestyle{empty}
% \pdfinfo{%
% 	/Creator		(TeX)
% 	/Producer		(pdfTeX)
% 	/Author			(Clea F.\ Rees)
% 	/Title			(fontscripts)
% 	/Subject		(TeX)
% 	/Keywords		(TeX,LaTeX,Clea,Rees)}
% \pdfcatalog{%
% 	/URL				()
% 	/PageMode	/UseOutlines}	
% \setlength{\parindent}{0pt}
% \setlength{\parskip}{0.5em}
% 
% 
% \begin{abstract}
%   \noindent
%   \lpack{fontscripts} provides variant font encodings, support metrics and Lua script fragments to automate the creation of \TeX{}/\LaTeXe{} font files for 8-bit engines using \lpack{l3build}.
%   A template-based system enables the automatic generation of font tables and \lpack{l3build} tests.
%   Variables support the addition of encoding subset declarations for the Text Companion (\texttt{TS1}) encoding and variable scaling factors in font definition files.\smallskip 
%
%   The script fragments\footnote{%
%     The main script fragments are written in \emph{extremely elementary} Lua.
%     This is the first thing I've ever attempted in Lua and I am not a programmer.%
% } make it possible to automate the generation of \TeX{} font metrics, virtual fonts, map files etc.\ and the conversion of fonts and encodings. 
%   For tools which do not otherwise support it, such as \lpack{fontinst}, the scripts enable the automatic addition of variable scaling in font definition files. 
%   A semi-automatic system tries to ensure font encoding names are unique.\smallskip
% 
%   The script fragments were originally designed for \lpack{fontinst}, but support for \texttt{afm2tfm} (intended for fonts consisting entirely of ornaments etc.) is provided and a modular approach enables easy modification for other tools.
%   The default configuration is intended to be cross-platform and requires only tools included in \TeX{} Live, but the documentation includes a simple adaption for integration with FontForge and \textsc{gnu} make.
% \end{abstract}
%
% \tableofcontents
%
% \section{Quick Start}\label{sec:start}
%
% \textbf{This assumes you are familiar with basic usage of \lpack{l3build}.
%   If not, please see \lpack{l3build}'s manual for basic setup.}
%
% Configuring your project for \lpack{l3build} includes creation of a file, \file{build.lua}, in your working directory.
% This should specify at least the \texttt{module} name, typically the name of your package.
% So you should have something like this,
% \iffalse
%<*verb>
% \fi
\begin{verbatim}
module = "fontastic"
\end{verbatim}
% \iffalse
%</verb>
% \fi
% We now need to edit this file to make \pkg{l3build} aware of the Lua \texttt{fntbuild} scripts provided by \lpack{fontscripts}.
% \iffalse
%<*verb>
% \fi
%^^A kpse.set_program_name("kpsewhich")
\begin{verbatim}
module = "fontastic"
require(kpse.lookup("fntbuild.lua"))
\end{verbatim}
% \iffalse
%</verb>
% \fi
% This should be sufficient to make the functions provided by the script available.
% To effectively use these functions, however, you will probably need to provide configuration details for your specific workflow and setup.
% For example, if you want to build fonts this way, you may need to specify which build tool you use and which files contain the code to compile or commands to execute.
% The remainder of this document explains how to do this\footnote{%^^A
%   Unless you are reading a version which includes the implementation section.
%   In that case, \emph{most} of the document is just code listing and it is the first bit of what remains which explains how to actually use the scripts.%^^A
% }.
% You also need to tell the scripts if you want to override the defaults e.g.~if you would like \texttt{fntbuild} to insert subset encoding declarations and/or variable scaling factors\footnote{%
%   \texttt{fontinst} only.
%   As far as I know, this is simply an oversight on \lpack{fontinst}'s part -- fixed scaling factors are supported out-of-the-box -- and I assume other tools have the ability to include variable scaling without \texttt{fntbuild}'s interference.%^^A
% } into font definition files or to auto-generate tests and/or font tables.
%
% \section{Lua Script Fragments}\label{sec:lua}
%
% \changes{v0.2}{2025-01-25}{\texttt{fontinst.lua} renamed to \texttt{fntbuild.lua}.}
% \file{fntbuild.lua} provides two custom \lpack{l3build} targets and five new Lua functions.
% In addition, it redefines three \lpack{l3build} functions (two of which do nothing by default).
% As well as those provided by \lpack{l3build}, a number of additional variables are used by these functions to determine what, where and how they should operate.
%
% \subsection{\lpack{l3build} Targets}\label{subsec:targs}
%
% \changes{v0.2}{2025-01-25}{\texttt{fnttarg} may now be ‘sandboxed’.
%   The default build recipes (based on \lpack{fontinst} and \texttt{afm2tfm}) are configured to do this and \texttt{fnt.buildsearch} is disabled by default.
%   At present, isolation only affects \TeX-based build tools. 
%   FontForge and \textsc{gnu} \texttt{make} are unaffected, for example.}
% \DescribeTarg{fnttarg}
% Runs \texttt{fnt.fontinst ()} by default or \verb|fnt.afm2tfm ()| if \texttt{fnt.afmtotfm} is \texttt{true}.
% 
% Builds traditional \TeX{} font files for 8-bit engines.
% The building is done in \texttt{fnt.fntdir}.
% The results are copied to \texttt{fnt.keepdir} and the intermediate results to \texttt{fnt.keeptempdir}.
% Files in the former are copied back into the \texttt{builddir} when building documentation, running tests etc.
%
% This target may be redefined if another tool chain is used for font creation\footnote{^^A
%   \verb|fnt.afm2tfm ()| in \file{fntbuild.lua} and \file{build.lua} in \lpack{berenisadf} provide examples.^^A
% }.
%
% If additional setup is required, this may be done by defining \verb|fnt.buildinit_hook ()|, which is empty by default.
% See \cref{subsubsec:new}.
% 
% \DescribeTarg{uniquifyencs} \meta{encoding tag}\par\noindent
% Tries to ensure that the names of encoding files and font encodings are unique by editing font definition files and map file fragments in \texttt{fnt.keepdir}.
% \meta{encoding tag} is optional.
% If no argument is given and \texttt{encodingtag} is empty, a suitable tag will be determined automatically\footnote{Hopefully.}.
%
% Note that it is not necessary to use this target if \textt{l3build fnttarg} is used with a tool for which built-in support is provided.
%
% \subsection{Lua Functions}\label{subsec:fns}
%
% \subsubsection{New Functions}\label{subsubsec:new}
%
% \changes{v0.2}{2025-01-25}{New function: \texttt{fnt.afm2tfm()}.}
% \DescribelFn{fnt.afm2tfm()} \texttt{fnt.afm2afm}\plarg{dir}\par\noindent
% A function to generate support files for simple symbolic fonts in \meta{dir} using \texttt{afm2tfm}.
% The function generates \tfm s, \fd s and a \map.
% It requires \afm s and, optionally, \enc s.
% If \meta{dir} is unspecified, \texttt{fnt.fntdir} is used.
%
% If \texttt{fnt.afmtotfm} is \texttt{true}, a call to \verb|fnt.afm2tfm()| replaces the default call to \texttt{fnt.fontinst()} when the \texttt{fnttarg} target is used.
%
% Like \texttt{fnt.fontinst()}, \verb|fnt.afm2tfm()| calls \verb|build_fnt()| and \texttt{fnt.fntkeeper()}, but it does \textbf{not} call \texttt{fnt.uniquify()}.
%
% Note this function is designed for only very simple cases.
% The target \TeX{} fonts must each correspond to a unique \file{.afm} (though it would not be hard to extend the function to support a one-many mapping).
%
% \changes{v0.2}{2025-01-26}{New function: \texttt{fnt.buildinit()}.}
% \DescribelFn{fnt.buildinit()} \texttt{fnt.buildinit}\plarg{}\par\noindent
% Sets up the build environment in preparation for building by unpacking and copying files, installing dependencies etc.\ in preparation for building.
% Workflows supported out-of-the-box use this function, so there is no need to invoke it explicitly in these cases\footnote{^^A
%   See \verb|fnt.fontinst()| and \verb|fnt.afm2tfm()| in \file{fntbuild-build.lua}.^^A
% }.
% This function is intended for use in defining custom build functions.
% See \cref{sec:custom} for an example.
% It is called, if needed, by those provided out-of-the-box such as \texttt{fnt.fontinst()}.
%
% If \texttt{fnt.needs_fontinst} is \texttt{true}, the files in \texttt{\$\{TEXMFDIST\}/tex/fontinst} are copied to \texttt{fnt.fntdir} before user-specified system files are processed.
% For files provided by \lpack{fontscripts}, the prefix \texttt{fontscripts-} is removed unless doing so conflicts with an existing file name.
%
% Note that \lpack{fontinst} files may be used by tools other than \lpack{fontinst}.
% For example, \lpack{otftotfm} can use converted \lpack{fontinst} encoding files.
%
% \changes{v0.2}{2025-01-25}{New function: \texttt{fnt.buildinit\_hook()}.}
% \DescribelFn{fnt.buildinit_hook()} \texttt{fnt.buildinit\_hook}\plarg{}\par\noindent
% A function which does nothing successfully by default, similar to the hooks provided by \lpack{l3build}.
% May be redefined to perform additional tasks when setting up the environment for \texttt{fnttarg}.
% The function should return an error level.
%
% The hook is executed after unpacking, copying etc.\ at the end of \texttt{fnt.buildinit()} so it can be used to modify those preparations, if necessary.
%
% \changes{v0.2}{2025-01-25}{New function: \texttt{fnt.build\_fnt()}.}
% \DescribelFn{fnt.build_fnt()} \verb|fnt.build_fnt|\plarg{dir,cmd,file}\par\noindent
% A wrapper around \texttt{runcmd()} which runs \meta{cmd} \meta{file} in \meta{dir}.
% If \texttt{fnt.buildsearch} is \texttt{false} (the default), the function first sets up a restricted environment designed to avoid contaminating the build with undesirable elements from \TeX{} search paths. 
%
% Note that the restrictions are designed to avoid contamination resulting from searches using \texttt{kpathsea}.
% If you use this function with a different tool chain, you may need to take additional steps to isolate the build.
% For example, the function does nothing to prevent a FontForge script accessing resources in \path{~/.local/share}, \path{/usr/local/fonts}, \path{/etc}, \path{/System/Library} etc\@.
%
% \textbf{In general, \lpack{l3build} ‘sandboxing’ uses a black-listing rather than a white-listing model.
%   Since this is inherently less reliable, you should not depend on the effectiveness of isolation.}
% This means you should check what actually happens e.g.~which paths actually end up in your \texttt{.log} files or test the effectiveness of the isolation.
% 
% The table \verb|fnt.build_fnt_env| may be used to selectively adapt the default sandboxing, which is sets the \texttt{kpathsea} variables shown in \cref{tab:build-env}.
% \begin{table}
%   \centering
%   \caption{Default environment for ‘sandboxed’ build.}\label{tab:build-env}
%   \begin{tabular}{>{\ttfamily\arraybackslash}l>{\ttfamily\arraybackslash}l}
%     \toprule
%       kpathsea \textsf{variable} & \textsf{value} \\
%     \midrule
%       TEXINPUTS & . fnt.localtexmf() \\
%       TEXMFAUXTREES & \{\} \\
%       TEXMFHOME & \{\} \\
%       TEXMFLOCAL & \{\} \\
%       TEXMFCONFIG & . \\
%       TEXMFVAR & . \\
%       VFFONTS & \verb|${TEXINPUTS}| \\
%       TFMFONTS & \verb|${TEXINPUTS}| \\
%       TEXFONTMAPS & \verb|${TEXINPUTS}| \\
%       T1FONTS & \verb|${TEXINPUTS}| \\
%       AFMFONTS & \verb|${TEXINPUTS}| \\
%       TTFFONTS & \verb|${TEXINPUTS}| \\
%       OPENTYPEFONTS & \verb|${TEXINPUTS}| \\
%       LIGFONTS & \verb|${TEXINPUTS}| \\
%       ENCFONTS & \verb|${TEXINPUTS}| \\
%     \bottomrule
%   \end{tabular}
% \end{table}
%
% \changes{v0.3}{2025-02-12}{New function: \texttt{fnt.buildinit\_tidy()}.}
% \DescribelFn{fnt.buildinit_tidy()} \texttt{fnt.buildinit\_tidy}\plarg{}\par\noindent
% Removes \texttt{fnt.buildsuppfiles_sys} from \texttt{fnt.fntdir}.
% Note that this does not remove files installed by \texttt{fnt.buildinit_fontinst()}.
%
% This function is intended for use in defining custom build functions.
% It is called, if needed, by those provided out-of-the-box such as \texttt{fnt.fontinst()}.
% 
% \changes{v0.2}{2025-02-05}{Renamed function: \texttt{fnt.finst()} (was \texttt{finst()}.}
% \DescribelFn{fnt.finst()} \texttt{fnt.finst}\plarg{patt,dir,mode}\par\noindent
% A wrapper around \verb|fnt.build_fnt()| which runs \verb|pdftex --interaction| \meta{mode} in directory \meta{dir} on each file in \meta{dir} matching the pattern \meta{patt}. 
%
% Called by \texttt{fnt.fontinst()}.
% There is usually no need to call this function directly.
% 
% \changes{v0.2}{2025-02-05}{Renamed function: \texttt{fnt.fontinst()} (was \texttt{fontinst()}.}
% \DescribelFn{fnt.fontinst()} \texttt{fnt.fontinst}\plarg{dir,mode}\par\noindent
% Executes a \lpack{fontinst} workflow to generate \TeX{} fonts and associated files in directory \meta{dir}, executing \texttt{pdftex} with the option \verb|--interaction| \meta{mode}.
% 
% This is the default build function called by \texttt{fnttarg}.
%
% The function calls \texttt{fnt.finst()}, \texttt{fnt.fntkeeper()} and \texttt{fnt.uniquify()} in addition to performing initial compilation.
% The process requires the usual \lpack{fontinst} setup i.e.~a driver to generate the initial files and a second file to create map file fragments.
%
% If a variable scaling factor is used, the font definition files will be edited to ensure this works, since \lpack{fontinst} apparently supports scaling only by a fixed factor.
% For a simple demonstration of how to set this up, see, for example, \lpack{baskervaldadf}'s driver which includes
% \changes{v0.0}{2025-06-24}{Fix code markup for scaling example; add lines from \texttt{.sty} for completeness.}
% \iffalse
%<*verb>
% \fi
\begin{verbatim}
\declaresize{}{<-> \string\ybv@@scale}
\end{verbatim}
% \iffalse
%</verb>
% \fi
% Shape declarations should then be written without scaling,
% \iffalse
%<*verb>
% \fi
\begin{verbatim}
\installfont{ybvr8t}{ybvr8r,ybvr8sr,newlatin}{t1-baskervald}{T1}{ybv}{m}{n}{}
\end{verbatim}
% \iffalse
%</verb>
% \fi
% since scaling will be added after the font definition files are created.
% For example, the result of the above lines in the font definition file is
% \iffalse
%<*verb>
% \fi
\begin{verbatim}
\expandafter\ifx\csname ybv@scale\endcsname\relax
  \let\ybv@@scale\@empty
\else
  \edef\ybv@@scale{s*[\csname ybv@scale\endcsname]}%
\fi

\DeclareFontFamily{T1}{ybv}{}

\DeclareFontShape{T1}{ybv}{m}{n}{
  <-> \ybv@@scale ybvr8t
}{}
\end{verbatim}
% \iffalse
%</verb>
% \fi
% which supports an option provided in the \texttt{sty} file.
% \iffalse
%<*verb>
% \fi
\begin{verbatim}
\keys_define:nn {  baskervald }
{
  scale .tl_set:N = \ybv@scale,
  scale .initial:V = \@empty,
}
\end{verbatim}
% \iffalse
%</verb>
% \fi
%
% If \lpack{fontinst} is setup to produce a shell script for converting \file{.pl} and \file{.vpl} files to \file{.tfm} and \file{.vf}, the function executes the commands contained in this script.
% In order to maintain compatibility on Windows, the commands are extracted and run by Lua rather than calling the shell to execute the commands directly.
% 
% \changes{v0.2}{2025-02-05}{New function: \texttt{fnt.fntsubsetter()}.}
% \DescribelFn{fnt.fntsubsetter ()} \texttt{fnt.fntsubsetter}\plarg{}\par\noindent
% Inserts font encoding subset declarations into font definition files.
% Explicit use is only required for custom build functions, as the standard options for \verb|fnt.build_fnt ()| incorporate this function.
% Returns an error level, but this may not be accurate.
%
% Note that, by default, the presence of this function does nothing because subset declarations are disabled by default.
% To insert declarations, not only must \texttt{fnt.fntsubsetter ()} be executed by the build function.
% \texttt{fnt.subset} must also be set to a non-\texttt{nil} value other than \texttt{false}.
%
% \changes{v0.2}{2025-02-05}{Renamed function: \texttt{fnt.fntkeeper()} (was \texttt{fntkeeper()}.}
% \DescribelFn{fnt.fntkeeper()} \texttt{fnt.fntkeeper}\plarg{dir}\par\noindent
% Copies generated files in \meta{dir} to \texttt{fnt.keepdir} and/or \texttt{fnt.keeptempdir} to prevent deletion by \texttt{l3build}.
% Returns an error level.
%
% This function is called by the function executed by the default functions called by \texttt{fnttarg()} and may be useful if that target is redefined to call a different function.
% 
% \changes{v0.2}{2025-01-25}{New function: \texttt{fnt.lsrdir()}.}
% \changes{v0.0}{2025-06-24}{Correct description of \texttt{fnt.lsrdir()}.}
% \DescribelFn{fnt.lsrdir()} \texttt{fnt.lsrdir}\plarg{path,filenames}\par\noindent
% Adds a recursive list of files in \texttt{path} to the list of files in the table \texttt{filenames} and returns the resulting table.
% \texttt{path} should be a string containing the fully-qualified path to a directory.
% If \texttt{filenames} is unspecified, a new table is returned.
% Otherwise, the function appends entries to \texttt{filenames} rather than replacing them.
%
% \changes{v0.2}{2025-02-05}{Renamed function: \texttt{fnt.test()} (was \texttt{fnt_test()}.}
% \DescribelFn{fnt.test()} \verb|fnt.test|\plarg{fntpkgname,fds,content,maps,fdsdir}\par\noindent
% Auto-generates \texttt{lvt} files suitable for use with \texttt{l3build check} from a template.
%
% \DescribelFn{fnt.uniquify()} \texttt{fnt.uniquify}\plarg{tag}\par\noindent
% Tries to ensure the names of font encodings and encoding files are unique by editing font definition files and map file fragments. 
%
% This function is used by \texttt{fnt.fontinst()} and \verb|fnt.afm2tfm ()|, may be utilised in a custom definition of \verb|fnt.build_fnt ()| or may be called directly using the \texttt{uniquifyencs} target.
% 
% \subsubsection{Redefined Functions}\label{subsubsec:redef}
% 
% \DescribelFn{checkinit_hook()} \verb|checkinit_hook|\plarg{}\par\noindent
% This is a standard \lpack{l3build} function which does nothing by default.
% \file{fntbuild.lua} redefines it to automatically generate test files suitable for use with \texttt{l3build check} if a test template is available.
% If \texttt{checksearch} is \texttt{false}, the hook also sets up an environment tailored to font-testing.
% 
% \changes{v0.2}{2025-01-25}{Disable \texttt{checksearch} by default.}
% \textbf{Note that \texttt{checksearch} is disabled by default.}
% Contrary to \lpack{l3build}, \texttt{fntbuild.lua} enables sandboxing during both building and testing.
% If you do not want this, reenable \texttt{checksearch} after \file{fntbuild.lua} is read or, better yet, create a custom \file{fntbuild-config.lua}.
%
% The default environment for testing does the following:
% \begin{itemize}
%   \item concatenates all files specified in \verb|fnt.mapfiles_sys| and writes the result to \file{pdftex.map} in \texttt{testdir};
%   \item copies the contents of \texttt{fnt.keepdir} to \texttt{testdir};
%   \item copies either the files specified in \verb|fnt.checksuppfiles_sys| or the contents of the directories listed in \cref{tab:check-suppdirs} to \texttt{testdir};
% \begin{table}
%   \centering
%   \caption{Default additions to ‘sandbox’ if none are specified.
%     For files, the file is added.
%     For directories, the contents are added.%
%   }\label{tab:check-suppdirs}
%   \begin{tabular}{*{3}T}
%     \toprule
%     \multicolumn{2}{l}{\sffamily Files} & \sffamily Directories  \\
%     \midrule
%     /tex/latex/base/  & article.cls     & /tex/latex/l3build  \\
%                       & fontenc.sty     & /tex/latex/l3backend \\
%                       & omlcmm.fd       & /fonts/enc/dvips/base \\
%                       & omlcmr.fd       & /fonts/enc/dvips/cm-super \\
%                       & omscmr.fd       & /fonts/type1/public/amsfonts/cm \\ 
%                       & omscmsy.fd      & /fonts/type1/public/cm-super \\
%                       & ot1cmr.fd       & /fonts/tfm/public/cm \\
%                       & ot1cmss.fd      & /fonts/tfm/jknappen/ec \\
%                       & ot1cmtt.fd		  & \\
%                       & size10.clo		  & \\
%                       & size11.clo   		& \\
%                       & size12.clo   		& \\
%                       & t1cmr.fd	      & \\
%                       & t1cmss.fd		    & \\
%                       & t1cmtt.fd    		& \\
%                       & tracefnt.sty 		& \\
%                       & ts1cmr.fd		    & \\
%                       & ts1cmss.fd	  	& \\
%                       & ts1cmtt.fd		  & \\
%                       & ucmr.fd	      	& \\
%                       & ucmss.fd		    & \\
%                       & ucmtt.fd		    & \\
%     /tex/latex/fonttable/  &  fonttable.sty		& \\
%     \bottomrule
%   \end{tabular}
% \end{table}
%   \item copies any files specified in \verb|fnt.checksuppfiles_add| to \texttt{testdir};
%   \item copies the file specified by \texttt{fnt.regress} to \texttt{testdir};
%   \item copies the template specified by \texttt{fnt.testtemp} to \texttt{unpackdir} for use in auto-generating tests;
%   \item restricts the environment by setting the variables listed in \cref{tab:check-env}.
% \end{itemize}
% \begin{table}
%   \centering
%   \caption{%^^A
%     Default environment for ‘sandboxed’ testing.%^^A
%     Note this configuration is applied \emph{in addition} to the steps taken by \texttt{l3build --check} to ‘isolate’ the test environment when \texttt{checksearch} is \texttt{false}.%^^A
%   }\label{tab:check-env}
%   \begin{tabular}{>{\ttfamily\arraybackslash}l>{\ttfamily\arraybackslash}l}
%     \toprule
%     \sffamily\texttt{kpathsea} variable & value \\
%     \midrule
%     TEXMFAUXTREES & \{\} \\
%     TEXMFHOME & \{\} \\
%     TEXMFLOCAL & \{\} \\
%     TEXMFCONFIG & . \\
%     TEXMFVAR & . \\
%     VFFONTS & .  fnt.localtexmf()   \\
%     TFMFONTS & . fnt.localtexmf()   \\
%     TEXFONTMAPS & . fnt.localtexmf()   \\
%     T1FONTS & . fnt.localtexmf()   \\
%     AFMFONTS & . fnt.localtexmf()   \\
%     TTFFONTS & . fnt.localtexmf()   \\
%     OPENTYPEFONTS & . fnt.localtexmf()   \\
%     LIGFONTS & . fnt.localtexmf()   \\
%     ENCFONTS & . fnt.localtexmf()  \\
%     \bottomrule
%   \end{tabular}
% \end{table}
% 
% \DescribelFn{copyctan()} \texttt{copyctan}\plarg{}\par\noindent
% This is extended to copy files from \texttt{fnt.keepdir} and to impose a single-layer of sub-directories of the kind required by \ctan{} for font distributions.
%
% \DescribelFn{docinit_hook()} \verb|docinit_hook|\plarg{}\par\noindent
% This is a standard \lpack{l3build} function which does nothing by default.
% \file{fntbuild.lua} redefines it to automatically generate font tables suitable for use with \texttt{l3build doc} from a template.
%
% \subsection{Variables}\label{subsec:vars}
%
% \newcolumntype{e}{>{\ttfamily =}c}
% \begin{table}
%   \centering
%   \caption{\lpack{fontscripts} defaults for \lpack{l3build} variables.}\label{tab:varia}
%   \begin{tabularx}\linewidth{>{\ttfamily}le>{\ttfamily\raggedright\arraybackslash}X}
%     \toprule
%     \normalfont\textbf{Variable} & \multicolumn{1}{c}{} & \normalfont\textbf{Value} \\
%     \midrule
%       binaryfiles && \{''*.pdf'', ''*.zip'', ''*.vf'', ''*.tfm'', ''*.pfb'', ''*.pfm'', ''*.ttf'', ''*.otf'', ''*.tar.gz''\}\\
%       checkdeps && \{maindir .. ''/fnt-tests''\}\\
%       checkengines && \{''pdftex''\}\\
%       checkformat && ''latex''\\
%       checksearch &&  false \\
%       cleanfiles && \{fnt.keeptempfiles\}\\
%       installfiles && \{''*.afm'', ''*.cls'', ''*.enc'', ''*.fd'', ''*.map'', ''*.otf'', ''*.pfb'', ''*.pfm'', ''*.sty'', ''*.tfm'', ''*.ttf'', ''*.vf''\}\\
%       sourcefiledir && sourcefiledir or ''.''\\
%       sourcefiles && \{''*.afm'', ''afm/*.afm'', ''*.pfb'', ''*.pfm'', ''*.dtx'', ''*.ins'', ''opentype/*.otf'', ''*.otf'', ''tfm/*.tfm'', ''truetype/*.ttf'', ''*.ttf'', ''type1/*.pfb'', ''type1/*.pfm''\}\\
%       tdslocations && \{%
%       	''fonts/afm/'' .. fnt.vendor .. ''/'' .. module .. ''/'' .. ''*.afm'',\newline
%       	''fonts/enc/dvips/'' .. module .. ''/'' .. ''*.enc'',\newline
%       	''fonts/map/dvips/'' .. module .. ''/'' .. ''*.map'',\newline
%       	''fonts/opentype/'' .. fnt.vendor .. ''/'' .. module .. ''/'' .. ''*.otf'',\newline
%       	''fonts/tfm/'' .. fnt.vendor .. ''/'' .. module .. ''/'' .. ''*.tfm'',\newline
%       	''fonts/truetype/'' .. fnt.vendor .. ''/'' .. module .. ''/'' .. ''*.ttf'',\newline
%       	''fonts/type1/'' .. fnt.vendor .. ''/'' .. module .. ''/'' .. ''*.pfb'',\newline
%       	''fonts/type1/'' .. fnt.vendor .. ''/'' .. module .. ''/'' .. ''*.pfm'',\newline
%       	''fonts/vf/'' .. fnt.vendor .. ''/'' .. module .. ''/'' .. ''*.vf'',\newline
%       	''source/fonts/'' .. module .. ''/'' .. ''*.etx'',\newline
%       	''source/fonts/'' .. module .. ''/'' .. ''*.mtx'',\newline
%       	''source/fonts/'' .. module .. ''/'' .. ''*-drv.tex'',\newline
%       	''source/fonts/'' .. module .. ''/'' .. ''*-map.tex'',\newline
%       	''tex/latex/'' .. module .. ''/'' .. ''*.fd'',\newline
%       	''tex/latex/'' .. module .. ''/'' .. ''*.sty''
%       \} \\
%       typesetexe && ''TEXMFDOTDIR=.:../local: pdflatex''\\
%       typesetfiles && typesetfiles or  \{''*.dtx'', ''*-tables.tex'', ''*-example.tex''\}\\
%       typesetsourcefiles && \{fnt.keepdir .. ''/*''\}\\
%     \bottomrule
%   \end{tabularx}
% \end{table}
% \begin{table}
%   \centering
%   \caption{Default values for \lpack{fontscripts}  variables.}\label{tab:varia-n}
%   \begin{tabularx}\linewidth{>{\ttfamily}le>{\ttfamily\raggedright\arraybackslash}X}
%     \toprule
%     \normalfont\textbf{Variable} & \multicolumn{1}{c}{} & \normalfont\textbf{Value} \\
%     \midrule
%       fnt.afmtotfm && nil \\
%       fnt.autotestfds && fnt.autotestfds or \{\}\\
%       fnt.binmakers && fnt.binmakers or \{''*-pltotf.sh''\}\\
%       fnt.builddeps && fnt.builddeps or \{\} \\
%       fnt.buildfiles && fnt.buildfiles or \{ ''*.afm'', ''*.enc'', ''*.etx'', ''*.fd'', ''*.lig'', ''*.make'', ''*.map'', ''*.mtx'', ''*.nam'', ''*.otf'', ''*.pe'', ''*.tex'' , ''*.tfm'' \} \\
%       fnt.buildsearch && false \\
%       fnt.buildsuppfiles\_sys && fnt.buildsuppfiles\_sys or \{\} \\
%       fnt.checksuppfiles\_sys && fnt.checksuppfiles\_sys or \{\} \\
%       fnt.checksuppfiles\_add && fnt.checksuppfiles\_add or \{\} \\
%       fnt.familymakers && fnt.familymakers or \{''*-drv.tex''\}\\
%       fnt.fntdir  && fnt.fntdir or builddir .. ''/fnt'' \\
%       fnt.fnttestfds && fnt.fnttestfds or \{\}\\
%       fnt.keepdir && fnt.keepdir or sourcefiledir .. ''/keep''\\
%       fnt.keepfiles && fnt.keepfiles or \{''*.enc'', ''*.fd'', ''*.map'', ''*.tfm'', ''*.vf''\}\\
%       fnt.keeptempdir && fnt.keeptempdir or sourcefiledir .. ''/keeptemp''\\
%       fnt.keeptempfiles && fnt.keeptempfiles or \{''*.mtx'', ''*.pl'', ''*-pltotf.sh'', ''*-rec.tex'', ''*.vpl'', ''*.zz''\}\\
%       fnt.mapmakers && fnt.mapmakers or \{''*-map.tex''\}\\
%       fnt.mapfiles\_sys && fnt.mapfiles\_sys or \{\} \\
%       fnt.mapfiles\_add && fnt.mapfiles\_add or \{\} \\
%       fnt.needs_fontinst  &&  true \\
%       fnt.pkgbase && fnt.pkgbase or '''' \\
%       fnt.regress && fnt.regress or ''fntbuild-regression-test.tex'' \\
%       fnt.subset && false \\
%       fnt.subsetdefns && fnt.subsetdefns or \{\} \\
%       fnt.subsetfiles && fnt.subsetfiles or \{\} \\
%       fnt.subsettemplate && fnt.subsettemplate or ''\textbackslash DeclareEncodingSubset\{TS1\}\{\$FONTFAMILY\}\{\$SUBSET\}'' \\
%       fnt.tablestemp && fnt.tablestemp or ''fntbuild-tables.tex'' \\
%       fnt.testtemp && fnt.testtemp or ''fntbuild-test.lvt'' \\
%       fnt.vendor && fnt.vendor or ''public''\\
%     \bottomrule
%   \end{tabularx}
% \end{table}
%
% Defaults assigned by \file{fntbuild.lua} to both \lpack{l3build} and \lpack{fontscripts} variables are listed in \cref{tab:varia} and \cref{tab:varia-n} respectively.
%
% \changes{v0.2}{2025-01-25}{New variable: \texttt{fnt.afmtotfm}.}
% \DescribelVar{fnt.afmtotfm}\texttt{fnt.afmtotfm}\isbool
% \texttt{nil} by default.
% Setting this to anything other than \texttt{false} or \texttt{nil} will use \texttt{afm2tfm} rather than \lpack{fontinst} to build the fonts.
% This is only designed for \textbf{\em extremely} simple fonts such as those which contain only text symbols.
%
% \changes{v0.2}{2025-02-05}{Renamed variable: \texttt{fnt.autotestfds} (was \texttt{autotestfds}).}
% \DescribelVar{fnt.autotestfds}\texttt{fnt.autotestfds} \istable{globs}
%
%^^A See also \texttt{testfds}.
%
% \changes{v0.2}{2025-02-05}{Renamed variable: \texttt{fnt.binmakers} (was \texttt{binmakers}).}
% \DescribelVar{fnt.binmakers}\texttt{fnt.binmakers} \istable{globs}
% Scripts to run to convert human-readable \TeX{} font metrics/virtual font metrics into binary \TeX{} font metrics and virtual fonts.
%
% \changes{v0.2}{2025-01-25}{New variable: \texttt{fnt.buildsearch}.}
% \DescribelVar{fnt.buildsearch}\texttt{fnt.buildsearch} \isbool
% Whether to search the \texttt{texmf} trees while generating font files.
%
% Defaults to \texttt{false}.
%
% Note that this is akin to \pkg{l3build}'s \texttt{checksearch} etc.\ but for the additional build step required when preparing fonts.
%
% \changes{v0.2}{2025-01-25}{New variable: \texttt{fnt.builddeps}.}
% \DescribelVar{fnt.builddeps}\texttt{fnt.builddeps} \istable{}
% Dependencies to install when building fonts in a sandbox.
% If empty, some basic dependencies are installed to ensure straightforward cases work out-of-the-box.
%
% \changes{v0.2}{2025-01-25}{New variable: \texttt{fnt.buildfiles}.}
% \DescribelVar{fnt.buildfiles}\texttt{fnt.buildfiles} \istable{globs}
% Source files to copy to the build directory when generating font files.
%
% \changes{v0.2}{2025-01-25}{New variable: \texttt{fnt.buildsuppfiles\_sys}.}
% \DescribelVar{fnt.buildsuppfiles_sys}\texttt{fnt.buildsuppfiles\_sys} \istable{globs}
% 
% \changes{v0.2}{2025-01-25}{New variable: \texttt{fnt.checksuppfiles\_add}.}
% \DescribelVar{fnt.checksuppfiles_add}\texttt{fnt.checksuppfiles\_add} \istable{globs}
% Dependencies to install when checking.
% Supplements default list.
% 
% \changes{v0.2}{2025-01-25}{New variable: \texttt{fnt.checksuppfiles\_sys}.}
% \DescribelVar{fnt.checksuppfiles_sys}\texttt{fnt.checksuppfiles\_sys} \istable{globs}
% Dependencies to install when checking.
% Overrides default list.
% 
% \changes{v0.2}{2025-02-05}{Renamed variable: \texttt{fnt.familymakers} (was \texttt{familymakers}).}
% \DescribelVar{fnt.familymakers}\texttt{fnt.familymakers} \istable{globs}
% Source files \texttt{fnt.fontinst()} should compile to generate \TeX{} support files.
% 
% For the default definition of \texttt{fnt.fontinst()} this variable should specify the driver or drivers to be compiled.
% 
% \changes{v0.2}{2025-01-25}{New variable: \texttt{fnt.fntdir}.}
% \DescribelVar{fnt.fntdir}\texttt{fnt.fntdir}\isstring{directory path}
% Directory in which to build fonts.
% 
% \changes{v0.2}{2025-02-05}{Renamed variable: \texttt{fnt.fnttestfds} (was \texttt{fnttestfds}).}
% \DescribelVar{fnt.fnttestfds}\texttt{fnt.fnttestfds} \istable{globs}
% Files to use when generating test files for \texttt{l3build}.
% 
% \changes{v0.2}{2025-02-05}{Renamed variable: \texttt{fnt.keepdir} (was \texttt{keepdir}).}
% \DescribelVar{fnt.keepdir}\texttt{fnt.keepdir} \isstring{dir}
% Directory to store final products of font creation e.g.~font definitions, map file fragments, \TeX{} font metrics, virtual fonts etc.
% 
% \changes{v0.2}{2025-02-05}{Renamed variable: \texttt{fnt.keepfiles} (was \texttt{keepfiles}).}
% \DescribelVar{fnt.keepfiles}\texttt{fnt.keepfiles} \istable{globs}
% Files to copy to \texttt{fnt.keepdir}.
% 
% \changes{v0.2}{2025-02-05}{Renamed variable: \texttt{fnt.keeptempdir} (was \texttt{keeptempdir}).}
% \DescribelVar{fnt.keeptempdir}\texttt{fnt.keeptempdir} \isstring{dir}
% Directory to store intermediat products of font creation e.g.~human-readable \TeX{} font metrics, virtual font metrics etc.
% 
% \changes{v0.2}{2025-02-05}{Renamed variable: \texttt{fnt.keeptempfiles} (was \texttt{keeptempfiles}).}
% \DescribelVar{fnt.keeptempfiles}\texttt{fnt.keeptempfiles} \istable{globs}
% Files to copy to \texttt{fnt.keeptempdir}.
% 
% \changes{v0.2}{2025-01-25}{New variable: \texttt{fnt.mapfiles\_sys}.}
% \DescribelVar{fnt.mapfiles_sys}\texttt{fnt.mapfiles\_sys} \istable{globs}
% Map file fragments to install when checking.
% If unset, \file{cm.map}, \file{cm-super-t1.map}, \file{cm-super-ts1.map} and \file{lm.map} are used.
% 
% \changes{v0.2}{2025-01-25}{New variable: \texttt{fnt.mapfiles\_add}.}
% \DescribelVar{fnt.mapfiles_add}\texttt{fnt.mapfiles\_add} \istable{globs}
% Additional map file fragments to install when checking.
% 
% \changes{v0.2}{2025-02-05}{Renamed variable: \texttt{fnt.mapmakers} (was \texttt{mapmakers}).}
% \DescribelVar{fnt.mapmakers}\texttt{fnt.mapmakers} \istable{globs}
% Source files \texttt{fnt.finst()} should compile to generate map file fragments etc.
% 
% \changes{v0.3}{2025-02-12}{New variable: \texttt{fnt.needs_fontinst}.}
% \DescribelVar{fnt.needs_fontinst}\texttt{fnt.needs_fontinst} \isbool \par\noindent
% \texttt{true} by default.
% Determines whether \file{.etx} and \file{.mtx} files under \path{tex/fontinst} in the distribution's \textsc{tds} tree are copied to \texttt{fnt.fntdir} if \texttt{fnt.buildinit()} is called.
% 
% Unless you are using files with conflicting names (or whose names would conflict if the prefix \texttt{fontscripts-} was removed from those supplied by this package), it is probably best to change the default only if you are certain you do not need the files, you have set \texttt{fnt.buildsearch} \texttt{true} or you run into trouble.
% Note that not only \texttt{fontinst} make use of the files installed by \lpack{fontinst} and \lpack{afm2pl}.
%
% \changes{v0.2}{2025-01-28}{Newly documented variable: \texttt{fnt.pkgbase}.}
% \DescribelVar{fnt.pkgbase}\texttt{fnt.pkgbase} \isstring{identifier}
% May be used to set a string identifying the package if \texttt{fntbuild} has trouble determining one automatically\footnote{%
%   None of the packages I've prepared with \texttt{fntbuild} have required this.
%   Usually, the script should do a reasonable job of discovering a suitable value automatically.%
% }.
%
% \changes{v0.2}{2025-02-04}{New variable: \texttt{fnt.regress}.}
% \DescribelVar{fnt.regress}\texttt{fnt.regress}\isstring{filename} \par\noindent
%
% The file containing regression tests for use in auto-generating tests.
% \file{fntbuild-regression-test.tex} provides a set used by default.
%
% \changes{v0.3}{2025-02-20}{Include warning re.\ use of subset declarations in \file{.fd} files. \href{https://github.com/latex3/latex2e/issues/1669}{GitHub \#1669}.}
% \changes{v0.4}{2025-06-25}{Update warning re.\ subset declarations with date of fix in format.}
% \textbf{Warning!
%   The implementation of subset declarations in font definition files is broken in \LaTeX{} prior to 2025-06-01\footnote{%
%     \href{https://github.com/latex3/latex2e/issues/1669}{GitHub \#1669}.%
%   }.
%   If your package should work on older versions of \LaTeX{}, do NOT remove declarations from \file{.sty} files unless present in the format itself.}
%
% \changes{v0.2}{2025-01-25}{New variable: \texttt{fnt.subset}.}
% \DescribelVar{fnt.subset}\texttt{fnt.subset}\isbool \par\noindent
% \texttt{nil} by default.
% Set to any value other than \texttt{nil} or \texttt{false} to enable subset declarations.
%
% \textbf{Note this variable must be given a non-default value if subset encoding declarations should be added.
%   Specifying definitions, files and templates only determines what will be done IF anything is done at all.}
%
% \changes{v0.2}{2025-01-25}{New variable: \texttt{fnt.subsetdefns}.}
% \DescribelVar{fnt.subsetdefns}\texttt{fnt.subsetdefns} \istable{}
% If different families should get different subset values, the various settings should be configured here.
%
% For example, the following code sets the subset for \meta{family 1} to \texttt{1} and that for \meta{family 2} to \texttt{3}.
% \iffalse
%<*verb>
% \fi
\begin{verbatim}
fnt.subsetdefns.<family 1> = "1"
fnt.subsetdefns.<family 2> = "3"
\end{verbatim}
% \iffalse
%</verb>
% \fi
% 
% \changes{v0.2}{2025-01-25}{New variable: \texttt{fnt.subsetfiles}.}
% \DescribelVar{fnt.subsetfiles}\texttt{fnt.subsetfiles} \istable{globs}
% Font definition files into which encoding subset declarations should be inserted.
% 
% \changes{v0.2}{2025-01-25}{New variable: \texttt{fnt.subsettemplate}.}
% \DescribelVar{fnt.subsettemplate}\texttt{fnt.subsettemplate} \isstring{}
% Template to use when constructing subset declarations.
% 
% By default, the template simply plugs family and value pairs into a standard declaration as shown in \cref{tab:varia-n}.
% However, it is also possible to use more complex declarations.
% For example, \pkg{cfr-lm} does not actually include any \texttt{TS1} fonts at all.
% Instead, it just says which Latin Modern family should be used for which \pkg{cfr-lm} family.
% So the only sensible approach is to use whichever subset declarations the relevant Latin Modern font definitions use.
% Unfortunately, these are provided by the \LaTeXe{} format rather than \pkg{lm}, besides which it is at least theoretically possible the appropriate subsets might change and \pkg{cfr-lm} has no control over this.
% For this reason, the package simply says ‘for \pkg{cfr-lm} family $x$, use whichever subset is declared for Latin Modern family $y$’ for each pair of families $x$ and $y$.
%
% This is setup using the following line in the \file{build.lua}:
% \iffalse
%<*verb>
% \fi
\begin{verbatim}
fnt.subsettemplate = 
  "\\ExpandArgs {nnc} \\DeclareEncodingSubset {TS1} {$FONTFAMILY} {TS1:$SUBSET}"
\end{verbatim}
% \iffalse
%</verb>
% \fi
% Rather than numerical values, therefore, \pkg{cfr-lm} declares subset encodings such as
% \iffalse
%<*verb>
% \fi
\begin{verbatim}
fnt.subsetdefns.clms = "lmss" 
\end{verbatim}
% \iffalse
%</verb>
% \fi
% The result is that \LaTeXe{} will use whichever subset is declared for \texttt{lmss} for \texttt{clms}, which, after all, is no more than a fancy alias for \texttt{lmss} in the \texttt{TS1} encoding.
%
% \changes{v0.2}{2025-02-04}{New variable: \texttt{fnt.tablestemp}.}
% \DescribelVar{fnt.tablestemp}\texttt{fnt.tablestemp}\isstring{filename} \par\noindent
%
% The file containing the template to use when auto-generating font tables.
% \file{fntbuild-tables.tex} provides the default template.
%
% \changes{v0.2}{2025-02-04}{New variable: \texttt{fnt.testtemp}.}
% \DescribelVar{fnt.testtemp}\texttt{fnt.testtemp}\isstring{filename} \par\noindent
%
% The file containing the template to use when auto-generating font tests.
% \file{fntbuild-test.lvt} provides the default template.
%
% \changes{v0.2}{2025-02-05}{Renamed variable: \texttt{fnt.vendor} (was \texttt{vendor}).}
% \DescribelVar{fnt.vendor}\texttt{fnt.vendor} \isstring{fnt.vendor}
% Vendor directory for font installation such as \texttt{public} or \texttt{arkandis}.
% 
% \section{Configuration}\label{sec:config}
%
% \changes{v0.2}{2025-01-25}{Support for configuration files. See \cref{sec:config}.}
%
% For simple cases, configuration may be done in \file{build.lua}, along with generic \pkg{l3build} customisation.
% For more elaborate modifications --- or cases where a subset of configuration options should be shared between modules --- settings may be placed in one or more \file{fntbuild-config.lua} files.
%
% Variables are set sequentially from the following files:
% \begin{enumerate}
%   \item \file{fntbuild-vars.lua} is loaded.
%     This ensures required variables get default values and modifies a number of \pkg{l3build} defaults.
%   \item A \texttt{kpse}-based search is performed.
%     If \file{fntbuild-config.lua} is found, it is loaded.
%   \item If it exists, \texttt{maindir/fntbuild-config.lua} is loaded.
%   \item If it exists, \texttt{sourcedir/fntbuild-config.lua} is loaded.
% \end{enumerate}
% So more specific settings will override more general ones.
%
% \section{Templates}\label{sec:templates}
%
% \changes{v0.2}{2025-02-05}{Default templates. Simplified template configuration and search.}
% By default, \file{fntbuild.lua} is able to utilise two kinds of \texttt{tex} template.
% The names of these files are stored in \texttt{fnt.testtemp} and \texttt{fnt.tablestemp}.
% If no local alternatives are provided, \texttt{fntbuild.lua} will use \lpack{kpathsea} to locate them, falling back to those included in the package and installed in \path{tex/latex/fontscripts/} under the \texttt{TEXMFDIST} tree by default.
%
% The default files may be overridden by either or both of the following methods:
% \begin{enumerate}
%   \item altering the values of \texttt{fnt.testtemp} and/or \texttt{fnt.tablestemp};
%   \item providing local copies of the files named in \texttt{fnt.testtemp} and/or \texttt{fnt.tablestemp}.
% \end{enumerate}
%
% The default contents are listed in \cref{subsec:temp}.
%
% \subsection{Font Tables}\label{subsec:tables}
%
% A template for producing font tables as part of package documentation is provided.
% The template is used in \file{fntbuild.lua}'s \texttt{doc\_init()} hook to generate \texttt{tex} files populated with font information from font definition files.
% This is then compiled by \texttt{l3build doc} to produce font tables.
%
% Alternative content may be provided by supplying a local copy of \file{fntbuild-tables.tex} or specifying an alternative template in \texttt{fnt.tablestemp}. 
% If \texttt{TABLES} occurs in the template, \file{fntbuild.lua} will replace it with a \env{document} environment containing a series of commands of the form \cs{sampletable}\marg{encoding}\marg{family}\marg{series}\marg{shape}, where \meta{encoding}, \meta{family}, \meta{series} and \meta{shape} are  derived from the font definition files.
%
% The default template is listed in \cref{subsubsec:tablestemp}.
%
% \subsection{Font Tests}\label{subsec:tests}
%
% If a template is found in the font test directory, it will be used in \file{fntbuild.lua}'s \texttt{check\_init()} hook to generate \texttt{lvt} files populated with font information from font definition files.
% These tests are then compiled by \texttt{l3build check} as part of the test suite.
% Certain file patterns are excluded from testing.
% In particular, separate tests are not generated for \texttt{ts1} \texttt{fd} files because these families are typically better tested along with their \texttt{t1} counterparts.
%
% Alternative content may be provided by supplying a local copy of \file{fntbuild-test.lvt} or specifying an alternative template in \texttt{fnt.testtemp}. 
% If \texttt{SAMP} occurs in the template, \file{fntbuild.lua} will replace it with a series of tests derived from the font definition files.
% These tests should typically use commands provided explicitly or implicitly by the file specified in \texttt{fnt.regress}.
% By default, tests utilise a macro \cs{sampler} defined to absorb four arguments of the form \marg{encoding}\marg{family}\marg{series}\marg{shape}.
% The result is a document containing various pieces supplied by \lpack{fonttable}.
%
% A custom test suite may be specified by supplying a local copy of \file{fntbuild-regression-test.tex}, giving an alternative in \texttt{fnt.regress} or defining a different set of tests in the template specified by \texttt{fnt.testtemp}.
%
% \textbf{Note that you MUST supply a custom template if you change \texttt{fnt.regress}.}
% It is, however, fine to supply a customised \file{fntbuild-regression-test.tex} for use with the default template.
% Only if you want to use a different filename for the test suite is a custom template required.
%
% The default template is listed in \cref{subsubsec:testtemp}.
% The default test suite is listed in \cref{subsec:regress}.
%
% \section{Customisation}\label{sec:custom}
%
% As in the case of \lpack{l3build}, you can replace functions and targets at will, albeit on a much more limited scale.
% Although it would be better to just not use \file{fntbuild.lua} at all if you want to redefine everything, it can make sense to replace \texttt{fnt.fontinst()} if, say, you want to use different font creation tools but make use of the functions for stashing generated files, generating font tables, testing etc.
%
% For example, \lpack{berenisadf} was built using a \file{build.lua} containing 
% \newcommand*\FancyVerbStopString{-- STOP doc eg}
% \newcommand*\FancyVerbStartString{-- START doc eg}
% \VerbatimInput[gobble=0,numbers=left,numberblanklines=true]{berenis-build.lua}
% \let\FancyVerbStopString\relax
% \let\FancyVerbStartString\relax
% That is, \lpack{berenisadf} doesn't use any built-in build function at all, but \texttt{fnt.fntmake()}, which simply invokes \texttt{gnu} \texttt{make} and calls \texttt{fnt.fntkeeper()}\footnote{%
%   It does not ensure encoding names are unique because it uses no custom encodings of its own.%
% }.
% Note the use of \texttt{fnt.buildinit()} to setup the build environment.
%
% \subsection{Examples}\label{subsec:eeiau}
% 
% The latest versions of the following packages were developed using \lpack{fontscripts} and build using \lpack{l3build}\footnote{%
%   Note that it is in no sense a dependency; it contains nothing ever required in typesetting.%
% }.
% Full details, including the use of templates and scaling are available on \codeberg{} or \github{} as part of the code repository.
% \href{https://codeberg.org/cfr/ebgaramond-maths}{\lpack{ebgaramond-maths}} provides a further example.
% \begin{itemize}
%   \item \textsc{Gust}:
%   \begin{itemize}
%     \item \lpack{cfr-lm}
%   \end{itemize}
%   \item \textsc{Arkandis}:
%   \begin{itemize}
%     \item \lpack{baskervaldadf}
%     \item \lpack{berenisadf}
%     \item \lpack{electrumadf}
%     \item \lpack{librisadf}
%     \item \lpack{romandeadf} 
%     \item \lpack{venturisadf}.
%   \end{itemize}
% \end{itemize}
%
% \subsection{Templates}\label{subsec:temp}
%
% Templates for auto-generating font tables and regression tests.
%
% \subsubsection{Font Tables Template}\label{subsubsec:tablestemp}
%
% \iffalse
%<*tables>
% \fi
% \begin{template}{fntbuild-tables.tex}
% Default content:
%    \begin{macrocode}
\pdftracingfonts=1
\RequirePackage{svn-prov}
\ProvidesFileSVN[fnt-tables.tex]{$Id: fontscripts.dtx 11047 2025-06-25 08:20:08Z cfrees $}[v0.2 \revinfo][\filebase: font table template]
\DefineFileInfoSVN
%    \end{macrocode}
% \iffalse
% ^^A Paid â defnyddio \GetFileInfoSVN*/\GetFileInfoSVN{} yn y fan hon!!
% \fi
%    \begin{macrocode}
\documentclass[10pt,a4paper]{article}
\usepackage{geometry}
\usepackage{fonttable}
\newcommand\sampletable[4]{%
  #1/#2/#3/#4:\par\noindent
  \xfonttable{#1}{#2}{#3}{#4}%
  \clearpage
}

TABLES

%    \end{macrocode}
% \end{template}
% \iffalse
%</tables>
% \fi
%
% \subsubsection{Test Template}\label{subsubsec:testtemp}
% 
% \iffalse
%<*lvt>
% \fi
% \begin{template}{fntbuild-test.lvt}
% Default content:
%    \begin{macrocode}
\RequirePackage{svn-prov}
\ProvidesFileSVN[fnt-test.lvt]{$Id: fontscripts.dtx 11047 2025-06-25 08:20:08Z cfrees $}[v0.0 \revinfo][fontscripts: l3build test template]
\listfiles
\input regression-test.tex\relax
\input fntbuild-regression-test.tex\relax
\setlength{\parindent}{0pt}
\textheight=250mm
\textwidth=160mm
\oddsidemargin=0mm
\evensidemargin=0mm
\headheight=0pt
\headsep=0pt
\topmargin=-10mm
\marginparwidth=0pt
\marginparsep=0pt

SAMP

%    \end{macrocode}
% \end{template}
% \iffalse
%</lvt>
% \fi
% 
% \subsection{Regression Tests}\label{subsec:regress}
% 
% \iffalse
%<*tests>
% \fi
% \begin{template}{fntbuild-regression-test.tex}
% Simple regression test suite for fonts focused on T1 and TS1 encodings.
%    \begin{macrocode}
\pdftracingfonts=1
\RequirePackage{svn-prov}
\ProvidesFileSVN[fnt-tests.tex]{$Id: fontscripts.dtx 11047 2025-06-25 08:20:08Z cfrees $}[v0.0 \revinfo][fontscripts: l3build tests]
\documentclass[10pt,a4paper]{article}
\usepackage{fonttable}
\usepackage{tracefnt}% infoshow is default; debugshow traces maths fonts, too
\PassOptionsToPackage{nfssext-cfr}{debug}
\newcounter{tccharcnt}
\NewDocumentCommand \tcfnttest { o m }
{%
  \iftctest
    \stepcounter{tccharcnt}{\normalfont\thetccharcnt:~}%
    % \texttt{\textbackslash #2}:
    \IfValueTF{#1}{%
      \csname #2\endcsname {#1}%
    }{%
      \csname #2\endcsname
    }%
  \fi
}
\newif\iftctest
\tctesttrue
\makeatletter
\newcommand\sampler{}
\def\sampler#1#2#3#4{%
  \setcounter{tccharcnt}{0}%
  % \tracingoutput=1
  #1 / #2 / #3 / #4 :

  \tracinglostchars\thr@@
  % \tracingonline\thr@@
  \showoutput
  \pikfont{#1}{#2}{#3}{#4}%
  \aztext
  
  \AZtext
  
  \digitstext 
  
  \punctext
  
  \knutext

  Ææ Ŵŵ Ŷŷ Šš € 

  percent: \%

  \begin{tabular}{llllllllll}
    \tcfnttest{S} & 
    \tcfnttest[I]{capitalacute} & 
    \tcfnttest[I]{capitalbreve} & 
    \tcfnttest[I]{capitalcaron} & 
    \tcfnttest[I]{capitalcedilla} & 
    \tcfnttest[I]{capitalcircumflex} & 
    \tcfnttest[I]{capitaldieresis} & 
    \tcfnttest[I]{capitaldotaccent} & 
    \tcfnttest[I]{capitalgrave} & 
    \tcfnttest[I]{capitalhungarumlaut} \\
    \tcfnttest[I]{capitalnewtie} & 
    \tcfnttest[I]{capitalogonek} & 
    \tcfnttest[I]{capitalring} & 
    \tcfnttest[I]{capitaltie} & 
    \tcfnttest[I]{capitaltilde} & 
    \tcfnttest{textacutedbl} & 
    \tcfnttest{textasciiacute} & 
    \tcfnttest{textasciibreve} & 
    \tcfnttest{textasciicaron} & 
    \tcfnttest{textasciicircum} \\
    \tcfnttest{textasciidieresis} & 
    \tcfnttest{textasciigrave} & 
    \tcfnttest{textasciimacron} & 
    \tcfnttest{textasciitilde} & 
    \tcfnttest{textasteriskcentered} & 
    \tcfnttest{textbackslash} & 
    \tcfnttest{textbaht} & 
    \tcfnttest{textbar} & 
    \tcfnttest{textbardbl} & 
    \tcfnttest{textbigcircle} \\
    \tcfnttest{textblank} & 
    \tcfnttest{textborn} & 
    \tcfnttest{textbraceleft} & 
    \tcfnttest{textbraceright} & 
    \tcfnttest{textbrokenbar} & 
    \tcfnttest{textbullet} & 
    \tcfnttest{textcelsius} & 
    \tcfnttest{textcent} & 
    \tcfnttest{textcentoldstyle} & 
    \tcfnttest{textcircledP} \\
    \tcfnttest{textcolonmonetary} & 
    \tcfnttest{textcompwordmark} & 
    \tcfnttest{textcopyleft} & 
    \tcfnttest{textcopyright} & 
    \tcfnttest{textcurrency} & 
    \tcfnttest{textdagger} & 
    \tcfnttest{textdaggerdbl} & 
    \tcfnttest{textdblhyphen} & 
    \tcfnttest{textdblhyphenchar} & 
    \tcfnttest{textdegree} \\
    \tcfnttest{textdied} & 
    \tcfnttest{textdiscount} & 
    \tcfnttest{textdiv} & 
    \tcfnttest{textdivorced} & 
    \tcfnttest{textdollar} & 
    \tcfnttest{textdollaroldstyle} & 
    \tcfnttest{textdong} & 
    \tcfnttest{textdownarrow} & 
    \tcfnttest{texteightoldstyle} & 
    \tcfnttest{textestimated} \\
    \tcfnttest{texteuro} & 
    \tcfnttest{textfiveoldstyle} & 
    \tcfnttest{textflorin} & 
    \tcfnttest{textfouroldstyle} & 
    \tcfnttest{textfractionsolidus} & 
    \tcfnttest{textgravedbl} & 
    \tcfnttest{textguarani} & 
    \tcfnttest{textinterrobang} & 
    \tcfnttest{textinterrobangdown} & 
    \tcfnttest{textlangle} \\
    \tcfnttest{textlbrackdbl} & 
    \tcfnttest{textleaf} & 
    \tcfnttest{textleftarrow} & 
    \tcfnttest{textlira} & 
    \tcfnttest{textlnot} & 
    \tcfnttest{textlquill} & 
    \tcfnttest{textmarried} & 
    \tcfnttest{textmho} & 
    \tcfnttest{textminus} & 
    \tcfnttest{textmu} \\
    \tcfnttest{textmusicalnote} & 
    \tcfnttest{textnaira} & 
    \tcfnttest{textnaira} & 
    \tcfnttest{textnineoldstyle} & 
    \tcfnttest{textnumero} & 
    \tcfnttest{textohm} & 
    \tcfnttest{textonehalf} & 
    \tcfnttest{textoneoldstyle} & 
    \tcfnttest{textonequarter} & 
    \tcfnttest{textonesuperior} \\
    \tcfnttest{textopenbullet} & 
    \tcfnttest{textordfeminine} & 
    \tcfnttest{textordmasculine} & 
    \tcfnttest{textparagraph} & 
    \tcfnttest{textperiodcentered} & 
    \tcfnttest{textpertenthousand} & 
    \tcfnttest{textperthousand} & 
    \tcfnttest{textpeso} & 
    \tcfnttest{textpilcrow} & 
    \tcfnttest{textpm} \\
    \tcfnttest{textquotedblleft} & 
    \tcfnttest{textquotedblright} & 
    \tcfnttest{textquoteleft} & 
    \tcfnttest{textquoteright} & 
    \tcfnttest{textquotesingle} & 
    \tcfnttest{textquotestraightbase} & 
    \tcfnttest{textquotestraightdblbase} & 
    \tcfnttest{textrangle} & 
    \tcfnttest{textrbrackdbl} & 
    \tcfnttest{textrecipe} \\
    \tcfnttest{textreferencemark} & 
    \tcfnttest{textregistered} & 
    \tcfnttest{textrightarrow} & 
    \tcfnttest{textrquill} & 
    \tcfnttest{textsection} & 
    \tcfnttest{textservicemark} & 
    \tcfnttest{textsevenoldstyle} & 
    \tcfnttest{textsixoldstyle} & 
    \tcfnttest{textsterling} & 
    \tcfnttest{textsurd} \\
    \tcfnttest{textthreeoldstyle} & 
    \tcfnttest{textthreequarters} & 
    \tcfnttest{textthreequartersemdash} & 
    \tcfnttest{textthreesuperior} & 
    \tcfnttest{texttildelow} & 
    \tcfnttest{texttimes} & 
    \tcfnttest{texttrademark} & 
    \tcfnttest{texttwelveudash} & 
    \tcfnttest{texttwooldstyle} & 
    \tcfnttest{texttwosuperior} \\
    \tcfnttest{textuparrow} & 
    \tcfnttest{textwon} & 
    \tcfnttest{textyen} & 
    \tcfnttest{textzerooldstyle} & 
                                 &
                                 &
                                 &
                                 &
                                 &
                                 \\
  \end{tabular}
  
  \vfil\break
}

\makeatother
%    \end{macrocode}
% \end{template}
% \iffalse
%</tests>
% \fi
% 
% 
%
% \MaybeStop{%
% \def\glossaryname{Change History}%
% \PrintChanges
% \PrintIndex
% }
% 
% \section{Implementation}
%
% \iffalse
%<*doc>
\RequirePackage{svn-prov}
\def\GetFileBaseName#1-#2\nil{#1}
\edef\MyFileBaseName{\expandafter\GetFileBaseName\jobname\nil}
\ProvidesFileSVN[\MyFileBaseName doc]{$Id: fontscripts.dtx 11047 2025-06-25 08:20:08Z cfrees $}[v0.0 \revinfo]
\DefineFileInfoSVN
\AddToHook{begindocument}{\OnlyDescription}
\input{\MyFileBaseName.dtx}
%</doc>
% \fi
% \iffalse
%<*doc-code>
\RequirePackage{svn-prov}
\def\GetFileBaseName#1-#2\nil{#1}
\edef\MyFileBaseName{\expandafter\GetFileBaseName\jobname\nil}
\ProvidesFileSVN[\MyFileBaseName code]{$Id: fontscripts.dtx 11047 2025-06-25 08:20:08Z cfrees $}[v0.0 \revinfo]
\DefineFileInfoSVN
\input{\MyFileBaseName.dtx}
%</doc-code>
% \fi
%
% 
% \subsection{Script Fragments}\label{subsec:lua}
% 
% These are not standalone scripts, but files to be read by \texttt{l3build} to provide additional and/or divergent functions.
% Note that several are based on LPPL files not on \ctan{}, as the \LaTeX{} Project does not package their build scripts. 
% 
% 
% \subsubsection{fntbuild.lua}\label{subsubsec:fntbuild-lua}
% 
% \begin{luafrag}{fntbuild.lua}
%   Use \texttt{dofile()} in \file{build.lua} to add to \lpack{l3build}.
%   \VerbatimInput[gobble=0,numbers=left]{fntbuild.lua}
% \end{luafrag}
%
% \changes{v0.2}{2025-01-25}{Split most of \texttt{fntbuild.lua} into other files, similar to the existing \texttt{fntbuild-ctan.lua}.}
% \subsubsection{fntbuild-build.lua}\label{subsubsec:fntbuild-build-lua}
% 
% \begin{luafrag}{fntbuild-build.lua}
%   Included by \file{fntbuild.lua}.
%   Functions used when building fonts for \texttt{fnttarg}.
%   \VerbatimInput[gobble=0,numbers=left]{fntbuild-build.lua}
% \end{luafrag} 
% 
% \subsubsection{fntbuild-check.lua}\label{subsubsec:fntbuild-check-lua}
% 
% \begin{luafrag}{fntbuild-check.lua}
%   Included by \file{fntbuild.lua}.
%   Configures \pkg{l3build} hook to auto-generate (additional) test files if templates are provided.
%   \VerbatimInput[gobble=0,numbers=left]{fntbuild-check.lua}
% \end{luafrag} 
% 
% \subsubsection{fntbuild-ctan.lua}\label{subsubsec:fntbuild-ctan-lua}
% 
% \begin{luafrag}{fntbuild-ctan.lua}
%   Included by \file{fntbuild.lua}.
%   Changes the way the \texttt{ctan} target builds the archive to match \ctan{} requirements for font distributions.
%   \VerbatimInput[gobble=0,numbers=left]{fntbuild-ctan.lua}
% \end{luafrag} 
% 
% \subsubsection{fntbuild-doc.lua}\label{subsubsec:fntbuild-doc-lua}
% 
% \begin{luafrag}{fntbuild-doc.lua}
%   Included by \file{fntbuild.lua}.
%   Configures \pkg{l3build}'s documentation hook to produce template-based documentation supplement e.g.~for auto-generating font tables.
%   \VerbatimInput[gobble=0,numbers=left]{fntbuild-doc.lua}
% \end{luafrag} 
% 
% \subsubsection{fntbuild-utils.lua}\label{subsubsec:fntbuild-utils-lua}
% 
% \begin{luafrag}{fntbuild-utils.lua}
%   Included by \file{fntbuild.lua}.
%   Internal functions used in various places.
%   \VerbatimInput[gobble=0,numbers=left]{fntbuild-utils.lua}
% \end{luafrag} 
% 
% \subsubsection{fntbuild-vars.lua}\label{subsubsec:fntbuild-vars-lua}
% 
% \begin{luafrag}{fntbuild-vars.lua}
%   Included by \file{fntbuild.lua}.
%   Additional variables and changes to default values of existing \pkg{l3build} variables.
%   These may be overridden in \file{fntbuild-config.lua} and/or \file{build.lua}.
%   \VerbatimInput[gobble=0,numbers=left]{fntbuild-vars.lua}
% \end{luafrag} 
% 
% 
% \subsection{Metrics}\label{subsec:mtx}
%
% These files influence the characters which end up in the \TeX{} fonts.
% For example, they may construct otherwise missing glyphs or adjust kerning pairs.
%
% 
% \subsubsection{fontscripts-dotscbuild.mtx}\label{subsubsec:dotscbuild}
% 
% \iffalse
%<*dotscbuild>
% \fi
% \begin{metrics}{fontscripts-dotscbuild.mtx}
% \changes{v0.0}{2025-02-10}{Filename prefix for Karl.}
%    \begin{macrocode}
\relax

\metrics

\needsfontinstversion{1.917}

\usemtxpackage{ltcmds}
\ProvidesMtxPackage{dotscbuild}

\begincomment
\section{Proper latin small capitals}

\subsection{Some utility commands}
\endcomment

\setcommand\setcsctopglyph#1#2#3#4{
   \ifareglyphs{#2.sc,#3}\then
      \setglyph{#1}
         \topaccent{#2.sc}{#3}{#4}
      \endsetglyph
      \setleftrightkerning{#1}{#2.sc}{1000}
   \Fi
}
\setcommand\setcscbotglyph#1#2#3#4{
   \ifareglyphs{#2.sc,#3}\then
      \setglyph{#1}
         \botaccent{#2.sc}{#3}{#4}
      \endsetglyph
      \setleftrightkerning{#1}{#2.sc}{1000}
   \Fi
}
\begincomment\medskip
A \textbf{Try: Set glyph} is an ordinary \textbf{Set glyph} which is 
conditional on that a set of glyphs (those used to construct the 
composite glyph) are available. It is technically e.g.
\begin{quotation}
  \setcsctopglyph{\macroparameter{1}}{\macroparameter{2}}%
    {\macroparameter{3}}{\macroparameter{4}}
  \setcscbotglyph{\macroparameter{1}}{\macroparameter{2}}%
    {\macroparameter{3}}{\macroparameter{4}}\par
\end{quotation}
(both of which are implemented in the code as simple four argument 
commands), but in the list of commands below those two commands will 
be typeset as
\resetcommand\setcsctopglyph#1#2#3#4{%
   \Aheading{Try: Set glyph `\TypesetStringExpression{#1}'}
   \topaccent{#2.sc}{#3}{#4}
   \setleftrightkerning{#1}{#2.sc}{1000}
}
\resetcommand\setcscbotglyph#1#2#3#4{%
   \Aheading{Try: Set glyph `\TypesetStringExpression{#1}'}
   \botaccent{#2.sc}{#3}{#4}
   \setleftrightkerning{#1}{#2.sc}{1000}
}
\begin{quotation}
  \setcsctopglyph{\macroparameter{1}}{\macroparameter{2}}%
    {\macroparameter{3}}{\macroparameter{4}}
  \setcscbotglyph{\macroparameter{1}}{\macroparameter{2}}%
    {\macroparameter{3}}{\macroparameter{4}}
\end{quotation}
\endcomment

\ifareglyphs{L,l.sc}\then
   \setint{smallcapsscale}{
     \div{\mul{1000}{\height{l.sc}}}{\height{L}}
   }
\Else
   \setint{smallcapsscale}{800}
\Fi


\setcsctopglyph{aacute.sc}{a}{acute}{500}
\setcsctopglyph{abreve.sc}{a}{breve}{500}
\setcsctopglyph{acircumflex.sc}{a}{circumflex}{500}
\setcsctopglyph{adieresis.sc}{a}{dieresis}{500}
\setcsctopglyph{agrave.sc}{a}{grave}{500}
\setcscbotglyph{aogonek.sc}{a}{ogonek}{900}
\setcsctopglyph{aring.sc}{a}{ring}{500}
\setcsctopglyph{atilde.sc}{a}{tilde}{500}

\setcsctopglyph{cacute.sc}{c}{acute}{500}
\setcsctopglyph{ccaron.sc}{c}{caron}{500}
\setcsctopglyph{ccedilla.sc}{c}{cedilla}{500}

\setcsctopglyph{dcaron.sc}{d}{caron}{500}

\setcsctopglyph{eacute.sc}{e}{acute}{500}
\setcsctopglyph{ecaron.sc}{e}{caron}{500}
\setcsctopglyph{ecircumflex.sc}{e}{circumflex}{500}
\setcsctopglyph{edieresis.sc}{e}{dieresis}{500}
\setcsctopglyph{egrave.sc}{e}{grave}{500}
\setcscbotglyph{eogonek.sc}{e}{ogonek}{850}

\setcsctopglyph{gbreve.sc}{g}{breve}{500}

\setcsctopglyph{iacute.sc}{i}{acute}{500}
\setcsctopglyph{icircumflex.sc}{i}{circumflex}{500}
\setcsctopglyph{idieresis.sc}{i}{dieresis}{500}
\setcsctopglyph{idotaccent.sc}{i}{dotaccent}{500}
\setcsctopglyph{igrave.sc}{i}{grave}{500}

\setcsctopglyph{lacute.sc}{l}{acute}{250}

\ifareglyphs{l.sc,quoteright}\then
   \setglyph{lcaron.sc}
      \glyph{l.sc}{1000}
      \ifisint{monowidth}\then\Else \movert{-100} \Fi
      \glyph{quoteright}{\int{smallcapsscale}}
   \endsetglyph
   \setleftkerning{lcaron.sc}{l.sc}{1000}
\Fi

\setcsctopglyph{nacute.sc}{n}{acute}{500}
\setcsctopglyph{ncaron.sc}{n}{caron}{500}
\setcsctopglyph{ntilde.sc}{n}{tilde}{500}

\setcsctopglyph{oacute.sc}{o}{acute}{500}
\setcsctopglyph{ocircumflex.sc}{o}{circumflex}{500}
\setcsctopglyph{odieresis.sc}{o}{dieresis}{500}
\setcsctopglyph{ograve.sc}{o}{grave}{500}
\setcsctopglyph{ohungarumlaut.sc}{o}{hungarumlaut}{500}
\setcsctopglyph{otilde.sc}{o}{tilde}{500}

\setcsctopglyph{racute.sc}{r}{acute}{500}
\setcsctopglyph{rcaron.sc}{r}{caron}{500}

\setcsctopglyph{sacute.sc}{s}{acute}{500}
\setcsctopglyph{scaron.sc}{s}{caron}{500}
\setcscbotglyph{scedilla.sc}{s}{cedilla}{500}

\setcsctopglyph{tcaron.sc}{t}{caron}{500}
\setcscbotglyph{tcedilla.sc}{t}{cedilla}{500}

\setcsctopglyph{uacute.sc}{u}{acute}{500}
\setcsctopglyph{ucircumflex.sc}{u}{circumflex}{500}
\setcsctopglyph{udieresis.sc}{u}{dieresis}{500}
\setcsctopglyph{ugrave.sc}{u}{grave}{500}
\setcsctopglyph{uhungarumlaut.sc}{u}{hungarumlaut}{500}
\setcsctopglyph{uring.sc}{u}{ring}{500}

\setcsctopglyph{yacute.sc}{y}{acute}{500}
\setcsctopglyph{ydieresis.sc}{y}{dieresis}{500}

\setcsctopglyph{zacute.sc}{z}{acute}{500}
\setcsctopglyph{zcaron.sc}{z}{caron}{500}
\setcsctopglyph{zdotaccent.sc}{z}{dotaccent}{500}


\begincomment
\subsection{Other glyphs that can be built}
\endcomment

\ifisglyph{dcroat.sc}\then
   \setglyph{dbar.sc}
      \glyph{dcroat.sc}{1000}
   \endsetglyph
   \setleftrightkerning{dbar.sc}{dcroat.sc}{1000}
\Else\ifisglyph{eth.sc}\then
   \setglyph{dbar.sc}
      \glyph{eth.sc}{1000}
   \endsetglyph
   \setleftrightkerning{dbar.sc}{eth.sc}{1000}
\Fi\Fi


\endmetrics
%    \end{macrocode}
% \end{metrics}
% \iffalse
%</dotscbuild>
% \fi
% 
% 
% \subsubsection{fontscripts-newlatin-dotsc.mtx}\label{subsubsec:newlatin-dotsc}
% 
% \iffalse
%<*newlatin-dotsc>
% \fi
% \begin{metrics}{fontscripts-newlatin-dotsc.mtx}
% based on \file{newlatin.mtx} - because I couldn't figure out how to pass options...
% \changes{v0.0}{2025-02-10}{Filename prefix for Karl.}
% \changes{v0.2}{2025-01-25}{Re-remove load of \file{lsfake}.}
%    \begin{macrocode}
\relax

\documentclass[twocolumn]{article}

\metrics

\needsfontinstversion{1.924}

\usemtxpackage{llbuild}

\usemtxpackage{lubuild}

\ifoption{nosc}\then \Else

\usemtxpackage{dotscbuild}
\usemtxpackage{dotscmisc}

\Fi

\usemtxpackage{ltpunct}


\usemtxpackage{ltcmds}

\unfakable{Gamma}
\unfakable{Delta}
\unfakable{Theta}
\unfakable{Lambda}
\unfakable{Xi}
\unfakable{Pi}
\unfakable{Sigma}
\unfakable{Upsilon}
\unfakable{Upsilon1}
\unfakable{Phi}
\unfakable{Psi}
\unfakable{Omega}

\foreach(accent){grave,acute,circumflex,tilde,dieresis,hungarumlaut,%
      ring,caron,breve,macron,dotaccent}
   \ifisglyph{\str{accent}}\then
      \resetglyph{\str{accent}}
         \glyph{\str{accent}}{1000}
         \resetdepth{0}
      \endresetglyph
   \Fi
\endfor(accent)


\setglyph{ringfitted}
   \movert{\half{\sub{\width{A}}{\width{ring}}}}
   \glyph{ring}{1000}
   \movert{\otherhalf{\sub{\width{A}}{\width{ring}}}}
\endsetglyph



\setleftkerning{less}{guillemotleft}{1000}
\setleftkerning{greater}{guillemotright}{1000}


\endmetrics

%    \end{macrocode}
% \end{metrics}
% \iffalse
%</newlatin-dotsc>
% \fi
% 
% 
% \subsubsection{fontscripts-dotscmisc.mtx}\label{subsubsec:dotscmisc}
% 
% \iffalse
%<*dotscmisc>
% \fi
% \begin{metrics}{fontscripts-dotscmisc.mtx}
% \changes{v0.0}{2025-02-10}{Filename prefix for Karl.}
% \changes{v0.2}{2025-01-25}{Re-remove spurious kerning adjustments for triple ligatures.}
%    \begin{macrocode}
\relax

\metrics

\needsfontinstversion{1.917}

\ProvidesMtxPackage{dotscmisc}

\ifisglyph{i.sc}\then
   \setglyph{dotlessi.sc}
      \glyph{i.sc}{1000}
      \setleftrightkerning{dotlessi.sc}{i.sc}{1000}
   \endsetglyph
\Fi

\ifisglyph{j.sc}\then
   \setglyph{dotlessj.sc}
      \glyph{j.sc}{1000}
      \setleftrightkerning{dotlessj.sc}{j.sc}{1000}
   \endsetglyph
\Fi


\setint{smallcapsspacing}{0}

\ifisglyph{f.sc}\then 

\setglyph{ff.sc}
   \glyph{f.sc}{1000}
   \movert{\add{\kerning{f.sc}{f.sc}}{\int{smallcapsspacing}}}
   \glyph{f.sc}{1000}
\endsetglyph
\setrightkerning{ff.sc}{f.sc}{1000}
\setglyph{f_f.sc}
	\glyph{ff.sc}{1000}
\endsetglyph
\setrightkerning{f_f.sc}{ff.sc}{1000}

\ifisglyph{i.sc}\then
   \setglyph{fi.sc}
      \glyph{f.sc}{1000}
      \movert{\add{\kerning{f.sc}{i.sc}}{\int{smallcapsspacing}}}
      \glyph{i.sc}{1000}
   \endsetglyph
   
   \setrightkerning{fi.sc}{i.sc}{1000}

   \setglyph{f_i.sc}
	\glyph{fi.sc}{1000}
   \endsetglyph
   \setrightkerning{f_i.sc}{fi.sc}{1000}

   \setglyph{ffi.sc}
      \glyph{ff.sc}{1000}
      \movert{\add{\kerning{f.sc}{i.sc}}{\int{smallcapsspacing}}}
      \glyph{i.sc}{1000}
   \endsetglyph

   \setglyph{f_f_i.sc}
	\glyph{ffi.sc}{1000}
   \endsetglyph
   \setrightkerning{f_f_i.sc}{ffi.sc}{1000}

   \setrightkerning{ffi.sc}{i.sc}{1000}
\fi

\ifisglyph{l.sc}\then
   \setglyph{fl.sc}
      \glyph{f.sc}{1000}
      \movert{\add{\kerning{f.sc}{l.sc}}{\int{smallcapsspacing}}}
      \glyph{l.sc}{1000}
   \endsetglyph

   \setrightkerning{fl.sc}{l.sc}{1000}

   \setglyph{f_l.sc}
	\glyph{fl.sc}{1000}
   \endsetglyph
   \setrightkerning{f_l.sc}{fl.sc}{1000}

   \setglyph{ffl.sc}
      \glyph{ff.sc}{1000}
      \movert{\add{\kerning{f.sc}{l.sc}}{\int{smallcapsspacing}}}
      \glyph{l.sc}{1000}
   \endsetglyph

   \setrightkerning{ffl.sc}{l.sc}{1000}

   \setglyph{f_f_l.sc}
	\glyph{ffl.sc}{1000}
   \endsetglyph
   \setrightkerning{f_f_l.sc}{ffl.sc}{1000}

\fi
\fi % ifisglyph{f.sc}


\ifareglyphs{i.sc,j.sc}\then
   \setglyph{ij.sc}
      \glyph{i.sc}{1000}
      \movert{\add{\kerning{i.sc}{j.sc}}{\int{smallcapsspacing}}}
      \glyph{j.sc}{1000}
   \endsetglyph
   \setrightkerning{ij.sc}{j.sc}{1000}
\fi

\ifisglyph{ss.sc}\then
	\setglyph{germandbls.sc}
		\glyph{ss.sc}{1000}
	\endsetglyph
	\setleftrightkerning{germandbls.sc}{ss.sc}{1000}
\Else
	\ifisglyph{s.sc}\then
   		\setglyph{germandbls.sc}
      			\glyph{s.sc}{1000}
      			\movert{\add{\kerning{s.sc}{s.sc}}{\int{smallcapsspacing}}}
      			\glyph{s.sc}{1000}
   		\endsetglyph
   		\setleftrightkerning{germandbls.sc}{s.sc}{1000}
	\Fi
\Fi

\endmetrics
%    \end{macrocode}
% \end{metrics}
% \iffalse
%</dotscmisc>
% \fi
% 
% \subsubsection{fontscripts-unfakable.mtx}\label{subsubsec:unfakable}
% 
% This metrics file prevents \pkg{fontinst} from inserting ‘gravestones’ into virtual fonts when a glyph is missing.
% \textbf{It should be used only for \texttt{TS1}.
% It should NOT be used when using the \texttt{T1} encoding.}
% 
% \iffalse
%<*unfakable>
% \fi
% \begin{metrics}{fontscripts-unfakable.mtx}
% \changes{v0.0}{2025-02-10}{Filename prefix for Karl.}
%    \begin{macrocode}
\relax
\metrics

\ProvidesMtxPackage{fontscripts-unfakable}

\setcommand\unfakableaccent#1{%
  \message{Missing glyph `#1'}%
}
\setcommand\unfakable#1{%
  \message{Missing glyph `#1'}%
}

\endmetrics
%    \end{macrocode}
% \end{metrics}
% \iffalse
%</unfakable>
% \fi
% 
%
% \subsection{Encodings}\label{subsec:etx}
% 
% None of these files are actually used in typesetting.
% Rather, they are converted to \texttt{.enc} files, renamed and automatically edited to ensure unique encoding names.
% These names are then substituted into \texttt{.map} files.
% 
% \subsubsection{fontscripts-dotoldstyle.etx}\label{subsubsec:dotoldstyle}
% 
% \iffalse
%<*dotoldstyle>
% \fi
% \begin{encoding}{fontscripts-dotoldstyle.etx}
% \changes{v0.0}{2025-02-10}{Filename prefix for Karl.}
%    \begin{macrocode}
\relax
\encoding
	\setcommand\digit#1{#1.oldstyle}
\endencoding
%    \end{macrocode}
% \end{encoding}
% \iffalse
%</dotoldstyle>
% \fi
% 
% 
% \subsubsection{fontscripts-dotsc2.etx}\label{subsubsec:dotsc2}
% 
% \iffalse
%<*dotsc2>
% \fi
% \begin{encoding}{fontscripts-dotsc2.etx}
% \changes{v0.0}{2025-02-10}{Filename prefix for Karl.}
%    \begin{macrocode}
\relax

\encoding

\setcommand\lc#1#2{#2.sc}
\setcommand\uc#1#2{#1}
\setcommand\lctop#1#2{#2.sc}
\setcommand\uctop#1#2{#1}
\setcommand\lclig#1#2{#2.sc}
\setcommand\uclig#1#2{#1spaced}


\ifisint{capspacing}\then
   \setint{letterspacing}{\int{capspacing}}
\fi

\endencoding
%    \end{macrocode}
% \end{encoding}
% \iffalse
%</dotsc2>
% \fi
% 
% 
% \subsubsection{fontscripts-dottaboldstyle.etx}\label{subsubsec:dottaboldstyle}
% 
% \iffalse
%<*dottaboldstyle>
% \fi
% \begin{encoding}{fontscripts-dottaboldstyle.etx}
% \changes{v0.0}{2025-02-10}{Filename prefix for Karl.}
%    \begin{macrocode}
\relax
\encoding
	\setcommand\digit#1{#1.taboldstyle}
\endencoding
%    \end{macrocode}
% \end{encoding}
% \iffalse
%</dottaboldstyle>
% \fi
% 
% 
% \subsubsection{fontscripts-lining.etx}\label{subsubsec:lining}
% 
% \iffalse
%<*lining>
% \fi
% \begin{encoding}{fontscripts-lining.etx}
% \changes{v0.0}{2025-02-10}{Filename prefix for Karl.}
%    \begin{macrocode}
\relax
\encoding
\setcommand\digit#1{#1lining}
\endencoding
%    \end{macrocode}
% \end{encoding}
% \iffalse
%</lining>
% \fi
% 
% 
% \subsubsection{fontscripts-oldstyle.etx}\label{subsubsec:oldstyle}
% 
% \iffalse
%<*oldstyle>
% \fi
% \begin{encoding}{fontscripts-oldstyle.etx}
% \changes{v0.0}{2025-02-10}{Filename prefix for Karl.}
%    \begin{macrocode}
\relax
\encoding
	\setcommand\digit#1{#1oldstyle}
\endencoding
%    \end{macrocode}
% \end{encoding}
% \iffalse
%</oldstyle>
% \fi
% 
% 
% \subsubsection{fontscripts-t1-cfr.etx}\label{subsubsec:t1-cfr}
% 
% \iffalse
%<*t1-cfr>
% \fi
% \begin{encoding}{fontscripts-t1-cfr.etx}
% \changes{v0.0}{2025-02-10}{Filename prefix for Karl.}
%    \begin{macrocode}
%%
%% - The commentary in the original is deleted in this version. For 
%% information about the T1 etc., typeset the original t1.etx 
%% included with fontinst.
%% - Slots are altered to accommodate characters which are named 
%% differently. For example, this encoding uses "endash" and "emdash" 
%% whereas t1.etx called for "rangedash" and "punctdash".
%% - The original notices at the top of that file concerning authors,
%% maintenance etc. are replaced by this notice.
%% - The file is renamed.
%% - The encoding name is modified.
%%
%%%%%%%%%%%%%%%%%%%%%%%%%%%%%%%%%%%%%%%%%%%%%%%%%
\relax
\encoding

\needsfontinstversion{1.910}

\setcommand\lc#1#2{#2}
\setcommand\uc#1#2{#1}
\setcommand\lctop#1#2{#2}
\setcommand\uctop#1#2{#1}
\setcommand\lclig#1#2{#2}
\ifisint{letterspacing}\then
   \ifnumber{\int{letterspacing}}={0}\then \Else
      \setcommand\uclig#1#2{#1spaced}
      \comment{Here we set \verb|\uclig#1#2| to \verb|#1spaced|, but 
      you can't see it as \verb|\setcommand| commands are invisible in 
      the typeset output.}
   \Fi
\Fi
\setcommand\uclig#1#2{#1}
\setcommand\digit#1{#1}

\ifisint{monowidth}\then
   \setint{ligaturing}{0}
\Else
   % The following empty line is *important* to get the formatting
   % right here (sigh)! (Remember that it is a \par token.)
   
   \ifisint{letterspacing}\then
      \ifnumber{\int{letterspacing}}={0}\then \Else
         \setint{ligaturing}{0}
      \Fi
   \Fi
	\setint{ligaturing}{1}
\Fi

\setint{italicslant}{0}
\setint{quad}{1000}
\setint{baselineskip}{1200}

\ifisglyph{x}\then
   \setint{xheight}{\height{x}}
\Else
   \setint{xheight}{500}
\Fi

\ifisglyph{space}\then
   \setint{interword}{\width{space}}
\Else\ifisglyph{i}\then
   \setint{interword}{\width{i}}
\Else
   \setint{interword}{333}
\Fi\Fi

\ifisint{monowidth}\then
   \setint{stretchword}{0}
   \setint{shrinkword}{0}
   \setint{extraspace}{\int{interword}}
\Else
   \setint{stretchword}{\scale{\int{interword}}{600}}
   \setint{shrinkword}{\scale{\int{interword}}{240}}
   \setint{extraspace}{\scale{\int{interword}}{240}}
\Fi

\ifisglyph{X}\then
   \setint{capheight}{\height{X}}
\Else
   \setint{capheight}{750}
\Fi

\ifisglyph{d}\then
   \setint{ascender}{\height{d}}
\Else\ifisint{capheight}\then
   \setint{ascender}{\int{capheight}}
\Else
   \setint{ascender}{750}
\Fi\Fi

\ifisglyph{Aring}\then
   \setint{acccapheight}{\height{Aring}}
\Else
   \setint{acccapheight}{999}
\Fi

\ifisint{descender_neg}\then
   \setint{descender}{\neg{\int{descender_neg}}}
\Else\ifisglyph{p}\then
   \setint{descender}{\depth{p}}
\Else
   \setint{descender}{250}
\Fi\Fi

\ifisglyph{Aring}\then
   \setint{maxheight}{\height{Aring}}
\Else
   \setint{maxheight}{1000}
\Fi

\ifisint{maxdepth_neg}\then
   \setint{maxdepth}{\neg{\int{maxdepth_neg}}}
\Else\ifisglyph{j}\then
   \setint{maxdepth}{\depth{j}}
\Else
   \setint{maxdepth}{250}
\Fi\Fi

\ifisglyph{six}\then
   \setint{digitwidth}{\width{six}}
\Else
   \setint{digitwidth}{500}
\Fi

\setint{capstem}{0} % not in AFM files

\setfontdimen{1}{italicslant}    % italic slant
\setfontdimen{2}{interword}      % interword space
\setfontdimen{3}{stretchword}    % interword stretch
\setfontdimen{4}{shrinkword}     % interword shrink
\setfontdimen{5}{xheight}        % x-height
\setfontdimen{6}{quad}           % quad
\setfontdimen{7}{extraspace}     % extra space after .
\setfontdimen{8}{capheight}      % cap height
\setfontdimen{9}{ascender}       % ascender
\setfontdimen{10}{acccapheight}  % accented cap height
\setfontdimen{11}{descender}     % descender's depth
\setfontdimen{12}{maxheight}     % max height
\setfontdimen{13}{maxdepth}      % max depth
\setfontdimen{14}{digitwidth}    % digit width
\setfontdimen{15}{verticalstem}  % dominant width of verical stems
\setfontdimen{16}{baselineskip}  % baselineskip

\ifnumber{\int{ligaturing}}<{0}\then 
   \comment{In this case, the codingscheme can be different from the 
     default, and therefore we refrain from setting it.}
\Else
   \setstr{codingscheme}{EXTENDED TEX FONT ENCODING - LATIN CFR}
\Fi

\setslot{\lc{Grave}{grave}}
   \comment{The grave accent `\`{}'.}
\endsetslot

\setslot{\lc{Acute}{acute}}
   \comment{The acute accent `\'{}'.}
\endsetslot

\setslot{\lc{Circumflex}{circumflex}}
   \comment{The circumflex accent `\^{}'.}
\endsetslot

\setslot{\lc{Tilde}{tilde}}
   \comment{The tilde accent `\~{}'.}
\endsetslot

\setslot{\lc{Dieresis}{dieresis}}
   \comment{The umlaut or dieresis accent `\"{}'.}
\endsetslot

\setslot{\lc{Hungarumlaut}{hungarumlaut}}
   \comment{The long Hungarian umlaut `\H{}'.}
\endsetslot

\setslot{\lc{Ring}{ring}}
   \comment{The ring accent `\r{}'.}
\endsetslot

\setslot{\lc{Caron}{caron}}
   \comment{The caron or h\'a\v cek accent `\v{}'.}
\endsetslot

\setslot{\lc{Breve}{breve}}
   \comment{The breve accent `\u{}'.}
\endsetslot

\setslot{\lc{Macron}{macron}}
   \comment{The macron accent `\={}'.}
\endsetslot

\setslot{\lc{Dotaccent}{dotaccent}}
   \comment{The dot accent `\.{}'.}
\endsetslot

\setslot{\lc{Cedilla}{cedilla}}
   \comment{The cedilla accent `\c {}'.}
\endsetslot

\setslot{\lc{Ogonek}{ogonek}}
   \comment{The ogonek accent `\k {}'.}
\endsetslot

\setslot{quotesinglbase}
  \comment{A German single quote mark `\quotesinglbase' similar to a comma,
      but with different sidebearings.}
\endsetslot

\setslot{guilsinglleft}
  \comment{A French single opening quote mark `\guilsinglleft',
      unavailable in \plain\ \TeX.}
\endsetslot

\setslot{guilsinglright}
  \comment{A French single closing quote mark `\guilsinglright',
      unavailable in \plain\ \TeX.}
\endsetslot

\setslot{quotedblleft}
  \comment{The English opening quote mark `\,\textquotedblleft\,'.}
\endsetslot

\setslot{quotedblright}
  \comment{The English closing quote mark `\,\textquotedblright\,'.}
\endsetslot

\setslot{quotedblbase}
  \comment{A German double quote mark `\quotedblbase' similar to two commas,
      but with tighter letterspacing and different sidebearings.}
\endsetslot

\setslot{guillemotleft}
  \comment{A French double opening quote mark `\guillemotleft',
      unavailable in \plain\ \TeX.}
\endsetslot

\setslot{guillemotright}
  \comment{A French closing opening quote mark `\guillemotright',
      unavailable in \plain\ \TeX.}
\endsetslot

\setslot{endash}
   \ligature{LIG}{hyphen}{emdash}
   \comment{The number range dash `1--9'. This is called `rangedash' by 
     fontinst's t1.etx, but it needs to be called `endash' to work right. The
     `\textendash'.  In a monowidth font, this might be set as 
   `\texttt{1{-}9}'.}
\endsetslot

\setslot{emdash}
   \comment{The punctuation dash `Oh---boy.' This is calle `punctdash' by 
     fontinst's t1.etx, but needs to be called `emdash' to work right. The
     `\textemdash'.  In a monowidth font, this might be set as
   `\texttt{Oh{-}{-}boy.}'}
\endsetslot

\setslot{compwordmark}
   \comment{An invisible glyph, with zero width and depth, but the
      height of lowercase letters without ascenders.
      It is used to stop ligaturing in words like `shelf{}ful'.}
\endsetslot

\setslot{perthousandzero}
   \comment{A glyph which is placed after `\%' to produce a
      `per-thousand', or twice to produce `per-ten-thousand'.
      Your guess is as good as mine as to what this glyph should look
      like in a monowidth font.}
\endsetslot

\setslot{\lc{dotlessI}{dotlessi}}
   \comment{A dotless i `\i', used to produce accented letters such as
      `\=\i'.}
\endsetslot

\setslot{\lc{dotlessJ}{dotlessj}}
   \comment{A dotless j `\j', used to produce accented letters such as
      `\=\j'.  Most non-\TeX\ fonts do not have this glyph.}
\endsetslot

\ifnumber{\int{ligaturing}}<{0}\then \skipslots{5}\Else

\setslot{\lclig{FF}{ff}}
   \ifnumber{\int{ligaturing}}>{0}\then
      \ligature{LIG}{\lc{I}{i}}{\lclig{FFI}{ffi}}
      \ligature{LIG}{\lc{L}{l}}{\lclig{FFL}{ffl}}
   \Fi
   \comment{The `ff' ligature.  It should be two characters wide in a
      monowidth font.}
\endsetslot

\setslot{\lclig{FI}{fi}}
   \comment{The `fi' ligature.  It should be two characters wide in a
      monowidth font.}
\endsetslot

\setslot{\lclig{FL}{fl}}
   \comment{The `fl' ligature.  It should be two characters wide in a
      monowidth font.}
\endsetslot

\setslot{\lclig{FFI}{ffi}}
   \comment{The `ffi' ligature.  It should be three characters wide in a
      monowidth font.}
\endsetslot

\setslot{\lclig{FFL}{ffl}}
   \comment{The `ffl' ligature.  It should be three characters wide in a
      monowidth font.}
\endsetslot

\Fi

\setslot{visiblespace}
   \comment{A visible space glyph `\textvisiblespace'.}
\endsetslot

\setslot{exclam}
   \ligature{LIG}{quoteleft}{exclamdown}
   \comment{The exclamation mark `!'.}
\endsetslot

\setslot{quotedbl}
  \comment{The `neutral' double quotation mark `\,\textquotedbl\,',
      included for use in monowidth fonts, or for setting computer
      programs.  Note that the inclusion of this glyph in this slot
      means that \TeX\ documents which used `{\tt\char`\"}' as an
      input character will no longer work.}
\endsetslot

\setslot{numbersign}
   \comment{The hash sign `\#'.}
\endsetslot

\setslot{dollar}
   \comment{The dollar sign `\$'.}
\endsetslot

\setslot{percent}
   \comment{The percent sign `\%'.}
\endsetslot

\setslot{ampersand}
   \comment{The ampersand sign `\&'.}
\endsetslot

\setslot{quoteright}
   \ligature{LIG}{quoteright}{quotedblright}
   \comment{The English closing single quote mark `\,\textquoteright\,'.}
\endsetslot

\setslot{parenleft}
   \comment{The opening parenthesis `('.}
\endsetslot

\setslot{parenright}
   \comment{The closing parenthesis `)'.}
\endsetslot

\setslot{asterisk}
   \comment{The raised asterisk `*'.}
\endsetslot

\setslot{plus}
   \comment{The addition sign `+'.}
\endsetslot

\setslot{comma}
   \ligature{LIG}{comma}{quotedblbase}
   \comment{The comma `,'.}
\endsetslot

\setslot{hyphen}
   \ligature{LIG}{hyphen}{endash}
   \ligature{LIG}{hyphenchar}{hyphenchar}
   \comment{The hyphen `-'.}
\endsetslot

\setslot{period}
   \comment{The period `.'.}
\endsetslot

\setslot{slash}
   \comment{The forward oblique `/'.}
\endsetslot

\setslot{\digit{zero}}
   \comment{The number `0'.  This (and all the other numerals) may be
      old style or ranging digits.}
\endsetslot

\setslot{\digit{one}}
   \comment{The number `1'.}
\endsetslot

\setslot{\digit{two}}
   \comment{The number `2'.}
\endsetslot

\setslot{\digit{three}}
   \comment{The number `3'.}
\endsetslot

\setslot{\digit{four}}
   \comment{The number `4'.}
\endsetslot

\setslot{\digit{five}}
   \comment{The number `5'.}
\endsetslot

\setslot{\digit{six}}
   \comment{The number `6'.}
\endsetslot

\setslot{\digit{seven}}
   \comment{The number `7'.}
\endsetslot

\setslot{\digit{eight}}
   \comment{The number `8'.}
\endsetslot

\setslot{\digit{nine}}
   \comment{The number `9'.}
\endsetslot

\setslot{colon}
   \comment{The colon punctuation mark `:'.}
\endsetslot

\setslot{semicolon}
   \comment{The semi-colon punctuation mark `;'.}
\endsetslot

\setslot{less}
   \ligature{LIG}{less}{guillemotleft}
   \comment{The less-than sign `\textless'.}
\endsetslot

\setslot{equal}
   \comment{The equals sign `='.}
\endsetslot

\setslot{greater}
   \ligature{LIG}{greater}{guillemotright}
   \comment{The greater-than sign `\textgreater'.}
\endsetslot

\setslot{question}
   \ligature{LIG}{quoteleft}{questiondown}
   \comment{The question mark `?'.}
\endsetslot

\setslot{at}
   \comment{The at sign `@'.}
\endsetslot

\setslot{\uc{A}{a}}
   \comment{The letter `{A}'.}
\endsetslot

\setslot{\uc{B}{b}}
   \comment{The letter `{B}'.}
\endsetslot

\setslot{\uc{C}{c}}
   \comment{The letter `{C}'.}
\endsetslot

\setslot{\uc{D}{d}}
   \comment{The letter `{D}'.}
\endsetslot

\setslot{\uc{E}{e}}
   \comment{The letter `{E}'.}
\endsetslot

\setslot{\uc{F}{f}}
   \comment{The letter `{F}'.}
\endsetslot

\setslot{\uc{G}{g}}
   \comment{The letter `{G}'.}
\endsetslot

\setslot{\uc{H}{h}}
   \comment{The letter `{H}'.}
\endsetslot

\ifnumber{\int{ligaturing}}<{-1}\then \skipslots{1}\Else

\setslot{\uc{I}{i}}
   \comment{The letter `{I}'.}
\endsetslot

\Fi

\setslot{\uc{J}{j}}
   \comment{The letter `{J}'.}
\endsetslot

\setslot{\uc{K}{k}}
   \comment{The letter `{K}'.}
\endsetslot

\setslot{\uc{L}{l}}
   \comment{The letter `{L}'.}
\endsetslot

\setslot{\uc{M}{m}}
   \comment{The letter `{M}'.}
\endsetslot

\setslot{\uc{N}{n}}
   \comment{The letter `{N}'.}
\endsetslot

\setslot{\uc{O}{o}}
   \comment{The letter `{O}'.}
\endsetslot

\setslot{\uc{P}{p}}
   \comment{The letter `{P}'.}
\endsetslot

\setslot{\uc{Q}{q}}
   \comment{The letter `{Q}'.}
\endsetslot

\setslot{\uc{R}{r}}
   \comment{The letter `{R}'.}
\endsetslot

\setslot{\uc{S}{s}}
   \comment{The letter `{S}'.}
\endsetslot

\setslot{\uc{T}{t}}
   \comment{The letter `{T}'.}
\endsetslot

\setslot{\uc{U}{u}}
   \comment{The letter `{U}'.}
\endsetslot

\setslot{\uc{V}{v}}
   \comment{The letter `{V}'.}
\endsetslot

\setslot{\uc{W}{w}}
   \comment{The letter `{W}'.}
\endsetslot

\setslot{\uc{X}{x}}
   \comment{The letter `{X}'.}
\endsetslot

\setslot{\uc{Y}{y}}
   \comment{The letter `{Y}'.}
\endsetslot

\setslot{\uc{Z}{z}}
   \comment{The letter `{Z}'.}
\endsetslot

\setslot{bracketleft}
   \comment{The opening square bracket `['.}
\endsetslot

\setslot{backslash}
   \comment{The backwards oblique `\textbackslash'.}
\endsetslot

\setslot{bracketright}
   \comment{The closing square bracket `]'.}
\endsetslot

\setslot{asciicircum}
   \comment{The ASCII upward-pointing arrow head `\textasciicircum'.
      This is included for compatibility with typewriter fonts used
      for computer listings.}
\endsetslot

\setslot{underscore}
   \comment{The ASCII underline character `\textunderscore', usually
      set on the baseline.
      This is included for compatibility with typewriter fonts used
      for computer listings.}
\endsetslot

\setslot{quoteleft}
   \ligature{LIG}{quoteleft}{quotedblleft}
   \comment{The English opening single quote mark `\,\textquoteleft\,'.}
\endsetslot

\setslot{\lc{A}{a}}
   \comment{The letter `{a}'.}
\endsetslot

\setslot{\lc{B}{b}}
   \comment{The letter `{b}'.}
\endsetslot

\ifnumber{\int{ligaturing}}<{-1}\then \skipslots{1}\Else

   \setslot{\lc{C}{c}}
      \comment{The letter `{c}'.}
   \endsetslot

\Fi

\setslot{\lc{D}{d}}
   \comment{The letter `{d}'.}
\endsetslot

\setslot{\lc{E}{e}}
   \comment{The letter `{e}'.}
\endsetslot

\ifnumber{\int{ligaturing}}<{-1}\then \skipslots{1}\Else

   \setslot{\lc{F}{f}}
      \ifnumber{\int{ligaturing}}>{0}\then
         \ligature{LIG}{\lc{I}{i}}{\lclig{FI}{fi}}
         \ligature{LIG}{\lc{F}{f}}{\lclig{FF}{ff}}
         \ligature{LIG}{\lc{L}{l}}{\lclig{FL}{fl}}
      \Fi
      \comment{The letter `{f}'.}
   \endsetslot

\Fi

\setslot{\lc{G}{g}}
   \comment{The letter `{g}'.}
\endsetslot

\setslot{\lc{H}{h}}
   \comment{The letter `{h}'.}
\endsetslot

\ifnumber{\int{ligaturing}}<{-1}\then \skipslots{1}\Else

   \setslot{\lc{I}{i}}
      \comment{The letter `{i}'.}
   \endsetslot

\Fi

\setslot{\lc{J}{j}}
   \comment{The letter `{j}'.}
\endsetslot

\setslot{\lc{K}{k}}
   \comment{The letter `{k}'.}
\endsetslot

\setslot{\lc{L}{l}}
   \comment{The letter `{l}'.}
\endsetslot

\setslot{\lc{M}{m}}
   \comment{The letter `{m}'.}
\endsetslot

\setslot{\lc{N}{n}}
   \comment{The letter `{n}'.}
\endsetslot

\setslot{\lc{O}{o}}
   \comment{The letter `{o}'.}
\endsetslot

\setslot{\lc{P}{p}}
   \comment{The letter `{p}'.}
\endsetslot

\setslot{\lc{Q}{q}}
   \comment{The letter `{q}'.}
\endsetslot

\setslot{\lc{R}{r}}
   \comment{The letter `{r}'.}
\endsetslot

\ifnumber{\int{ligaturing}}<{-1}\then \skipslots{1}\Else

   \setslot{\lc{S}{s}}
      \comment{The letter `{s}'.}
   \endsetslot

\Fi

\setslot{\lc{T}{t}}
   \comment{The letter `{t}'.}
\endsetslot

\setslot{\lc{U}{u}}
   \comment{The letter `{u}'.}
\endsetslot

\setslot{\lc{V}{v}}
   \comment{The letter `{v}'.}
\endsetslot

\setslot{\lc{W}{w}}
   \comment{The letter `{w}'.}
\endsetslot

\setslot{\lc{X}{x}}
   \comment{The letter `{x}'.}
\endsetslot

\setslot{\lc{Y}{y}}
   \comment{The letter `{y}'.}
\endsetslot

\setslot{\lc{Z}{z}}
   \comment{The letter `{z}'.}
\endsetslot

\setslot{braceleft}
   \comment{The opening curly brace `\textbraceleft'.}
\endsetslot

\setslot{bar}
   \comment{The ASCII vertical bar `\textbar'.
      This is included for compatibility with typewriter fonts used
      for computer listings.}
\endsetslot

\setslot{braceright}
   \comment{The closing curly brace `\textbraceright'.}
\endsetslot

\setslot{asciitilde}
   \comment{The ASCII tilde `\textasciitilde'.
      This is included for compatibility with typewriter fonts used
      for computer listings.}
\endsetslot

\setslot{hyphenchar}
   \comment{The glyph used for hyphenation in this font, which will
      almost always be the same as `hyphen'.}
\endsetslot

\setslot{\uctop{Abreve}{abreve}}
   \comment{The letter `\u A'.}
\endsetslot

\setslot{\uc{Aogonek}{aogonek}}
   \comment{The letter `\k A'.}
\endsetslot

\setslot{\uctop{Cacute}{cacute}}
   \comment{The letter `\' C'.}
\endsetslot

\setslot{\uctop{Ccaron}{ccaron}}
   \comment{The letter `\v C'.}
\endsetslot

\setslot{\uctop{Dcaron}{dcaron}}
   \comment{The letter `\v D'.}
\endsetslot

\setslot{\uctop{Ecaron}{ecaron}}
   \comment{The letter `\v E'.}
\endsetslot

\setslot{\uc{Eogonek}{eogonek}}
   \comment{The letter `\k E'.}
\endsetslot

\setslot{\uctop{Gbreve}{gbreve}}
   \comment{The letter `\u G'.}
\endsetslot

\setslot{\uctop{Lacute}{lacute}}
   \comment{The letter `\' L'.}
\endsetslot

\setslot{\uc{Lcaron}{lcaron}}
   \comment{The letter `\v L'.}
\endsetslot

\setslot{\uc{Lslash}{lslash}}
   \comment{The letter `\L'.}
\endsetslot

\setslot{\uctop{Nacute}{nacute}}
   \comment{The letter `\' N'.}
\endsetslot

\setslot{\uctop{Ncaron}{ncaron}}
   \comment{The letter `\v N'.}
\endsetslot

\setslot{\uc{Eng}{eng}}
   \comment{The Sami letter `\NG'.  It is unavailable in \plain\ \TeX. 
   This needs to be called `Eng'/`eng' rather than `Ng'/`ng' as in t1.etx in 
    most cases, it seems.}
\endsetslot

\setslot{\uctop{Ohungarumlaut}{ohungarumlaut}}
   \comment{The letter `\H O'.}
\endsetslot

\setslot{\uctop{Racute}{racute}}
   \comment{The letter `\' R'.}
\endsetslot

\setslot{\uctop{Rcaron}{rcaron}}
   \comment{The letter `\v R'.}
\endsetslot

\setslot{\uctop{Sacute}{sacute}}
   \comment{The letter `\' S'.}
\endsetslot

\setslot{\uctop{Scaron}{scaron}}
   \comment{The letter `\v S'.}
\endsetslot

\setslot{\uc{Scedilla}{scedilla}}
   \comment{The letter `\c S'.}
\endsetslot

\setslot{\uctop{Tcaron}{tcaron}}
   \comment{The letter `\v T'.}
\endsetslot

\setslot{\uc{Tcedilla}{tcedilla}}
   \comment{The letter `\c T'.}
\endsetslot

\setslot{\uctop{Uhungarumlaut}{uhungarumlaut}}
   \comment{The letter `\H U'.}
\endsetslot

\setslot{\uctop{Uring}{uring}}
   \comment{The letter `\r U'.}
\endsetslot

\setslot{\uctop{Ydieresis}{ydieresis}}
   \comment{The letter `\" Y'.}
\endsetslot

\setslot{\uctop{Zacute}{zacute}}
   \comment{The letter `\' Z'.}
\endsetslot

\setslot{\uctop{Zcaron}{zcaron}}
   \comment{The letter `\v Z'.}
\endsetslot

\setslot{\uctop{Zdotaccent}{zdotaccent}}
   \comment{The letter `\. Z'.}
\endsetslot

\ifnumber{\int{ligaturing}}<{0}\then \skipslots{1}\Else

   \setslot{\uclig{IJ}{ij}}
      \comment{The letter `IJ'.  This is a single letter, and in a 
        monowidth font should ideally be one letter wide.}
   \endsetslot

\Fi

\setslot{\uctop{Idotaccent}{idotaccent}}
   \comment{The letter `\. I'.}
\endsetslot

\setslot{\lc{Dbar}{dbar}}
   \comment{The letter `\dj'.}
\endsetslot

\setslot{section}
   \comment{The section mark `\textsection'.}
\endsetslot

\setslot{\lctop{Abreve}{abreve}}
   \comment{The letter `\u a'.}
\endsetslot

\setslot{\lc{Aogonek}{aogonek}}
   \comment{The letter `\k a'.}
\endsetslot

\setslot{\lctop{Cacute}{cacute}}
   \comment{The letter `\' c'.}
\endsetslot

\setslot{\lctop{Ccaron}{ccaron}}
   \comment{The letter `\v c'.}
\endsetslot

\setslot{\lctop{Dcaron}{dcaron}}
   \comment{The letter `\v d'.}
\endsetslot

\setslot{\lctop{Ecaron}{ecaron}}
   \comment{The letter `\v e'.}
\endsetslot

\setslot{\lc{Eogonek}{eogonek}}
   \comment{The letter `\k e'.}
\endsetslot

\setslot{\lctop{Gbreve}{gbreve}}
   \comment{The letter `\u g'.}
\endsetslot

\setslot{\lctop{Lacute}{lacute}}
   \comment{The letter `\' l'.}
\endsetslot

\setslot{\lc{Lcaron}{lcaron}}
   \comment{The letter `\v l'.}
\endsetslot

\setslot{\lc{Lslash}{lslash}}
   \comment{The letter `\l'.}
\endsetslot

\setslot{\lctop{Nacute}{nacute}}
   \comment{The letter `\' n'.}
\endsetslot

\setslot{\lctop{Ncaron}{ncaron}}
   \comment{The letter `\v n'.}
\endsetslot

\setslot{\lc{Eng}{eng}}
   \comment{The Sami letter `\ng'.  It is unavailable in \plain\ \TeX. This needs to be called `Eng'/`eng' rather than `Ng'/`ng' as it is in t1.etx in most cases, it seems.}
\endsetslot

\setslot{\lctop{Ohungarumlaut}{ohungarumlaut}}
   \comment{The letter `\H o'.}
\endsetslot

\setslot{\lctop{Racute}{racute}}
   \comment{The letter `\' r'.}
\endsetslot

\setslot{\lctop{Rcaron}{rcaron}}
   \comment{The letter `\v r'.}
\endsetslot

\setslot{\lctop{Sacute}{sacute}}
   \comment{The letter `\' s'.}
\endsetslot

\setslot{\lctop{Scaron}{scaron}}
   \comment{The letter `\v s'.}
\endsetslot

\setslot{\lc{Scedilla}{scedilla}}
   \comment{The letter `\c s'.}
\endsetslot

\setslot{\lctop{Tcaron}{tcaron}}
   \comment{The letter `\v t'.}
\endsetslot

\setslot{\lc{Tcedilla}{tcedilla}}
   \comment{The letter `\c t'.}
\endsetslot

\setslot{\lctop{Uhungarumlaut}{uhungarumlaut}}
   \comment{The letter `\H u'.}
\endsetslot

\setslot{\lctop{Uring}{uring}}
   \comment{The letter `\r u'.}
\endsetslot

\setslot{\lctop{Ydieresis}{ydieresis}}
   \comment{The letter `\" y'.}
\endsetslot

\setslot{\lctop{Zacute}{zacute}}
   \comment{The letter `\' z'.}
\endsetslot

\setslot{\lctop{Zcaron}{zcaron}}
   \comment{The letter `\v z'.}
\endsetslot

\setslot{\lctop{Zdotaccent}{zdotaccent}}
   \comment{The letter `\. z'.}
\endsetslot

\ifnumber{\int{ligaturing}}<{0}\then \skipslots{1}\Else

   \setslot{\lclig{IJ}{ij}}
      \comment{The letter `ij'.  This is a single letter, and in a 
        monowidth font should ideally be one letter wide.}
   \endsetslot

\Fi

\setslot{exclamdown}
   \comment{The Spanish punctuation mark `!`'.}
\endsetslot

\setslot{questiondown}
   \comment{The Spanish punctuation mark `?`'.}
\endsetslot

\setslot{sterling}
   \comment{The British currency mark `\textsterling'.}
\endsetslot

\setslot{\uctop{Agrave}{agrave}}
   \comment{The letter `\` A'.}
\endsetslot

\setslot{\uctop{Aacute}{aacute}}
   \comment{The letter `\' A'.}
\endsetslot

\setslot{\uctop{Acircumflex}{acircumflex}}
   \comment{The letter `\^ A'.}
\endsetslot

\setslot{\uctop{Atilde}{atilde}}
   \comment{The letter `\~ A'.}
\endsetslot

\setslot{\uctop{Adieresis}{adieresis}}
   \comment{The letter `\" A'.}
\endsetslot

\setslot{\uctop{Aring}{aring}}
   \comment{The letter `\r A'.}
\endsetslot

\setslot{\uc{AE}{ae}}
   \comment{The letter `\AE'.  This is a single letter, and should not be
      faked with `AE'.}
\endsetslot

\setslot{\uc{Ccedilla}{ccedilla}}
   \comment{The letter `\c C'.}
\endsetslot

\setslot{\uctop{Egrave}{egrave}}
   \comment{The letter `\` E'.}
\endsetslot

\setslot{\uctop{Eacute}{eacute}}
   \comment{The letter `\' E'.}
\endsetslot

\setslot{\uctop{Ecircumflex}{ecircumflex}}
   \comment{The letter `\^ E'.}
\endsetslot

\setslot{\uctop{Edieresis}{edieresis}}
   \comment{The letter `\" E'.}
\endsetslot

\setslot{\uctop{Igrave}{igrave}}
   \comment{The letter `\` I'.}
\endsetslot

\setslot{\uctop{Iacute}{iacute}}
   \comment{The letter `\' I'.}
\endsetslot

\setslot{\uctop{Icircumflex}{icircumflex}}
   \comment{The letter `\^ I'.}
\endsetslot

\setslot{\uctop{Idieresis}{idieresis}}
   \comment{The letter `\" I'.}
\endsetslot

\setslot{\uc{Eth}{eth}}
   \comment{The uppercase Icelandic letter `Eth' similar to a `D'
      with a horizontal bar through the stem.  It is unavailable
      in \plain\ \TeX.}
\endsetslot

\setslot{\uctop{Ntilde}{ntilde}}
   \comment{The letter `\~ N'.}
\endsetslot

\setslot{\uctop{Ograve}{ograve}}
   \comment{The letter `\` O'.}
\endsetslot

\setslot{\uctop{Oacute}{oacute}}
   \comment{The letter `\' O'.}
\endsetslot

\setslot{\uctop{Ocircumflex}{ocircumflex}}
   \comment{The letter `\^ O'.}
\endsetslot

\setslot{\uctop{Otilde}{otilde}}
   \comment{The letter `\~ O'.}
\endsetslot

\setslot{\uctop{Odieresis}{odieresis}}
   \comment{The letter `\" O'.}
\endsetslot

\setslot{\uc{OE}{oe}}
   \comment{The letter `\OE'.  This is a single letter, and should not be
      faked with `OE'.}
\endsetslot

\setslot{\uc{Oslash}{oslash}}
   \comment{The letter `\O'.}
\endsetslot

\setslot{\uctop{Ugrave}{ugrave}}
   \comment{The letter `\` U'.}
\endsetslot

\setslot{\uctop{Uacute}{uacute}}
   \comment{The letter `\' U'.}
\endsetslot

\setslot{\uctop{Ucircumflex}{ucircumflex}}
   \comment{The letter `\^ U'.}
\endsetslot

\setslot{\uctop{Udieresis}{udieresis}}
   \comment{The letter `\" U'.}
\endsetslot

\setslot{\uctop{Yacute}{yacute}}
   \comment{The letter `\' Y'.}
\endsetslot

\setslot{\uc{Thorn}{thorn}}
   \comment{The Icelandic capital letter Thorn, similar to a `P'
      with the bowl moved down.  It is unavailable in \plain\ \TeX.}
\endsetslot

\setslot{\uclig{SS}{germandbls}}
   \comment{The ligature `SS', used to give an upper case `\ss'.
      In a monowidth font it should be two letters wide.}
\endsetslot

\setslot{\lctop{Agrave}{agrave}}
   \comment{The letter `\` a'.}
\endsetslot

\setslot{\lctop{Aacute}{aacute}}
   \comment{The letter `\' a'.}
\endsetslot

\setslot{\lctop{Acircumflex}{acircumflex}}
   \comment{The letter `\^ a'.}
\endsetslot

\setslot{\lctop{Atilde}{atilde}}
   \comment{The letter `\~ a'.}
\endsetslot

\setslot{\lctop{Adieresis}{adieresis}}
   \comment{The letter `\" a'.}
\endsetslot

\setslot{\lctop{Aring}{aring}}
   \comment{The letter `\r a'.}
\endsetslot

\setslot{\lc{AE}{ae}}
   \comment{The letter `\ae'.  This is a single letter, and should not be
      faked with `ae'.}
\endsetslot

\setslot{\lc{Ccedilla}{ccedilla}}
   \comment{The letter `\c c'.}
\endsetslot

\setslot{\lctop{Egrave}{egrave}}
   \comment{The letter `\` e'.}
\endsetslot

\setslot{\lctop{Eacute}{eacute}}
   \comment{The letter `\' e'.}
\endsetslot

\setslot{\lctop{Ecircumflex}{ecircumflex}}
   \comment{The letter `\^ e'.}
\endsetslot

\setslot{\lctop{Edieresis}{edieresis}}
   \comment{The letter `\" e'.}
\endsetslot

\setslot{\lctop{Igrave}{igrave}}
   \comment{The letter `\`\i'.}
\endsetslot

\setslot{\lctop{Iacute}{iacute}}
   \comment{The letter `\'\i'.}
\endsetslot

\setslot{\lctop{Icircumflex}{icircumflex}}
   \comment{The letter `\^\i'.}
\endsetslot

\setslot{\lctop{Idieresis}{idieresis}}
   \comment{The letter `\"\i'.}
\endsetslot

\setslot{\lc{Eth}{eth}}
   \comment{The Icelandic lowercase letter `eth' similar to
     a `$\partial$' with an oblique bar through the stem.
     It is unavailable in \plain\ \TeX.}
\endsetslot

\setslot{\lctop{Ntilde}{ntilde}}
   \comment{The letter `\~ n'.}
\endsetslot

\setslot{\lctop{Ograve}{ograve}}
   \comment{The letter `\` o'.}
\endsetslot

\setslot{\lctop{Oacute}{oacute}}
   \comment{The letter `\' o'.}
\endsetslot

\setslot{\lctop{Ocircumflex}{ocircumflex}}
   \comment{The letter `\^ o'.}
\endsetslot

\setslot{\lctop{Otilde}{otilde}}
   \comment{The letter `\~ o'.}
\endsetslot

\setslot{\lctop{Odieresis}{odieresis}}
   \comment{The letter `\" o'.}
\endsetslot

\setslot{\lc{OE}{oe}}
   \comment{The letter `\oe'.  This is a single letter, and should not be
      faked with `oe'.}
\endsetslot

\setslot{\lc{Oslash}{oslash}}
   \comment{The letter `\o'.}
\endsetslot

\setslot{\lctop{Ugrave}{ugrave}}
   \comment{The letter `\` u'.}
\endsetslot

\setslot{\lctop{Uacute}{uacute}}
   \comment{The letter `\' u'.}
\endsetslot

\setslot{\lctop{Ucircumflex}{ucircumflex}}
   \comment{The letter `\^ u'.}
\endsetslot

\setslot{\lctop{Udieresis}{udieresis}}
   \comment{The letter `\" u'.}
\endsetslot

\setslot{\lctop{Yacute}{yacute}}
   \comment{The letter `\' y'.}
\endsetslot

\setslot{\lc{Thorn}{thorn}}
   \comment{The Icelandic lowercase letter `thorn', similar to a `p'
      with an ascender rising from the stem.  It is unavailable
      in \plain\ \TeX.}
\endsetslot

\setslot{\lc{SS}{germandbls}}
   \comment{The letter `\ss'.}
\endsetslot

\endencoding
%    \end{macrocode}
% \end{encoding}
% \iffalse
%</t1-cfr>
% \fi
% 
% 
% \subsubsection{fontscripts-t1-dotalt-f\_f.etx}\label{subsubsec:t1-dotalt-f-f}
% 
% \iffalse
%<*t1-dotalt-f-f>
% \fi
% \begin{encoding}{t1-dotalt-f_f}
% \changes{v0.0}{2025-02-10}{Filename prefix for Karl.}
%    \begin{macrocode}
%%
%% - The commentary in the original is deleted in this version. For
%% information about the T1 etc., typeset the original t1.etx
%% included with fontinst.
%% - Slots are altered to accommodate characters which are named
%% differently. For example, this encoding uses "endash" and "emdash"
%% whereas t1.etx called for "rangedash" and "punctdash".
%% - The original notices at the top of that file concerning authors,
%% maintenance etc. are replaced by this notice.
%% - The file is renamed.
%% - The encoding name is modified.
%% - f_f, f_f_i and f_f_l replace ff, ffi and ffl.
%% - lc, uc and accented lc, uc are set to characters named "a.alt" etc.
%%
%%%%%%%%%%%%%%%%%%%%%%%%%%%%%%%%%%%%%%%%%%%%%%%%%
\relax
\encoding

\needsfontinstversion{1.910}

\setcommand\lc#1#2{#2.alt}
\setcommand\uc#1#2{#1.alt}
\setcommand\lctop#1#2{#2.alt}
\setcommand\uctop#1#2{#1.alt}
\setcommand\lclig#1#2{#2}
\ifisint{letterspacing}\then
   \ifnumber{\int{letterspacing}}={0}\then \Else
      \setcommand\uclig#1#2{#1spaced}
      \comment{Here we set \verb|\uclig#1#2| to \verb|#1spaced|, but 
      you can't see it as \verb|\setcommand| commands are invisible in 
      the typeset output.}
   \Fi
\Fi
\setcommand\uclig#1#2{#1}
\setcommand\digit#1{#1}

\ifisint{monowidth}\then
   \setint{ligaturing}{0}
\Else
   % The following empty line is *important* to get the formatting
   % right here (sigh)! (Remember that it is a \par token.)
   
   \ifisint{letterspacing}\then
      \ifnumber{\int{letterspacing}}={0}\then \Else
         \setint{ligaturing}{0}
      \Fi
   \Fi
	\setint{ligaturing}{1}
\Fi

\setint{italicslant}{0}
\setint{quad}{1000}
\setint{baselineskip}{1200}

\ifisglyph{x}\then
   \setint{xheight}{\height{x}}
\Else
   \setint{xheight}{500}
\Fi

\ifisglyph{space}\then
   \setint{interword}{\width{space}}
\Else\ifisglyph{i}\then
   \setint{interword}{\width{i}}
\Else
   \setint{interword}{333}
\Fi\Fi

\ifisint{monowidth}\then
   \setint{stretchword}{0}
   \setint{shrinkword}{0}
   \setint{extraspace}{\int{interword}}
\Else
   \setint{stretchword}{\scale{\int{interword}}{600}}
   \setint{shrinkword}{\scale{\int{interword}}{240}}
   \setint{extraspace}{\scale{\int{interword}}{240}}
\Fi

\ifisglyph{X}\then
   \setint{capheight}{\height{X}}
\Else
   \setint{capheight}{750}
\Fi

\ifisglyph{d}\then
   \setint{ascender}{\height{d}}
\Else\ifisint{capheight}\then
   \setint{ascender}{\int{capheight}}
\Else
   \setint{ascender}{750}
\Fi\Fi

\ifisglyph{Aring}\then
   \setint{acccapheight}{\height{Aring}}
\Else
   \setint{acccapheight}{999}
\Fi

\ifisint{descender_neg}\then
   \setint{descender}{\neg{\int{descender_neg}}}
\Else\ifisglyph{p}\then
   \setint{descender}{\depth{p}}
\Else
   \setint{descender}{250}
\Fi\Fi

\ifisglyph{Aring}\then
   \setint{maxheight}{\height{Aring}}
\Else
   \setint{maxheight}{1000}
\Fi

\ifisint{maxdepth_neg}\then
   \setint{maxdepth}{\neg{\int{maxdepth_neg}}}
\Else\ifisglyph{j}\then
   \setint{maxdepth}{\depth{j}}
\Else
   \setint{maxdepth}{250}
\Fi\Fi

\ifisglyph{six}\then
   \setint{digitwidth}{\width{six}}
\Else
   \setint{digitwidth}{500}
\Fi

\setint{capstem}{0} % not in AFM files

\setfontdimen{1}{italicslant}    % italic slant
\setfontdimen{2}{interword}      % interword space
\setfontdimen{3}{stretchword}    % interword stretch
\setfontdimen{4}{shrinkword}     % interword shrink
\setfontdimen{5}{xheight}        % x-height
\setfontdimen{6}{quad}           % quad
\setfontdimen{7}{extraspace}     % extra space after .
\setfontdimen{8}{capheight}      % cap height
\setfontdimen{9}{ascender}       % ascender
\setfontdimen{10}{acccapheight}  % accented cap height
\setfontdimen{11}{descender}     % descender's depth
\setfontdimen{12}{maxheight}     % max height
\setfontdimen{13}{maxdepth}      % max depth
\setfontdimen{14}{digitwidth}    % digit width
\setfontdimen{15}{verticalstem}  % dominant width of verical stems
\setfontdimen{16}{baselineskip}  % baselineskip

\ifnumber{\int{ligaturing}}<{0}\then 
   \comment{In this case, the codingscheme can be different from the 
     default, and therefore we refrain from setting it.}
\Else
   \setstr{codingscheme}{EXTENDED TEX ENC - DOTALT F_F}
\Fi

\setslot{\lc{Grave}{grave}}
   \comment{The grave accent `\`{}'.}
\endsetslot

\setslot{\lc{Acute}{acute}}
   \comment{The acute accent `\'{}'.}
\endsetslot

\setslot{\lc{Circumflex}{circumflex}}
   \comment{The circumflex accent `\^{}'.}
\endsetslot

\setslot{\lc{Tilde}{tilde}}
   \comment{The tilde accent `\~{}'.}
\endsetslot

\setslot{\lc{Dieresis}{dieresis}}
   \comment{The umlaut or dieresis accent `\"{}'.}
\endsetslot

\setslot{\lc{Hungarumlaut}{hungarumlaut}}
   \comment{The long Hungarian umlaut `\H{}'.}
\endsetslot

\setslot{\lc{Ring}{ring}}
   \comment{The ring accent `\r{}'.}
\endsetslot

\setslot{\lc{Caron}{caron}}
   \comment{The caron or h\'a\v cek accent `\v{}'.}
\endsetslot

\setslot{\lc{Breve}{breve}}
   \comment{The breve accent `\u{}'.}
\endsetslot

\setslot{\lc{Macron}{macron}}
   \comment{The macron accent `\={}'.}
\endsetslot

\setslot{\lc{Dotaccent}{dotaccent}}
   \comment{The dot accent `\.{}'.}
\endsetslot

\setslot{\lc{Cedilla}{cedilla}}
   \comment{The cedilla accent `\c {}'.}
\endsetslot

\setslot{\lc{Ogonek}{ogonek}}
   \comment{The ogonek accent `\k {}'.}
\endsetslot

\setslot{quotesinglbase}
  \comment{A German single quote mark `\quotesinglbase' similar to a comma,
      but with different sidebearings.}
\endsetslot

\setslot{guilsinglleft}
  \comment{A French single opening quote mark `\guilsinglleft',
      unavailable in \plain\ \TeX.}
\endsetslot

\setslot{guilsinglright}
  \comment{A French single closing quote mark `\guilsinglright',
      unavailable in \plain\ \TeX.}
\endsetslot

\setslot{quotedblleft}
  \comment{The English opening quote mark `\,\textquotedblleft\,'.}
\endsetslot

\setslot{quotedblright}
  \comment{The English closing quote mark `\,\textquotedblright\,'.}
\endsetslot

\setslot{quotedblbase}
  \comment{A German double quote mark `\quotedblbase' similar to two commas,
      but with tighter letterspacing and different sidebearings.}
\endsetslot

\setslot{guillemotleft}
  \comment{A French double opening quote mark `\guillemotleft',
      unavailable in \plain\ \TeX.}
\endsetslot

\setslot{guillemotright}
  \comment{A French closing opening quote mark `\guillemotright',
      unavailable in \plain\ \TeX.}
\endsetslot

\setslot{endash}
   \ligature{LIG}{hyphen}{emdash}
   \comment{The number range dash `1--9'. 
     This is called `rangedash' by fontinst's t1.etx, but it needs to be 
     called `endash' to work right. 
     The `\textendash'.  In a monowidth font, this might be set as 
      `\texttt{1{-}9}'.}
\endsetslot

\setslot{emdash}
   \comment{The punctuation dash `Oh---boy.' 
     This is calle `punctdash' by fontinst's t1.etx, but needs to be 
     called `emdash' to work right. 
     The `\textemdash'.  
     In a monowidth font, this might be set as `\texttt{Oh{-}{-}boy.}'}
\endsetslot

\setslot{compwordmark}
   \comment{An invisible glyph, with zero width and depth, but the
      height of lowercase letters without ascenders.
      It is used to stop ligaturing in words like `shelf{}ful'.}
\endsetslot

\setslot{perthousandzero}
   \comment{A glyph which is placed after `\%' to produce a
      `per-thousand', or twice to produce `per-ten-thousand'.
      Your guess is as good as mine as to what this glyph should look
      like in a monowidth font.}
\endsetslot

\setslot{\lc{dotlessI}{dotlessi}}
   \comment{A dotless i `\i', used to produce accented letters such as
      `\=\i'.}
\endsetslot

\setslot{\lc{dotlessJ}{dotlessj}}
   \comment{A dotless j `\j', used to produce accented letters such as
      `\=\j'.  Most non-\TeX\ fonts do not have this glyph.}
\endsetslot

\ifnumber{\int{ligaturing}}<{0}\then \skipslots{5}\Else

\setslot{\lclig{FF}{f_f}}
   \ifnumber{\int{ligaturing}}>{0}\then
      \ligature{LIG}{\lc{I}{i}}{\lclig{FFI}{f_f_i}}
      \ligature{LIG}{\lc{L}{l}}{\lclig{FFL}{f_f_l}}
   \Fi
   \comment{The `ff' ligature.  It should be two characters wide in a
      monowidth font.}
\endsetslot

\setslot{\lclig{FI}{fi}}
   \comment{The `fi' ligature.  It should be two characters wide in a
      monowidth font.}
\endsetslot

\setslot{\lclig{FL}{fl}}
   \comment{The `fl' ligature.  It should be two characters wide in a
      monowidth font.}
\endsetslot

\setslot{\lclig{FFI}{f_f_i}}
   \comment{The `ffi' ligature.  It should be three characters wide in a
      monowidth font.}
\endsetslot

\setslot{\lclig{FFL}{f_f_l}}
   \comment{The `ffl' ligature.  It should be three characters wide in a
      monowidth font.}
\endsetslot

\Fi

\setslot{visiblespace}
   \comment{A visible space glyph `\textvisiblespace'.}
\endsetslot

\setslot{exclam}
   \ligature{LIG}{quoteleft}{exclamdown}
   \comment{The exclamation mark `!'.}
\endsetslot

\setslot{quotedbl}
  \comment{The `neutral' double quotation mark `\,\textquotedbl\,',
      included for use in monowidth fonts, or for setting computer
      programs.  Note that the inclusion of this glyph in this slot
      means that \TeX\ documents which used `{\tt\char`\"}' as an
      input character will no longer work.}
\endsetslot

\setslot{numbersign}
   \comment{The hash sign `\#'.}
\endsetslot

\setslot{dollar}
   \comment{The dollar sign `\$'.}
\endsetslot

\setslot{percent}
   \comment{The percent sign `\%'.}
\endsetslot

\setslot{ampersand}
   \comment{The ampersand sign `\&'.}
\endsetslot

\setslot{quoteright}
   \ligature{LIG}{quoteright}{quotedblright}
   \comment{The English closing single quote mark `\,\textquoteright\,'.}
\endsetslot

\setslot{parenleft}
   \comment{The opening parenthesis `('.}
\endsetslot

\setslot{parenright}
   \comment{The closing parenthesis `)'.}
\endsetslot

\setslot{asterisk}
   \comment{The raised asterisk `*'.}
\endsetslot

\setslot{plus}
   \comment{The addition sign `+'.}
\endsetslot

\setslot{comma}
   \ligature{LIG}{comma}{quotedblbase}
   \comment{The comma `,'.}
\endsetslot

\setslot{hyphen}
   \ligature{LIG}{hyphen}{endash}
   \ligature{LIG}{hyphenchar}{hyphenchar}
   \comment{The hyphen `-'.}
\endsetslot

\setslot{period}
   \comment{The period `.'.}
\endsetslot

\setslot{slash}
   \comment{The forward oblique `/'.}
\endsetslot

\setslot{\digit{zero}}
   \comment{The number `0'.  This (and all the other numerals) may be
      old style or ranging digits.}
\endsetslot

\setslot{\digit{one}}
   \comment{The number `1'.}
\endsetslot

\setslot{\digit{two}}
   \comment{The number `2'.}
\endsetslot

\setslot{\digit{three}}
   \comment{The number `3'.}
\endsetslot

\setslot{\digit{four}}
   \comment{The number `4'.}
\endsetslot

\setslot{\digit{five}}
   \comment{The number `5'.}
\endsetslot

\setslot{\digit{six}}
   \comment{The number `6'.}
\endsetslot

\setslot{\digit{seven}}
   \comment{The number `7'.}
\endsetslot

\setslot{\digit{eight}}
   \comment{The number `8'.}
\endsetslot

\setslot{\digit{nine}}
   \comment{The number `9'.}
\endsetslot

\setslot{colon}
   \comment{The colon punctuation mark `:'.}
\endsetslot

\setslot{semicolon}
   \comment{The semi-colon punctuation mark `;'.}
\endsetslot

\setslot{less}
   \ligature{LIG}{less}{guillemotleft}
   \comment{The less-than sign `\textless'.}
\endsetslot

\setslot{equal}
   \comment{The equals sign `='.}
\endsetslot

\setslot{greater}
   \ligature{LIG}{greater}{guillemotright}
   \comment{The greater-than sign `\textgreater'.}
\endsetslot

\setslot{question}
   \ligature{LIG}{quoteleft}{questiondown}
   \comment{The question mark `?'.}
\endsetslot

\setslot{at}
   \comment{The at sign `@'.}
\endsetslot

\setslot{\uc{A}{a}}
   \comment{The letter `{A}'.}
\endsetslot

\setslot{\uc{B}{b}}
   \comment{The letter `{B}'.}
\endsetslot

\setslot{\uc{C}{c}}
   \comment{The letter `{C}'.}
\endsetslot

\setslot{\uc{D}{d}}
   \comment{The letter `{D}'.}
\endsetslot

\setslot{\uc{E}{e}}
   \comment{The letter `{E}'.}
\endsetslot

\setslot{\uc{F}{f}}
   \comment{The letter `{F}'.}
\endsetslot

\setslot{\uc{G}{g}}
   \comment{The letter `{G}'.}
\endsetslot

\setslot{\uc{H}{h}}
   \comment{The letter `{H}'.}
\endsetslot

\ifnumber{\int{ligaturing}}<{-1}\then \skipslots{1}\Else

\setslot{\uc{I}{i}}
   \comment{The letter `{I}'.}
\endsetslot

\Fi

\setslot{\uc{J}{j}}
   \comment{The letter `{J}'.}
\endsetslot

\setslot{\uc{K}{k}}
   \comment{The letter `{K}'.}
\endsetslot

\setslot{\uc{L}{l}}
   \comment{The letter `{L}'.}
\endsetslot

\setslot{\uc{M}{m}}
   \comment{The letter `{M}'.}
\endsetslot

\setslot{\uc{N}{n}}
   \comment{The letter `{N}'.}
\endsetslot

\setslot{\uc{O}{o}}
   \comment{The letter `{O}'.}
\endsetslot

\setslot{\uc{P}{p}}
   \comment{The letter `{P}'.}
\endsetslot

\setslot{\uc{Q}{q}}
   \comment{The letter `{Q}'.}
\endsetslot

\setslot{\uc{R}{r}}
   \comment{The letter `{R}'.}
\endsetslot

\setslot{\uc{S}{s}}
   \comment{The letter `{S}'.}
\endsetslot

\setslot{\uc{T}{t}}
   \comment{The letter `{T}'.}
\endsetslot

\setslot{\uc{U}{u}}
   \comment{The letter `{U}'.}
\endsetslot

\setslot{\uc{V}{v}}
   \comment{The letter `{V}'.}
\endsetslot

\setslot{\uc{W}{w}}
   \comment{The letter `{W}'.}
\endsetslot

\setslot{\uc{X}{x}}
   \comment{The letter `{X}'.}
\endsetslot

\setslot{\uc{Y}{y}}
   \comment{The letter `{Y}'.}
\endsetslot

\setslot{\uc{Z}{z}}
   \comment{The letter `{Z}'.}
\endsetslot

\setslot{bracketleft}
   \comment{The opening square bracket `['.}
\endsetslot

\setslot{backslash}
   \comment{The backwards oblique `\textbackslash'.}
\endsetslot

\setslot{bracketright}
   \comment{The closing square bracket `]'.}
\endsetslot

\setslot{asciicircum}
   \comment{The ASCII upward-pointing arrow head `\textasciicircum'.
      This is included for compatibility with typewriter fonts used
      for computer listings.}
\endsetslot

\setslot{underscore}
   \comment{The ASCII underline character `\textunderscore', usually
      set on the baseline.
      This is included for compatibility with typewriter fonts used
      for computer listings.}
\endsetslot

\setslot{quoteleft}
   \ligature{LIG}{quoteleft}{quotedblleft}
   \comment{The English opening single quote mark `\,\textquoteleft\,'.}
\endsetslot

\setslot{\lc{A}{a}}
   \comment{The letter `{a}'.}
\endsetslot

\setslot{\lc{B}{b}}
   \comment{The letter `{b}'.}
\endsetslot

\ifnumber{\int{ligaturing}}<{-1}\then \skipslots{1}\Else

   \setslot{\lc{C}{c}}
      \comment{The letter `{c}'.}
   \endsetslot

\Fi

\setslot{\lc{D}{d}}
   \comment{The letter `{d}'.}
\endsetslot

\setslot{\lc{E}{e}}
   \comment{The letter `{e}'.}
\endsetslot

\ifnumber{\int{ligaturing}}<{-1}\then \skipslots{1}\Else

   \setslot{\lc{F}{f}}
      \ifnumber{\int{ligaturing}}>{0}\then
         \ligature{LIG}{\lc{I}{i}}{\lclig{FI}{fi}}
         \ligature{LIG}{\lc{F}{f}}{\lclig{FF}{f_f}}
         \ligature{LIG}{\lc{L}{l}}{\lclig{FL}{fl}}
      \Fi
      \comment{The letter `{f}'.}
   \endsetslot

\Fi

\setslot{\lc{G}{g}}
   \comment{The letter `{g}'.}
\endsetslot

\setslot{\lc{H}{h}}
   \comment{The letter `{h}'.}
\endsetslot

\ifnumber{\int{ligaturing}}<{-1}\then \skipslots{1}\Else

   \setslot{\lc{I}{i}}
      \comment{The letter `{i}'.}
   \endsetslot

\Fi

\setslot{\lc{J}{j}}
   \comment{The letter `{j}'.}
\endsetslot

\setslot{\lc{K}{k}}
   \comment{The letter `{k}'.}
\endsetslot

\setslot{\lc{L}{l}}
   \comment{The letter `{l}'.}
\endsetslot

\setslot{\lc{M}{m}}
   \comment{The letter `{m}'.}
\endsetslot

\setslot{\lc{N}{n}}
   \comment{The letter `{n}'.}
\endsetslot

\setslot{\lc{O}{o}}
   \comment{The letter `{o}'.}
\endsetslot

\setslot{\lc{P}{p}}
   \comment{The letter `{p}'.}
\endsetslot

\setslot{\lc{Q}{q}}
   \comment{The letter `{q}'.}
\endsetslot

\setslot{\lc{R}{r}}
   \comment{The letter `{r}'.}
\endsetslot

\ifnumber{\int{ligaturing}}<{-1}\then \skipslots{1}\Else

   \setslot{\lc{S}{s}}
      \comment{The letter `{s}'.}
   \endsetslot

\Fi

\setslot{\lc{T}{t}}
   \comment{The letter `{t}'.}
\endsetslot

\setslot{\lc{U}{u}}
   \comment{The letter `{u}'.}
\endsetslot

\setslot{\lc{V}{v}}
   \comment{The letter `{v}'.}
\endsetslot

\setslot{\lc{W}{w}}
   \comment{The letter `{w}'.}
\endsetslot

\setslot{\lc{X}{x}}
   \comment{The letter `{x}'.}
\endsetslot

\setslot{\lc{Y}{y}}
   \comment{The letter `{y}'.}
\endsetslot

\setslot{\lc{Z}{z}}
   \comment{The letter `{z}'.}
\endsetslot

\setslot{braceleft}
   \comment{The opening curly brace `\textbraceleft'.}
\endsetslot

\setslot{bar}
   \comment{The ASCII vertical bar `\textbar'.
      This is included for compatibility with typewriter fonts used
      for computer listings.}
\endsetslot

\setslot{braceright}
   \comment{The closing curly brace `\textbraceright'.}
\endsetslot

\setslot{asciitilde}
   \comment{The ASCII tilde `\textasciitilde'.
      This is included for compatibility with typewriter fonts used
      for computer listings.}
\endsetslot

\setslot{hyphenchar}
   \comment{The glyph used for hyphenation in this font, which will
      almost always be the same as `hyphen'.}
\endsetslot

\setslot{\uctop{Abreve}{abreve}}
   \comment{The letter `\u A'.}
\endsetslot

\setslot{\uc{Aogonek}{aogonek}}
   \comment{The letter `\k A'.}
\endsetslot

\setslot{\uctop{Cacute}{cacute}}
   \comment{The letter `\' C'.}
\endsetslot

\setslot{\uctop{Ccaron}{ccaron}}
   \comment{The letter `\v C'.}
\endsetslot

\setslot{\uctop{Dcaron}{dcaron}}
   \comment{The letter `\v D'.}
\endsetslot

\setslot{\uctop{Ecaron}{ecaron}}
   \comment{The letter `\v E'.}
\endsetslot

\setslot{\uc{Eogonek}{eogonek}}
   \comment{The letter `\k E'.}
\endsetslot

\setslot{\uctop{Gbreve}{gbreve}}
   \comment{The letter `\u G'.}
\endsetslot

\setslot{\uctop{Lacute}{lacute}}
   \comment{The letter `\' L'.}
\endsetslot

\setslot{\uc{Lcaron}{lcaron}}
   \comment{The letter `\v L'.}
\endsetslot

\setslot{\uc{Lslash}{lslash}}
   \comment{The letter `\L'.}
\endsetslot

\setslot{\uctop{Nacute}{nacute}}
   \comment{The letter `\' N'.}
\endsetslot

\setslot{\uctop{Ncaron}{ncaron}}
   \comment{The letter `\v N'.}
\endsetslot

\setslot{\uc{Eng}{eng}}
   \comment{The Sami letter `\NG'.  It is unavailable in \plain\ \TeX. This needs to be called `Eng'/`eng' rather than `Ng'/`ng' as in t1.etx in most cases, it seems.}
\endsetslot

\setslot{\uctop{Ohungarumlaut}{ohungarumlaut}}
   \comment{The letter `\H O'.}
\endsetslot

\setslot{\uctop{Racute}{racute}}
   \comment{The letter `\' R'.}
\endsetslot

\setslot{\uctop{Rcaron}{rcaron}}
   \comment{The letter `\v R'.}
\endsetslot

\setslot{\uctop{Sacute}{sacute}}
   \comment{The letter `\' S'.}
\endsetslot

\setslot{\uctop{Scaron}{scaron}}
   \comment{The letter `\v S'.}
\endsetslot

\setslot{\uc{Scedilla}{scedilla}}
   \comment{The letter `\c S'.}
\endsetslot

\setslot{\uctop{Tcaron}{tcaron}}
   \comment{The letter `\v T'.}
\endsetslot

\setslot{\uc{Tcedilla}{tcedilla}}
   \comment{The letter `\c T'.}
\endsetslot

\setslot{\uctop{Uhungarumlaut}{uhungarumlaut}}
   \comment{The letter `\H U'.}
\endsetslot

\setslot{\uctop{Uring}{uring}}
   \comment{The letter `\r U'.}
\endsetslot

\setslot{\uctop{Ydieresis}{ydieresis}}
   \comment{The letter `\" Y'.}
\endsetslot

\setslot{\uctop{Zacute}{zacute}}
   \comment{The letter `\' Z'.}
\endsetslot

\setslot{\uctop{Zcaron}{zcaron}}
   \comment{The letter `\v Z'.}
\endsetslot

\setslot{\uctop{Zdotaccent}{zdotaccent}}
   \comment{The letter `\. Z'.}
\endsetslot

\ifnumber{\int{ligaturing}}<{0}\then \skipslots{1}\Else

   \setslot{\uclig{IJ}{ij}}
      \comment{The letter `IJ'.  This is a single letter, and in a 
        monowidth font should ideally be one letter wide.}
   \endsetslot

\Fi

\setslot{\uctop{Idotaccent}{idotaccent}}
   \comment{The letter `\. I'.}
\endsetslot

\setslot{\lc{Dbar}{dbar}}
   \comment{The letter `\dj'.}
\endsetslot

\setslot{section}
   \comment{The section mark `\textsection'.}
\endsetslot

\setslot{\lctop{Abreve}{abreve}}
   \comment{The letter `\u a'.}
\endsetslot

\setslot{\lc{Aogonek}{aogonek}}
   \comment{The letter `\k a'.}
\endsetslot

\setslot{\lctop{Cacute}{cacute}}
   \comment{The letter `\' c'.}
\endsetslot

\setslot{\lctop{Ccaron}{ccaron}}
   \comment{The letter `\v c'.}
\endsetslot

\setslot{\lctop{Dcaron}{dcaron}}
   \comment{The letter `\v d'.}
\endsetslot

\setslot{\lctop{Ecaron}{ecaron}}
   \comment{The letter `\v e'.}
\endsetslot

\setslot{\lc{Eogonek}{eogonek}}
   \comment{The letter `\k e'.}
\endsetslot

\setslot{\lctop{Gbreve}{gbreve}}
   \comment{The letter `\u g'.}
\endsetslot

\setslot{\lctop{Lacute}{lacute}}
   \comment{The letter `\' l'.}
\endsetslot

\setslot{\lc{Lcaron}{lcaron}}
   \comment{The letter `\v l'.}
\endsetslot

\setslot{\lc{Lslash}{lslash}}
   \comment{The letter `\l'.}
\endsetslot

\setslot{\lctop{Nacute}{nacute}}
   \comment{The letter `\' n'.}
\endsetslot

\setslot{\lctop{Ncaron}{ncaron}}
   \comment{The letter `\v n'.}
\endsetslot

\setslot{\lc{Eng}{eng}}
   \comment{The Sami letter `\ng'.  It is unavailable in \plain\ \TeX. This needs to be called `Eng'/`eng' rather than `Ng'/`ng' as it is in t1.etx in most cases, it seems.}
\endsetslot

\setslot{\lctop{Ohungarumlaut}{ohungarumlaut}}
   \comment{The letter `\H o'.}
\endsetslot

\setslot{\lctop{Racute}{racute}}
   \comment{The letter `\' r'.}
\endsetslot

\setslot{\lctop{Rcaron}{rcaron}}
   \comment{The letter `\v r'.}
\endsetslot

\setslot{\lctop{Sacute}{sacute}}
   \comment{The letter `\' s'.}
\endsetslot

\setslot{\lctop{Scaron}{scaron}}
   \comment{The letter `\v s'.}
\endsetslot

\setslot{\lc{Scedilla}{scedilla}}
   \comment{The letter `\c s'.}
\endsetslot

\setslot{\lctop{Tcaron}{tcaron}}
   \comment{The letter `\v t'.}
\endsetslot

\setslot{\lc{Tcedilla}{tcedilla}}
   \comment{The letter `\c t'.}
\endsetslot

\setslot{\lctop{Uhungarumlaut}{uhungarumlaut}}
   \comment{The letter `\H u'.}
\endsetslot

\setslot{\lctop{Uring}{uring}}
   \comment{The letter `\r u'.}
\endsetslot

\setslot{\lctop{Ydieresis}{ydieresis}}
   \comment{The letter `\" y'.}
\endsetslot

\setslot{\lctop{Zacute}{zacute}}
   \comment{The letter `\' z'.}
\endsetslot

\setslot{\lctop{Zcaron}{zcaron}}
   \comment{The letter `\v z'.}
\endsetslot

\setslot{\lctop{Zdotaccent}{zdotaccent}}
   \comment{The letter `\. z'.}
\endsetslot

\ifnumber{\int{ligaturing}}<{0}\then \skipslots{1}\Else

   \setslot{\lclig{IJ}{ij}}
      \comment{The letter `ij'.  This is a single letter, and in a 
        monowidth font should ideally be one letter wide.}
   \endsetslot

\Fi

\setslot{exclamdown}
   \comment{The Spanish punctuation mark `!`'.}
\endsetslot

\setslot{questiondown}
   \comment{The Spanish punctuation mark `?`'.}
\endsetslot

\setslot{sterling}
   \comment{The British currency mark `\textsterling'.}
\endsetslot

\setslot{\uctop{Agrave}{agrave}}
   \comment{The letter `\` A'.}
\endsetslot

\setslot{\uctop{Aacute}{aacute}}
   \comment{The letter `\' A'.}
\endsetslot

\setslot{\uctop{Acircumflex}{acircumflex}}
   \comment{The letter `\^ A'.}
\endsetslot

\setslot{\uctop{Atilde}{atilde}}
   \comment{The letter `\~ A'.}
\endsetslot

\setslot{\uctop{Adieresis}{adieresis}}
   \comment{The letter `\" A'.}
\endsetslot

\setslot{\uctop{Aring}{aring}}
   \comment{The letter `\r A'.}
\endsetslot

\setslot{\uc{AE}{ae}}
   \comment{The letter `\AE'.  This is a single letter, and should not be
      faked with `AE'.}
\endsetslot

\setslot{\uc{Ccedilla}{ccedilla}}
   \comment{The letter `\c C'.}
\endsetslot

\setslot{\uctop{Egrave}{egrave}}
   \comment{The letter `\` E'.}
\endsetslot

\setslot{\uctop{Eacute}{eacute}}
   \comment{The letter `\' E'.}
\endsetslot

\setslot{\uctop{Ecircumflex}{ecircumflex}}
   \comment{The letter `\^ E'.}
\endsetslot

\setslot{\uctop{Edieresis}{edieresis}}
   \comment{The letter `\" E'.}
\endsetslot

\setslot{\uctop{Igrave}{igrave}}
   \comment{The letter `\` I'.}
\endsetslot

\setslot{\uctop{Iacute}{iacute}}
   \comment{The letter `\' I'.}
\endsetslot

\setslot{\uctop{Icircumflex}{icircumflex}}
   \comment{The letter `\^ I'.}
\endsetslot

\setslot{\uctop{Idieresis}{idieresis}}
   \comment{The letter `\" I'.}
\endsetslot

\setslot{\uc{Eth}{eth}}
   \comment{The uppercase Icelandic letter `Eth' similar to a `D'
      with a horizontal bar through the stem.  It is unavailable
      in \plain\ \TeX.}
\endsetslot

\setslot{\uctop{Ntilde}{ntilde}}
   \comment{The letter `\~ N'.}
\endsetslot

\setslot{\uctop{Ograve}{ograve}}
   \comment{The letter `\` O'.}
\endsetslot

\setslot{\uctop{Oacute}{oacute}}
   \comment{The letter `\' O'.}
\endsetslot

\setslot{\uctop{Ocircumflex}{ocircumflex}}
   \comment{The letter `\^ O'.}
\endsetslot

\setslot{\uctop{Otilde}{otilde}}
   \comment{The letter `\~ O'.}
\endsetslot

\setslot{\uctop{Odieresis}{odieresis}}
   \comment{The letter `\" O'.}
\endsetslot

\setslot{\uc{OE}{oe}}
   \comment{The letter `\OE'.  This is a single letter, and should not be
      faked with `OE'.}
\endsetslot

\setslot{\uc{Oslash}{oslash}}
   \comment{The letter `\O'.}
\endsetslot

\setslot{\uctop{Ugrave}{ugrave}}
   \comment{The letter `\` U'.}
\endsetslot

\setslot{\uctop{Uacute}{uacute}}
   \comment{The letter `\' U'.}
\endsetslot

\setslot{\uctop{Ucircumflex}{ucircumflex}}
   \comment{The letter `\^ U'.}
\endsetslot

\setslot{\uctop{Udieresis}{udieresis}}
   \comment{The letter `\" U'.}
\endsetslot

\setslot{\uctop{Yacute}{yacute}}
   \comment{The letter `\' Y'.}
\endsetslot

\setslot{\uc{Thorn}{thorn}}
   \comment{The Icelandic capital letter Thorn, similar to a `P'
      with the bowl moved down.  It is unavailable in \plain\ \TeX.}
\endsetslot

\setslot{\uclig{SS}{germandbls}}
   \comment{The ligature `SS', used to give an upper case `\ss'.
      In a monowidth font it should be two letters wide.}
\endsetslot

\setslot{\lctop{Agrave}{agrave}}
   \comment{The letter `\` a'.}
\endsetslot

\setslot{\lctop{Aacute}{aacute}}
   \comment{The letter `\' a'.}
\endsetslot

\setslot{\lctop{Acircumflex}{acircumflex}}
   \comment{The letter `\^ a'.}
\endsetslot

\setslot{\lctop{Atilde}{atilde}}
   \comment{The letter `\~ a'.}
\endsetslot

\setslot{\lctop{Adieresis}{adieresis}}
   \comment{The letter `\" a'.}
\endsetslot

\setslot{\lctop{Aring}{aring}}
   \comment{The letter `\r a'.}
\endsetslot

\setslot{\lc{AE}{ae}}
   \comment{The letter `\ae'.  This is a single letter, and should not be
      faked with `ae'.}
\endsetslot

\setslot{\lc{Ccedilla}{ccedilla}}
   \comment{The letter `\c c'.}
\endsetslot

\setslot{\lctop{Egrave}{egrave}}
   \comment{The letter `\` e'.}
\endsetslot

\setslot{\lctop{Eacute}{eacute}}
   \comment{The letter `\' e'.}
\endsetslot

\setslot{\lctop{Ecircumflex}{ecircumflex}}
   \comment{The letter `\^ e'.}
\endsetslot

\setslot{\lctop{Edieresis}{edieresis}}
   \comment{The letter `\" e'.}
\endsetslot

\setslot{\lctop{Igrave}{igrave}}
   \comment{The letter `\`\i'.}
\endsetslot

\setslot{\lctop{Iacute}{iacute}}
   \comment{The letter `\'\i'.}
\endsetslot

\setslot{\lctop{Icircumflex}{icircumflex}}
   \comment{The letter `\^\i'.}
\endsetslot

\setslot{\lctop{Idieresis}{idieresis}}
   \comment{The letter `\"\i'.}
\endsetslot

\setslot{\lc{Eth}{eth}}
   \comment{The Icelandic lowercase letter `eth' similar to
     a `$\partial$' with an oblique bar through the stem.
     It is unavailable in \plain\ \TeX.}
\endsetslot

\setslot{\lctop{Ntilde}{ntilde}}
   \comment{The letter `\~ n'.}
\endsetslot

\setslot{\lctop{Ograve}{ograve}}
   \comment{The letter `\` o'.}
\endsetslot

\setslot{\lctop{Oacute}{oacute}}
   \comment{The letter `\' o'.}
\endsetslot

\setslot{\lctop{Ocircumflex}{ocircumflex}}
   \comment{The letter `\^ o'.}
\endsetslot

\setslot{\lctop{Otilde}{otilde}}
   \comment{The letter `\~ o'.}
\endsetslot

\setslot{\lctop{Odieresis}{odieresis}}
   \comment{The letter `\" o'.}
\endsetslot

\setslot{\lc{OE}{oe}}
   \comment{The letter `\oe'.  This is a single letter, and should not be
      faked with `oe'.}
\endsetslot

\setslot{\lc{Oslash}{oslash}}
   \comment{The letter `\o'.}
\endsetslot

\setslot{\lctop{Ugrave}{ugrave}}
   \comment{The letter `\` u'.}
\endsetslot

\setslot{\lctop{Uacute}{uacute}}
   \comment{The letter `\' u'.}
\endsetslot

\setslot{\lctop{Ucircumflex}{ucircumflex}}
   \comment{The letter `\^ u'.}
\endsetslot

\setslot{\lctop{Udieresis}{udieresis}}
   \comment{The letter `\" u'.}
\endsetslot

\setslot{\lctop{Yacute}{yacute}}
   \comment{The letter `\' y'.}
\endsetslot

\setslot{\lc{Thorn}{thorn}}
   \comment{The Icelandic lowercase letter `thorn', similar to a `p'
      with an ascender rising from the stem.  It is unavailable
      in \plain\ \TeX.}
\endsetslot

\setslot{\lc{SS}{germandbls}}
   \comment{The letter `\ss'.}
\endsetslot

\endencoding
%    \end{macrocode}
% \end{encoding}
% \iffalse
%</t1-dotalt-f-f>
% \fi
% 
% 
% \subsubsection{fontscripts-t1-dotinf.etx}\label{subsubsec:t1-dotinf}
% 
% \iffalse
%<*t1-dotinf>
% \fi
% \begin{encoding}{fontscripts-t1-dotinf.etx}
% \changes{v0.0}{2025-02-10}{Filename prefix for Karl.}
%    \begin{macrocode}
%%
%% - The original notices at the top of that file concerning authors,
%% maintenance etc. are replaced by this notice.
%% - The file is renamed.
%% - The encoding name is modified.
%% - The file is modified to accommodate differences in glyph names.
%% - The file is modified for use in encoding inferiors.
%%
%%%%%%%%%%%%%%%%%%%%%%%%%%%%%%%%%%%%%%%%%%%%%%%%%
\relax
\encoding

\needsfontinstversion{1.910}

\setcommand\lc#1#2{#2.inferior}
\setcommand\uc#1#2{#1.inferior}
\setcommand\lctop#1#2{#2.inferior}
\setcommand\uctop#1#2{#1.inferior}
\setcommand\lclig#1#2{#2.inferior}
\ifisint{letterspacing}\then
   \ifnumber{\int{letterspacing}}={0}\then \Else
      \setcommand\uclig#1#2{#1spaced}
      \comment{Here we set \verb|\uclig#1#2| to \verb|#1spaced|, but 
      you can't see it as \verb|\setcommand| commands are invisible in 
      the typeset output.}
   \Fi
\Fi
\setcommand\uclig#1#2{#1.inferior}
\setcommand\digit#1{#1.inferior}

\ifisint{monowidth}\then
   \setint{ligaturing}{0}
\Else
   % The following empty line is *important* to get the formatting
   % right here (sigh)! (Remember that it is a \par token.)
   
   \ifisint{letterspacing}\then
      \ifnumber{\int{letterspacing}}={0}\then \Else
         \setint{ligaturing}{0}
      \Fi
   \Fi
	\setint{ligaturing}{1}
\Fi

\setint{italicslant}{0}
\setint{quad}{1000}
\setint{baselineskip}{1200}

\ifisglyph{x}\then
   \setint{xheight}{\height{x}}
\Else
   \setint{xheight}{500}
\Fi

\ifisglyph{space}\then
   \setint{interword}{\width{space}}
\Else\ifisglyph{i}\then
   \setint{interword}{\width{i}}
\Else
   \setint{interword}{333}
\Fi\Fi

\ifisint{monowidth}\then
   \setint{stretchword}{0}
   \setint{shrinkword}{0}
   \setint{extraspace}{\int{interword}}
\Else
   \setint{stretchword}{\scale{\int{interword}}{600}}
   \setint{shrinkword}{\scale{\int{interword}}{240}}
   \setint{extraspace}{\scale{\int{interword}}{240}}
\Fi

\ifisglyph{X}\then
   \setint{capheight}{\height{X}}
\Else
   \setint{capheight}{750}
\Fi

\ifisglyph{d}\then
   \setint{ascender}{\height{d}}
\Else\ifisint{capheight}\then
   \setint{ascender}{\int{capheight}}
\Else
   \setint{ascender}{750}
\Fi\Fi

\ifisglyph{Aring}\then
   \setint{acccapheight}{\height{Aring}}
\Else
   \setint{acccapheight}{999}
\Fi

\ifisint{descender_neg}\then
   \setint{descender}{\neg{\int{descender_neg}}}
\Else\ifisglyph{p}\then
   \setint{descender}{\depth{p}}
\Else
   \setint{descender}{250}
\Fi\Fi

\ifisglyph{Aring}\then
   \setint{maxheight}{\height{Aring}}
\Else
   \setint{maxheight}{1000}
\Fi

\ifisint{maxdepth_neg}\then
   \setint{maxdepth}{\neg{\int{maxdepth_neg}}}
\Else\ifisglyph{j}\then
   \setint{maxdepth}{\depth{j}}
\Else
   \setint{maxdepth}{250}
\Fi\Fi

\ifisglyph{six}\then
   \setint{digitwidth}{\width{six}}
\Else
   \setint{digitwidth}{500}
\Fi

\setint{capstem}{0} % not in AFM files

\setfontdimen{1}{italicslant}    % italic slant
\setfontdimen{2}{interword}      % interword space
\setfontdimen{3}{stretchword}    % interword stretch
\setfontdimen{4}{shrinkword}     % interword shrink
\setfontdimen{5}{xheight}        % x-height
\setfontdimen{6}{quad}           % quad
\setfontdimen{7}{extraspace}     % extra space after .
\setfontdimen{8}{capheight}      % cap height
\setfontdimen{9}{ascender}       % ascender
\setfontdimen{10}{acccapheight}  % accented cap height
\setfontdimen{11}{descender}     % descender's depth
\setfontdimen{12}{maxheight}     % max height
\setfontdimen{13}{maxdepth}      % max depth
\setfontdimen{14}{digitwidth}    % digit width
\setfontdimen{15}{verticalstem}  % dominant width of verical stems
\setfontdimen{16}{baselineskip}  % baselineskip

\ifnumber{\int{ligaturing}}<{0}\then 
   \comment{In this case, the codingscheme can be different from the 
     default, and therefore we refrain from setting it.}
\Else
   \setstr{codingscheme}{EXTENDED TEX FONT ENCODING - DOTINF}
\Fi

\setslot{\lc{Grave}{grave}}
   \comment{The grave accent `\`{}'.}
\endsetslot

\setslot{\lc{Acute}{acute}}
   \comment{The acute accent `\'{}'.}
\endsetslot

\setslot{\lc{Circumflex}{circumflex}}
   \comment{The circumflex accent `\^{}'.}
\endsetslot

\setslot{\lc{Tilde}{tilde}}
   \comment{The tilde accent `\~{}'.}
\endsetslot

\setslot{\lc{Dieresis}{dieresis}}
   \comment{The umlaut or dieresis accent `\"{}'.}
\endsetslot

\setslot{\lc{Hungarumlaut}{hungarumlaut}}
   \comment{The long Hungarian umlaut `\H{}'.}
\endsetslot

\setslot{\lc{Ring}{ring}}
   \comment{The ring accent `\r{}'.}
\endsetslot

\setslot{\lc{Caron}{caron}}
   \comment{The caron or h\'a\v cek accent `\v{}'.}
\endsetslot

\setslot{\lc{Breve}{breve}}
   \comment{The breve accent `\u{}'.}
\endsetslot

\setslot{\lc{Macron}{macron}}
   \comment{The macron accent `\={}'.}
\endsetslot

\setslot{\lc{Dotaccent}{dotaccent}}
   \comment{The dot accent `\.{}'.}
\endsetslot

\setslot{\lc{Cedilla}{cedilla}}
   \comment{The cedilla accent `\c {}'.}
\endsetslot

\setslot{\lc{Ogonek}{ogonek}}
   \comment{The ogonek accent `\k {}'.}
\endsetslot

\setslot{quotesinglbase.inferior}
  \comment{A German single quote mark `\quotesinglbase' similar to a comma,
      but with different sidebearings.}
\endsetslot

\setslot{guilsinglleft.inferior}
  \comment{A French single opening quote mark `\guilsinglleft',
      unavailable in \plain\ \TeX.}
\endsetslot

\setslot{guilsinglright.inferior}
  \comment{A French single closing quote mark `\guilsinglright',
      unavailable in \plain\ \TeX.}
\endsetslot

\setslot{quotedblleft.inferior}
  \comment{The English opening quote mark `\,\textquotedblleft\,'.}
\endsetslot

\setslot{quotedblright.inferior}
  \comment{The English closing quote mark `\,\textquotedblright\,'.}
\endsetslot

\setslot{quotedblbase.inferior}
  \comment{A German double quote mark `\quotedblbase' similar to two commas,
      but with tighter letterspacing and different sidebearings.}
\endsetslot

\setslot{guillemotleft.inferior}
  \comment{A French double opening quote mark `\guillemotleft',
      unavailable in \plain\ \TeX.}
\endsetslot

\setslot{guillemotright.inferior}
  \comment{A French closing opening quote mark `\guillemotright',
      unavailable in \plain\ \TeX.}
\endsetslot

\setslot{endash.inferior}
   \ligature{LIG}{hyphen.inferior}{emdash.inferior}
   \comment{The number range dash `1--9'. 
     This is called `rangedash' by fontinst's t1.etx, but it needs to be 
     called `endash' to work right. 
     The `\textendash'.  
     In a monowidth font, this might be set as `\texttt{1{-}9}'.}
\endsetslot

\setslot{emdash.inferior}
   \comment{The punctuation dash `Oh---boy.' 
     This is calle `punctdash' by fontinst's t1.etx, but needs to be 
     called `emdash' to work right. 
     The `\textemdash'.  
     In a monowidth font, this might be set as `\texttt{Oh{-}{-}boy.}'}
\endsetslot

\setslot{compwordmark.inferior}
   \comment{An invisible glyph, with zero width and depth, but the
      height of lowercase letters without ascenders.
      It is used to stop ligaturing in words like `shelf{}ful'.}
\endsetslot

\setslot{perthousandzero.inferior}
   \comment{A glyph which is placed after `\%' to produce a
      `per-thousand', or twice to produce `per-ten-thousand'.
      Your guess is as good as mine as to what this glyph should look
      like in a monowidth font.}
\endsetslot

\setslot{\lc{dotlessI}{dotlessi}}
   \comment{A dotless i `\i', used to produce accented letters such as
      `\=\i'.}
\endsetslot

\setslot{\lc{dotlessJ}{dotlessj}}
   \comment{A dotless j `\j', used to produce accented letters such as
      `\=\j'.  Most non-\TeX\ fonts do not have this glyph.}
\endsetslot

\ifnumber{\int{ligaturing}}<{0}\then \skipslots{5}\Else

\setslot{\lclig{FF}{ff}}
   \ifnumber{\int{ligaturing}}>{0}\then
      \ligature{LIG}{\lc{I}{i}}{\lclig{FFI}{ffi}}
      \ligature{LIG}{\lc{L}{l}}{\lclig{FFL}{ffl}}
   \Fi
   \comment{The `ff' ligature.  It should be two characters wide in a
      monowidth font.}
\endsetslot

\setslot{\lclig{FI}{fi}}
   \comment{The `fi' ligature.  It should be two characters wide in a
      monowidth font.}
\endsetslot

\setslot{\lclig{FL}{fl}}
   \comment{The `fl' ligature.  It should be two characters wide in a
      monowidth font.}
\endsetslot

\setslot{\lclig{FFI}{ffi}}
   \comment{The `ffi' ligature.  It should be three characters wide in a
      monowidth font.}
\endsetslot

\setslot{\lclig{FFL}{ffl}}
   \comment{The `ffl' ligature.  It should be three characters wide in a
      monowidth font.}
\endsetslot

\Fi

\setslot{visiblespace.inferior}
   \comment{A visible space glyph `\textvisiblespace'.}
\endsetslot

\setslot{exclam.inferior}
   \ligature{LIG}{quoteleft.inferior}{exclamdown.inferior}
   \comment{The exclamation mark `!'.}
\endsetslot

\setslot{quotedbl.inferior}
  \comment{The `neutral' double quotation mark `\,\textquotedbl\,',
      included for use in monowidth fonts, or for setting computer
      programs.  Note that the inclusion of this glyph in this slot
      means that \TeX\ documents which used `{\tt\char`\"}' as an
      input character will no longer work.}
\endsetslot

\setslot{numbersign.inferior}
   \comment{The hash sign `\#'.}
\endsetslot

\setslot{dollar.inferior}
   \comment{The dollar sign `\$'.}
\endsetslot

\setslot{percent.inferior}
   \comment{The percent sign `\%'.}
\endsetslot

\setslot{ampersand.inferior}
   \comment{The ampersand sign `\&'.}
\endsetslot

\setslot{quoteright.inferior}
   \ligature{LIG}{quoteright.inferior}{quotedblright.inferior}
   \comment{The English closing single quote mark `\,\textquoteright\,'.}
\endsetslot

\setslot{parenleft.inferior}
   \comment{The opening parenthesis `('.}
\endsetslot

\setslot{parenright.inferior}
   \comment{The closing parenthesis `)'.}
\endsetslot

\setslot{asterisk.inferior}
   \comment{The raised asterisk `*'.}
\endsetslot

\setslot{plus.inferior}
   \comment{The addition sign `+'.}
\endsetslot

\setslot{comma.inferior}
   \ligature{LIG}{comma.inferior}{quotedblbase.inferior}
   \comment{The comma `,'.}
\endsetslot

\setslot{hyphen.inferior}
   \ligature{LIG}{hyphen.inferior}{endash.inferior}
   \ligature{LIG}{hyphenchar.inferior}{hyphenchar.inferior}
   \comment{The hyphen `-'.}
\endsetslot

\setslot{period.inferior}
   \comment{The period `.'.}
\endsetslot

\setslot{slash.inferior}
   \comment{The forward oblique `/'.}
\endsetslot

\setslot{\digit{zero}}
   \comment{The number `0'.  This (and all the other numerals) may be
      old style or ranging digits.}
\endsetslot

\setslot{\digit{one}}
   \comment{The number `1'.}
\endsetslot

\setslot{\digit{two}}
   \comment{The number `2'.}
\endsetslot

\setslot{\digit{three}}
   \comment{The number `3'.}
\endsetslot

\setslot{\digit{four}}
   \comment{The number `4'.}
\endsetslot

\setslot{\digit{five}}
   \comment{The number `5'.}
\endsetslot

\setslot{\digit{six}}
   \comment{The number `6'.}
\endsetslot

\setslot{\digit{seven}}
   \comment{The number `7'.}
\endsetslot

\setslot{\digit{eight}}
   \comment{The number `8'.}
\endsetslot

\setslot{\digit{nine}}
   \comment{The number `9'.}
\endsetslot

\setslot{colon.inferior}
   \comment{The colon punctuation mark `:'.}
\endsetslot

\setslot{semicolon.inferior}
   \comment{The semi-colon punctuation mark `;'.}
\endsetslot

\setslot{less.inferior}
   \ligature{LIG}{less.inferior}{guillemotleft.inferior}
   \comment{The less-than sign `\textless'.}
\endsetslot

\setslot{equal.inferior}
   \comment{The equals sign `='.}
\endsetslot

\setslot{greater.inferior}
   \ligature{LIG}{greater.inferior}{guillemotright.inferior}
   \comment{The greater-than sign `\textgreater'.}
\endsetslot

\setslot{question.inferior}
   \ligature{LIG}{quoteleft.inferior}{questiondown.inferior}
   \comment{The question mark `?'.}
\endsetslot

\setslot{at.inferior}
   \comment{The at sign `@'.}
\endsetslot

\setslot{\uc{A}{a}}
   \comment{The letter `{A}'.}
\endsetslot

\setslot{\uc{B}{b}}
   \comment{The letter `{B}'.}
\endsetslot

\setslot{\uc{C}{c}}
   \comment{The letter `{C}'.}
\endsetslot

\setslot{\uc{D}{d}}
   \comment{The letter `{D}'.}
\endsetslot

\setslot{\uc{E}{e}}
   \comment{The letter `{E}'.}
\endsetslot

\setslot{\uc{F}{f}}
   \comment{The letter `{F}'.}
\endsetslot

\setslot{\uc{G}{g}}
   \comment{The letter `{G}'.}
\endsetslot

\setslot{\uc{H}{h}}
   \comment{The letter `{H}'.}
\endsetslot

\ifnumber{\int{ligaturing}}<{-1}\then \skipslots{1}\Else

\setslot{\uc{I}{i}}
   \comment{The letter `{I}'.}
\endsetslot

\Fi

\setslot{\uc{J}{j}}
   \comment{The letter `{J}'.}
\endsetslot

\setslot{\uc{K}{k}}
   \comment{The letter `{K}'.}
\endsetslot

\setslot{\uc{L}{l}}
   \comment{The letter `{L}'.}
\endsetslot

\setslot{\uc{M}{m}}
   \comment{The letter `{M}'.}
\endsetslot

\setslot{\uc{N}{n}}
   \comment{The letter `{N}'.}
\endsetslot

\setslot{\uc{O}{o}}
   \comment{The letter `{O}'.}
\endsetslot

\setslot{\uc{P}{p}}
   \comment{The letter `{P}'.}
\endsetslot

\setslot{\uc{Q}{q}}
   \comment{The letter `{Q}'.}
\endsetslot

\setslot{\uc{R}{r}}
   \comment{The letter `{R}'.}
\endsetslot

\setslot{\uc{S}{s}}
   \comment{The letter `{S}'.}
\endsetslot

\setslot{\uc{T}{t}}
   \comment{The letter `{T}'.}
\endsetslot

\setslot{\uc{U}{u}}
   \comment{The letter `{U}'.}
\endsetslot

\setslot{\uc{V}{v}}
   \comment{The letter `{V}'.}
\endsetslot

\setslot{\uc{W}{w}}
   \comment{The letter `{W}'.}
\endsetslot

\setslot{\uc{X}{x}}
   \comment{The letter `{X}'.}
\endsetslot

\setslot{\uc{Y}{y}}
   \comment{The letter `{Y}'.}
\endsetslot

\setslot{\uc{Z}{z}}
   \comment{The letter `{Z}'.}
\endsetslot

\setslot{bracketleft.inferior}
   \comment{The opening square bracket `['.}
\endsetslot

\setslot{backslash.inferior}
   \comment{The backwards oblique `\textbackslash'.}
\endsetslot

\setslot{bracketright.inferior}
   \comment{The closing square bracket `]'.}
\endsetslot

\setslot{asciicircum.inferior}
   \comment{The ASCII upward-pointing arrow head `\textasciicircum'.
      This is included for compatibility with typewriter fonts used
      for computer listings.}
\endsetslot

\setslot{underscore.inferior}
   \comment{The ASCII underline character `\textunderscore', usually
      set on the baseline.
      This is included for compatibility with typewriter fonts used
      for computer listings.}
\endsetslot

\setslot{quoteleft.inferior}
   \ligature{LIG}{quoteleft.inferior}{quotedblleft.inferior}
   \comment{The English opening single quote mark `\,\textquoteleft\,'.}
\endsetslot

\setslot{\lc{A}{a}}
   \comment{The letter `{a}'.}
\endsetslot

\setslot{\lc{B}{b}}
   \comment{The letter `{b}'.}
\endsetslot

\ifnumber{\int{ligaturing}}<{-1}\then \skipslots{1}\Else

   \setslot{\lc{C}{c}}
      \comment{The letter `{c}'.}
   \endsetslot

\Fi

\setslot{\lc{D}{d}}
   \comment{The letter `{d}'.}
\endsetslot

\setslot{\lc{E}{e}}
   \comment{The letter `{e}'.}
\endsetslot

\ifnumber{\int{ligaturing}}<{-1}\then \skipslots{1}\Else

   \setslot{\lc{F}{f}}
      \ifnumber{\int{ligaturing}}>{0}\then
         \ligature{LIG}{\lc{I}{i}}{\lclig{FI}{fi}}
         \ligature{LIG}{\lc{F}{f}}{\lclig{FF}{ff}}
         \ligature{LIG}{\lc{L}{l}}{\lclig{FL}{fl}}
      \Fi
      \comment{The letter `{f}'.}
   \endsetslot

\Fi

\setslot{\lc{G}{g}}
   \comment{The letter `{g}'.}
\endsetslot

\setslot{\lc{H}{h}}
   \comment{The letter `{h}'.}
\endsetslot

\ifnumber{\int{ligaturing}}<{-1}\then \skipslots{1}\Else

   \setslot{\lc{I}{i}}
      \comment{The letter `{i}'.}
   \endsetslot

\Fi

\setslot{\lc{J}{j}}
   \comment{The letter `{j}'.}
\endsetslot

\setslot{\lc{K}{k}}
   \comment{The letter `{k}'.}
\endsetslot

\setslot{\lc{L}{l}}
   \comment{The letter `{l}'.}
\endsetslot

\setslot{\lc{M}{m}}
   \comment{The letter `{m}'.}
\endsetslot

\setslot{\lc{N}{n}}
   \comment{The letter `{n}'.}
\endsetslot

\setslot{\lc{O}{o}}
   \comment{The letter `{o}'.}
\endsetslot

\setslot{\lc{P}{p}}
   \comment{The letter `{p}'.}
\endsetslot

\setslot{\lc{Q}{q}}
   \comment{The letter `{q}'.}
\endsetslot

\setslot{\lc{R}{r}}
   \comment{The letter `{r}'.}
\endsetslot

\ifnumber{\int{ligaturing}}<{-1}\then \skipslots{1}\Else

   \setslot{\lc{S}{s}}
      \comment{The letter `{s}'.}
   \endsetslot

\Fi

\setslot{\lc{T}{t}}
   \comment{The letter `{t}'.}
\endsetslot

\setslot{\lc{U}{u}}
   \comment{The letter `{u}'.}
\endsetslot

\setslot{\lc{V}{v}}
   \comment{The letter `{v}'.}
\endsetslot

\setslot{\lc{W}{w}}
   \comment{The letter `{w}'.}
\endsetslot

\setslot{\lc{X}{x}}
   \comment{The letter `{x}'.}
\endsetslot

\setslot{\lc{Y}{y}}
   \comment{The letter `{y}'.}
\endsetslot

\setslot{\lc{Z}{z}}
   \comment{The letter `{z}'.}
\endsetslot

\setslot{braceleft.inferior}
   \comment{The opening curly brace `\textbraceleft'.}
\endsetslot

\setslot{bar.inferior}
   \comment{The ASCII vertical bar `\textbar'.
      This is included for compatibility with typewriter fonts used
      for computer listings.}
\endsetslot

\setslot{braceright.inferior}
   \comment{The closing curly brace `\textbraceright'.}
\endsetslot

\setslot{asciitilde.inferior}
   \comment{The ASCII tilde `\textasciitilde'.
      This is included for compatibility with typewriter fonts used
      for computer listings.}
\endsetslot

\setslot{hyphenchar.inferior}
   \comment{The glyph used for hyphenation in this font, which will
      almost always be the same as `hyphen'.}
\endsetslot

\setslot{\uctop{Abreve}{abreve}}
   \comment{The letter `\u A'.}
\endsetslot

\setslot{\uc{Aogonek}{aogonek}}
   \comment{The letter `\k A'.}
\endsetslot

\setslot{\uctop{Cacute}{cacute}}
   \comment{The letter `\' C'.}
\endsetslot

\setslot{\uctop{Ccaron}{ccaron}}
   \comment{The letter `\v C'.}
\endsetslot

\setslot{\uctop{Dcaron}{dcaron}}
   \comment{The letter `\v D'.}
\endsetslot

\setslot{\uctop{Ecaron}{ecaron}}
   \comment{The letter `\v E'.}
\endsetslot

\setslot{\uc{Eogonek}{eogonek}}
   \comment{The letter `\k E'.}
\endsetslot

\setslot{\uctop{Gbreve}{gbreve}}
   \comment{The letter `\u G'.}
\endsetslot

\setslot{\uctop{Lacute}{lacute}}
   \comment{The letter `\' L'.}
\endsetslot

\setslot{\uc{Lcaron}{lcaron}}
   \comment{The letter `\v L'.}
\endsetslot

\setslot{\uc{Lslash}{lslash}}
   \comment{The letter `\L'.}
\endsetslot

\setslot{\uctop{Nacute}{nacute}}
   \comment{The letter `\' N'.}
\endsetslot

\setslot{\uctop{Ncaron}{ncaron}}
   \comment{The letter `\v N'.}
\endsetslot

\setslot{\uc{Eng}{eng}}
   \comment{The Sami letter `\NG'.  It is unavailable in \plain\ \TeX. This needs to be called `Eng'/`eng' rather than `Ng'/`ng' as in t1.etx in most cases, it seems.}
\endsetslot

\setslot{\uctop{Ohungarumlaut}{ohungarumlaut}}
   \comment{The letter `\H O'.}
\endsetslot

\setslot{\uctop{Racute}{racute}}
   \comment{The letter `\' R'.}
\endsetslot

\setslot{\uctop{Rcaron}{rcaron}}
   \comment{The letter `\v R'.}
\endsetslot

\setslot{\uctop{Sacute}{sacute}}
   \comment{The letter `\' S'.}
\endsetslot

\setslot{\uctop{Scaron}{scaron}}
   \comment{The letter `\v S'.}
\endsetslot

\setslot{\uc{Scedilla}{scedilla}}
   \comment{The letter `\c S'.}
\endsetslot

\setslot{\uctop{Tcaron}{tcaron}}
   \comment{The letter `\v T'.}
\endsetslot

\setslot{\uc{Tcedilla}{tcedilla}}
   \comment{The letter `\c T'.}
\endsetslot

\setslot{\uctop{Uhungarumlaut}{uhungarumlaut}}
   \comment{The letter `\H U'.}
\endsetslot

\setslot{\uctop{Uring}{uring}}
   \comment{The letter `\r U'.}
\endsetslot

\setslot{\uctop{Ydieresis}{ydieresis}}
   \comment{The letter `\" Y'.}
\endsetslot

\setslot{\uctop{Zacute}{zacute}}
   \comment{The letter `\' Z'.}
\endsetslot

\setslot{\uctop{Zcaron}{zcaron}}
   \comment{The letter `\v Z'.}
\endsetslot

\setslot{\uctop{Zdotaccent}{zdotaccent}}
   \comment{The letter `\. Z'.}
\endsetslot

\ifnumber{\int{ligaturing}}<{0}\then \skipslots{1}\Else

   \setslot{\uclig{IJ}{ij}}
      \comment{The letter `IJ'.  This is a single letter, and in a 
        monowidth font should ideally be one letter wide.}
   \endsetslot

\Fi

\setslot{\uctop{Idotaccent}{idotaccent}}
   \comment{The letter `\. I'.}
\endsetslot

\setslot{\lc{Dbar}{dbar}}
   \comment{The letter `\dj'.}
\endsetslot

\setslot{section.inferior}
   \comment{The section mark `\textsection'.}
\endsetslot

\setslot{\lctop{Abreve}{abreve}}
   \comment{The letter `\u a'.}
\endsetslot

\setslot{\lc{Aogonek}{aogonek}}
   \comment{The letter `\k a'.}
\endsetslot

\setslot{\lctop{Cacute}{cacute}}
   \comment{The letter `\' c'.}
\endsetslot

\setslot{\lctop{Ccaron}{ccaron}}
   \comment{The letter `\v c'.}
\endsetslot

\setslot{\lctop{Dcaron}{dcaron}}
   \comment{The letter `\v d'.}
\endsetslot

\setslot{\lctop{Ecaron}{ecaron}}
   \comment{The letter `\v e'.}
\endsetslot

\setslot{\lc{Eogonek}{eogonek}}
   \comment{The letter `\k e'.}
\endsetslot

\setslot{\lctop{Gbreve}{gbreve}}
   \comment{The letter `\u g'.}
\endsetslot

\setslot{\lctop{Lacute}{lacute}}
   \comment{The letter `\' l'.}
\endsetslot

\setslot{\lc{Lcaron}{lcaron}}
   \comment{The letter `\v l'.}
\endsetslot

\setslot{\lc{Lslash}{lslash}}
   \comment{The letter `\l'.}
\endsetslot

\setslot{\lctop{Nacute}{nacute}}
   \comment{The letter `\' n'.}
\endsetslot

\setslot{\lctop{Ncaron}{ncaron}}
   \comment{The letter `\v n'.}
\endsetslot

\setslot{\lc{Eng}{eng}}
   \comment{The Sami letter `\ng'.  It is unavailable in \plain\ \TeX. This needs to be called `Eng'/`eng' rather than `Ng'/`ng' as it is in t1.etx in most cases, it seems.}
\endsetslot

\setslot{\lctop{Ohungarumlaut}{ohungarumlaut}}
   \comment{The letter `\H o'.}
\endsetslot

\setslot{\lctop{Racute}{racute}}
   \comment{The letter `\' r'.}
\endsetslot

\setslot{\lctop{Rcaron}{rcaron}}
   \comment{The letter `\v r'.}
\endsetslot

\setslot{\lctop{Sacute}{sacute}}
   \comment{The letter `\' s'.}
\endsetslot

\setslot{\lctop{Scaron}{scaron}}
   \comment{The letter `\v s'.}
\endsetslot

\setslot{\lc{Scedilla}{scedilla}}
   \comment{The letter `\c s'.}
\endsetslot

\setslot{\lctop{Tcaron}{tcaron}}
   \comment{The letter `\v t'.}
\endsetslot

\setslot{\lc{Tcedilla}{tcedilla}}
   \comment{The letter `\c t'.}
\endsetslot

\setslot{\lctop{Uhungarumlaut}{uhungarumlaut}}
   \comment{The letter `\H u'.}
\endsetslot

\setslot{\lctop{Uring}{uring}}
   \comment{The letter `\r u'.}
\endsetslot

\setslot{\lctop{Ydieresis}{ydieresis}}
   \comment{The letter `\" y'.}
\endsetslot

\setslot{\lctop{Zacute}{zacute}}
   \comment{The letter `\' z'.}
\endsetslot

\setslot{\lctop{Zcaron}{zcaron}}
   \comment{The letter `\v z'.}
\endsetslot

\setslot{\lctop{Zdotaccent}{zdotaccent}}
   \comment{The letter `\. z'.}
\endsetslot

\ifnumber{\int{ligaturing}}<{0}\then \skipslots{1}\Else

   \setslot{\lclig{IJ}{ij}}
      \comment{The letter `ij'.  This is a single letter, and in a 
        monowidth font should ideally be one letter wide.}
   \endsetslot

\Fi

\setslot{exclamdown.inferior}
   \comment{The Spanish punctuation mark `!`'.}
\endsetslot

\setslot{questiondown.inferior}
   \comment{The Spanish punctuation mark `?`'.}
\endsetslot

\setslot{sterling.inferior}
   \comment{The British currency mark `\textsterling'.}
\endsetslot

\setslot{\uctop{Agrave}{agrave}}
   \comment{The letter `\` A'.}
\endsetslot

\setslot{\uctop{Aacute}{aacute}}
   \comment{The letter `\' A'.}
\endsetslot

\setslot{\uctop{Acircumflex}{acircumflex}}
   \comment{The letter `\^ A'.}
\endsetslot

\setslot{\uctop{Atilde}{atilde}}
   \comment{The letter `\~ A'.}
\endsetslot

\setslot{\uctop{Adieresis}{adieresis}}
   \comment{The letter `\" A'.}
\endsetslot

\setslot{\uctop{Aring}{aring}}
   \comment{The letter `\r A'.}
\endsetslot

\setslot{\uc{AE}{ae}}
   \comment{The letter `\AE'.  This is a single letter, and should not be
      faked with `AE'.}
\endsetslot

\setslot{\uc{Ccedilla}{ccedilla}}
   \comment{The letter `\c C'.}
\endsetslot

\setslot{\uctop{Egrave}{egrave}}
   \comment{The letter `\` E'.}
\endsetslot

\setslot{\uctop{Eacute}{eacute}}
   \comment{The letter `\' E'.}
\endsetslot

\setslot{\uctop{Ecircumflex}{ecircumflex}}
   \comment{The letter `\^ E'.}
\endsetslot

\setslot{\uctop{Edieresis}{edieresis}}
   \comment{The letter `\" E'.}
\endsetslot

\setslot{\uctop{Igrave}{igrave}}
   \comment{The letter `\` I'.}
\endsetslot

\setslot{\uctop{Iacute}{iacute}}
   \comment{The letter `\' I'.}
\endsetslot

\setslot{\uctop{Icircumflex}{icircumflex}}
   \comment{The letter `\^ I'.}
\endsetslot

\setslot{\uctop{Idieresis}{idieresis}}
   \comment{The letter `\" I'.}
\endsetslot

\setslot{\uc{Eth}{eth}}
   \comment{The uppercase Icelandic letter `Eth' similar to a `D'
      with a horizontal bar through the stem.  It is unavailable
      in \plain\ \TeX.}
\endsetslot

\setslot{\uctop{Ntilde}{ntilde}}
   \comment{The letter `\~ N'.}
\endsetslot

\setslot{\uctop{Ograve}{ograve}}
   \comment{The letter `\` O'.}
\endsetslot

\setslot{\uctop{Oacute}{oacute}}
   \comment{The letter `\' O'.}
\endsetslot

\setslot{\uctop{Ocircumflex}{ocircumflex}}
   \comment{The letter `\^ O'.}
\endsetslot

\setslot{\uctop{Otilde}{otilde}}
   \comment{The letter `\~ O'.}
\endsetslot

\setslot{\uctop{Odieresis}{odieresis}}
   \comment{The letter `\" O'.}
\endsetslot

\setslot{\uc{OE}{oe}}
   \comment{The letter `\OE'.  This is a single letter, and should not be
      faked with `OE'.}
\endsetslot

\setslot{\uc{Oslash}{oslash}}
   \comment{The letter `\O'.}
\endsetslot

\setslot{\uctop{Ugrave}{ugrave}}
   \comment{The letter `\` U'.}
\endsetslot

\setslot{\uctop{Uacute}{uacute}}
   \comment{The letter `\' U'.}
\endsetslot

\setslot{\uctop{Ucircumflex}{ucircumflex}}
   \comment{The letter `\^ U'.}
\endsetslot

\setslot{\uctop{Udieresis}{udieresis}}
   \comment{The letter `\" U'.}
\endsetslot

\setslot{\uctop{Yacute}{yacute}}
   \comment{The letter `\' Y'.}
\endsetslot

\setslot{\uc{Thorn}{thorn}}
   \comment{The Icelandic capital letter Thorn, similar to a `P'
      with the bowl moved down.  It is unavailable in \plain\ \TeX.}
\endsetslot

\setslot{\uclig{SS}{germandbls}}
   \comment{The ligature `SS', used to give an upper case `\ss'.
      In a monowidth font it should be two letters wide.}
\endsetslot

\setslot{\lctop{Agrave}{agrave}}
   \comment{The letter `\` a'.}
\endsetslot

\setslot{\lctop{Aacute}{aacute}}
   \comment{The letter `\' a'.}
\endsetslot

\setslot{\lctop{Acircumflex}{acircumflex}}
   \comment{The letter `\^ a'.}
\endsetslot

\setslot{\lctop{Atilde}{atilde}}
   \comment{The letter `\~ a'.}
\endsetslot

\setslot{\lctop{Adieresis}{adieresis}}
   \comment{The letter `\" a'.}
\endsetslot

\setslot{\lctop{Aring}{aring}}
   \comment{The letter `\r a'.}
\endsetslot

\setslot{\lc{AE}{ae}}
   \comment{The letter `\ae'.  This is a single letter, and should not be
      faked with `ae'.}
\endsetslot

\setslot{\lc{Ccedilla}{ccedilla}}
   \comment{The letter `\c c'.}
\endsetslot

\setslot{\lctop{Egrave}{egrave}}
   \comment{The letter `\` e'.}
\endsetslot

\setslot{\lctop{Eacute}{eacute}}
   \comment{The letter `\' e'.}
\endsetslot

\setslot{\lctop{Ecircumflex}{ecircumflex}}
   \comment{The letter `\^ e'.}
\endsetslot

\setslot{\lctop{Edieresis}{edieresis}}
   \comment{The letter `\" e'.}
\endsetslot

\setslot{\lctop{Igrave}{igrave}}
   \comment{The letter `\`\i'.}
\endsetslot

\setslot{\lctop{Iacute}{iacute}}
   \comment{The letter `\'\i'.}
\endsetslot

\setslot{\lctop{Icircumflex}{icircumflex}}
   \comment{The letter `\^\i'.}
\endsetslot

\setslot{\lctop{Idieresis}{idieresis}}
   \comment{The letter `\"\i'.}
\endsetslot

\setslot{\lc{Eth}{eth}}
   \comment{The Icelandic lowercase letter `eth' similar to
     a `$\partial$' with an oblique bar through the stem.
     It is unavailable in \plain\ \TeX.}
\endsetslot

\setslot{\lctop{Ntilde}{ntilde}}
   \comment{The letter `\~ n'.}
\endsetslot

\setslot{\lctop{Ograve}{ograve}}
   \comment{The letter `\` o'.}
\endsetslot

\setslot{\lctop{Oacute}{oacute}}
   \comment{The letter `\' o'.}
\endsetslot

\setslot{\lctop{Ocircumflex}{ocircumflex}}
   \comment{The letter `\^ o'.}
\endsetslot

\setslot{\lctop{Otilde}{otilde}}
   \comment{The letter `\~ o'.}
\endsetslot

\setslot{\lctop{Odieresis}{odieresis}}
   \comment{The letter `\" o'.}
\endsetslot

\setslot{\lc{OE}{oe}}
   \comment{The letter `\oe'.  This is a single letter, and should not be
      faked with `oe'.}
\endsetslot

\setslot{\lc{Oslash}{oslash}}
   \comment{The letter `\o'.}
\endsetslot

\setslot{\lctop{Ugrave}{ugrave}}
   \comment{The letter `\` u'.}
\endsetslot

\setslot{\lctop{Uacute}{uacute}}
   \comment{The letter `\' u'.}
\endsetslot

\setslot{\lctop{Ucircumflex}{ucircumflex}}
   \comment{The letter `\^ u'.}
\endsetslot

\setslot{\lctop{Udieresis}{udieresis}}
   \comment{The letter `\" u'.}
\endsetslot

\setslot{\lctop{Yacute}{yacute}}
   \comment{The letter `\' y'.}
\endsetslot

\setslot{\lc{Thorn}{thorn}}
   \comment{The Icelandic lowercase letter `thorn', similar to a `p'
      with an ascender rising from the stem.  It is unavailable
      in \plain\ \TeX.}
\endsetslot

\setslot{\lc{SS}{germandbls}}
   \comment{The letter `\ss'.}
\endsetslot

\endencoding
%    \end{macrocode}
% \end{encoding}
% \iffalse
%</t1-dotinf>
% \fi
% 
% 
% \subsubsection{fontscripts-t1-dotinferior.etx}\label{subsubsec:t1-dotinferior}
% 
% \iffalse
%<*t1-dotinferior>
% \fi
% \begin{encoding}{fontscripts-t1-dotinferior.etx}
% \changes{v0.0}{2025-02-10}{Filename prefix for Karl.}
%    \begin{macrocode}
%%
%% - The original notices at the top of that file concerning authors,
%% maintenance etc. are replaced by this notice.
%% - The file is renamed.
%% - The encoding name is modified.
%% - The file is modified to accommodate differences in glyph names.
%% - The file is modified for use in encoding inferiors.
%%
%%%%%%%%%%%%%%%%%%%%%%%%%%%%%%%%%%%%%%%%%%%%%%%%%
\relax
\encoding

\needsfontinstversion{1.910}

\setcommand\lc#1#2{#2.inferior}
\setcommand\uc#1#2{#1.inferior}
\setcommand\lctop#1#2{#2.inferior}
\setcommand\uctop#1#2{#1.inferior}
\setcommand\lclig#1#2{#2.inferior}
\ifisint{letterspacing}\then
   \ifnumber{\int{letterspacing}}={0}\then \Else
      \setcommand\uclig#1#2{#1spaced}
      \comment{Here we set \verb|\uclig#1#2| to \verb|#1spaced|, but 
      you can't see it as \verb|\setcommand| commands are invisible in 
      the typeset output.}
   \Fi
\Fi
\setcommand\uclig#1#2{#1.inferior}
\setcommand\digit#1{#1.inferior}

\ifisint{monowidth}\then
   \setint{ligaturing}{0}
\Else
   % The following empty line is *important* to get the formatting
   % right here (sigh)! (Remember that it is a \par token.)
   
   \ifisint{letterspacing}\then
      \ifnumber{\int{letterspacing}}={0}\then \Else
         \setint{ligaturing}{0}
      \Fi
   \Fi
	\setint{ligaturing}{1}
\Fi

\setint{italicslant}{0}
\setint{quad}{1000}
\setint{baselineskip}{1200}

\ifisglyph{x}\then
   \setint{xheight}{\height{x}}
\Else
   \setint{xheight}{500}
\Fi

\ifisglyph{space}\then
   \setint{interword}{\width{space}}
\Else\ifisglyph{i}\then
   \setint{interword}{\width{i}}
\Else
   \setint{interword}{333}
\Fi\Fi

\ifisint{monowidth}\then
   \setint{stretchword}{0}
   \setint{shrinkword}{0}
   \setint{extraspace}{\int{interword}}
\Else
   \setint{stretchword}{\scale{\int{interword}}{600}}
   \setint{shrinkword}{\scale{\int{interword}}{240}}
   \setint{extraspace}{\scale{\int{interword}}{240}}
\Fi

\ifisglyph{X}\then
   \setint{capheight}{\height{X}}
\Else
   \setint{capheight}{750}
\Fi

\ifisglyph{d}\then
   \setint{ascender}{\height{d}}
\Else\ifisint{capheight}\then
   \setint{ascender}{\int{capheight}}
\Else
   \setint{ascender}{750}
\Fi\Fi

\ifisglyph{Aring}\then
   \setint{acccapheight}{\height{Aring}}
\Else
   \setint{acccapheight}{999}
\Fi

\ifisint{descender_neg}\then
   \setint{descender}{\neg{\int{descender_neg}}}
\Else\ifisglyph{p}\then
   \setint{descender}{\depth{p}}
\Else
   \setint{descender}{250}
\Fi\Fi

\ifisglyph{Aring}\then
   \setint{maxheight}{\height{Aring}}
\Else
   \setint{maxheight}{1000}
\Fi

\ifisint{maxdepth_neg}\then
   \setint{maxdepth}{\neg{\int{maxdepth_neg}}}
\Else\ifisglyph{j}\then
   \setint{maxdepth}{\depth{j}}
\Else
   \setint{maxdepth}{250}
\Fi\Fi

\ifisglyph{six}\then
   \setint{digitwidth}{\width{six}}
\Else
   \setint{digitwidth}{500}
\Fi

\setint{capstem}{0} % not in AFM files

\setfontdimen{1}{italicslant}    % italic slant
\setfontdimen{2}{interword}      % interword space
\setfontdimen{3}{stretchword}    % interword stretch
\setfontdimen{4}{shrinkword}     % interword shrink
\setfontdimen{5}{xheight}        % x-height
\setfontdimen{6}{quad}           % quad
\setfontdimen{7}{extraspace}     % extra space after .
\setfontdimen{8}{capheight}      % cap height
\setfontdimen{9}{ascender}       % ascender
\setfontdimen{10}{acccapheight}  % accented cap height
\setfontdimen{11}{descender}     % descender's depth
\setfontdimen{12}{maxheight}     % max height
\setfontdimen{13}{maxdepth}      % max depth
\setfontdimen{14}{digitwidth}    % digit width
\setfontdimen{15}{verticalstem}  % dominant width of verical stems
\setfontdimen{16}{baselineskip}  % baselineskip

\ifnumber{\int{ligaturing}}<{0}\then 
   \comment{In this case, the codingscheme can be different from the 
     default, and therefore we refrain from setting it.}
\Else
   \setstr{codingscheme}{EXTENDED TEX FONT ENCODING - DOTINFERIOR}
\Fi

\setslot{\lc{Grave}{grave}}
   \comment{The grave accent `\`{}'.}
\endsetslot

\setslot{\lc{Acute}{acute}}
   \comment{The acute accent `\'{}'.}
\endsetslot

\setslot{\lc{Circumflex}{circumflex}}
   \comment{The circumflex accent `\^{}'.}
\endsetslot

\setslot{\lc{Tilde}{tilde}}
   \comment{The tilde accent `\~{}'.}
\endsetslot

\setslot{\lc{Dieresis}{dieresis}}
   \comment{The umlaut or dieresis accent `\"{}'.}
\endsetslot

\setslot{\lc{Hungarumlaut}{hungarumlaut}}
   \comment{The long Hungarian umlaut `\H{}'.}
\endsetslot

\setslot{\lc{Ring}{ring}}
   \comment{The ring accent `\r{}'.}
\endsetslot

\setslot{\lc{Caron}{caron}}
   \comment{The caron or h\'a\v cek accent `\v{}'.}
\endsetslot

\setslot{\lc{Breve}{breve}}
   \comment{The breve accent `\u{}'.}
\endsetslot

\setslot{\lc{Macron}{macron}}
   \comment{The macron accent `\={}'.}
\endsetslot

\setslot{\lc{Dotaccent}{dotaccent}}
   \comment{The dot accent `\.{}'.}
\endsetslot

\setslot{\lc{Cedilla}{cedilla}}
   \comment{The cedilla accent `\c {}'.}
\endsetslot

\setslot{\lc{Ogonek}{ogonek}}
   \comment{The ogonek accent `\k {}'.}
\endsetslot

\setslot{quotesinglbase.inferior}
  \comment{A German single quote mark `\quotesinglbase' similar to a comma,
      but with different sidebearings.}
\endsetslot

\setslot{guilsinglleft.inferior}
  \comment{A French single opening quote mark `\guilsinglleft',
      unavailable in \plain\ \TeX.}
\endsetslot

\setslot{guilsinglright.inferior}
  \comment{A French single closing quote mark `\guilsinglright',
      unavailable in \plain\ \TeX.}
\endsetslot

\setslot{quotedblleft.inferior}
  \comment{The English opening quote mark `\,\textquotedblleft\,'.}
\endsetslot

\setslot{quotedblright.inferior}
  \comment{The English closing quote mark `\,\textquotedblright\,'.}
\endsetslot

\setslot{quotedblbase.inferior}
  \comment{A German double quote mark `\quotedblbase' similar to two commas,
      but with tighter letterspacing and different sidebearings.}
\endsetslot

\setslot{guillemotleft.inferior}
  \comment{A French double opening quote mark `\guillemotleft',
      unavailable in \plain\ \TeX.}
\endsetslot

\setslot{guillemotright.inferior}
  \comment{A French closing opening quote mark `\guillemotright',
      unavailable in \plain\ \TeX.}
\endsetslot

\setslot{endash.inferior}
   \ligature{LIG}{hyphen.inferior}{emdash.inferior}
   \comment{The number range dash `1--9'. 
     This is called `rangedash' by fontinst's t1.etx, but it needs to be 
     called `endash' to work right. 
     The `\textendash'.  
     In a monowidth font, this might be set as `\texttt{1{-}9}'.}
\endsetslot

\setslot{emdash.inferior}
   \comment{The punctuation dash `Oh---boy.' 
     This is calle `punctdash' by fontinst's t1.etx, but needs to be 
     called `emdash' to work right. 
     The `\textemdash'.  
     In a monowidth font, this might be set as `\texttt{Oh{-}{-}boy.}'}
\endsetslot

\setslot{compwordmark.inferior}
   \comment{An invisible glyph, with zero width and depth, but the
      height of lowercase letters without ascenders.
      It is used to stop ligaturing in words like `shelf{}ful'.}
\endsetslot

\setslot{perthousandzero.inferior}
   \comment{A glyph which is placed after `\%' to produce a
      `per-thousand', or twice to produce `per-ten-thousand'.
      Your guess is as good as mine as to what this glyph should look
      like in a monowidth font.}
\endsetslot

\setslot{\lc{dotlessI}{dotlessi}}
   \comment{A dotless i `\i', used to produce accented letters such as
      `\=\i'.}
\endsetslot

\setslot{\lc{dotlessJ}{dotlessj}}
   \comment{A dotless j `\j', used to produce accented letters such as
      `\=\j'.  Most non-\TeX\ fonts do not have this glyph.}
\endsetslot

\ifnumber{\int{ligaturing}}<{0}\then \skipslots{5}\Else

\setslot{\lclig{FF}{ff}}
   \ifnumber{\int{ligaturing}}>{0}\then
      \ligature{LIG}{\lc{I}{i}}{\lclig{FFI}{ffi}}
      \ligature{LIG}{\lc{L}{l}}{\lclig{FFL}{ffl}}
   \Fi
   \comment{The `ff' ligature.  It should be two characters wide in a
      monowidth font.}
\endsetslot

\setslot{\lclig{FI}{fi}}
   \comment{The `fi' ligature.  It should be two characters wide in a
      monowidth font.}
\endsetslot

\setslot{\lclig{FL}{fl}}
   \comment{The `fl' ligature.  It should be two characters wide in a
      monowidth font.}
\endsetslot

\setslot{\lclig{FFI}{ffi}}
   \comment{The `ffi' ligature.  It should be three characters wide in a
      monowidth font.}
\endsetslot

\setslot{\lclig{FFL}{ffl}}
   \comment{The `ffl' ligature.  It should be three characters wide in a
      monowidth font.}
\endsetslot

\Fi

\setslot{visiblespace.inferior}
   \comment{A visible space glyph `\textvisiblespace'.}
\endsetslot

\setslot{exclam.inferior}
   \ligature{LIG}{quoteleft.inferior}{exclamdown.inferior}
   \comment{The exclamation mark `!'.}
\endsetslot

\setslot{quotedbl.inferior}
  \comment{The `neutral' double quotation mark `\,\textquotedbl\,',
      included for use in monowidth fonts, or for setting computer
      programs.  Note that the inclusion of this glyph in this slot
      means that \TeX\ documents which used `{\tt\char`\"}' as an
      input character will no longer work.}
\endsetslot

\setslot{numbersign.inferior}
   \comment{The hash sign `\#'.}
\endsetslot

\setslot{dollar.inferior}
   \comment{The dollar sign `\$'.}
\endsetslot

\setslot{percent.inferior}
   \comment{The percent sign `\%'.}
\endsetslot

\setslot{ampersand.inferior}
   \comment{The ampersand sign `\&'.}
\endsetslot

\setslot{quoteright.inferior}
   \ligature{LIG}{quoteright.inferior}{quotedblright.inferior}
   \comment{The English closing single quote mark `\,\textquoteright\,'.}
\endsetslot

\setslot{parenleft.inferior}
   \comment{The opening parenthesis `('.}
\endsetslot

\setslot{parenright.inferior}
   \comment{The closing parenthesis `)'.}
\endsetslot

\setslot{asterisk.inferior}
   \comment{The raised asterisk `*'.}
\endsetslot

\setslot{plus.inferior}
   \comment{The addition sign `+'.}
\endsetslot

\setslot{comma.inferior}
   \ligature{LIG}{comma.inferior}{quotedblbase.inferior}
   \comment{The comma `,'.}
\endsetslot

\setslot{hyphen.inferior}
   \ligature{LIG}{hyphen.inferior}{endash.inferior}
   \ligature{LIG}{hyphenchar.inferior}{hyphenchar.inferior}
   \comment{The hyphen `-'.}
\endsetslot

\setslot{period.inferior}
   \comment{The period `.'.}
\endsetslot

\setslot{slash.inferior}
   \comment{The forward oblique `/'.}
\endsetslot

\setslot{\digit{zero}}
   \comment{The number `0'.  This (and all the other numerals) may be
      old style or ranging digits.}
\endsetslot

\setslot{\digit{one}}
   \comment{The number `1'.}
\endsetslot

\setslot{\digit{two}}
   \comment{The number `2'.}
\endsetslot

\setslot{\digit{three}}
   \comment{The number `3'.}
\endsetslot

\setslot{\digit{four}}
   \comment{The number `4'.}
\endsetslot

\setslot{\digit{five}}
   \comment{The number `5'.}
\endsetslot

\setslot{\digit{six}}
   \comment{The number `6'.}
\endsetslot

\setslot{\digit{seven}}
   \comment{The number `7'.}
\endsetslot

\setslot{\digit{eight}}
   \comment{The number `8'.}
\endsetslot

\setslot{\digit{nine}}
   \comment{The number `9'.}
\endsetslot

\setslot{colon.inferior}
   \comment{The colon punctuation mark `:'.}
\endsetslot

\setslot{semicolon.inferior}
   \comment{The semi-colon punctuation mark `;'.}
\endsetslot

\setslot{less.inferior}
   \ligature{LIG}{less.inferior}{guillemotleft.inferior}
   \comment{The less-than sign `\textless'.}
\endsetslot

\setslot{equal.inferior}
   \comment{The equals sign `='.}
\endsetslot

\setslot{greater.inferior}
   \ligature{LIG}{greater.inferior}{guillemotright.inferior}
   \comment{The greater-than sign `\textgreater'.}
\endsetslot

\setslot{question.inferior}
   \ligature{LIG}{quoteleft.inferior}{questiondown.inferior}
   \comment{The question mark `?'.}
\endsetslot

\setslot{at.inferior}
   \comment{The at sign `@'.}
\endsetslot

\setslot{\uc{A}{a}}
   \comment{The letter `{A}'.}
\endsetslot

\setslot{\uc{B}{b}}
   \comment{The letter `{B}'.}
\endsetslot

\setslot{\uc{C}{c}}
   \comment{The letter `{C}'.}
\endsetslot

\setslot{\uc{D}{d}}
   \comment{The letter `{D}'.}
\endsetslot

\setslot{\uc{E}{e}}
   \comment{The letter `{E}'.}
\endsetslot

\setslot{\uc{F}{f}}
   \comment{The letter `{F}'.}
\endsetslot

\setslot{\uc{G}{g}}
   \comment{The letter `{G}'.}
\endsetslot

\setslot{\uc{H}{h}}
   \comment{The letter `{H}'.}
\endsetslot

\ifnumber{\int{ligaturing}}<{-1}\then \skipslots{1}\Else

\setslot{\uc{I}{i}}
   \comment{The letter `{I}'.}
\endsetslot

\Fi

\setslot{\uc{J}{j}}
   \comment{The letter `{J}'.}
\endsetslot

\setslot{\uc{K}{k}}
   \comment{The letter `{K}'.}
\endsetslot

\setslot{\uc{L}{l}}
   \comment{The letter `{L}'.}
\endsetslot

\setslot{\uc{M}{m}}
   \comment{The letter `{M}'.}
\endsetslot

\setslot{\uc{N}{n}}
   \comment{The letter `{N}'.}
\endsetslot

\setslot{\uc{O}{o}}
   \comment{The letter `{O}'.}
\endsetslot

\setslot{\uc{P}{p}}
   \comment{The letter `{P}'.}
\endsetslot

\setslot{\uc{Q}{q}}
   \comment{The letter `{Q}'.}
\endsetslot

\setslot{\uc{R}{r}}
   \comment{The letter `{R}'.}
\endsetslot

\setslot{\uc{S}{s}}
   \comment{The letter `{S}'.}
\endsetslot

\setslot{\uc{T}{t}}
   \comment{The letter `{T}'.}
\endsetslot

\setslot{\uc{U}{u}}
   \comment{The letter `{U}'.}
\endsetslot

\setslot{\uc{V}{v}}
   \comment{The letter `{V}'.}
\endsetslot

\setslot{\uc{W}{w}}
   \comment{The letter `{W}'.}
\endsetslot

\setslot{\uc{X}{x}}
   \comment{The letter `{X}'.}
\endsetslot

\setslot{\uc{Y}{y}}
   \comment{The letter `{Y}'.}
\endsetslot

\setslot{\uc{Z}{z}}
   \comment{The letter `{Z}'.}
\endsetslot

\setslot{bracketleft.inferior}
   \comment{The opening square bracket `['.}
\endsetslot

\setslot{backslash.inferior}
   \comment{The backwards oblique `\textbackslash'.}
\endsetslot

\setslot{bracketright.inferior}
   \comment{The closing square bracket `]'.}
\endsetslot

\setslot{asciicircum.inferior}
   \comment{The ASCII upward-pointing arrow head `\textasciicircum'.
      This is included for compatibility with typewriter fonts used
      for computer listings.}
\endsetslot

\setslot{underscore.inferior}
   \comment{The ASCII underline character `\textunderscore', usually
      set on the baseline.
      This is included for compatibility with typewriter fonts used
      for computer listings.}
\endsetslot

\setslot{quoteleft.inferior}
   \ligature{LIG}{quoteleft.inferior}{quotedblleft.inferior}
   \comment{The English opening single quote mark `\,\textquoteleft\,'.}
\endsetslot

\setslot{\lc{A}{a}}
   \comment{The letter `{a}'.}
\endsetslot

\setslot{\lc{B}{b}}
   \comment{The letter `{b}'.}
\endsetslot

\ifnumber{\int{ligaturing}}<{-1}\then \skipslots{1}\Else

   \setslot{\lc{C}{c}}
      \comment{The letter `{c}'.}
   \endsetslot

\Fi

\setslot{\lc{D}{d}}
   \comment{The letter `{d}'.}
\endsetslot

\setslot{\lc{E}{e}}
   \comment{The letter `{e}'.}
\endsetslot

\ifnumber{\int{ligaturing}}<{-1}\then \skipslots{1}\Else

   \setslot{\lc{F}{f}}
      \ifnumber{\int{ligaturing}}>{0}\then
         \ligature{LIG}{\lc{I}{i}}{\lclig{FI}{fi}}
         \ligature{LIG}{\lc{F}{f}}{\lclig{FF}{ff}}
         \ligature{LIG}{\lc{L}{l}}{\lclig{FL}{fl}}
      \Fi
      \comment{The letter `{f}'.}
   \endsetslot

\Fi

\setslot{\lc{G}{g}}
   \comment{The letter `{g}'.}
\endsetslot

\setslot{\lc{H}{h}}
   \comment{The letter `{h}'.}
\endsetslot

\ifnumber{\int{ligaturing}}<{-1}\then \skipslots{1}\Else

   \setslot{\lc{I}{i}}
      \comment{The letter `{i}'.}
   \endsetslot

\Fi

\setslot{\lc{J}{j}}
   \comment{The letter `{j}'.}
\endsetslot

\setslot{\lc{K}{k}}
   \comment{The letter `{k}'.}
\endsetslot

\setslot{\lc{L}{l}}
   \comment{The letter `{l}'.}
\endsetslot

\setslot{\lc{M}{m}}
   \comment{The letter `{m}'.}
\endsetslot

\setslot{\lc{N}{n}}
   \comment{The letter `{n}'.}
\endsetslot

\setslot{\lc{O}{o}}
   \comment{The letter `{o}'.}
\endsetslot

\setslot{\lc{P}{p}}
   \comment{The letter `{p}'.}
\endsetslot

\setslot{\lc{Q}{q}}
   \comment{The letter `{q}'.}
\endsetslot

\setslot{\lc{R}{r}}
   \comment{The letter `{r}'.}
\endsetslot

\ifnumber{\int{ligaturing}}<{-1}\then \skipslots{1}\Else

   \setslot{\lc{S}{s}}
      \comment{The letter `{s}'.}
   \endsetslot

\Fi

\setslot{\lc{T}{t}}
   \comment{The letter `{t}'.}
\endsetslot

\setslot{\lc{U}{u}}
   \comment{The letter `{u}'.}
\endsetslot

\setslot{\lc{V}{v}}
   \comment{The letter `{v}'.}
\endsetslot

\setslot{\lc{W}{w}}
   \comment{The letter `{w}'.}
\endsetslot

\setslot{\lc{X}{x}}
   \comment{The letter `{x}'.}
\endsetslot

\setslot{\lc{Y}{y}}
   \comment{The letter `{y}'.}
\endsetslot

\setslot{\lc{Z}{z}}
   \comment{The letter `{z}'.}
\endsetslot

\setslot{braceleft.inferior}
   \comment{The opening curly brace `\textbraceleft'.}
\endsetslot

\setslot{bar.inferior}
   \comment{The ASCII vertical bar `\textbar'.
      This is included for compatibility with typewriter fonts used
      for computer listings.}
\endsetslot

\setslot{braceright.inferior}
   \comment{The closing curly brace `\textbraceright'.}
\endsetslot

\setslot{asciitilde.inferior}
   \comment{The ASCII tilde `\textasciitilde'.
      This is included for compatibility with typewriter fonts used
      for computer listings.}
\endsetslot

\setslot{hyphenchar.inferior}
   \comment{The glyph used for hyphenation in this font, which will
      almost always be the same as `hyphen'.}
\endsetslot

\setslot{\uctop{Abreve}{abreve}}
   \comment{The letter `\u A'.}
\endsetslot

\setslot{\uc{Aogonek}{aogonek}}
   \comment{The letter `\k A'.}
\endsetslot

\setslot{\uctop{Cacute}{cacute}}
   \comment{The letter `\' C'.}
\endsetslot

\setslot{\uctop{Ccaron}{ccaron}}
   \comment{The letter `\v C'.}
\endsetslot

\setslot{\uctop{Dcaron}{dcaron}}
   \comment{The letter `\v D'.}
\endsetslot

\setslot{\uctop{Ecaron}{ecaron}}
   \comment{The letter `\v E'.}
\endsetslot

\setslot{\uc{Eogonek}{eogonek}}
   \comment{The letter `\k E'.}
\endsetslot

\setslot{\uctop{Gbreve}{gbreve}}
   \comment{The letter `\u G'.}
\endsetslot

\setslot{\uctop{Lacute}{lacute}}
   \comment{The letter `\' L'.}
\endsetslot

\setslot{\uc{Lcaron}{lcaron}}
   \comment{The letter `\v L'.}
\endsetslot

\setslot{\uc{Lslash}{lslash}}
   \comment{The letter `\L'.}
\endsetslot

\setslot{\uctop{Nacute}{nacute}}
   \comment{The letter `\' N'.}
\endsetslot

\setslot{\uctop{Ncaron}{ncaron}}
   \comment{The letter `\v N'.}
\endsetslot

\setslot{\uc{Eng}{eng}}
   \comment{The Sami letter `\NG'.  It is unavailable in \plain\ \TeX. This needs to be called `Eng'/`eng' rather than `Ng'/`ng' as in t1.etx in most cases, it seems.}
\endsetslot

\setslot{\uctop{Ohungarumlaut}{ohungarumlaut}}
   \comment{The letter `\H O'.}
\endsetslot

\setslot{\uctop{Racute}{racute}}
   \comment{The letter `\' R'.}
\endsetslot

\setslot{\uctop{Rcaron}{rcaron}}
   \comment{The letter `\v R'.}
\endsetslot

\setslot{\uctop{Sacute}{sacute}}
   \comment{The letter `\' S'.}
\endsetslot

\setslot{\uctop{Scaron}{scaron}}
   \comment{The letter `\v S'.}
\endsetslot

\setslot{\uc{Scedilla}{scedilla}}
   \comment{The letter `\c S'.}
\endsetslot

\setslot{\uctop{Tcaron}{tcaron}}
   \comment{The letter `\v T'.}
\endsetslot

\setslot{\uc{Tcedilla}{tcedilla}}
   \comment{The letter `\c T'.}
\endsetslot

\setslot{\uctop{Uhungarumlaut}{uhungarumlaut}}
   \comment{The letter `\H U'.}
\endsetslot

\setslot{\uctop{Uring}{uring}}
   \comment{The letter `\r U'.}
\endsetslot

\setslot{\uctop{Ydieresis}{ydieresis}}
   \comment{The letter `\" Y'.}
\endsetslot

\setslot{\uctop{Zacute}{zacute}}
   \comment{The letter `\' Z'.}
\endsetslot

\setslot{\uctop{Zcaron}{zcaron}}
   \comment{The letter `\v Z'.}
\endsetslot

\setslot{\uctop{Zdotaccent}{zdotaccent}}
   \comment{The letter `\. Z'.}
\endsetslot

\ifnumber{\int{ligaturing}}<{0}\then \skipslots{1}\Else

   \setslot{\uclig{IJ}{ij}}
      \comment{The letter `IJ'.  This is a single letter, and in a 
        monowidth font should ideally be one letter wide.}
   \endsetslot

\Fi

\setslot{\uctop{Idotaccent}{idotaccent}}
   \comment{The letter `\. I'.}
\endsetslot

\setslot{\lc{Dbar}{dbar}}
   \comment{The letter `\dj'.}
\endsetslot

\setslot{section.inferior}
   \comment{The section mark `\textsection'.}
\endsetslot

\setslot{\lctop{Abreve}{abreve}}
   \comment{The letter `\u a'.}
\endsetslot

\setslot{\lc{Aogonek}{aogonek}}
   \comment{The letter `\k a'.}
\endsetslot

\setslot{\lctop{Cacute}{cacute}}
   \comment{The letter `\' c'.}
\endsetslot

\setslot{\lctop{Ccaron}{ccaron}}
   \comment{The letter `\v c'.}
\endsetslot

\setslot{\lctop{Dcaron}{dcaron}}
   \comment{The letter `\v d'.}
\endsetslot

\setslot{\lctop{Ecaron}{ecaron}}
   \comment{The letter `\v e'.}
\endsetslot

\setslot{\lc{Eogonek}{eogonek}}
   \comment{The letter `\k e'.}
\endsetslot

\setslot{\lctop{Gbreve}{gbreve}}
   \comment{The letter `\u g'.}
\endsetslot

\setslot{\lctop{Lacute}{lacute}}
   \comment{The letter `\' l'.}
\endsetslot

\setslot{\lc{Lcaron}{lcaron}}
   \comment{The letter `\v l'.}
\endsetslot

\setslot{\lc{Lslash}{lslash}}
   \comment{The letter `\l'.}
\endsetslot

\setslot{\lctop{Nacute}{nacute}}
   \comment{The letter `\' n'.}
\endsetslot

\setslot{\lctop{Ncaron}{ncaron}}
   \comment{The letter `\v n'.}
\endsetslot

\setslot{\lc{Eng}{eng}}
   \comment{The Sami letter `\ng'.  It is unavailable in \plain\ \TeX. This needs to be called `Eng'/`eng' rather than `Ng'/`ng' as it is in t1.etx in most cases, it seems.}
\endsetslot

\setslot{\lctop{Ohungarumlaut}{ohungarumlaut}}
   \comment{The letter `\H o'.}
\endsetslot

\setslot{\lctop{Racute}{racute}}
   \comment{The letter `\' r'.}
\endsetslot

\setslot{\lctop{Rcaron}{rcaron}}
   \comment{The letter `\v r'.}
\endsetslot

\setslot{\lctop{Sacute}{sacute}}
   \comment{The letter `\' s'.}
\endsetslot

\setslot{\lctop{Scaron}{scaron}}
   \comment{The letter `\v s'.}
\endsetslot

\setslot{\lc{Scedilla}{scedilla}}
   \comment{The letter `\c s'.}
\endsetslot

\setslot{\lctop{Tcaron}{tcaron}}
   \comment{The letter `\v t'.}
\endsetslot

\setslot{\lc{Tcedilla}{tcedilla}}
   \comment{The letter `\c t'.}
\endsetslot

\setslot{\lctop{Uhungarumlaut}{uhungarumlaut}}
   \comment{The letter `\H u'.}
\endsetslot

\setslot{\lctop{Uring}{uring}}
   \comment{The letter `\r u'.}
\endsetslot

\setslot{\lctop{Ydieresis}{ydieresis}}
   \comment{The letter `\" y'.}
\endsetslot

\setslot{\lctop{Zacute}{zacute}}
   \comment{The letter `\' z'.}
\endsetslot

\setslot{\lctop{Zcaron}{zcaron}}
   \comment{The letter `\v z'.}
\endsetslot

\setslot{\lctop{Zdotaccent}{zdotaccent}}
   \comment{The letter `\. z'.}
\endsetslot

\ifnumber{\int{ligaturing}}<{0}\then \skipslots{1}\Else

   \setslot{\lclig{IJ}{ij}}
      \comment{The letter `ij'.  This is a single letter, and in a 
        monowidth font should ideally be one letter wide.}
   \endsetslot

\Fi

\setslot{exclamdown.inferior}
   \comment{The Spanish punctuation mark `!`'.}
\endsetslot

\setslot{questiondown.inferior}
   \comment{The Spanish punctuation mark `?`'.}
\endsetslot

\setslot{sterling.inferior}
   \comment{The British currency mark `\textsterling'.}
\endsetslot

\setslot{\uctop{Agrave}{agrave}}
   \comment{The letter `\` A'.}
\endsetslot

\setslot{\uctop{Aacute}{aacute}}
   \comment{The letter `\' A'.}
\endsetslot

\setslot{\uctop{Acircumflex}{acircumflex}}
   \comment{The letter `\^ A'.}
\endsetslot

\setslot{\uctop{Atilde}{atilde}}
   \comment{The letter `\~ A'.}
\endsetslot

\setslot{\uctop{Adieresis}{adieresis}}
   \comment{The letter `\" A'.}
\endsetslot

\setslot{\uctop{Aring}{aring}}
   \comment{The letter `\r A'.}
\endsetslot

\setslot{\uc{AE}{ae}}
   \comment{The letter `\AE'.  This is a single letter, and should not be
      faked with `AE'.}
\endsetslot

\setslot{\uc{Ccedilla}{ccedilla}}
   \comment{The letter `\c C'.}
\endsetslot

\setslot{\uctop{Egrave}{egrave}}
   \comment{The letter `\` E'.}
\endsetslot

\setslot{\uctop{Eacute}{eacute}}
   \comment{The letter `\' E'.}
\endsetslot

\setslot{\uctop{Ecircumflex}{ecircumflex}}
   \comment{The letter `\^ E'.}
\endsetslot

\setslot{\uctop{Edieresis}{edieresis}}
   \comment{The letter `\" E'.}
\endsetslot

\setslot{\uctop{Igrave}{igrave}}
   \comment{The letter `\` I'.}
\endsetslot

\setslot{\uctop{Iacute}{iacute}}
   \comment{The letter `\' I'.}
\endsetslot

\setslot{\uctop{Icircumflex}{icircumflex}}
   \comment{The letter `\^ I'.}
\endsetslot

\setslot{\uctop{Idieresis}{idieresis}}
   \comment{The letter `\" I'.}
\endsetslot

\setslot{\uc{Eth}{eth}}
   \comment{The uppercase Icelandic letter `Eth' similar to a `D'
      with a horizontal bar through the stem.  It is unavailable
      in \plain\ \TeX.}
\endsetslot

\setslot{\uctop{Ntilde}{ntilde}}
   \comment{The letter `\~ N'.}
\endsetslot

\setslot{\uctop{Ograve}{ograve}}
   \comment{The letter `\` O'.}
\endsetslot

\setslot{\uctop{Oacute}{oacute}}
   \comment{The letter `\' O'.}
\endsetslot

\setslot{\uctop{Ocircumflex}{ocircumflex}}
   \comment{The letter `\^ O'.}
\endsetslot

\setslot{\uctop{Otilde}{otilde}}
   \comment{The letter `\~ O'.}
\endsetslot

\setslot{\uctop{Odieresis}{odieresis}}
   \comment{The letter `\" O'.}
\endsetslot

\setslot{\uc{OE}{oe}}
   \comment{The letter `\OE'.  This is a single letter, and should not be
      faked with `OE'.}
\endsetslot

\setslot{\uc{Oslash}{oslash}}
   \comment{The letter `\O'.}
\endsetslot

\setslot{\uctop{Ugrave}{ugrave}}
   \comment{The letter `\` U'.}
\endsetslot

\setslot{\uctop{Uacute}{uacute}}
   \comment{The letter `\' U'.}
\endsetslot

\setslot{\uctop{Ucircumflex}{ucircumflex}}
   \comment{The letter `\^ U'.}
\endsetslot

\setslot{\uctop{Udieresis}{udieresis}}
   \comment{The letter `\" U'.}
\endsetslot

\setslot{\uctop{Yacute}{yacute}}
   \comment{The letter `\' Y'.}
\endsetslot

\setslot{\uc{Thorn}{thorn}}
   \comment{The Icelandic capital letter Thorn, similar to a `P'
      with the bowl moved down.  It is unavailable in \plain\ \TeX.}
\endsetslot

\setslot{\uclig{SS}{germandbls}}
   \comment{The ligature `SS', used to give an upper case `\ss'.
      In a monowidth font it should be two letters wide.}
\endsetslot

\setslot{\lctop{Agrave}{agrave}}
   \comment{The letter `\` a'.}
\endsetslot

\setslot{\lctop{Aacute}{aacute}}
   \comment{The letter `\' a'.}
\endsetslot

\setslot{\lctop{Acircumflex}{acircumflex}}
   \comment{The letter `\^ a'.}
\endsetslot

\setslot{\lctop{Atilde}{atilde}}
   \comment{The letter `\~ a'.}
\endsetslot

\setslot{\lctop{Adieresis}{adieresis}}
   \comment{The letter `\" a'.}
\endsetslot

\setslot{\lctop{Aring}{aring}}
   \comment{The letter `\r a'.}
\endsetslot

\setslot{\lc{AE}{ae}}
   \comment{The letter `\ae'.  This is a single letter, and should not be
      faked with `ae'.}
\endsetslot

\setslot{\lc{Ccedilla}{ccedilla}}
   \comment{The letter `\c c'.}
\endsetslot

\setslot{\lctop{Egrave}{egrave}}
   \comment{The letter `\` e'.}
\endsetslot

\setslot{\lctop{Eacute}{eacute}}
   \comment{The letter `\' e'.}
\endsetslot

\setslot{\lctop{Ecircumflex}{ecircumflex}}
   \comment{The letter `\^ e'.}
\endsetslot

\setslot{\lctop{Edieresis}{edieresis}}
   \comment{The letter `\" e'.}
\endsetslot

\setslot{\lctop{Igrave}{igrave}}
   \comment{The letter `\`\i'.}
\endsetslot

\setslot{\lctop{Iacute}{iacute}}
   \comment{The letter `\'\i'.}
\endsetslot

\setslot{\lctop{Icircumflex}{icircumflex}}
   \comment{The letter `\^\i'.}
\endsetslot

\setslot{\lctop{Idieresis}{idieresis}}
   \comment{The letter `\"\i'.}
\endsetslot

\setslot{\lc{Eth}{eth}}
   \comment{The Icelandic lowercase letter `eth' similar to
     a `$\partial$' with an oblique bar through the stem.
     It is unavailable in \plain\ \TeX.}
\endsetslot

\setslot{\lctop{Ntilde}{ntilde}}
   \comment{The letter `\~ n'.}
\endsetslot

\setslot{\lctop{Ograve}{ograve}}
   \comment{The letter `\` o'.}
\endsetslot

\setslot{\lctop{Oacute}{oacute}}
   \comment{The letter `\' o'.}
\endsetslot

\setslot{\lctop{Ocircumflex}{ocircumflex}}
   \comment{The letter `\^ o'.}
\endsetslot

\setslot{\lctop{Otilde}{otilde}}
   \comment{The letter `\~ o'.}
\endsetslot

\setslot{\lctop{Odieresis}{odieresis}}
   \comment{The letter `\" o'.}
\endsetslot

\setslot{\lc{OE}{oe}}
   \comment{The letter `\oe'.  This is a single letter, and should not be
      faked with `oe'.}
\endsetslot

\setslot{\lc{Oslash}{oslash}}
   \comment{The letter `\o'.}
\endsetslot

\setslot{\lctop{Ugrave}{ugrave}}
   \comment{The letter `\` u'.}
\endsetslot

\setslot{\lctop{Uacute}{uacute}}
   \comment{The letter `\' u'.}
\endsetslot

\setslot{\lctop{Ucircumflex}{ucircumflex}}
   \comment{The letter `\^ u'.}
\endsetslot

\setslot{\lctop{Udieresis}{udieresis}}
   \comment{The letter `\" u'.}
\endsetslot

\setslot{\lctop{Yacute}{yacute}}
   \comment{The letter `\' y'.}
\endsetslot

\setslot{\lc{Thorn}{thorn}}
   \comment{The Icelandic lowercase letter `thorn', similar to a `p'
      with an ascender rising from the stem.  It is unavailable
      in \plain\ \TeX.}
\endsetslot

\setslot{\lc{SS}{germandbls}}
   \comment{The letter `\ss'.}
\endsetslot

\endencoding
%    \end{macrocode}
% \end{encoding}
% \iffalse
%</t1-dotinferior>
% \fi
% 
% 
% \subsubsection{fontscripts-t1-dotsup.etx}\label{subsubsec:t1-dotsup}
% 
% \iffalse
%<*t1-dotsup>
% \fi
% \begin{encoding}{fontscripts-t1-dotsup.etx}
% \changes{v0.0}{2025-02-10}{Filename prefix for Karl.}
%    \begin{macrocode}
%%
%% - The original notices at the top of that file concerning authors,
%% maintenance etc. are replaced by this notice.
%% - The file is renamed.
%% - The encoding name is modified.
%% - The file is modified to accommodate differences in glyph names.
%% - The file is modified for use in encoding superiors.
%%
%%%%%%%%%%%%%%%%%%%%%%%%%%%%%%%%%%%%%%%%%%%%%%%%%
\relax
\encoding

\needsfontinstversion{1.910}

\setcommand\lc#1#2{#2.superior}
\setcommand\uc#1#2{#1.superior}
\setcommand\lctop#1#2{#2.superior}
\setcommand\uctop#1#2{#1.superior}
\setcommand\lclig#1#2{#2.superior}
\ifisint{letterspacing}\then
   \ifnumber{\int{letterspacing}}={0}\then \Else
      \setcommand\uclig#1#2{#1spaced}
      \comment{Here we set \verb|\uclig#1#2| to \verb|#1spaced|, but 
      you can't see it as \verb|\setcommand| commands are invisible in 
      the typeset output.}
   \Fi
\Fi
\setcommand\uclig#1#2{#1.superior}
\setcommand\digit#1{#1.superior}

\ifisint{monowidth}\then
   \setint{ligaturing}{0}
\Else
   % The following empty line is *important* to get the formatting
   % right here (sigh)! (Remember that it is a \par token.)
   
   \ifisint{letterspacing}\then
      \ifnumber{\int{letterspacing}}={0}\then \Else
         \setint{ligaturing}{0}
      \Fi
   \Fi
	\setint{ligaturing}{1}
\Fi

\setint{italicslant}{0}
\setint{quad}{1000}
\setint{baselineskip}{1200}

\ifisglyph{x}\then
   \setint{xheight}{\height{x}}
\Else
   \setint{xheight}{500}
\Fi

\ifisglyph{space}\then
   \setint{interword}{\width{space}}
\Else\ifisglyph{i}\then
   \setint{interword}{\width{i}}
\Else
   \setint{interword}{333}
\Fi\Fi

\ifisint{monowidth}\then
   \setint{stretchword}{0}
   \setint{shrinkword}{0}
   \setint{extraspace}{\int{interword}}
\Else
   \setint{stretchword}{\scale{\int{interword}}{600}}
   \setint{shrinkword}{\scale{\int{interword}}{240}}
   \setint{extraspace}{\scale{\int{interword}}{240}}
\Fi

\ifisglyph{X}\then
   \setint{capheight}{\height{X}}
\Else
   \setint{capheight}{750}
\Fi

\ifisglyph{d}\then
   \setint{ascender}{\height{d}}
\Else\ifisint{capheight}\then
   \setint{ascender}{\int{capheight}}
\Else
   \setint{ascender}{750}
\Fi\Fi

\ifisglyph{Aring}\then
   \setint{acccapheight}{\height{Aring}}
\Else
   \setint{acccapheight}{999}
\Fi

\ifisint{descender_neg}\then
   \setint{descender}{\neg{\int{descender_neg}}}
\Else\ifisglyph{p}\then
   \setint{descender}{\depth{p}}
\Else
   \setint{descender}{250}
\Fi\Fi

\ifisglyph{Aring}\then
   \setint{maxheight}{\height{Aring}}
\Else
   \setint{maxheight}{1000}
\Fi

\ifisint{maxdepth_neg}\then
   \setint{maxdepth}{\neg{\int{maxdepth_neg}}}
\Else\ifisglyph{j}\then
   \setint{maxdepth}{\depth{j}}
\Else
   \setint{maxdepth}{250}
\Fi\Fi

\ifisglyph{six}\then
   \setint{digitwidth}{\width{six}}
\Else
   \setint{digitwidth}{500}
\Fi

\setint{capstem}{0} % not in AFM files

\setfontdimen{1}{italicslant}    % italic slant
\setfontdimen{2}{interword}      % interword space
\setfontdimen{3}{stretchword}    % interword stretch
\setfontdimen{4}{shrinkword}     % interword shrink
\setfontdimen{5}{xheight}        % x-height
\setfontdimen{6}{quad}           % quad
\setfontdimen{7}{extraspace}     % extra space after .
\setfontdimen{8}{capheight}      % cap height
\setfontdimen{9}{ascender}       % ascender
\setfontdimen{10}{acccapheight}  % accented cap height
\setfontdimen{11}{descender}     % descender's depth
\setfontdimen{12}{maxheight}     % max height
\setfontdimen{13}{maxdepth}      % max depth
\setfontdimen{14}{digitwidth}    % digit width
\setfontdimen{15}{verticalstem}  % dominant width of verical stems
\setfontdimen{16}{baselineskip}  % baselineskip

\ifnumber{\int{ligaturing}}<{0}\then 
   \comment{In this case, the codingscheme can be different from the 
     default, and therefore we refrain from setting it.}
\Else
   \setstr{codingscheme}{EXTENDED TEX FONT ENCODING - DOTSUP}
\Fi

\setslot{\lc{Grave}{grave}}
   \comment{The grave accent `\`{}'.}
\endsetslot

\setslot{\lc{Acute}{acute}}
   \comment{The acute accent `\'{}'.}
\endsetslot

\setslot{\lc{Circumflex}{circumflex}}
   \comment{The circumflex accent `\^{}'.}
\endsetslot

\setslot{\lc{Tilde}{tilde}}
   \comment{The tilde accent `\~{}'.}
\endsetslot

\setslot{\lc{Dieresis}{dieresis}}
   \comment{The umlaut or dieresis accent `\"{}'.}
\endsetslot

\setslot{\lc{Hungarumlaut}{hungarumlaut}}
   \comment{The long Hungarian umlaut `\H{}'.}
\endsetslot

\setslot{\lc{Ring}{ring}}
   \comment{The ring accent `\r{}'.}
\endsetslot

\setslot{\lc{Caron}{caron}}
   \comment{The caron or h\'a\v cek accent `\v{}'.}
\endsetslot

\setslot{\lc{Breve}{breve}}
   \comment{The breve accent `\u{}'.}
\endsetslot

\setslot{\lc{Macron}{macron}}
   \comment{The macron accent `\={}'.}
\endsetslot

\setslot{\lc{Dotaccent}{dotaccent}}
   \comment{The dot accent `\.{}'.}
\endsetslot

\setslot{\lc{Cedilla}{cedilla}}
   \comment{The cedilla accent `\c {}'.}
\endsetslot

\setslot{\lc{Ogonek}{ogonek}}
   \comment{The ogonek accent `\k {}'.}
\endsetslot

\setslot{quotesinglbase.superior}
  \comment{A German single quote mark `\quotesinglbase' similar to a comma,
      but with different sidebearings.}
\endsetslot

\setslot{guilsinglleft.superior}
  \comment{A French single opening quote mark `\guilsinglleft',
      unavailable in \plain\ \TeX.}
\endsetslot

\setslot{guilsinglright.superior}
  \comment{A French single closing quote mark `\guilsinglright',
      unavailable in \plain\ \TeX.}
\endsetslot

\setslot{quotedblleft.superior}
  \comment{The English opening quote mark `\,\textquotedblleft\,'.}
\endsetslot

\setslot{quotedblright.superior}
  \comment{The English closing quote mark `\,\textquotedblright\,'.}
\endsetslot

\setslot{quotedblbase.superior}
  \comment{A German double quote mark `\quotedblbase' similar to two commas,
      but with tighter letterspacing and different sidebearings.}
\endsetslot

\setslot{guillemotleft.superior}
  \comment{A French double opening quote mark `\guillemotleft',
      unavailable in \plain\ \TeX.}
\endsetslot

\setslot{guillemotright.superior}
  \comment{A French closing opening quote mark `\guillemotright',
      unavailable in \plain\ \TeX.}
\endsetslot

\setslot{endash.superior}
   \ligature{LIG}{hyphen.superior}{emdash.superior}
   \comment{The number range dash `1--9'. 
     This is called `rangedash' by fontinst's t1.etx, but it needs to be 
     called `endash' to work right. 
     The `\textendash'.  
     In a monowidth font, this might be set as `\texttt{1{-}9}'.}
\endsetslot

\setslot{emdash.superior}
   \comment{The punctuation dash `Oh---boy.' 
     This is calle `punctdash' by fontinst's t1.etx, but needs to be 
     called `emdash' to work right. 
     The `\textemdash'.  
     In a monowidth font, this might be set as `\texttt{Oh{-}{-}boy.}'}
\endsetslot

\setslot{compwordmark.superior}
   \comment{An invisible glyph, with zero width and depth, but the
      height of lowercase letters without ascenders.
      It is used to stop ligaturing in words like `shelf{}ful'.}
\endsetslot

\setslot{perthousandzero.superior}
   \comment{A glyph which is placed after `\%' to produce a
      `per-thousand', or twice to produce `per-ten-thousand'.
      Your guess is as good as mine as to what this glyph should look
      like in a monowidth font.}
\endsetslot

\setslot{\lc{dotlessI}{dotlessi}}
   \comment{A dotless i `\i', used to produce accented letters such as
      `\=\i'.}
\endsetslot

\setslot{\lc{dotlessJ}{dotlessj}}
   \comment{A dotless j `\j', used to produce accented letters such as
      `\=\j'.  Most non-\TeX\ fonts do not have this glyph.}
\endsetslot

\ifnumber{\int{ligaturing}}<{0}\then \skipslots{5}\Else

\setslot{\lclig{FF}{ff}}
   \ifnumber{\int{ligaturing}}>{0}\then
      \ligature{LIG}{\lc{I}{i}}{\lclig{FFI}{ffi}}
      \ligature{LIG}{\lc{L}{l}}{\lclig{FFL}{ffl}}
   \Fi
   \comment{The `ff' ligature.  It should be two characters wide in a
      monowidth font.}
\endsetslot

\setslot{\lclig{FI}{fi}}
   \comment{The `fi' ligature.  It should be two characters wide in a
      monowidth font.}
\endsetslot

\setslot{\lclig{FL}{fl}}
   \comment{The `fl' ligature.  It should be two characters wide in a
      monowidth font.}
\endsetslot

\setslot{\lclig{FFI}{ffi}}
   \comment{The `ffi' ligature.  It should be three characters wide in a
      monowidth font.}
\endsetslot

\setslot{\lclig{FFL}{ffl}}
   \comment{The `ffl' ligature.  It should be three characters wide in a
      monowidth font.}
\endsetslot

\Fi

\setslot{visiblespace.superior}
   \comment{A visible space glyph `\textvisiblespace'.}
\endsetslot

\setslot{exclam.superior}
   \ligature{LIG}{quoteleft.superior}{exclamdown.superior}
   \comment{The exclamation mark `!'.}
\endsetslot

\setslot{quotedbl.superior}
  \comment{The `neutral' double quotation mark `\,\textquotedbl\,',
      included for use in monowidth fonts, or for setting computer
      programs.  Note that the inclusion of this glyph in this slot
      means that \TeX\ documents which used `{\tt\char`\"}' as an
      input character will no longer work.}
\endsetslot

\setslot{numbersign.superior}
   \comment{The hash sign `\#'.}
\endsetslot

\setslot{dollar.superior}
   \comment{The dollar sign `\$'.}
\endsetslot

\setslot{percent.superior}
   \comment{The percent sign `\%'.}
\endsetslot

\setslot{ampersand.superior}
   \comment{The ampersand sign `\&'.}
\endsetslot

\setslot{quoteright.superior}
   \ligature{LIG}{quoteright.superior}{quotedblright.superior}
   \comment{The English closing single quote mark `\,\textquoteright\,'.}
\endsetslot

\setslot{parenleft.superior}
   \comment{The opening parenthesis `('.}
\endsetslot

\setslot{parenright.superior}
   \comment{The closing parenthesis `)'.}
\endsetslot

\setslot{asterisk.superior}
   \comment{The raised asterisk `*'.}
\endsetslot

\setslot{plus.superior}
   \comment{The addition sign `+'.}
\endsetslot

\setslot{comma.superior}
   \ligature{LIG}{comma.superior}{quotedblbase.superior}
   \comment{The comma `,'.}
\endsetslot

\setslot{hyphen.superior}
   \ligature{LIG}{hyphen.superior}{endash.superior}
   \ligature{LIG}{hyphenchar.superior}{hyphenchar.superior}
   \comment{The hyphen `-'.}
\endsetslot

\setslot{period.superior}
   \comment{The period `.'.}
\endsetslot

\setslot{slash.superior}
   \comment{The forward oblique `/'.}
\endsetslot

\setslot{\digit{zero}}
   \comment{The number `0'.  This (and all the other numerals) may be
      old style or ranging digits.}
\endsetslot

\setslot{\digit{one}}
   \comment{The number `1'.}
\endsetslot

\setslot{\digit{two}}
   \comment{The number `2'.}
\endsetslot

\setslot{\digit{three}}
   \comment{The number `3'.}
\endsetslot

\setslot{\digit{four}}
   \comment{The number `4'.}
\endsetslot

\setslot{\digit{five}}
   \comment{The number `5'.}
\endsetslot

\setslot{\digit{six}}
   \comment{The number `6'.}
\endsetslot

\setslot{\digit{seven}}
   \comment{The number `7'.}
\endsetslot

\setslot{\digit{eight}}
   \comment{The number `8'.}
\endsetslot

\setslot{\digit{nine}}
   \comment{The number `9'.}
\endsetslot

\setslot{colon.superior}
   \comment{The colon punctuation mark `:'.}
\endsetslot

\setslot{semicolon.superior}
   \comment{The semi-colon punctuation mark `;'.}
\endsetslot

\setslot{less.superior}
   \ligature{LIG}{less.superior}{guillemotleft.superior}
   \comment{The less-than sign `\textless'.}
\endsetslot

\setslot{equal.superior}
   \comment{The equals sign `='.}
\endsetslot

\setslot{greater.superior}
   \ligature{LIG}{greater.superior}{guillemotright.superior}
   \comment{The greater-than sign `\textgreater'.}
\endsetslot

\setslot{question.superior}
   \ligature{LIG}{quoteleft.superior}{questiondown.superior}
   \comment{The question mark `?'.}
\endsetslot

\setslot{at.superior}
   \comment{The at sign `@'.}
\endsetslot

\setslot{\uc{A}{a}}
   \comment{The letter `{A}'.}
\endsetslot

\setslot{\uc{B}{b}}
   \comment{The letter `{B}'.}
\endsetslot

\setslot{\uc{C}{c}}
   \comment{The letter `{C}'.}
\endsetslot

\setslot{\uc{D}{d}}
   \comment{The letter `{D}'.}
\endsetslot

\setslot{\uc{E}{e}}
   \comment{The letter `{E}'.}
\endsetslot

\setslot{\uc{F}{f}}
   \comment{The letter `{F}'.}
\endsetslot

\setslot{\uc{G}{g}}
   \comment{The letter `{G}'.}
\endsetslot

\setslot{\uc{H}{h}}
   \comment{The letter `{H}'.}
\endsetslot

\ifnumber{\int{ligaturing}}<{-1}\then \skipslots{1}\Else

\setslot{\uc{I}{i}}
   \comment{The letter `{I}'.}
\endsetslot

\Fi

\setslot{\uc{J}{j}}
   \comment{The letter `{J}'.}
\endsetslot

\setslot{\uc{K}{k}}
   \comment{The letter `{K}'.}
\endsetslot

\setslot{\uc{L}{l}}
   \comment{The letter `{L}'.}
\endsetslot

\setslot{\uc{M}{m}}
   \comment{The letter `{M}'.}
\endsetslot

\setslot{\uc{N}{n}}
   \comment{The letter `{N}'.}
\endsetslot

\setslot{\uc{O}{o}}
   \comment{The letter `{O}'.}
\endsetslot

\setslot{\uc{P}{p}}
   \comment{The letter `{P}'.}
\endsetslot

\setslot{\uc{Q}{q}}
   \comment{The letter `{Q}'.}
\endsetslot

\setslot{\uc{R}{r}}
   \comment{The letter `{R}'.}
\endsetslot

\setslot{\uc{S}{s}}
   \comment{The letter `{S}'.}
\endsetslot

\setslot{\uc{T}{t}}
   \comment{The letter `{T}'.}
\endsetslot

\setslot{\uc{U}{u}}
   \comment{The letter `{U}'.}
\endsetslot

\setslot{\uc{V}{v}}
   \comment{The letter `{V}'.}
\endsetslot

\setslot{\uc{W}{w}}
   \comment{The letter `{W}'.}
\endsetslot

\setslot{\uc{X}{x}}
   \comment{The letter `{X}'.}
\endsetslot

\setslot{\uc{Y}{y}}
   \comment{The letter `{Y}'.}
\endsetslot

\setslot{\uc{Z}{z}}
   \comment{The letter `{Z}'.}
\endsetslot

\setslot{bracketleft.superior}
   \comment{The opening square bracket `['.}
\endsetslot

\setslot{backslash.superior}
   \comment{The backwards oblique `\textbackslash'.}
\endsetslot

\setslot{bracketright.superior}
   \comment{The closing square bracket `]'.}
\endsetslot

\setslot{asciicircum.superior}
   \comment{The ASCII upward-pointing arrow head `\textasciicircum'.
      This is included for compatibility with typewriter fonts used
      for computer listings.}
\endsetslot

\setslot{underscore.superior}
   \comment{The ASCII underline character `\textunderscore', usually
      set on the baseline.
      This is included for compatibility with typewriter fonts used
      for computer listings.}
\endsetslot

\setslot{quoteleft.superior}
   \ligature{LIG}{quoteleft.superior}{quotedblleft.superior}
   \comment{The English opening single quote mark `\,\textquoteleft\,'.}
\endsetslot

\setslot{\lc{A}{a}}
   \comment{The letter `{a}'.}
\endsetslot

\setslot{\lc{B}{b}}
   \comment{The letter `{b}'.}
\endsetslot

\ifnumber{\int{ligaturing}}<{-1}\then \skipslots{1}\Else

   \setslot{\lc{C}{c}}
      \comment{The letter `{c}'.}
   \endsetslot

\Fi

\setslot{\lc{D}{d}}
   \comment{The letter `{d}'.}
\endsetslot

\setslot{\lc{E}{e}}
   \comment{The letter `{e}'.}
\endsetslot

\ifnumber{\int{ligaturing}}<{-1}\then \skipslots{1}\Else

   \setslot{\lc{F}{f}}
      \ifnumber{\int{ligaturing}}>{0}\then
         \ligature{LIG}{\lc{I}{i}}{\lclig{FI}{fi}}
         \ligature{LIG}{\lc{F}{f}}{\lclig{FF}{ff}}
         \ligature{LIG}{\lc{L}{l}}{\lclig{FL}{fl}}
      \Fi
      \comment{The letter `{f}'.}
   \endsetslot

\Fi

\setslot{\lc{G}{g}}
   \comment{The letter `{g}'.}
\endsetslot

\setslot{\lc{H}{h}}
   \comment{The letter `{h}'.}
\endsetslot

\ifnumber{\int{ligaturing}}<{-1}\then \skipslots{1}\Else

   \setslot{\lc{I}{i}}
      \comment{The letter `{i}'.}
   \endsetslot

\Fi

\setslot{\lc{J}{j}}
   \comment{The letter `{j}'.}
\endsetslot

\setslot{\lc{K}{k}}
   \comment{The letter `{k}'.}
\endsetslot

\setslot{\lc{L}{l}}
   \comment{The letter `{l}'.}
\endsetslot

\setslot{\lc{M}{m}}
   \comment{The letter `{m}'.}
\endsetslot

\setslot{\lc{N}{n}}
   \comment{The letter `{n}'.}
\endsetslot

\setslot{\lc{O}{o}}
   \comment{The letter `{o}'.}
\endsetslot

\setslot{\lc{P}{p}}
   \comment{The letter `{p}'.}
\endsetslot

\setslot{\lc{Q}{q}}
   \comment{The letter `{q}'.}
\endsetslot

\setslot{\lc{R}{r}}
   \comment{The letter `{r}'.}
\endsetslot

\ifnumber{\int{ligaturing}}<{-1}\then \skipslots{1}\Else

   \setslot{\lc{S}{s}}
      \comment{The letter `{s}'.}
   \endsetslot

\Fi

\setslot{\lc{T}{t}}
   \comment{The letter `{t}'.}
\endsetslot

\setslot{\lc{U}{u}}
   \comment{The letter `{u}'.}
\endsetslot

\setslot{\lc{V}{v}}
   \comment{The letter `{v}'.}
\endsetslot

\setslot{\lc{W}{w}}
   \comment{The letter `{w}'.}
\endsetslot

\setslot{\lc{X}{x}}
   \comment{The letter `{x}'.}
\endsetslot

\setslot{\lc{Y}{y}}
   \comment{The letter `{y}'.}
\endsetslot

\setslot{\lc{Z}{z}}
   \comment{The letter `{z}'.}
\endsetslot

\setslot{braceleft.superior}
   \comment{The opening curly brace `\textbraceleft'.}
\endsetslot

\setslot{bar.superior}
   \comment{The ASCII vertical bar `\textbar'.
      This is included for compatibility with typewriter fonts used
      for computer listings.}
\endsetslot

\setslot{braceright.superior}
   \comment{The closing curly brace `\textbraceright'.}
\endsetslot

\setslot{asciitilde.superior}
   \comment{The ASCII tilde `\textasciitilde'.
      This is included for compatibility with typewriter fonts used
      for computer listings.}
\endsetslot

\setslot{hyphenchar.superior}
   \comment{The glyph used for hyphenation in this font, which will
      almost always be the same as `hyphen'.}
\endsetslot

\setslot{\uctop{Abreve}{abreve}}
   \comment{The letter `\u A'.}
\endsetslot

\setslot{\uc{Aogonek}{aogonek}}
   \comment{The letter `\k A'.}
\endsetslot

\setslot{\uctop{Cacute}{cacute}}
   \comment{The letter `\' C'.}
\endsetslot

\setslot{\uctop{Ccaron}{ccaron}}
   \comment{The letter `\v C'.}
\endsetslot

\setslot{\uctop{Dcaron}{dcaron}}
   \comment{The letter `\v D'.}
\endsetslot

\setslot{\uctop{Ecaron}{ecaron}}
   \comment{The letter `\v E'.}
\endsetslot

\setslot{\uc{Eogonek}{eogonek}}
   \comment{The letter `\k E'.}
\endsetslot

\setslot{\uctop{Gbreve}{gbreve}}
   \comment{The letter `\u G'.}
\endsetslot

\setslot{\uctop{Lacute}{lacute}}
   \comment{The letter `\' L'.}
\endsetslot

\setslot{\uc{Lcaron}{lcaron}}
   \comment{The letter `\v L'.}
\endsetslot

\setslot{\uc{Lslash}{lslash}}
   \comment{The letter `\L'.}
\endsetslot

\setslot{\uctop{Nacute}{nacute}}
   \comment{The letter `\' N'.}
\endsetslot

\setslot{\uctop{Ncaron}{ncaron}}
   \comment{The letter `\v N'.}
\endsetslot

\setslot{\uc{Eng}{eng}}
   \comment{The Sami letter `\NG'.  It is unavailable in \plain\ \TeX. This needs to be called `Eng'/`eng' rather than `Ng'/`ng' as in t1.etx in most cases, it seems.}
\endsetslot

\setslot{\uctop{Ohungarumlaut}{ohungarumlaut}}
   \comment{The letter `\H O'.}
\endsetslot

\setslot{\uctop{Racute}{racute}}
   \comment{The letter `\' R'.}
\endsetslot

\setslot{\uctop{Rcaron}{rcaron}}
   \comment{The letter `\v R'.}
\endsetslot

\setslot{\uctop{Sacute}{sacute}}
   \comment{The letter `\' S'.}
\endsetslot

\setslot{\uctop{Scaron}{scaron}}
   \comment{The letter `\v S'.}
\endsetslot

\setslot{\uc{Scedilla}{scedilla}}
   \comment{The letter `\c S'.}
\endsetslot

\setslot{\uctop{Tcaron}{tcaron}}
   \comment{The letter `\v T'.}
\endsetslot

\setslot{\uc{Tcedilla}{tcedilla}}
   \comment{The letter `\c T'.}
\endsetslot

\setslot{\uctop{Uhungarumlaut}{uhungarumlaut}}
   \comment{The letter `\H U'.}
\endsetslot

\setslot{\uctop{Uring}{uring}}
   \comment{The letter `\r U'.}
\endsetslot

\setslot{\uctop{Ydieresis}{ydieresis}}
   \comment{The letter `\" Y'.}
\endsetslot

\setslot{\uctop{Zacute}{zacute}}
   \comment{The letter `\' Z'.}
\endsetslot

\setslot{\uctop{Zcaron}{zcaron}}
   \comment{The letter `\v Z'.}
\endsetslot

\setslot{\uctop{Zdotaccent}{zdotaccent}}
   \comment{The letter `\. Z'.}
\endsetslot

\ifnumber{\int{ligaturing}}<{0}\then \skipslots{1}\Else

   \setslot{\uclig{IJ}{ij}}
      \comment{The letter `IJ'.  This is a single letter, and in a 
        monowidth font should ideally be one letter wide.}
   \endsetslot

\Fi

\setslot{\uctop{Idotaccent}{idotaccent}}
   \comment{The letter `\. I'.}
\endsetslot

\setslot{\lc{Dbar}{dbar}}
   \comment{The letter `\dj'.}
\endsetslot

\setslot{section.superior}
   \comment{The section mark `\textsection'.}
\endsetslot

\setslot{\lctop{Abreve}{abreve}}
   \comment{The letter `\u a'.}
\endsetslot

\setslot{\lc{Aogonek}{aogonek}}
   \comment{The letter `\k a'.}
\endsetslot

\setslot{\lctop{Cacute}{cacute}}
   \comment{The letter `\' c'.}
\endsetslot

\setslot{\lctop{Ccaron}{ccaron}}
   \comment{The letter `\v c'.}
\endsetslot

\setslot{\lctop{Dcaron}{dcaron}}
   \comment{The letter `\v d'.}
\endsetslot

\setslot{\lctop{Ecaron}{ecaron}}
   \comment{The letter `\v e'.}
\endsetslot

\setslot{\lc{Eogonek}{eogonek}}
   \comment{The letter `\k e'.}
\endsetslot

\setslot{\lctop{Gbreve}{gbreve}}
   \comment{The letter `\u g'.}
\endsetslot

\setslot{\lctop{Lacute}{lacute}}
   \comment{The letter `\' l'.}
\endsetslot

\setslot{\lc{Lcaron}{lcaron}}
   \comment{The letter `\v l'.}
\endsetslot

\setslot{\lc{Lslash}{lslash}}
   \comment{The letter `\l'.}
\endsetslot

\setslot{\lctop{Nacute}{nacute}}
   \comment{The letter `\' n'.}
\endsetslot

\setslot{\lctop{Ncaron}{ncaron}}
   \comment{The letter `\v n'.}
\endsetslot

\setslot{\lc{Eng}{eng}}
   \comment{The Sami letter `\ng'.  It is unavailable in \plain\ \TeX. This needs to be called `Eng'/`eng' rather than `Ng'/`ng' as it is in t1.etx in most cases, it seems.}
\endsetslot

\setslot{\lctop{Ohungarumlaut}{ohungarumlaut}}
   \comment{The letter `\H o'.}
\endsetslot

\setslot{\lctop{Racute}{racute}}
   \comment{The letter `\' r'.}
\endsetslot

\setslot{\lctop{Rcaron}{rcaron}}
   \comment{The letter `\v r'.}
\endsetslot

\setslot{\lctop{Sacute}{sacute}}
   \comment{The letter `\' s'.}
\endsetslot

\setslot{\lctop{Scaron}{scaron}}
   \comment{The letter `\v s'.}
\endsetslot

\setslot{\lc{Scedilla}{scedilla}}
   \comment{The letter `\c s'.}
\endsetslot

\setslot{\lctop{Tcaron}{tcaron}}
   \comment{The letter `\v t'.}
\endsetslot

\setslot{\lc{Tcedilla}{tcedilla}}
   \comment{The letter `\c t'.}
\endsetslot

\setslot{\lctop{Uhungarumlaut}{uhungarumlaut}}
   \comment{The letter `\H u'.}
\endsetslot

\setslot{\lctop{Uring}{uring}}
   \comment{The letter `\r u'.}
\endsetslot

\setslot{\lctop{Ydieresis}{ydieresis}}
   \comment{The letter `\" y'.}
\endsetslot

\setslot{\lctop{Zacute}{zacute}}
   \comment{The letter `\' z'.}
\endsetslot

\setslot{\lctop{Zcaron}{zcaron}}
   \comment{The letter `\v z'.}
\endsetslot

\setslot{\lctop{Zdotaccent}{zdotaccent}}
   \comment{The letter `\. z'.}
\endsetslot

\ifnumber{\int{ligaturing}}<{0}\then \skipslots{1}\Else

   \setslot{\lclig{IJ}{ij}}
      \comment{The letter `ij'.  This is a single letter, and in a 
        monowidth font should ideally be one letter wide.}
   \endsetslot

\Fi

\setslot{exclamdown.superior}
   \comment{The Spanish punctuation mark `!`'.}
\endsetslot

\setslot{questiondown.superior}
   \comment{The Spanish punctuation mark `?`'.}
\endsetslot

\setslot{sterling.superior}
   \comment{The British currency mark `\textsterling'.}
\endsetslot

\setslot{\uctop{Agrave}{agrave}}
   \comment{The letter `\` A'.}
\endsetslot

\setslot{\uctop{Aacute}{aacute}}
   \comment{The letter `\' A'.}
\endsetslot

\setslot{\uctop{Acircumflex}{acircumflex}}
   \comment{The letter `\^ A'.}
\endsetslot

\setslot{\uctop{Atilde}{atilde}}
   \comment{The letter `\~ A'.}
\endsetslot

\setslot{\uctop{Adieresis}{adieresis}}
   \comment{The letter `\" A'.}
\endsetslot

\setslot{\uctop{Aring}{aring}}
   \comment{The letter `\r A'.}
\endsetslot

\setslot{\uc{AE}{ae}}
   \comment{The letter `\AE'.  This is a single letter, and should not be
      faked with `AE'.}
\endsetslot

\setslot{\uc{Ccedilla}{ccedilla}}
   \comment{The letter `\c C'.}
\endsetslot

\setslot{\uctop{Egrave}{egrave}}
   \comment{The letter `\` E'.}
\endsetslot

\setslot{\uctop{Eacute}{eacute}}
   \comment{The letter `\' E'.}
\endsetslot

\setslot{\uctop{Ecircumflex}{ecircumflex}}
   \comment{The letter `\^ E'.}
\endsetslot

\setslot{\uctop{Edieresis}{edieresis}}
   \comment{The letter `\" E'.}
\endsetslot

\setslot{\uctop{Igrave}{igrave}}
   \comment{The letter `\` I'.}
\endsetslot

\setslot{\uctop{Iacute}{iacute}}
   \comment{The letter `\' I'.}
\endsetslot

\setslot{\uctop{Icircumflex}{icircumflex}}
   \comment{The letter `\^ I'.}
\endsetslot

\setslot{\uctop{Idieresis}{idieresis}}
   \comment{The letter `\" I'.}
\endsetslot

\setslot{\uc{Eth}{eth}}
   \comment{The uppercase Icelandic letter `Eth' similar to a `D'
      with a horizontal bar through the stem.  It is unavailable
      in \plain\ \TeX.}
\endsetslot

\setslot{\uctop{Ntilde}{ntilde}}
   \comment{The letter `\~ N'.}
\endsetslot

\setslot{\uctop{Ograve}{ograve}}
   \comment{The letter `\` O'.}
\endsetslot

\setslot{\uctop{Oacute}{oacute}}
   \comment{The letter `\' O'.}
\endsetslot

\setslot{\uctop{Ocircumflex}{ocircumflex}}
   \comment{The letter `\^ O'.}
\endsetslot

\setslot{\uctop{Otilde}{otilde}}
   \comment{The letter `\~ O'.}
\endsetslot

\setslot{\uctop{Odieresis}{odieresis}}
   \comment{The letter `\" O'.}
\endsetslot

\setslot{\uc{OE}{oe}}
   \comment{The letter `\OE'.  This is a single letter, and should not be
      faked with `OE'.}
\endsetslot

\setslot{\uc{Oslash}{oslash}}
   \comment{The letter `\O'.}
\endsetslot

\setslot{\uctop{Ugrave}{ugrave}}
   \comment{The letter `\` U'.}
\endsetslot

\setslot{\uctop{Uacute}{uacute}}
   \comment{The letter `\' U'.}
\endsetslot

\setslot{\uctop{Ucircumflex}{ucircumflex}}
   \comment{The letter `\^ U'.}
\endsetslot

\setslot{\uctop{Udieresis}{udieresis}}
   \comment{The letter `\" U'.}
\endsetslot

\setslot{\uctop{Yacute}{yacute}}
   \comment{The letter `\' Y'.}
\endsetslot

\setslot{\uc{Thorn}{thorn}}
   \comment{The Icelandic capital letter Thorn, similar to a `P'
      with the bowl moved down.  It is unavailable in \plain\ \TeX.}
\endsetslot

\setslot{\uclig{SS}{germandbls}}
   \comment{The ligature `SS', used to give an upper case `\ss'.
      In a monowidth font it should be two letters wide.}
\endsetslot

\setslot{\lctop{Agrave}{agrave}}
   \comment{The letter `\` a'.}
\endsetslot

\setslot{\lctop{Aacute}{aacute}}
   \comment{The letter `\' a'.}
\endsetslot

\setslot{\lctop{Acircumflex}{acircumflex}}
   \comment{The letter `\^ a'.}
\endsetslot

\setslot{\lctop{Atilde}{atilde}}
   \comment{The letter `\~ a'.}
\endsetslot

\setslot{\lctop{Adieresis}{adieresis}}
   \comment{The letter `\" a'.}
\endsetslot

\setslot{\lctop{Aring}{aring}}
   \comment{The letter `\r a'.}
\endsetslot

\setslot{\lc{AE}{ae}}
   \comment{The letter `\ae'.  This is a single letter, and should not be
      faked with `ae'.}
\endsetslot

\setslot{\lc{Ccedilla}{ccedilla}}
   \comment{The letter `\c c'.}
\endsetslot

\setslot{\lctop{Egrave}{egrave}}
   \comment{The letter `\` e'.}
\endsetslot

\setslot{\lctop{Eacute}{eacute}}
   \comment{The letter `\' e'.}
\endsetslot

\setslot{\lctop{Ecircumflex}{ecircumflex}}
   \comment{The letter `\^ e'.}
\endsetslot

\setslot{\lctop{Edieresis}{edieresis}}
   \comment{The letter `\" e'.}
\endsetslot

\setslot{\lctop{Igrave}{igrave}}
   \comment{The letter `\`\i'.}
\endsetslot

\setslot{\lctop{Iacute}{iacute}}
   \comment{The letter `\'\i'.}
\endsetslot

\setslot{\lctop{Icircumflex}{icircumflex}}
   \comment{The letter `\^\i'.}
\endsetslot

\setslot{\lctop{Idieresis}{idieresis}}
   \comment{The letter `\"\i'.}
\endsetslot

\setslot{\lc{Eth}{eth}}
   \comment{The Icelandic lowercase letter `eth' similar to
     a `$\partial$' with an oblique bar through the stem.
     It is unavailable in \plain\ \TeX.}
\endsetslot

\setslot{\lctop{Ntilde}{ntilde}}
   \comment{The letter `\~ n'.}
\endsetslot

\setslot{\lctop{Ograve}{ograve}}
   \comment{The letter `\` o'.}
\endsetslot

\setslot{\lctop{Oacute}{oacute}}
   \comment{The letter `\' o'.}
\endsetslot

\setslot{\lctop{Ocircumflex}{ocircumflex}}
   \comment{The letter `\^ o'.}
\endsetslot

\setslot{\lctop{Otilde}{otilde}}
   \comment{The letter `\~ o'.}
\endsetslot

\setslot{\lctop{Odieresis}{odieresis}}
   \comment{The letter `\" o'.}
\endsetslot

\setslot{\lc{OE}{oe}}
   \comment{The letter `\oe'.  This is a single letter, and should not be
      faked with `oe'.}
\endsetslot

\setslot{\lc{Oslash}{oslash}}
   \comment{The letter `\o'.}
\endsetslot

\setslot{\lctop{Ugrave}{ugrave}}
   \comment{The letter `\` u'.}
\endsetslot

\setslot{\lctop{Uacute}{uacute}}
   \comment{The letter `\' u'.}
\endsetslot

\setslot{\lctop{Ucircumflex}{ucircumflex}}
   \comment{The letter `\^ u'.}
\endsetslot

\setslot{\lctop{Udieresis}{udieresis}}
   \comment{The letter `\" u'.}
\endsetslot

\setslot{\lctop{Yacute}{yacute}}
   \comment{The letter `\' y'.}
\endsetslot

\setslot{\lc{Thorn}{thorn}}
   \comment{The Icelandic lowercase letter `thorn', similar to a `p'
      with an ascender rising from the stem.  It is unavailable
      in \plain\ \TeX.}
\endsetslot

\setslot{\lc{SS}{germandbls}}
   \comment{The letter `\ss'.}
\endsetslot

\endencoding
%    \end{macrocode}
% \end{encoding}
% \iffalse
%</t1-dotsup>
% \fi
% 
% 
% \subsubsection{fontscripts-t1-dotsuperior.etx}\label{subsubsec:t1-dotsuperior}
% 
% \iffalse
%<*t1-dotsuperior>
% \fi
% \begin{encoding}{fontscripts-t1-dotsuperior.etx}
% \changes{v0.0}{2025-02-10}{Filename prefix for Karl.}
%    \begin{macrocode}
%%
%% - The original notices at the top of that file concerning authors,
%% maintenance etc. are replaced by this notice.
%% - The file is renamed.
%% - The encoding name is modified.
%% - The file is modified to accommodate differences in glyph names.
%% - The file is modified for use in encoding superiors.
%%
%%%%%%%%%%%%%%%%%%%%%%%%%%%%%%%%%%%%%%%%%%%%%%%%%
\relax
\encoding

\needsfontinstversion{1.910}

\setcommand\lc#1#2{#2.superior}
\setcommand\uc#1#2{#1.superior}
\setcommand\lctop#1#2{#2.superior}
\setcommand\uctop#1#2{#1.superior}
\setcommand\lclig#1#2{#2.superior}
\ifisint{letterspacing}\then
   \ifnumber{\int{letterspacing}}={0}\then \Else
      \setcommand\uclig#1#2{#1spaced}
      \comment{Here we set \verb|\uclig#1#2| to \verb|#1spaced|, but 
      you can't see it as \verb|\setcommand| commands are invisible in 
      the typeset output.}
   \Fi
\Fi
\setcommand\uclig#1#2{#1.superior}
\setcommand\digit#1{#1.superior}

\ifisint{monowidth}\then
   \setint{ligaturing}{0}
\Else
   % The following empty line is *important* to get the formatting
   % right here (sigh)! (Remember that it is a \par token.)
   
   \ifisint{letterspacing}\then
      \ifnumber{\int{letterspacing}}={0}\then \Else
         \setint{ligaturing}{0}
      \Fi
   \Fi
	\setint{ligaturing}{1}
\Fi

\setint{italicslant}{0}
\setint{quad}{1000}
\setint{baselineskip}{1200}

\ifisglyph{x}\then
   \setint{xheight}{\height{x}}
\Else
   \setint{xheight}{500}
\Fi

\ifisglyph{space}\then
   \setint{interword}{\width{space}}
\Else\ifisglyph{i}\then
   \setint{interword}{\width{i}}
\Else
   \setint{interword}{333}
\Fi\Fi

\ifisint{monowidth}\then
   \setint{stretchword}{0}
   \setint{shrinkword}{0}
   \setint{extraspace}{\int{interword}}
\Else
   \setint{stretchword}{\scale{\int{interword}}{600}}
   \setint{shrinkword}{\scale{\int{interword}}{240}}
   \setint{extraspace}{\scale{\int{interword}}{240}}
\Fi

\ifisglyph{X}\then
   \setint{capheight}{\height{X}}
\Else
   \setint{capheight}{750}
\Fi

\ifisglyph{d}\then
   \setint{ascender}{\height{d}}
\Else\ifisint{capheight}\then
   \setint{ascender}{\int{capheight}}
\Else
   \setint{ascender}{750}
\Fi\Fi

\ifisglyph{Aring}\then
   \setint{acccapheight}{\height{Aring}}
\Else
   \setint{acccapheight}{999}
\Fi

\ifisint{descender_neg}\then
   \setint{descender}{\neg{\int{descender_neg}}}
\Else\ifisglyph{p}\then
   \setint{descender}{\depth{p}}
\Else
   \setint{descender}{250}
\Fi\Fi

\ifisglyph{Aring}\then
   \setint{maxheight}{\height{Aring}}
\Else
   \setint{maxheight}{1000}
\Fi

\ifisint{maxdepth_neg}\then
   \setint{maxdepth}{\neg{\int{maxdepth_neg}}}
\Else\ifisglyph{j}\then
   \setint{maxdepth}{\depth{j}}
\Else
   \setint{maxdepth}{250}
\Fi\Fi

\ifisglyph{six}\then
   \setint{digitwidth}{\width{six}}
\Else
   \setint{digitwidth}{500}
\Fi

\setint{capstem}{0} % not in AFM files

\setfontdimen{1}{italicslant}    % italic slant
\setfontdimen{2}{interword}      % interword space
\setfontdimen{3}{stretchword}    % interword stretch
\setfontdimen{4}{shrinkword}     % interword shrink
\setfontdimen{5}{xheight}        % x-height
\setfontdimen{6}{quad}           % quad
\setfontdimen{7}{extraspace}     % extra space after .
\setfontdimen{8}{capheight}      % cap height
\setfontdimen{9}{ascender}       % ascender
\setfontdimen{10}{acccapheight}  % accented cap height
\setfontdimen{11}{descender}     % descender's depth
\setfontdimen{12}{maxheight}     % max height
\setfontdimen{13}{maxdepth}      % max depth
\setfontdimen{14}{digitwidth}    % digit width
\setfontdimen{15}{verticalstem}  % dominant width of verical stems
\setfontdimen{16}{baselineskip}  % baselineskip

\ifnumber{\int{ligaturing}}<{0}\then 
   \comment{In this case, the codingscheme can be different from the 
     default, and therefore we refrain from setting it.}
\Else
   \setstr{codingscheme}{EXTENDED TEX FONT ENCODING - DOTSUPERIOR}
\Fi

\setslot{\lc{Grave}{grave}}
   \comment{The grave accent `\`{}'.}
\endsetslot

\setslot{\lc{Acute}{acute}}
   \comment{The acute accent `\'{}'.}
\endsetslot

\setslot{\lc{Circumflex}{circumflex}}
   \comment{The circumflex accent `\^{}'.}
\endsetslot

\setslot{\lc{Tilde}{tilde}}
   \comment{The tilde accent `\~{}'.}
\endsetslot

\setslot{\lc{Dieresis}{dieresis}}
   \comment{The umlaut or dieresis accent `\"{}'.}
\endsetslot

\setslot{\lc{Hungarumlaut}{hungarumlaut}}
   \comment{The long Hungarian umlaut `\H{}'.}
\endsetslot

\setslot{\lc{Ring}{ring}}
   \comment{The ring accent `\r{}'.}
\endsetslot

\setslot{\lc{Caron}{caron}}
   \comment{The caron or h\'a\v cek accent `\v{}'.}
\endsetslot

\setslot{\lc{Breve}{breve}}
   \comment{The breve accent `\u{}'.}
\endsetslot

\setslot{\lc{Macron}{macron}}
   \comment{The macron accent `\={}'.}
\endsetslot

\setslot{\lc{Dotaccent}{dotaccent}}
   \comment{The dot accent `\.{}'.}
\endsetslot

\setslot{\lc{Cedilla}{cedilla}}
   \comment{The cedilla accent `\c {}'.}
\endsetslot

\setslot{\lc{Ogonek}{ogonek}}
   \comment{The ogonek accent `\k {}'.}
\endsetslot

\setslot{quotesinglbase.superior}
  \comment{A German single quote mark `\quotesinglbase' similar to a comma,
      but with different sidebearings.}
\endsetslot

\setslot{guilsinglleft.superior}
  \comment{A French single opening quote mark `\guilsinglleft',
      unavailable in \plain\ \TeX.}
\endsetslot

\setslot{guilsinglright.superior}
  \comment{A French single closing quote mark `\guilsinglright',
      unavailable in \plain\ \TeX.}
\endsetslot

\setslot{quotedblleft.superior}
  \comment{The English opening quote mark `\,\textquotedblleft\,'.}
\endsetslot

\setslot{quotedblright.superior}
  \comment{The English closing quote mark `\,\textquotedblright\,'.}
\endsetslot

\setslot{quotedblbase.superior}
  \comment{A German double quote mark `\quotedblbase' similar to two commas,
      but with tighter letterspacing and different sidebearings.}
\endsetslot

\setslot{guillemotleft.superior}
  \comment{A French double opening quote mark `\guillemotleft',
      unavailable in \plain\ \TeX.}
\endsetslot

\setslot{guillemotright.superior}
  \comment{A French closing opening quote mark `\guillemotright',
      unavailable in \plain\ \TeX.}
\endsetslot

\setslot{endash.superior}
   \ligature{LIG}{hyphen.superior}{emdash.superior}
   \comment{The number range dash `1--9'. 
     This is called `rangedash' by fontinst's t1.etx, but it needs to be 
     called `endash' to work right. 
     The `\textendash'.  In a monowidth font, this might be set as 
     `\texttt{1{-}9}'.}
\endsetslot

\setslot{emdash.superior}
   \comment{The punctuation dash `Oh---boy.' 
     This is calle `punctdash' by fontinst's t1.etx, but needs to be called 
     `emdash' to work right. 
     The `\textemdash'.  In a monowidth font, this might be set as 
      `\texttt{Oh{-}{-}boy.}'}
\endsetslot

\setslot{compwordmark.superior}
   \comment{An invisible glyph, with zero width and depth, but the
      height of lowercase letters without ascenders.
      It is used to stop ligaturing in words like `shelf{}ful'.}
\endsetslot

\setslot{perthousandzero.superior}
   \comment{A glyph which is placed after `\%' to produce a
      `per-thousand', or twice to produce `per-ten-thousand'.
      Your guess is as good as mine as to what this glyph should look
      like in a monowidth font.}
\endsetslot

\setslot{\lc{dotlessI}{dotlessi}}
   \comment{A dotless i `\i', used to produce accented letters such as
      `\=\i'.}
\endsetslot

\setslot{\lc{dotlessJ}{dotlessj}}
   \comment{A dotless j `\j', used to produce accented letters such as
      `\=\j'.  Most non-\TeX\ fonts do not have this glyph.}
\endsetslot

\ifnumber{\int{ligaturing}}<{0}\then \skipslots{5}\Else

\setslot{\lclig{FF}{ff}}
   \ifnumber{\int{ligaturing}}>{0}\then
      \ligature{LIG}{\lc{I}{i}}{\lclig{FFI}{ffi}}
      \ligature{LIG}{\lc{L}{l}}{\lclig{FFL}{ffl}}
   \Fi
   \comment{The `ff' ligature.  It should be two characters wide in a
      monowidth font.}
\endsetslot

\setslot{\lclig{FI}{fi}}
   \comment{The `fi' ligature.  It should be two characters wide in a
      monowidth font.}
\endsetslot

\setslot{\lclig{FL}{fl}}
   \comment{The `fl' ligature.  It should be two characters wide in a
      monowidth font.}
\endsetslot

\setslot{\lclig{FFI}{ffi}}
   \comment{The `ffi' ligature.  It should be three characters wide in a
      monowidth font.}
\endsetslot

\setslot{\lclig{FFL}{ffl}}
   \comment{The `ffl' ligature.  It should be three characters wide in a
      monowidth font.}
\endsetslot

\Fi

\setslot{visiblespace.superior}
   \comment{A visible space glyph `\textvisiblespace'.}
\endsetslot

\setslot{exclam.superior}
   \ligature{LIG}{quoteleft.superior}{exclamdown.superior}
   \comment{The exclamation mark `!'.}
\endsetslot

\setslot{quotedbl.superior}
  \comment{The `neutral' double quotation mark `\,\textquotedbl\,',
      included for use in monowidth fonts, or for setting computer
      programs.  Note that the inclusion of this glyph in this slot
      means that \TeX\ documents which used `{\tt\char`\"}' as an
      input character will no longer work.}
\endsetslot

\setslot{numbersign.superior}
   \comment{The hash sign `\#'.}
\endsetslot

\setslot{dollar.superior}
   \comment{The dollar sign `\$'.}
\endsetslot

\setslot{percent.superior}
   \comment{The percent sign `\%'.}
\endsetslot

\setslot{ampersand.superior}
   \comment{The ampersand sign `\&'.}
\endsetslot

\setslot{quoteright.superior}
   \ligature{LIG}{quoteright.superior}{quotedblright.superior}
   \comment{The English closing single quote mark `\,\textquoteright\,'.}
\endsetslot

\setslot{parenleft.superior}
   \comment{The opening parenthesis `('.}
\endsetslot

\setslot{parenright.superior}
   \comment{The closing parenthesis `)'.}
\endsetslot

\setslot{asterisk.superior}
   \comment{The raised asterisk `*'.}
\endsetslot

\setslot{plus.superior}
   \comment{The addition sign `+'.}
\endsetslot

\setslot{comma.superior}
   \ligature{LIG}{comma.superior}{quotedblbase.superior}
   \comment{The comma `,'.}
\endsetslot

\setslot{hyphen.superior}
   \ligature{LIG}{hyphen.superior}{endash.superior}
   \ligature{LIG}{hyphenchar.superior}{hyphenchar.superior}
   \comment{The hyphen `-'.}
\endsetslot

\setslot{period.superior}
   \comment{The period `.'.}
\endsetslot

\setslot{slash.superior}
   \comment{The forward oblique `/'.}
\endsetslot

\setslot{\digit{zero}}
   \comment{The number `0'.  This (and all the other numerals) may be
      old style or ranging digits.}
\endsetslot

\setslot{\digit{one}}
   \comment{The number `1'.}
\endsetslot

\setslot{\digit{two}}
   \comment{The number `2'.}
\endsetslot

\setslot{\digit{three}}
   \comment{The number `3'.}
\endsetslot

\setslot{\digit{four}}
   \comment{The number `4'.}
\endsetslot

\setslot{\digit{five}}
   \comment{The number `5'.}
\endsetslot

\setslot{\digit{six}}
   \comment{The number `6'.}
\endsetslot

\setslot{\digit{seven}}
   \comment{The number `7'.}
\endsetslot

\setslot{\digit{eight}}
   \comment{The number `8'.}
\endsetslot

\setslot{\digit{nine}}
   \comment{The number `9'.}
\endsetslot

\setslot{colon.superior}
   \comment{The colon punctuation mark `:'.}
\endsetslot

\setslot{semicolon.superior}
   \comment{The semi-colon punctuation mark `;'.}
\endsetslot

\setslot{less.superior}
   \ligature{LIG}{less.superior}{guillemotleft.superior}
   \comment{The less-than sign `\textless'.}
\endsetslot

\setslot{equal.superior}
   \comment{The equals sign `='.}
\endsetslot

\setslot{greater.superior}
   \ligature{LIG}{greater.superior}{guillemotright.superior}
   \comment{The greater-than sign `\textgreater'.}
\endsetslot

\setslot{question.superior}
   \ligature{LIG}{quoteleft.superior}{questiondown.superior}
   \comment{The question mark `?'.}
\endsetslot

\setslot{at.superior}
   \comment{The at sign `@'.}
\endsetslot

\setslot{\uc{A}{a}}
   \comment{The letter `{A}'.}
\endsetslot

\setslot{\uc{B}{b}}
   \comment{The letter `{B}'.}
\endsetslot

\setslot{\uc{C}{c}}
   \comment{The letter `{C}'.}
\endsetslot

\setslot{\uc{D}{d}}
   \comment{The letter `{D}'.}
\endsetslot

\setslot{\uc{E}{e}}
   \comment{The letter `{E}'.}
\endsetslot

\setslot{\uc{F}{f}}
   \comment{The letter `{F}'.}
\endsetslot

\setslot{\uc{G}{g}}
   \comment{The letter `{G}'.}
\endsetslot

\setslot{\uc{H}{h}}
   \comment{The letter `{H}'.}
\endsetslot

\ifnumber{\int{ligaturing}}<{-1}\then \skipslots{1}\Else

\setslot{\uc{I}{i}}
   \comment{The letter `{I}'.}
\endsetslot

\Fi

\setslot{\uc{J}{j}}
   \comment{The letter `{J}'.}
\endsetslot

\setslot{\uc{K}{k}}
   \comment{The letter `{K}'.}
\endsetslot

\setslot{\uc{L}{l}}
   \comment{The letter `{L}'.}
\endsetslot

\setslot{\uc{M}{m}}
   \comment{The letter `{M}'.}
\endsetslot

\setslot{\uc{N}{n}}
   \comment{The letter `{N}'.}
\endsetslot

\setslot{\uc{O}{o}}
   \comment{The letter `{O}'.}
\endsetslot

\setslot{\uc{P}{p}}
   \comment{The letter `{P}'.}
\endsetslot

\setslot{\uc{Q}{q}}
   \comment{The letter `{Q}'.}
\endsetslot

\setslot{\uc{R}{r}}
   \comment{The letter `{R}'.}
\endsetslot

\setslot{\uc{S}{s}}
   \comment{The letter `{S}'.}
\endsetslot

\setslot{\uc{T}{t}}
   \comment{The letter `{T}'.}
\endsetslot

\setslot{\uc{U}{u}}
   \comment{The letter `{U}'.}
\endsetslot

\setslot{\uc{V}{v}}
   \comment{The letter `{V}'.}
\endsetslot

\setslot{\uc{W}{w}}
   \comment{The letter `{W}'.}
\endsetslot

\setslot{\uc{X}{x}}
   \comment{The letter `{X}'.}
\endsetslot

\setslot{\uc{Y}{y}}
   \comment{The letter `{Y}'.}
\endsetslot

\setslot{\uc{Z}{z}}
   \comment{The letter `{Z}'.}
\endsetslot

\setslot{bracketleft.superior}
   \comment{The opening square bracket `['.}
\endsetslot

\setslot{backslash.superior}
   \comment{The backwards oblique `\textbackslash'.}
\endsetslot

\setslot{bracketright.superior}
   \comment{The closing square bracket `]'.}
\endsetslot

\setslot{asciicircum.superior}
   \comment{The ASCII upward-pointing arrow head `\textasciicircum'.
      This is included for compatibility with typewriter fonts used
      for computer listings.}
\endsetslot

\setslot{underscore.superior}
   \comment{The ASCII underline character `\textunderscore', usually
      set on the baseline.
      This is included for compatibility with typewriter fonts used
      for computer listings.}
\endsetslot

\setslot{quoteleft.superior}
   \ligature{LIG}{quoteleft.superior}{quotedblleft.superior}
   \comment{The English opening single quote mark `\,\textquoteleft\,'.}
\endsetslot

\setslot{\lc{A}{a}}
   \comment{The letter `{a}'.}
\endsetslot

\setslot{\lc{B}{b}}
   \comment{The letter `{b}'.}
\endsetslot

\ifnumber{\int{ligaturing}}<{-1}\then \skipslots{1}\Else

   \setslot{\lc{C}{c}}
      \comment{The letter `{c}'.}
   \endsetslot

\Fi

\setslot{\lc{D}{d}}
   \comment{The letter `{d}'.}
\endsetslot

\setslot{\lc{E}{e}}
   \comment{The letter `{e}'.}
\endsetslot

\ifnumber{\int{ligaturing}}<{-1}\then \skipslots{1}\Else

   \setslot{\lc{F}{f}}
      \ifnumber{\int{ligaturing}}>{0}\then
         \ligature{LIG}{\lc{I}{i}}{\lclig{FI}{fi}}
         \ligature{LIG}{\lc{F}{f}}{\lclig{FF}{ff}}
         \ligature{LIG}{\lc{L}{l}}{\lclig{FL}{fl}}
      \Fi
      \comment{The letter `{f}'.}
   \endsetslot

\Fi

\setslot{\lc{G}{g}}
   \comment{The letter `{g}'.}
\endsetslot

\setslot{\lc{H}{h}}
   \comment{The letter `{h}'.}
\endsetslot

\ifnumber{\int{ligaturing}}<{-1}\then \skipslots{1}\Else

   \setslot{\lc{I}{i}}
      \comment{The letter `{i}'.}
   \endsetslot

\Fi

\setslot{\lc{J}{j}}
   \comment{The letter `{j}'.}
\endsetslot

\setslot{\lc{K}{k}}
   \comment{The letter `{k}'.}
\endsetslot

\setslot{\lc{L}{l}}
   \comment{The letter `{l}'.}
\endsetslot

\setslot{\lc{M}{m}}
   \comment{The letter `{m}'.}
\endsetslot

\setslot{\lc{N}{n}}
   \comment{The letter `{n}'.}
\endsetslot

\setslot{\lc{O}{o}}
   \comment{The letter `{o}'.}
\endsetslot

\setslot{\lc{P}{p}}
   \comment{The letter `{p}'.}
\endsetslot

\setslot{\lc{Q}{q}}
   \comment{The letter `{q}'.}
\endsetslot

\setslot{\lc{R}{r}}
   \comment{The letter `{r}'.}
\endsetslot

\ifnumber{\int{ligaturing}}<{-1}\then \skipslots{1}\Else

   \setslot{\lc{S}{s}}
      \comment{The letter `{s}'.}
   \endsetslot

\Fi

\setslot{\lc{T}{t}}
   \comment{The letter `{t}'.}
\endsetslot

\setslot{\lc{U}{u}}
   \comment{The letter `{u}'.}
\endsetslot

\setslot{\lc{V}{v}}
   \comment{The letter `{v}'.}
\endsetslot

\setslot{\lc{W}{w}}
   \comment{The letter `{w}'.}
\endsetslot

\setslot{\lc{X}{x}}
   \comment{The letter `{x}'.}
\endsetslot

\setslot{\lc{Y}{y}}
   \comment{The letter `{y}'.}
\endsetslot

\setslot{\lc{Z}{z}}
   \comment{The letter `{z}'.}
\endsetslot

\setslot{braceleft.superior}
   \comment{The opening curly brace `\textbraceleft'.}
\endsetslot

\setslot{bar.superior}
   \comment{The ASCII vertical bar `\textbar'.
      This is included for compatibility with typewriter fonts used
      for computer listings.}
\endsetslot

\setslot{braceright.superior}
   \comment{The closing curly brace `\textbraceright'.}
\endsetslot

\setslot{asciitilde.superior}
   \comment{The ASCII tilde `\textasciitilde'.
      This is included for compatibility with typewriter fonts used
      for computer listings.}
\endsetslot

\setslot{hyphenchar.superior}
   \comment{The glyph used for hyphenation in this font, which will
      almost always be the same as `hyphen'.}
\endsetslot

\setslot{\uctop{Abreve}{abreve}}
   \comment{The letter `\u A'.}
\endsetslot

\setslot{\uc{Aogonek}{aogonek}}
   \comment{The letter `\k A'.}
\endsetslot

\setslot{\uctop{Cacute}{cacute}}
   \comment{The letter `\' C'.}
\endsetslot

\setslot{\uctop{Ccaron}{ccaron}}
   \comment{The letter `\v C'.}
\endsetslot

\setslot{\uctop{Dcaron}{dcaron}}
   \comment{The letter `\v D'.}
\endsetslot

\setslot{\uctop{Ecaron}{ecaron}}
   \comment{The letter `\v E'.}
\endsetslot

\setslot{\uc{Eogonek}{eogonek}}
   \comment{The letter `\k E'.}
\endsetslot

\setslot{\uctop{Gbreve}{gbreve}}
   \comment{The letter `\u G'.}
\endsetslot

\setslot{\uctop{Lacute}{lacute}}
   \comment{The letter `\' L'.}
\endsetslot

\setslot{\uc{Lcaron}{lcaron}}
   \comment{The letter `\v L'.}
\endsetslot

\setslot{\uc{Lslash}{lslash}}
   \comment{The letter `\L'.}
\endsetslot

\setslot{\uctop{Nacute}{nacute}}
   \comment{The letter `\' N'.}
\endsetslot

\setslot{\uctop{Ncaron}{ncaron}}
   \comment{The letter `\v N'.}
\endsetslot

\setslot{\uc{Eng}{eng}}
   \comment{The Sami letter `\NG'.  It is unavailable in \plain\ \TeX. This needs to be called `Eng'/`eng' rather than `Ng'/`ng' as in t1.etx in most cases, it seems.}
\endsetslot

\setslot{\uctop{Ohungarumlaut}{ohungarumlaut}}
   \comment{The letter `\H O'.}
\endsetslot

\setslot{\uctop{Racute}{racute}}
   \comment{The letter `\' R'.}
\endsetslot

\setslot{\uctop{Rcaron}{rcaron}}
   \comment{The letter `\v R'.}
\endsetslot

\setslot{\uctop{Sacute}{sacute}}
   \comment{The letter `\' S'.}
\endsetslot

\setslot{\uctop{Scaron}{scaron}}
   \comment{The letter `\v S'.}
\endsetslot

\setslot{\uc{Scedilla}{scedilla}}
   \comment{The letter `\c S'.}
\endsetslot

\setslot{\uctop{Tcaron}{tcaron}}
   \comment{The letter `\v T'.}
\endsetslot

\setslot{\uc{Tcedilla}{tcedilla}}
   \comment{The letter `\c T'.}
\endsetslot

\setslot{\uctop{Uhungarumlaut}{uhungarumlaut}}
   \comment{The letter `\H U'.}
\endsetslot

\setslot{\uctop{Uring}{uring}}
   \comment{The letter `\r U'.}
\endsetslot

\setslot{\uctop{Ydieresis}{ydieresis}}
   \comment{The letter `\" Y'.}
\endsetslot

\setslot{\uctop{Zacute}{zacute}}
   \comment{The letter `\' Z'.}
\endsetslot

\setslot{\uctop{Zcaron}{zcaron}}
   \comment{The letter `\v Z'.}
\endsetslot

\setslot{\uctop{Zdotaccent}{zdotaccent}}
   \comment{The letter `\. Z'.}
\endsetslot

\ifnumber{\int{ligaturing}}<{0}\then \skipslots{1}\Else

   \setslot{\uclig{IJ}{ij}}
      \comment{The letter `IJ'.  This is a single letter, and in a 
        monowidth font should ideally be one letter wide.}
   \endsetslot

\Fi

\setslot{\uctop{Idotaccent}{idotaccent}}
   \comment{The letter `\. I'.}
\endsetslot

\setslot{\lc{Dbar}{dbar}}
   \comment{The letter `\dj'.}
\endsetslot

\setslot{section.superior}
   \comment{The section mark `\textsection'.}
\endsetslot

\setslot{\lctop{Abreve}{abreve}}
   \comment{The letter `\u a'.}
\endsetslot

\setslot{\lc{Aogonek}{aogonek}}
   \comment{The letter `\k a'.}
\endsetslot

\setslot{\lctop{Cacute}{cacute}}
   \comment{The letter `\' c'.}
\endsetslot

\setslot{\lctop{Ccaron}{ccaron}}
   \comment{The letter `\v c'.}
\endsetslot

\setslot{\lctop{Dcaron}{dcaron}}
   \comment{The letter `\v d'.}
\endsetslot

\setslot{\lctop{Ecaron}{ecaron}}
   \comment{The letter `\v e'.}
\endsetslot

\setslot{\lc{Eogonek}{eogonek}}
   \comment{The letter `\k e'.}
\endsetslot

\setslot{\lctop{Gbreve}{gbreve}}
   \comment{The letter `\u g'.}
\endsetslot

\setslot{\lctop{Lacute}{lacute}}
   \comment{The letter `\' l'.}
\endsetslot

\setslot{\lc{Lcaron}{lcaron}}
   \comment{The letter `\v l'.}
\endsetslot

\setslot{\lc{Lslash}{lslash}}
   \comment{The letter `\l'.}
\endsetslot

\setslot{\lctop{Nacute}{nacute}}
   \comment{The letter `\' n'.}
\endsetslot

\setslot{\lctop{Ncaron}{ncaron}}
   \comment{The letter `\v n'.}
\endsetslot

\setslot{\lc{Eng}{eng}}
   \comment{The Sami letter `\ng'.  It is unavailable in \plain\ \TeX. This needs to be called `Eng'/`eng' rather than `Ng'/`ng' as it is in t1.etx in most cases, it seems.}
\endsetslot

\setslot{\lctop{Ohungarumlaut}{ohungarumlaut}}
   \comment{The letter `\H o'.}
\endsetslot

\setslot{\lctop{Racute}{racute}}
   \comment{The letter `\' r'.}
\endsetslot

\setslot{\lctop{Rcaron}{rcaron}}
   \comment{The letter `\v r'.}
\endsetslot

\setslot{\lctop{Sacute}{sacute}}
   \comment{The letter `\' s'.}
\endsetslot

\setslot{\lctop{Scaron}{scaron}}
   \comment{The letter `\v s'.}
\endsetslot

\setslot{\lc{Scedilla}{scedilla}}
   \comment{The letter `\c s'.}
\endsetslot

\setslot{\lctop{Tcaron}{tcaron}}
   \comment{The letter `\v t'.}
\endsetslot

\setslot{\lc{Tcedilla}{tcedilla}}
   \comment{The letter `\c t'.}
\endsetslot

\setslot{\lctop{Uhungarumlaut}{uhungarumlaut}}
   \comment{The letter `\H u'.}
\endsetslot

\setslot{\lctop{Uring}{uring}}
   \comment{The letter `\r u'.}
\endsetslot

\setslot{\lctop{Ydieresis}{ydieresis}}
   \comment{The letter `\" y'.}
\endsetslot

\setslot{\lctop{Zacute}{zacute}}
   \comment{The letter `\' z'.}
\endsetslot

\setslot{\lctop{Zcaron}{zcaron}}
   \comment{The letter `\v z'.}
\endsetslot

\setslot{\lctop{Zdotaccent}{zdotaccent}}
   \comment{The letter `\. z'.}
\endsetslot

\ifnumber{\int{ligaturing}}<{0}\then \skipslots{1}\Else

   \setslot{\lclig{IJ}{ij}}
      \comment{The letter `ij'.  This is a single letter, and in a 
        monowidth font should ideally be one letter wide.}
   \endsetslot

\Fi

\setslot{exclamdown.superior}
   \comment{The Spanish punctuation mark `!`'.}
\endsetslot

\setslot{questiondown.superior}
   \comment{The Spanish punctuation mark `?`'.}
\endsetslot

\setslot{sterling.superior}
   \comment{The British currency mark `\textsterling'.}
\endsetslot

\setslot{\uctop{Agrave}{agrave}}
   \comment{The letter `\` A'.}
\endsetslot

\setslot{\uctop{Aacute}{aacute}}
   \comment{The letter `\' A'.}
\endsetslot

\setslot{\uctop{Acircumflex}{acircumflex}}
   \comment{The letter `\^ A'.}
\endsetslot

\setslot{\uctop{Atilde}{atilde}}
   \comment{The letter `\~ A'.}
\endsetslot

\setslot{\uctop{Adieresis}{adieresis}}
   \comment{The letter `\" A'.}
\endsetslot

\setslot{\uctop{Aring}{aring}}
   \comment{The letter `\r A'.}
\endsetslot

\setslot{\uc{AE}{ae}}
   \comment{The letter `\AE'.  This is a single letter, and should not be
      faked with `AE'.}
\endsetslot

\setslot{\uc{Ccedilla}{ccedilla}}
   \comment{The letter `\c C'.}
\endsetslot

\setslot{\uctop{Egrave}{egrave}}
   \comment{The letter `\` E'.}
\endsetslot

\setslot{\uctop{Eacute}{eacute}}
   \comment{The letter `\' E'.}
\endsetslot

\setslot{\uctop{Ecircumflex}{ecircumflex}}
   \comment{The letter `\^ E'.}
\endsetslot

\setslot{\uctop{Edieresis}{edieresis}}
   \comment{The letter `\" E'.}
\endsetslot

\setslot{\uctop{Igrave}{igrave}}
   \comment{The letter `\` I'.}
\endsetslot

\setslot{\uctop{Iacute}{iacute}}
   \comment{The letter `\' I'.}
\endsetslot

\setslot{\uctop{Icircumflex}{icircumflex}}
   \comment{The letter `\^ I'.}
\endsetslot

\setslot{\uctop{Idieresis}{idieresis}}
   \comment{The letter `\" I'.}
\endsetslot

\setslot{\uc{Eth}{eth}}
   \comment{The uppercase Icelandic letter `Eth' similar to a `D'
      with a horizontal bar through the stem.  It is unavailable
      in \plain\ \TeX.}
\endsetslot

\setslot{\uctop{Ntilde}{ntilde}}
   \comment{The letter `\~ N'.}
\endsetslot

\setslot{\uctop{Ograve}{ograve}}
   \comment{The letter `\` O'.}
\endsetslot

\setslot{\uctop{Oacute}{oacute}}
   \comment{The letter `\' O'.}
\endsetslot

\setslot{\uctop{Ocircumflex}{ocircumflex}}
   \comment{The letter `\^ O'.}
\endsetslot

\setslot{\uctop{Otilde}{otilde}}
   \comment{The letter `\~ O'.}
\endsetslot

\setslot{\uctop{Odieresis}{odieresis}}
   \comment{The letter `\" O'.}
\endsetslot

\setslot{\uc{OE}{oe}}
   \comment{The letter `\OE'.  This is a single letter, and should not be
      faked with `OE'.}
\endsetslot

\setslot{\uc{Oslash}{oslash}}
   \comment{The letter `\O'.}
\endsetslot

\setslot{\uctop{Ugrave}{ugrave}}
   \comment{The letter `\` U'.}
\endsetslot

\setslot{\uctop{Uacute}{uacute}}
   \comment{The letter `\' U'.}
\endsetslot

\setslot{\uctop{Ucircumflex}{ucircumflex}}
   \comment{The letter `\^ U'.}
\endsetslot

\setslot{\uctop{Udieresis}{udieresis}}
   \comment{The letter `\" U'.}
\endsetslot

\setslot{\uctop{Yacute}{yacute}}
   \comment{The letter `\' Y'.}
\endsetslot

\setslot{\uc{Thorn}{thorn}}
   \comment{The Icelandic capital letter Thorn, similar to a `P'
      with the bowl moved down.  It is unavailable in \plain\ \TeX.}
\endsetslot

\setslot{\uclig{SS}{germandbls}}
   \comment{The ligature `SS', used to give an upper case `\ss'.
      In a monowidth font it should be two letters wide.}
\endsetslot

\setslot{\lctop{Agrave}{agrave}}
   \comment{The letter `\` a'.}
\endsetslot

\setslot{\lctop{Aacute}{aacute}}
   \comment{The letter `\' a'.}
\endsetslot

\setslot{\lctop{Acircumflex}{acircumflex}}
   \comment{The letter `\^ a'.}
\endsetslot

\setslot{\lctop{Atilde}{atilde}}
   \comment{The letter `\~ a'.}
\endsetslot

\setslot{\lctop{Adieresis}{adieresis}}
   \comment{The letter `\" a'.}
\endsetslot

\setslot{\lctop{Aring}{aring}}
   \comment{The letter `\r a'.}
\endsetslot

\setslot{\lc{AE}{ae}}
   \comment{The letter `\ae'.  This is a single letter, and should not be
      faked with `ae'.}
\endsetslot

\setslot{\lc{Ccedilla}{ccedilla}}
   \comment{The letter `\c c'.}
\endsetslot

\setslot{\lctop{Egrave}{egrave}}
   \comment{The letter `\` e'.}
\endsetslot

\setslot{\lctop{Eacute}{eacute}}
   \comment{The letter `\' e'.}
\endsetslot

\setslot{\lctop{Ecircumflex}{ecircumflex}}
   \comment{The letter `\^ e'.}
\endsetslot

\setslot{\lctop{Edieresis}{edieresis}}
   \comment{The letter `\" e'.}
\endsetslot

\setslot{\lctop{Igrave}{igrave}}
   \comment{The letter `\`\i'.}
\endsetslot

\setslot{\lctop{Iacute}{iacute}}
   \comment{The letter `\'\i'.}
\endsetslot

\setslot{\lctop{Icircumflex}{icircumflex}}
   \comment{The letter `\^\i'.}
\endsetslot

\setslot{\lctop{Idieresis}{idieresis}}
   \comment{The letter `\"\i'.}
\endsetslot

\setslot{\lc{Eth}{eth}}
   \comment{The Icelandic lowercase letter `eth' similar to
     a `$\partial$' with an oblique bar through the stem.
     It is unavailable in \plain\ \TeX.}
\endsetslot

\setslot{\lctop{Ntilde}{ntilde}}
   \comment{The letter `\~ n'.}
\endsetslot

\setslot{\lctop{Ograve}{ograve}}
   \comment{The letter `\` o'.}
\endsetslot

\setslot{\lctop{Oacute}{oacute}}
   \comment{The letter `\' o'.}
\endsetslot

\setslot{\lctop{Ocircumflex}{ocircumflex}}
   \comment{The letter `\^ o'.}
\endsetslot

\setslot{\lctop{Otilde}{otilde}}
   \comment{The letter `\~ o'.}
\endsetslot

\setslot{\lctop{Odieresis}{odieresis}}
   \comment{The letter `\" o'.}
\endsetslot

\setslot{\lc{OE}{oe}}
   \comment{The letter `\oe'.  This is a single letter, and should not be
      faked with `oe'.}
\endsetslot

\setslot{\lc{Oslash}{oslash}}
   \comment{The letter `\o'.}
\endsetslot

\setslot{\lctop{Ugrave}{ugrave}}
   \comment{The letter `\` u'.}
\endsetslot

\setslot{\lctop{Uacute}{uacute}}
   \comment{The letter `\' u'.}
\endsetslot

\setslot{\lctop{Ucircumflex}{ucircumflex}}
   \comment{The letter `\^ u'.}
\endsetslot

\setslot{\lctop{Udieresis}{udieresis}}
   \comment{The letter `\" u'.}
\endsetslot

\setslot{\lctop{Yacute}{yacute}}
   \comment{The letter `\' y'.}
\endsetslot

\setslot{\lc{Thorn}{thorn}}
   \comment{The Icelandic lowercase letter `thorn', similar to a `p'
      with an ascender rising from the stem.  It is unavailable
      in \plain\ \TeX.}
\endsetslot

\setslot{\lc{SS}{germandbls}}
   \comment{The letter `\ss'.}
\endsetslot

\endencoding
%    \end{macrocode}
% \end{encoding}
% \iffalse
%</t1-dotsuperior>
% \fi
% 
% 
% \subsubsection{fontscripts-t1-f\_f.etx}\label{subsubsec:t1-f-f}
% 
% \iffalse
%<*t1-f-f>
% \fi
% \begin{encoding}{fontscripts-t1-f_f.etx}
% \changes{v0.0}{2025-02-10}{Filename prefix for Karl.}
%    \begin{macrocode}
%%
%% - The commentary in the original is deleted in this version. For 
%% information about the T1 etc., typeset the original t1.etx 
%% included with fontinst.
%% - Slots are altered to accommodate characters which are named 
%% differently. For example, this encoding uses "endash" and "emdash" 
%% whereas t1.etx called for "rangedash" and "punctdash".
%% - The original notices at the top of that file concerning authors,
%% maintenance etc. are replaced by this notice.
%% - The file is renamed.
%% - The encoding name is modified.
%% - f_f, f_f_i and f_f_l replace ff, ffi and ffl.
%%
%%%%%%%%%%%%%%%%%%%%%%%%%%%%%%%%%%%%%%%%%%%%%%%%%
\relax
\encoding

\needsfontinstversion{1.910}

\setcommand\lc#1#2{#2}
\setcommand\uc#1#2{#1}
\setcommand\lctop#1#2{#2}
\setcommand\uctop#1#2{#1}
\setcommand\lclig#1#2{#2}
\ifisint{letterspacing}\then
   \ifnumber{\int{letterspacing}}={0}\then \Else
      \setcommand\uclig#1#2{#1spaced}
      \comment{Here we set \verb|\uclig#1#2| to \verb|#1spaced|, but 
      you can't see it as \verb|\setcommand| commands are invisible in 
      the typeset output.}
   \Fi
\Fi
\setcommand\uclig#1#2{#1}
\setcommand\digit#1{#1}

\ifisint{monowidth}\then
   \setint{ligaturing}{0}
\Else
   % The following empty line is *important* to get the formatting
   % right here (sigh)! (Remember that it is a \par token.)
   
   \ifisint{letterspacing}\then
      \ifnumber{\int{letterspacing}}={0}\then \Else
         \setint{ligaturing}{0}
      \Fi
   \Fi
	\setint{ligaturing}{1}
\Fi

\setint{italicslant}{0}
\setint{quad}{1000}
\setint{baselineskip}{1200}

\ifisglyph{x}\then
   \setint{xheight}{\height{x}}
\Else
   \setint{xheight}{500}
\Fi

\ifisglyph{space}\then
   \setint{interword}{\width{space}}
\Else\ifisglyph{i}\then
   \setint{interword}{\width{i}}
\Else
   \setint{interword}{333}
\Fi\Fi

\ifisint{monowidth}\then
   \setint{stretchword}{0}
   \setint{shrinkword}{0}
   \setint{extraspace}{\int{interword}}
\Else
   \setint{stretchword}{\scale{\int{interword}}{600}}
   \setint{shrinkword}{\scale{\int{interword}}{240}}
   \setint{extraspace}{\scale{\int{interword}}{240}}
\Fi

\ifisglyph{X}\then
   \setint{capheight}{\height{X}}
\Else
   \setint{capheight}{750}
\Fi

\ifisglyph{d}\then
   \setint{ascender}{\height{d}}
\Else\ifisint{capheight}\then
   \setint{ascender}{\int{capheight}}
\Else
   \setint{ascender}{750}
\Fi\Fi

\ifisglyph{Aring}\then
   \setint{acccapheight}{\height{Aring}}
\Else
   \setint{acccapheight}{999}
\Fi

\ifisint{descender_neg}\then
   \setint{descender}{\neg{\int{descender_neg}}}
\Else\ifisglyph{p}\then
   \setint{descender}{\depth{p}}
\Else
   \setint{descender}{250}
\Fi\Fi

\ifisglyph{Aring}\then
   \setint{maxheight}{\height{Aring}}
\Else
   \setint{maxheight}{1000}
\Fi

\ifisint{maxdepth_neg}\then
   \setint{maxdepth}{\neg{\int{maxdepth_neg}}}
\Else\ifisglyph{j}\then
   \setint{maxdepth}{\depth{j}}
\Else
   \setint{maxdepth}{250}
\Fi\Fi

\ifisglyph{six}\then
   \setint{digitwidth}{\width{six}}
\Else
   \setint{digitwidth}{500}
\Fi

\setint{capstem}{0} % not in AFM files

\setfontdimen{1}{italicslant}    % italic slant
\setfontdimen{2}{interword}      % interword space
\setfontdimen{3}{stretchword}    % interword stretch
\setfontdimen{4}{shrinkword}     % interword shrink
\setfontdimen{5}{xheight}        % x-height
\setfontdimen{6}{quad}           % quad
\setfontdimen{7}{extraspace}     % extra space after .
\setfontdimen{8}{capheight}      % cap height
\setfontdimen{9}{ascender}       % ascender
\setfontdimen{10}{acccapheight}  % accented cap height
\setfontdimen{11}{descender}     % descender's depth
\setfontdimen{12}{maxheight}     % max height
\setfontdimen{13}{maxdepth}      % max depth
\setfontdimen{14}{digitwidth}    % digit width
\setfontdimen{15}{verticalstem}  % dominant width of verical stems
\setfontdimen{16}{baselineskip}  % baselineskip

\ifnumber{\int{ligaturing}}<{0}\then 
   \comment{In this case, the codingscheme can be different from the 
     default, and therefore we refrain from setting it.}
\Else
   \setstr{codingscheme}{EXTENDED TEX ENC - F_F}
\Fi

\setslot{\lc{Grave}{grave}}
   \comment{The grave accent `\`{}'.}
\endsetslot

\setslot{\lc{Acute}{acute}}
   \comment{The acute accent `\'{}'.}
\endsetslot

\setslot{\lc{Circumflex}{circumflex}}
   \comment{The circumflex accent `\^{}'.}
\endsetslot

\setslot{\lc{Tilde}{tilde}}
   \comment{The tilde accent `\~{}'.}
\endsetslot

\setslot{\lc{Dieresis}{dieresis}}
   \comment{The umlaut or dieresis accent `\"{}'.}
\endsetslot

\setslot{\lc{Hungarumlaut}{hungarumlaut}}
   \comment{The long Hungarian umlaut `\H{}'.}
\endsetslot

\setslot{\lc{Ring}{ring}}
   \comment{The ring accent `\r{}'.}
\endsetslot

\setslot{\lc{Caron}{caron}}
   \comment{The caron or h\'a\v cek accent `\v{}'.}
\endsetslot

\setslot{\lc{Breve}{breve}}
   \comment{The breve accent `\u{}'.}
\endsetslot

\setslot{\lc{Macron}{macron}}
   \comment{The macron accent `\={}'.}
\endsetslot

\setslot{\lc{Dotaccent}{dotaccent}}
   \comment{The dot accent `\.{}'.}
\endsetslot

\setslot{\lc{Cedilla}{cedilla}}
   \comment{The cedilla accent `\c {}'.}
\endsetslot

\setslot{\lc{Ogonek}{ogonek}}
   \comment{The ogonek accent `\k {}'.}
\endsetslot

\setslot{quotesinglbase}
  \comment{A German single quote mark `\quotesinglbase' similar to a comma,
      but with different sidebearings.}
\endsetslot

\setslot{guilsinglleft}
  \comment{A French single opening quote mark `\guilsinglleft',
      unavailable in \plain\ \TeX.}
\endsetslot

\setslot{guilsinglright}
  \comment{A French single closing quote mark `\guilsinglright',
      unavailable in \plain\ \TeX.}
\endsetslot

\setslot{quotedblleft}
  \comment{The English opening quote mark `\,\textquotedblleft\,'.}
\endsetslot

\setslot{quotedblright}
  \comment{The English closing quote mark `\,\textquotedblright\,'.}
\endsetslot

\setslot{quotedblbase}
  \comment{A German double quote mark `\quotedblbase' similar to two commas,
      but with tighter letterspacing and different sidebearings.}
\endsetslot

\setslot{guillemotleft}
  \comment{A French double opening quote mark `\guillemotleft',
      unavailable in \plain\ \TeX.}
\endsetslot

\setslot{guillemotright}
  \comment{A French closing opening quote mark `\guillemotright',
      unavailable in \plain\ \TeX.}
\endsetslot

\setslot{endash}
   \ligature{LIG}{hyphen}{emdash}
   \comment{The number range dash `1--9'. 
     This is called `rangedash' by fontinst's t1.etx, but it needs to be 
     called `endash' to work right. 
     The `\textendash'.  In a monowidth font, this might be set as 
     `\texttt{1{-}9}'.}
\endsetslot

\setslot{emdash}
   \comment{The punctuation dash `Oh---boy.' 
     This is calle `punctdash' by fontinst's t1.etx, but needs to be 
     called `emdash' to work right.  
     The `\textemdash'.  
     In a monowidth font, this might be set as `\texttt{Oh{-}{-}boy.}'}
\endsetslot

\setslot{compwordmark}
   \comment{An invisible glyph, with zero width and depth, but the
      height of lowercase letters without ascenders.
      It is used to stop ligaturing in words like `shelf{}ful'.}
\endsetslot

\setslot{perthousandzero}
   \comment{A glyph which is placed after `\%' to produce a
      `per-thousand', or twice to produce `per-ten-thousand'.
      Your guess is as good as mine as to what this glyph should look
      like in a monowidth font.}
\endsetslot

\setslot{\lc{dotlessI}{dotlessi}}
   \comment{A dotless i `\i', used to produce accented letters such as
      `\=\i'.}
\endsetslot

\setslot{\lc{dotlessJ}{dotlessj}}
   \comment{A dotless j `\j', used to produce accented letters such as
      `\=\j'.  Most non-\TeX\ fonts do not have this glyph.}
\endsetslot

\ifnumber{\int{ligaturing}}<{0}\then \skipslots{5}\Else

\setslot{\lclig{FF}{f_f}}
   \ifnumber{\int{ligaturing}}>{0}\then
      \ligature{LIG}{\lc{I}{i}}{\lclig{FFI}{f_f_i}}
      \ligature{LIG}{\lc{L}{l}}{\lclig{FFL}{f_f_l}}
   \Fi
   \comment{The `ff' ligature.  It should be two characters wide in a
      monowidth font.}
\endsetslot

\setslot{\lclig{FI}{fi}}
   \comment{The `fi' ligature.  It should be two characters wide in a
      monowidth font.}
\endsetslot

\setslot{\lclig{FL}{fl}}
   \comment{The `fl' ligature.  It should be two characters wide in a
      monowidth font.}
\endsetslot

\setslot{\lclig{FFI}{f_f_i}}
   \comment{The `ffi' ligature.  It should be three characters wide in a
      monowidth font.}
\endsetslot

\setslot{\lclig{FFL}{f_f_l}}
   \comment{The `ffl' ligature.  It should be three characters wide in a
      monowidth font.}
\endsetslot

\Fi

\setslot{visiblespace}
   \comment{A visible space glyph `\textvisiblespace'.}
\endsetslot

\setslot{exclam}
   \ligature{LIG}{quoteleft}{exclamdown}
   \comment{The exclamation mark `!'.}
\endsetslot

\setslot{quotedbl}
  \comment{The `neutral' double quotation mark `\,\textquotedbl\,',
      included for use in monowidth fonts, or for setting computer
      programs.  Note that the inclusion of this glyph in this slot
      means that \TeX\ documents which used `{\tt\char`\"}' as an
      input character will no longer work.}
\endsetslot

\setslot{numbersign}
   \comment{The hash sign `\#'.}
\endsetslot

\setslot{dollar}
   \comment{The dollar sign `\$'.}
\endsetslot

\setslot{percent}
   \comment{The percent sign `\%'.}
\endsetslot

\setslot{ampersand}
   \comment{The ampersand sign `\&'.}
\endsetslot

\setslot{quoteright}
   \ligature{LIG}{quoteright}{quotedblright}
   \comment{The English closing single quote mark `\,\textquoteright\,'.}
\endsetslot

\setslot{parenleft}
   \comment{The opening parenthesis `('.}
\endsetslot

\setslot{parenright}
   \comment{The closing parenthesis `)'.}
\endsetslot

\setslot{asterisk}
   \comment{The raised asterisk `*'.}
\endsetslot

\setslot{plus}
   \comment{The addition sign `+'.}
\endsetslot

\setslot{comma}
   \ligature{LIG}{comma}{quotedblbase}
   \comment{The comma `,'.}
\endsetslot

\setslot{hyphen}
   \ligature{LIG}{hyphen}{endash}
   \ligature{LIG}{hyphenchar}{hyphenchar}
   \comment{The hyphen `-'.}
\endsetslot

\setslot{period}
   \comment{The period `.'.}
\endsetslot

\setslot{slash}
   \comment{The forward oblique `/'.}
\endsetslot

\setslot{\digit{zero}}
   \comment{The number `0'.  This (and all the other numerals) may be
      old style or ranging digits.}
\endsetslot

\setslot{\digit{one}}
   \comment{The number `1'.}
\endsetslot

\setslot{\digit{two}}
   \comment{The number `2'.}
\endsetslot

\setslot{\digit{three}}
   \comment{The number `3'.}
\endsetslot

\setslot{\digit{four}}
   \comment{The number `4'.}
\endsetslot

\setslot{\digit{five}}
   \comment{The number `5'.}
\endsetslot

\setslot{\digit{six}}
   \comment{The number `6'.}
\endsetslot

\setslot{\digit{seven}}
   \comment{The number `7'.}
\endsetslot

\setslot{\digit{eight}}
   \comment{The number `8'.}
\endsetslot

\setslot{\digit{nine}}
   \comment{The number `9'.}
\endsetslot

\setslot{colon}
   \comment{The colon punctuation mark `:'.}
\endsetslot

\setslot{semicolon}
   \comment{The semi-colon punctuation mark `;'.}
\endsetslot

\setslot{less}
   \ligature{LIG}{less}{guillemotleft}
   \comment{The less-than sign `\textless'.}
\endsetslot

\setslot{equal}
   \comment{The equals sign `='.}
\endsetslot

\setslot{greater}
   \ligature{LIG}{greater}{guillemotright}
   \comment{The greater-than sign `\textgreater'.}
\endsetslot

\setslot{question}
   \ligature{LIG}{quoteleft}{questiondown}
   \comment{The question mark `?'.}
\endsetslot

\setslot{at}
   \comment{The at sign `@'.}
\endsetslot

\setslot{\uc{A}{a}}
   \comment{The letter `{A}'.}
\endsetslot

\setslot{\uc{B}{b}}
   \comment{The letter `{B}'.}
\endsetslot

\setslot{\uc{C}{c}}
   \comment{The letter `{C}'.}
\endsetslot

\setslot{\uc{D}{d}}
   \comment{The letter `{D}'.}
\endsetslot

\setslot{\uc{E}{e}}
   \comment{The letter `{E}'.}
\endsetslot

\setslot{\uc{F}{f}}
   \comment{The letter `{F}'.}
\endsetslot

\setslot{\uc{G}{g}}
   \comment{The letter `{G}'.}
\endsetslot

\setslot{\uc{H}{h}}
   \comment{The letter `{H}'.}
\endsetslot

\ifnumber{\int{ligaturing}}<{-1}\then \skipslots{1}\Else

\setslot{\uc{I}{i}}
   \comment{The letter `{I}'.}
\endsetslot

\Fi

\setslot{\uc{J}{j}}
   \comment{The letter `{J}'.}
\endsetslot

\setslot{\uc{K}{k}}
   \comment{The letter `{K}'.}
\endsetslot

\setslot{\uc{L}{l}}
   \comment{The letter `{L}'.}
\endsetslot

\setslot{\uc{M}{m}}
   \comment{The letter `{M}'.}
\endsetslot

\setslot{\uc{N}{n}}
   \comment{The letter `{N}'.}
\endsetslot

\setslot{\uc{O}{o}}
   \comment{The letter `{O}'.}
\endsetslot

\setslot{\uc{P}{p}}
   \comment{The letter `{P}'.}
\endsetslot

\setslot{\uc{Q}{q}}
   \comment{The letter `{Q}'.}
\endsetslot

\setslot{\uc{R}{r}}
   \comment{The letter `{R}'.}
\endsetslot

\setslot{\uc{S}{s}}
   \comment{The letter `{S}'.}
\endsetslot

\setslot{\uc{T}{t}}
   \comment{The letter `{T}'.}
\endsetslot

\setslot{\uc{U}{u}}
   \comment{The letter `{U}'.}
\endsetslot

\setslot{\uc{V}{v}}
   \comment{The letter `{V}'.}
\endsetslot

\setslot{\uc{W}{w}}
   \comment{The letter `{W}'.}
\endsetslot

\setslot{\uc{X}{x}}
   \comment{The letter `{X}'.}
\endsetslot

\setslot{\uc{Y}{y}}
   \comment{The letter `{Y}'.}
\endsetslot

\setslot{\uc{Z}{z}}
   \comment{The letter `{Z}'.}
\endsetslot

\setslot{bracketleft}
   \comment{The opening square bracket `['.}
\endsetslot

\setslot{backslash}
   \comment{The backwards oblique `\textbackslash'.}
\endsetslot

\setslot{bracketright}
   \comment{The closing square bracket `]'.}
\endsetslot

\setslot{asciicircum}
   \comment{The ASCII upward-pointing arrow head `\textasciicircum'.
      This is included for compatibility with typewriter fonts used
      for computer listings.}
\endsetslot

\setslot{underscore}
   \comment{The ASCII underline character `\textunderscore', usually
      set on the baseline.
      This is included for compatibility with typewriter fonts used
      for computer listings.}
\endsetslot

\setslot{quoteleft}
   \ligature{LIG}{quoteleft}{quotedblleft}
   \comment{The English opening single quote mark `\,\textquoteleft\,'.}
\endsetslot

\setslot{\lc{A}{a}}
   \comment{The letter `{a}'.}
\endsetslot

\setslot{\lc{B}{b}}
   \comment{The letter `{b}'.}
\endsetslot

\ifnumber{\int{ligaturing}}<{-1}\then \skipslots{1}\Else

   \setslot{\lc{C}{c}}
      \comment{The letter `{c}'.}
   \endsetslot

\Fi

\setslot{\lc{D}{d}}
   \comment{The letter `{d}'.}
\endsetslot

\setslot{\lc{E}{e}}
   \comment{The letter `{e}'.}
\endsetslot

\ifnumber{\int{ligaturing}}<{-1}\then \skipslots{1}\Else

   \setslot{\lc{F}{f}}
      \ifnumber{\int{ligaturing}}>{0}\then
         \ligature{LIG}{\lc{I}{i}}{\lclig{FI}{fi}}
         \ligature{LIG}{\lc{F}{f}}{\lclig{FF}{f_f}}
         \ligature{LIG}{\lc{L}{l}}{\lclig{FL}{fl}}
      \Fi
      \comment{The letter `{f}'.}
   \endsetslot

\Fi

\setslot{\lc{G}{g}}
   \comment{The letter `{g}'.}
\endsetslot

\setslot{\lc{H}{h}}
   \comment{The letter `{h}'.}
\endsetslot

\ifnumber{\int{ligaturing}}<{-1}\then \skipslots{1}\Else

   \setslot{\lc{I}{i}}
      \comment{The letter `{i}'.}
   \endsetslot

\Fi

\setslot{\lc{J}{j}}
   \comment{The letter `{j}'.}
\endsetslot

\setslot{\lc{K}{k}}
   \comment{The letter `{k}'.}
\endsetslot

\setslot{\lc{L}{l}}
   \comment{The letter `{l}'.}
\endsetslot

\setslot{\lc{M}{m}}
   \comment{The letter `{m}'.}
\endsetslot

\setslot{\lc{N}{n}}
   \comment{The letter `{n}'.}
\endsetslot

\setslot{\lc{O}{o}}
   \comment{The letter `{o}'.}
\endsetslot

\setslot{\lc{P}{p}}
   \comment{The letter `{p}'.}
\endsetslot

\setslot{\lc{Q}{q}}
   \comment{The letter `{q}'.}
\endsetslot

\setslot{\lc{R}{r}}
   \comment{The letter `{r}'.}
\endsetslot

\ifnumber{\int{ligaturing}}<{-1}\then \skipslots{1}\Else

   \setslot{\lc{S}{s}}
      \comment{The letter `{s}'.}
   \endsetslot

\Fi

\setslot{\lc{T}{t}}
   \comment{The letter `{t}'.}
\endsetslot

\setslot{\lc{U}{u}}
   \comment{The letter `{u}'.}
\endsetslot

\setslot{\lc{V}{v}}
   \comment{The letter `{v}'.}
\endsetslot

\setslot{\lc{W}{w}}
   \comment{The letter `{w}'.}
\endsetslot

\setslot{\lc{X}{x}}
   \comment{The letter `{x}'.}
\endsetslot

\setslot{\lc{Y}{y}}
   \comment{The letter `{y}'.}
\endsetslot

\setslot{\lc{Z}{z}}
   \comment{The letter `{z}'.}
\endsetslot

\setslot{braceleft}
   \comment{The opening curly brace `\textbraceleft'.}
\endsetslot

\setslot{bar}
   \comment{The ASCII vertical bar `\textbar'.
      This is included for compatibility with typewriter fonts used
      for computer listings.}
\endsetslot

\setslot{braceright}
   \comment{The closing curly brace `\textbraceright'.}
\endsetslot

\setslot{asciitilde}
   \comment{The ASCII tilde `\textasciitilde'.
      This is included for compatibility with typewriter fonts used
      for computer listings.}
\endsetslot

\setslot{hyphenchar}
   \comment{The glyph used for hyphenation in this font, which will
      almost always be the same as `hyphen'.}
\endsetslot

\setslot{\uctop{Abreve}{abreve}}
   \comment{The letter `\u A'.}
\endsetslot

\setslot{\uc{Aogonek}{aogonek}}
   \comment{The letter `\k A'.}
\endsetslot

\setslot{\uctop{Cacute}{cacute}}
   \comment{The letter `\' C'.}
\endsetslot

\setslot{\uctop{Ccaron}{ccaron}}
   \comment{The letter `\v C'.}
\endsetslot

\setslot{\uctop{Dcaron}{dcaron}}
   \comment{The letter `\v D'.}
\endsetslot

\setslot{\uctop{Ecaron}{ecaron}}
   \comment{The letter `\v E'.}
\endsetslot

\setslot{\uc{Eogonek}{eogonek}}
   \comment{The letter `\k E'.}
\endsetslot

\setslot{\uctop{Gbreve}{gbreve}}
   \comment{The letter `\u G'.}
\endsetslot

\setslot{\uctop{Lacute}{lacute}}
   \comment{The letter `\' L'.}
\endsetslot

\setslot{\uc{Lcaron}{lcaron}}
   \comment{The letter `\v L'.}
\endsetslot

\setslot{\uc{Lslash}{lslash}}
   \comment{The letter `\L'.}
\endsetslot

\setslot{\uctop{Nacute}{nacute}}
   \comment{The letter `\' N'.}
\endsetslot

\setslot{\uctop{Ncaron}{ncaron}}
   \comment{The letter `\v N'.}
\endsetslot

\setslot{\uc{Eng}{eng}}
   \comment{The Sami letter `\NG'.  It is unavailable in \plain\ \TeX. This needs to be called `Eng'/`eng' rather than `Ng'/`ng' as in t1.etx in most cases, it seems.}
\endsetslot

\setslot{\uctop{Ohungarumlaut}{ohungarumlaut}}
   \comment{The letter `\H O'.}
\endsetslot

\setslot{\uctop{Racute}{racute}}
   \comment{The letter `\' R'.}
\endsetslot

\setslot{\uctop{Rcaron}{rcaron}}
   \comment{The letter `\v R'.}
\endsetslot

\setslot{\uctop{Sacute}{sacute}}
   \comment{The letter `\' S'.}
\endsetslot

\setslot{\uctop{Scaron}{scaron}}
   \comment{The letter `\v S'.}
\endsetslot

\setslot{\uc{Scedilla}{scedilla}}
   \comment{The letter `\c S'.}
\endsetslot

\setslot{\uctop{Tcaron}{tcaron}}
   \comment{The letter `\v T'.}
\endsetslot

\setslot{\uc{Tcedilla}{tcedilla}}
   \comment{The letter `\c T'.}
\endsetslot

\setslot{\uctop{Uhungarumlaut}{uhungarumlaut}}
   \comment{The letter `\H U'.}
\endsetslot

\setslot{\uctop{Uring}{uring}}
   \comment{The letter `\r U'.}
\endsetslot

\setslot{\uctop{Ydieresis}{ydieresis}}
   \comment{The letter `\" Y'.}
\endsetslot

\setslot{\uctop{Zacute}{zacute}}
   \comment{The letter `\' Z'.}
\endsetslot

\setslot{\uctop{Zcaron}{zcaron}}
   \comment{The letter `\v Z'.}
\endsetslot

\setslot{\uctop{Zdotaccent}{zdotaccent}}
   \comment{The letter `\. Z'.}
\endsetslot

\ifnumber{\int{ligaturing}}<{0}\then \skipslots{1}\Else

   \setslot{\uclig{IJ}{ij}}
      \comment{The letter `IJ'.  This is a single letter, and in a 
        monowidth font should ideally be one letter wide.}
   \endsetslot

\Fi

\setslot{\uctop{Idotaccent}{idotaccent}}
   \comment{The letter `\. I'.}
\endsetslot

\setslot{\lc{Dbar}{dbar}}
   \comment{The letter `\dj'.}
\endsetslot

\setslot{section}
   \comment{The section mark `\textsection'.}
\endsetslot

\setslot{\lctop{Abreve}{abreve}}
   \comment{The letter `\u a'.}
\endsetslot

\setslot{\lc{Aogonek}{aogonek}}
   \comment{The letter `\k a'.}
\endsetslot

\setslot{\lctop{Cacute}{cacute}}
   \comment{The letter `\' c'.}
\endsetslot

\setslot{\lctop{Ccaron}{ccaron}}
   \comment{The letter `\v c'.}
\endsetslot

\setslot{\lctop{Dcaron}{dcaron}}
   \comment{The letter `\v d'.}
\endsetslot

\setslot{\lctop{Ecaron}{ecaron}}
   \comment{The letter `\v e'.}
\endsetslot

\setslot{\lc{Eogonek}{eogonek}}
   \comment{The letter `\k e'.}
\endsetslot

\setslot{\lctop{Gbreve}{gbreve}}
   \comment{The letter `\u g'.}
\endsetslot

\setslot{\lctop{Lacute}{lacute}}
   \comment{The letter `\' l'.}
\endsetslot

\setslot{\lc{Lcaron}{lcaron}}
   \comment{The letter `\v l'.}
\endsetslot

\setslot{\lc{Lslash}{lslash}}
   \comment{The letter `\l'.}
\endsetslot

\setslot{\lctop{Nacute}{nacute}}
   \comment{The letter `\' n'.}
\endsetslot

\setslot{\lctop{Ncaron}{ncaron}}
   \comment{The letter `\v n'.}
\endsetslot

\setslot{\lc{Eng}{eng}}
   \comment{The Sami letter `\ng'.  
    It is unavailable in \plain\ \TeX. 
    This needs to be called `Eng'/`eng' rather than `Ng'/`ng' as it is in 
    t1.etx in most cases, it seems.}
\endsetslot

\setslot{\lctop{Ohungarumlaut}{ohungarumlaut}}
   \comment{The letter `\H o'.}
\endsetslot

\setslot{\lctop{Racute}{racute}}
   \comment{The letter `\' r'.}
\endsetslot

\setslot{\lctop{Rcaron}{rcaron}}
   \comment{The letter `\v r'.}
\endsetslot

\setslot{\lctop{Sacute}{sacute}}
   \comment{The letter `\' s'.}
\endsetslot

\setslot{\lctop{Scaron}{scaron}}
   \comment{The letter `\v s'.}
\endsetslot

\setslot{\lc{Scedilla}{scedilla}}
   \comment{The letter `\c s'.}
\endsetslot

\setslot{\lctop{Tcaron}{tcaron}}
   \comment{The letter `\v t'.}
\endsetslot

\setslot{\lc{Tcedilla}{tcedilla}}
   \comment{The letter `\c t'.}
\endsetslot

\setslot{\lctop{Uhungarumlaut}{uhungarumlaut}}
   \comment{The letter `\H u'.}
\endsetslot

\setslot{\lctop{Uring}{uring}}
   \comment{The letter `\r u'.}
\endsetslot

\setslot{\lctop{Ydieresis}{ydieresis}}
   \comment{The letter `\" y'.}
\endsetslot

\setslot{\lctop{Zacute}{zacute}}
   \comment{The letter `\' z'.}
\endsetslot

\setslot{\lctop{Zcaron}{zcaron}}
   \comment{The letter `\v z'.}
\endsetslot

\setslot{\lctop{Zdotaccent}{zdotaccent}}
   \comment{The letter `\. z'.}
\endsetslot

\ifnumber{\int{ligaturing}}<{0}\then \skipslots{1}\Else

   \setslot{\lclig{IJ}{ij}}
      \comment{The letter `ij'.  This is a single letter, and in a 
        monowidth font should ideally be one letter wide.}
   \endsetslot

\Fi

\setslot{exclamdown}
   \comment{The Spanish punctuation mark `!`'.}
\endsetslot

\setslot{questiondown}
   \comment{The Spanish punctuation mark `?`'.}
\endsetslot

\setslot{sterling}
   \comment{The British currency mark `\textsterling'.}
\endsetslot

\setslot{\uctop{Agrave}{agrave}}
   \comment{The letter `\` A'.}
\endsetslot

\setslot{\uctop{Aacute}{aacute}}
   \comment{The letter `\' A'.}
\endsetslot

\setslot{\uctop{Acircumflex}{acircumflex}}
   \comment{The letter `\^ A'.}
\endsetslot

\setslot{\uctop{Atilde}{atilde}}
   \comment{The letter `\~ A'.}
\endsetslot

\setslot{\uctop{Adieresis}{adieresis}}
   \comment{The letter `\" A'.}
\endsetslot

\setslot{\uctop{Aring}{aring}}
   \comment{The letter `\r A'.}
\endsetslot

\setslot{\uc{AE}{ae}}
   \comment{The letter `\AE'.  This is a single letter, and should not be
      faked with `AE'.}
\endsetslot

\setslot{\uc{Ccedilla}{ccedilla}}
   \comment{The letter `\c C'.}
\endsetslot

\setslot{\uctop{Egrave}{egrave}}
   \comment{The letter `\` E'.}
\endsetslot

\setslot{\uctop{Eacute}{eacute}}
   \comment{The letter `\' E'.}
\endsetslot

\setslot{\uctop{Ecircumflex}{ecircumflex}}
   \comment{The letter `\^ E'.}
\endsetslot

\setslot{\uctop{Edieresis}{edieresis}}
   \comment{The letter `\" E'.}
\endsetslot

\setslot{\uctop{Igrave}{igrave}}
   \comment{The letter `\` I'.}
\endsetslot

\setslot{\uctop{Iacute}{iacute}}
   \comment{The letter `\' I'.}
\endsetslot

\setslot{\uctop{Icircumflex}{icircumflex}}
   \comment{The letter `\^ I'.}
\endsetslot

\setslot{\uctop{Idieresis}{idieresis}}
   \comment{The letter `\" I'.}
\endsetslot

\setslot{\uc{Eth}{eth}}
   \comment{The uppercase Icelandic letter `Eth' similar to a `D'
      with a horizontal bar through the stem.  It is unavailable
      in \plain\ \TeX.}
\endsetslot

\setslot{\uctop{Ntilde}{ntilde}}
   \comment{The letter `\~ N'.}
\endsetslot

\setslot{\uctop{Ograve}{ograve}}
   \comment{The letter `\` O'.}
\endsetslot

\setslot{\uctop{Oacute}{oacute}}
   \comment{The letter `\' O'.}
\endsetslot

\setslot{\uctop{Ocircumflex}{ocircumflex}}
   \comment{The letter `\^ O'.}
\endsetslot

\setslot{\uctop{Otilde}{otilde}}
   \comment{The letter `\~ O'.}
\endsetslot

\setslot{\uctop{Odieresis}{odieresis}}
   \comment{The letter `\" O'.}
\endsetslot

\setslot{\uc{OE}{oe}}
   \comment{The letter `\OE'.  This is a single letter, and should not be
      faked with `OE'.}
\endsetslot

\setslot{\uc{Oslash}{oslash}}
   \comment{The letter `\O'.}
\endsetslot

\setslot{\uctop{Ugrave}{ugrave}}
   \comment{The letter `\` U'.}
\endsetslot

\setslot{\uctop{Uacute}{uacute}}
   \comment{The letter `\' U'.}
\endsetslot

\setslot{\uctop{Ucircumflex}{ucircumflex}}
   \comment{The letter `\^ U'.}
\endsetslot

\setslot{\uctop{Udieresis}{udieresis}}
   \comment{The letter `\" U'.}
\endsetslot

\setslot{\uctop{Yacute}{yacute}}
   \comment{The letter `\' Y'.}
\endsetslot

\setslot{\uc{Thorn}{thorn}}
   \comment{The Icelandic capital letter Thorn, similar to a `P'
      with the bowl moved down.  It is unavailable in \plain\ \TeX.}
\endsetslot

\setslot{\uclig{SS}{germandbls}}
   \comment{The ligature `SS', used to give an upper case `\ss'.
      In a monowidth font it should be two letters wide.}
\endsetslot

\setslot{\lctop{Agrave}{agrave}}
   \comment{The letter `\` a'.}
\endsetslot

\setslot{\lctop{Aacute}{aacute}}
   \comment{The letter `\' a'.}
\endsetslot

\setslot{\lctop{Acircumflex}{acircumflex}}
   \comment{The letter `\^ a'.}
\endsetslot

\setslot{\lctop{Atilde}{atilde}}
   \comment{The letter `\~ a'.}
\endsetslot

\setslot{\lctop{Adieresis}{adieresis}}
   \comment{The letter `\" a'.}
\endsetslot

\setslot{\lctop{Aring}{aring}}
   \comment{The letter `\r a'.}
\endsetslot

\setslot{\lc{AE}{ae}}
   \comment{The letter `\ae'.  This is a single letter, and should not be
      faked with `ae'.}
\endsetslot

\setslot{\lc{Ccedilla}{ccedilla}}
   \comment{The letter `\c c'.}
\endsetslot

\setslot{\lctop{Egrave}{egrave}}
   \comment{The letter `\` e'.}
\endsetslot

\setslot{\lctop{Eacute}{eacute}}
   \comment{The letter `\' e'.}
\endsetslot

\setslot{\lctop{Ecircumflex}{ecircumflex}}
   \comment{The letter `\^ e'.}
\endsetslot

\setslot{\lctop{Edieresis}{edieresis}}
   \comment{The letter `\" e'.}
\endsetslot

\setslot{\lctop{Igrave}{igrave}}
   \comment{The letter `\`\i'.}
\endsetslot

\setslot{\lctop{Iacute}{iacute}}
   \comment{The letter `\'\i'.}
\endsetslot

\setslot{\lctop{Icircumflex}{icircumflex}}
   \comment{The letter `\^\i'.}
\endsetslot

\setslot{\lctop{Idieresis}{idieresis}}
   \comment{The letter `\"\i'.}
\endsetslot

\setslot{\lc{Eth}{eth}}
   \comment{The Icelandic lowercase letter `eth' similar to
     a `$\partial$' with an oblique bar through the stem.
     It is unavailable in \plain\ \TeX.}
\endsetslot

\setslot{\lctop{Ntilde}{ntilde}}
   \comment{The letter `\~ n'.}
\endsetslot

\setslot{\lctop{Ograve}{ograve}}
   \comment{The letter `\` o'.}
\endsetslot

\setslot{\lctop{Oacute}{oacute}}
   \comment{The letter `\' o'.}
\endsetslot

\setslot{\lctop{Ocircumflex}{ocircumflex}}
   \comment{The letter `\^ o'.}
\endsetslot

\setslot{\lctop{Otilde}{otilde}}
   \comment{The letter `\~ o'.}
\endsetslot

\setslot{\lctop{Odieresis}{odieresis}}
   \comment{The letter `\" o'.}
\endsetslot

\setslot{\lc{OE}{oe}}
   \comment{The letter `\oe'.  This is a single letter, and should not be
      faked with `oe'.}
\endsetslot

\setslot{\lc{Oslash}{oslash}}
   \comment{The letter `\o'.}
\endsetslot

\setslot{\lctop{Ugrave}{ugrave}}
   \comment{The letter `\` u'.}
\endsetslot

\setslot{\lctop{Uacute}{uacute}}
   \comment{The letter `\' u'.}
\endsetslot

\setslot{\lctop{Ucircumflex}{ucircumflex}}
   \comment{The letter `\^ u'.}
\endsetslot

\setslot{\lctop{Udieresis}{udieresis}}
   \comment{The letter `\" u'.}
\endsetslot

\setslot{\lctop{Yacute}{yacute}}
   \comment{The letter `\' y'.}
\endsetslot

\setslot{\lc{Thorn}{thorn}}
   \comment{The Icelandic lowercase letter `thorn', similar to a `p'
      with an ascender rising from the stem.  It is unavailable
      in \plain\ \TeX.}
\endsetslot

\setslot{\lc{SS}{germandbls}}
   \comment{The letter `\ss'.}
\endsetslot

\endencoding
%    \end{macrocode}
% \end{encoding}
% \iffalse
%</t1-f-f>
% \fi
% 
% 
% \subsubsection{fontscripts-t1j-f\_f.etx}\label{subsubsec:t1j-f-f}
% 
% \iffalse
%<*t1j-f-f>
% \fi
% \begin{encoding}{fontscripts-t1j-f_f.etx}
% \changes{v0.0}{2025-02-10}{Filename prefix for Karl.}
%    \begin{macrocode}
\relax

%    \end{macrocode}
% \file{fontscripts-t1j-f\_f.etx} -- install a T1-encoded roman font with oldstyle 
% and f-ligatures named "f\_f" etc.
%
% We do not need to distinguish between roman and italic in T1,
% hence we simply call fontscripts-t1-f\_f.etx with oldstyle parameters.
%    \begin{macrocode}

\encoding

\setcommand\lc#1#2{#2}
\setcommand\uc#1#2{#1}
\setcommand\lctop#1#2{#2}
\setcommand\uctop#1#2{#1}
\setcommand\lclig#1#2{#2}
\setcommand\uclig#1#2{#1}
\setcommand\digit#1{#1oldstyle}

\inputetx{t1-f_f}

\endencoding
%    \end{macrocode}
% \end{encoding}
% \iffalse
%</t1j-f-f>
% \fi
% 
% 
% \subsubsection{fontscripts-ts1-dotinf.etx}\label{subsubsec:ts1-dotinf}
% 
% \iffalse
%<*ts1-dotinf>
% \fi
% \begin{encoding}{fontscripts-ts1-dotinf.etx}
% \changes{v0.0}{2025-02-10}{Filename prefix for Karl.}
%    \begin{macrocode}
%%
%% - The commentary in the original is deleted in this version. For 
%% information about the TS1 etc., typeset the original ts1.etx 
%% included with fontinst.
%% - The original notices at the top of that file concerning authors,
%% maintenance etc. are replaced by this notice.
%% - The file is renamed.
%% - The encoding name is modified.
%% - The file is modified to accommodate differences in glyph names.
%% - The file may be modified for use in encoding other characters.
%%
%%%%%%%%%%%%%%%%%%%%%%%%%%%%%%%%%%%%%%%%%%%%%%%%%
\relax

\encoding

\setstr{codingscheme}{TEX TEXT COMPANION 1---TS1 DOTINF}

\ifisglyph{x}\then
   \setint{xheight}{\height{x}}
\else
   \setint{xheight}{500}
\fi

\ifisglyph{space}\then
   \setint{interword}{\width{space}}
\else\ifisglyph{i}\then
   \setint{interword}{\width{i}}
\else
   \setint{interword}{333}
\fi\fi

\setint{italicslant}{0}


\setint{fontdimen(1)}{\int{italicslant}}              % italic slant
\setint{fontdimen(2)}{\int{interword}}                % interword space
\setint{fontdimen(3)}{0}                              % interword stretch
\setint{fontdimen(4)}{0}                              % interword shrink
\setint{fontdimen(5)}{\int{xheight}}                  % x-height
\setint{fontdimen(6)}{1000}                           % quad
\setint{fontdimen(7)}{\int{interword}}                % extra space after .

\nextslot{0}
\setslot{capitalgrave.inferior}
   \comment{The grave accent `\capitalgrave{}', intended for use with
      capital letters.}
\endsetslot

\setslot{capitalacute.inferior}
   \comment{The acute accent `\capitalacute{}', intended for use with
      capital letters.}
\endsetslot

\setslot{capitalcircumflex.inferior}
   \comment{The circumflex accent `\capitalcircumflex{}', intended for
      use with capital letters.}
\endsetslot

\setslot{capitaltilde.inferior}
   \comment{The tilde accent `\capitaltilde{}', intended for use with
      capital letters.}
\endsetslot

\setslot{capitaldieresis.inferior}
   \comment{The umlaut or dieresis accent `\capitaldieresis{}',
      intended for use with capital letters.}
\endsetslot

\setslot{capitalhungarumlaut.inferior}
   \comment{The long Hungarian umlaut `\capitalhungarumlaut{}',
      intended for use with capital letters.}
\endsetslot

\setslot{capitalring.inferior}
   \comment{The ring accent `\capitalring{}', intended for use with
      capital letters.}
\endsetslot

\setslot{capitalcaron.inferior}
   \comment{The caron or h\'a\v cek accent `\capitalcaron{}', intended
      for use with capital letters.}
\endsetslot

\setslot{capitalbreve.inferior}
   \comment{The breve accent `\capitalbreve{}', intended for use with
      capital letters.}
\endsetslot

\setslot{capitalmacron.inferior}
   \comment{The macron accent `\capitalmacron{}', intended for use with
      capital letters.}
\endsetslot

\setslot{capitaldotaccent.inferior}
   \comment{The dot accent `\capitaldotaccent{}', intended for use with
      capital letters.}
\endsetslot

\setslot{cedilla.inferior}
   \comment{The cedilla accent `\capitalcedilla{}', intended for use
      with capital letters.}
\endsetslot

\setslot{ogonek.inferior}
   \comment{The ogonek accent `\capitalogonek{}', intended for use with
      capital letters.}
\endsetslot

\nextslot{13}
\setslot{quotesinglbase.inferior}
   \comment{A straight single quote mark on the baseline,
      `\textquotestraightbase'.}
\endsetslot

\nextslot{18}
\setslot{quotedblbase.inferior}
   \comment{A straight double quote mark on the baseline,
      `\textquotestraightdblbase'.}
\endsetslot

\nextslot{21}
\setslot{twelveudash.inferior}
   \comment{A 2/3~em dash, `\texttwelveudash'.}
\endsetslot

\setslot{threequartersemdash.inferior}
   \comment{A 3/4~em dash, `\textthreequartersemdash'.}
\endsetslot

\nextslot{23}
\setslot{capitalcompwordmark.inferior}
    \comment{An invisible glyph, with zero width and depth, but the
      height of capital letters.
      It is used to stop ligaturing in words like `shelf{}ful'.}
\endsetslot

\nextslot{24}
\setslot{arrowleft.inferior}
   \comment{A left pointing arrow, `\textleftarrow', unavailable in
      most PostScript fonts.}
\endsetslot

\setslot{arrowright.inferior}
   \comment{A right pointing arrow, `\textrightarrow', unavailable in
      most PostScript fonts.}
\endsetslot

\nextslot{26}
\setslot{tieaccentlowercase.inferior}
   \comment{The original tie accent `\t{}', intended for use with
      lowercase letters.}
\endsetslot

\setslot{tieaccentcapital.inferior}
   \comment{The tie accent `\capitaltie{}', intended for use with
      capital letters.}
\endsetslot

\setslot{newtieaccentlowercase.inferior}
   \comment{A new tie accent `\newtie{}', intended for use with
      lowercase letters.}
\endsetslot

\setslot{newtieaccentcapital.inferior}
   \comment{A new tie accent `\capitalnewtie{}', intended for use
      with capital letters.}
\endsetslot

\nextslot{31}
\setslot{ascendercompwordmark.inferior}
    \comment{An invisible glyph, with zero width and depth, but the
      height of lowercase letters with ascenders.
      It is used to stop ligaturing in words like `shelf{}ful'.}
\endsetslot

\nextslot{32}
\setslot{blank.inferior}
   \comment{The blank indicator `\textblank', similar to the letter `b'
      with an oblique bar throgh the stem.}
\endsetslot

\nextslot{36}
\setslot{dollar.inferior}
   \comment{The dollar sign `\textdollar'.}
\endsetslot

\nextslot{39}
\setslot{quotesingle.inferior}
   \comment{A straight single quote mark, `\textquotesingle'.}
\endsetslot

\nextslot{42}
\setslot{asteriskcentered.inferior}
   \comment{The centered asterisk, `\textasteriskcentered'.}
\endsetslot

\nextslot{44}
\setslot{comma.inferior}
   \comment{The decimal comma `,'.}
\endsetslot

\nextslot{45}
\setslot{hyphendbl.inferior}
   \comment{An alternate double hyphen, `\textdblhyphen'.}
\endsetslot

\nextslot{46}
\setslot{period.inferior}
   \comment{The decimal point `.'.}
\endsetslot

\nextslot{47}
\setslot{fraction.inferior}
   \comment{The fraction slash `\textfractionsolidus'.}
\endsetslot

\nextslot{48}
\setslot{zerooldstyle.inferior}
   \comment{The oldstyle number `\oldstylenums{0}'.}
\endsetslot

\setslot{oneoldstyle.inferior}
   \comment{The oldstyle number `\oldstylenums{1}'.}
\endsetslot

\setslot{twooldstyle.inferior}
   \comment{The oldstyle number `\oldstylenums{2}'.}
\endsetslot

\setslot{threeoldstyle.inferior}
   \comment{The oldstyle number `\oldstylenums{3}'.}
\endsetslot

\setslot{fouroldstyle.inferior}
   \comment{The oldstyle number `\oldstylenums{4}'.}
\endsetslot

\setslot{fiveoldstyle.inferior}
   \comment{The oldstyle number `\oldstylenums{5}'.}
\endsetslot

\setslot{sixoldstyle.inferior}
   \comment{The oldstyle number `\oldstylenums{6}'.}
\endsetslot

\setslot{sevenoldstyle.inferior}
   \comment{The oldstyle number `\oldstylenums{7}'.}
\endsetslot

\setslot{eightoldstyle.inferior}
   \comment{The oldstyle number `\oldstylenums{8}'.}
\endsetslot

\setslot{nineoldstyle.inferior}
   \comment{The oldstyle number `\oldstylenums{9}'.}
\endsetslot

\nextslot{60}
\setslot{angbracketleft.inferior}
   \comment{The opening angle bracket `\textlangle', unavailable in
      most PostScript fonts.}
\endsetslot

\nextslot{61}
\setslot{minus.inferior}
   \comment{The subtraction sign `\textminus'.}
\endsetslot

\nextslot{62}
\setslot{angbracketright.inferior}
   \comment{The closing angle bracket `\textrangle', unavailable in
      most PostScript fonts.}
\endsetslot

\nextslot{77}
\setslot{Omegainv.inferior}
   \comment{The inverted Ohm sign `\textmho', unavailable in most fonts.}
\endsetslot

\nextslot{79}
   \comment{A circle `\textbigcircle', big enough to enclose a letter
      as in `\textcopyright' or `\textregistered'.}
\setslot{bigcircle.inferior}
\endsetslot

\nextslot{87}
\setslot{Omega.inferior}
   \comment{The upright Ohm sign `\textohm', unavailable in most fonts.
      Even if it is available in Mac-encoded fonts, it isn't directly
      accessible in the 8r or 8y encodings.}
\endsetslot

\nextslot{91}
\setslot{openbracketleft.inferior}
   \comment{The opening double square bracket `\textlbrackdbl',
      unavailable in most PostScript fonts.}
\endsetslot

\nextslot{93}
\setslot{openbracketright.inferior}
   \comment{The closing double square bracket `\textrbrackdbl',
      unavailable in most PostScript fonts.}
\endsetslot

\nextslot{94}
\setslot{arrowup.inferior}
   \comment{An upwards pointing arrow `\textuparrow', unavailable in
      most PostScript fonts.}
\endsetslot

\nextslot{95}
\setslot{arrowdown.inferior}
   \comment{An downwards pointing arrow `\textdownarrow', unavailable
      in most PostScript fonts.}
\endsetslot

\nextslot{96}
\setslot{asciigrave.inferior}
   \comment{An ASCII-style grave `\textasciigrave'. This is supposed
      to be a character by itself rather than a combining accents.}
\endsetslot

\nextslot{98}
\setslot{born.inferior}
   \comment{The born symbol `\textborn', unavailable in most PostScript
      fonts.}
\endsetslot

\nextslot{99}
\setslot{divorced.inferior}
   \comment{The divorced symbol `\textdivorced', unavailable in most
      PostScript fonts.}
\endsetslot

\nextslot{100}
\setslot{died.inferior}
   \comment{The died symbol `\textdied', unavailable in most PostScript
      fonts.}
\endsetslot

\nextslot{108}
\setslot{leaf.inferior}
   \comment{The leaf symbol `\textleaf', unavailable in most PostScript
      fonts.}
\endsetslot

\nextslot{109}
\setslot{married.inferior}
   \comment{The married symbol `\textmarried', unavailable in most
      PostScript  fonts.}
\endsetslot

\nextslot{110}
\setslot{musicalnote.inferior}
   \comment{A musical note symbol `\textmusicalnote', unavailable in
      most PostScript fonts.}
\endsetslot

\nextslot{126}
\setslot{tildelow.inferior}
   \comment{A lowered tilde `\texttildelow'.  In most PostScript fonts
      it can be substituted by `asciitilde', while `\textasciitilde'
      is supposed to be a raised `tilde'.}
\endsetslot

\nextslot{127}
\setslot{hyphendblchar.inferior}
    \comment{The glyph used for hyphenation in this font, which will
      almost always be the same as `hyphendbl'.}
\endsetslot

\nextslot{128}
\setslot{asciibreve.inferior}
   \comment{An ASCII-style breve `\textasciibreve'. This is supposed
      to be a character by itself rather than a combining accents.}
\endsetslot

\setslot{asciicaron.inferior}
   \comment{An ASCII-style caron `\textasciicaron'. This is supposed
      to be a character by itself rather than a combining accents.}
\endsetslot

\setslot{asciiacutedbl.inferior}
   \comment{An ASCII-style double tick mark, `\textacutedbl'.}
\endsetslot

\setslot{asciigravedbl.inferior}
   \comment{An ASCII-style double backtick mark, `\textgravedbl'.}
\endsetslot

\setslot{dagger.inferior}
   \comment{The single dagger `\textdagger'.}
\endsetslot

\setslot{daggerdbl.inferior}
   \comment{The double dagger `\textdaggerdbl'.}
\endsetslot

\setslot{bardbl.inferior}
   \comment{The double vertical bar `\textbardbl'.}
\endsetslot

\setslot{perthousand.inferior}
   \comment{The perthousand sign `\textperthousand'.}
\endsetslot

\setslot{bullet.inferior}
   \comment{The centered bullet `\textbullet'.}
\endsetslot

\setslot{centigrade.inferior}
   \comment{The degree centigrade symbol `\textcelsius'.}
\endsetslot

\setslot{dollaroldstyle.inferior}
   \comment{An oldstyle dollar sign `\textdollaroldstyle'.}
\endsetslot

\setslot{centoldstyle.inferior}
   \comment{An oldstyle cent sign `\textcentoldstyle'.}
\endsetslot

\setslot{florin.inferior}
   \comment{The florin sign `\textflorin'.}
\endsetslot

\setslot{colonmonetary.inferior}
   \comment{The Colon currency sign `\textcolonmonetary', similar to
      a capital `C' with a vertical bar through the middle.}
\endsetslot

\setslot{won.inferior}
   \comment{The Won currency sign `\textwon', similar to a capital `W'
      with two horizontal bars.}
\endsetslot

\setslot{naira.inferior}
   \comment{The Naira currency sign `\textnaira', similar to a
      capital `N' with two horizontal bars.}
\endsetslot

\setslot{guarani.inferior}
   \comment{The Guarani currency sign `\textguarani',  similar to
      a capital `G' with a vertical bar through the middle.}
\endsetslot

\setslot{peso.inferior}
   \comment{The Peso currency sign `\textpeso', similar to a capital `P'
      with a horizontal bar through the bowl or below the bowl.}
\endsetslot

\setslot{lira.inferior}
   \comment{The Lira currency sign `\textlira', similar to a sterling
      sign `\textsterling' with two horizontal bars.}
\endsetslot

\setslot{recipe.inferior}
   \comment{The recipe symbol `\textrecipe', similar to a capital `R'
      with an oblique bar through the tail.}
\endsetslot

\setslot{interrobang.inferior}
   \comment{The interrobang symbol `\textinterrobang', similar to
      a combination of an exclamation mark and a question mark.}
\endsetslot

\setslot{interrobangdown.inferior}
   \comment{The inverted interrobang symbol `\textinterrobangdown',
      similar to a combination of an inverted exclamation mark
      and an inverted question mark.}
\endsetslot

\setslot{dong.inferior}
   \comment{The Dong currency sign `\textdong', similar to a lowercase
      `d'  with a horizontal bar through the stem and another bar below
      the letter.}
\endsetslot

\setslot{trademark.inferior}
   \comment{The trademark sign `\texttrademark', similar to the raised
     letters `TM'.}
\endsetslot

\setslot{pertenthousand.inferior}
   \comment{The pertenthousand sign `\textpertenthousand', unavailable
     in most PostScript fonts.}
\endsetslot

\setslot{pilcrow.inferior}
   \comment{The pilcrow mark `\textpilcrow', similar to a paragraph
      mark `\textparagraph' with a single stem.}
\endsetslot

\setslot{baht.inferior}
   \comment{The Baht currency sign `\textbaht', similar to a capital `B'
      with a vertical bar through the middle.}
\endsetslot

\setslot{numero.inferior}
   \comment{The numero sign `\textnumero', similar to the letter `N'
      with a raised `o', unavailable in most PostScript fonts.}
\endsetslot

\setslot{discount.inferior}
   \comment{The discount sign `\textdiscount', similar to a stylized
      percent sign, unavailable in most PostScript fonts.}
\endsetslot

\setslot{estimated.inferior}
   \comment{The estimated sign `\textestimated', similar to an enlarged
      lowercase `e', unavailable in most PostScript fonts.}
\endsetslot

\setslot{openbullet.inferior}
   \comment{The centered open bullet `\textopenbullet'', unavailable
      in most PostScript fonts.}
\endsetslot

\setslot{servicemark.inferior}
   \comment{The service mark sign `\textservicemark', similar to the
      raised letters `SM', unavailable in most PostScript fonts.}
\endsetslot

\nextslot{160}
\setslot{quillbracketleft.inferior}
   \comment{The opening quill bracket `\textlquill', unavailable in
      most PostScript fonts.}
\endsetslot

\setslot{quillbracketright.inferior}
   \comment{The closing quill bracket `\textrquill', unavailable in
      most PostScript fonts.}
\endsetslot

\setslot{cent.inferior}
   \comment{The cent sign `\textcent'.}
\endsetslot

\setslot{sterling.inferior}
   \comment{The British currency sign, `\textsterling'.}
\endsetslot

\setslot{currency.inferior}
   \comment{The international currency sign, `\textcurrency'.}
\endsetslot

\setslot{yen.inferior}
   \comment{The Japanese currency sign, `\textyen'.}
\endsetslot

\setslot{brokenbar.inferior}
   \comment{A broken vertical bar, `\textbrokenbar', similar to
      `\textbar' with a gap through the middle.}
\endsetslot

\setslot{section.inferior}
   \comment{The section mark `\textsection'.}
\endsetslot

\setslot{asciidieresis.inferior}
   \comment{An ASCII-style dieresis `\textasciidieresis'. This is
       supposed to be character by itself  rather than an accents.}
\endsetslot

\setslot{copyright.inferior}
   \comment{The copyright sign `\textcopyright',  similar to a small
       letter `C' enclosed by a circle.}
\endsetslot

\setslot{ordfeminine.inferior}
   \comment{The raised letter `\textordfeminine'.}
\endsetslot

\setslot{copyleft.inferior}
   \comment{The reversed copyright sign `\textcopyleft', similar to
      a small reversed `C' enclosed by a circle.}
\endsetslot

\setslot{logicalnot.inferior}
   \comment{The logical not sign `\textlnot'.}
\endsetslot

\setslot{circledP.inferior}
   \comment{A small letter `P' enclosed by a circle, `\textcircledP',
      unavailable in most fonts.}
\endsetslot

\setslot{registered.inferior}
   \comment{The registered trademark sign `\textregistered', similar to
      a small letter `R' enclosed by a circle.}
\endsetslot

\setslot{asciimacron.inferior}
   \comment{An ASCII-style macron `\textasciimacron'. This is supposed
      to be a character by itself rather than a combining accents.}
\endsetslot

\setslot{degree.inferior}
   \comment{The degree sign `\textdegree'.}
\endsetslot

\setslot{plusminus.inferior}
   \comment{The plus or minus sign `\textpm'.}
\endsetslot

\setslot{two.superior}
   \comment{The raised digit `\texttwosuperior'.}
\endsetslot

\setslot{three.superior}
   \comment{The raised digit `\textthreesuperior'.}
\endsetslot

\setslot{asciiacute.inferior}
   \comment{An ASCII-style acute `\textasciiacute'. This is supposed
      to be a character by itself rather than a combining accents.}
\endsetslot

\setslot{mu.inferior}
   \comment{The lowercase Greek letter `\textmu', intended  for use as
      a prefix `micro' in physical units.}
\endsetslot

\setslot{paragraph.inferior}
   \comment{The paragraph mark `\textparagraph'.}
\endsetslot

\setslot{periodcentered.inferior}
   \comment{The centered period `\textperiodcentered'.}
\endsetslot

\setslot{referencemark.inferior}
   \comment{The reference mark `\textreferencemark', similar to
      a combination of the `multiply' and `divide' symbols.}
\endsetslot

\setslot{one.superior}
   \comment{The raised digit `\textonesuperior'.}
\endsetslot

\setslot{ordmasculine.inferior}
   \comment{The raised letter `\textordmasculine'.}
\endsetslot

\setslot{radical.inferior}
   \comment{The radical sign `\textsurd', unavailable in most fonts.
      Even if it is available in Mac-encoded fonts, it isn't directly
      accessible in the 8r or 8y encodings.}
\endsetslot

\setslot{onequarter.inferior}
   \comment{The fraction `\textonequarter'.}
\endsetslot

\setslot{onehalf.inferior}
   \comment{The fraction `\textonehalf'.}
\endsetslot

\setslot{threequarters.inferior}
   \comment{The fraction `\textthreequarters'.}
\endsetslot

\setslot{Euro.inferior}
   \comment{The European currency sign, similar to `\texteuro'.}
\endsetslot


\nextslot{214}
\setslot{multiply.inferior}
   \comment{The multiplication sign `\texttimes'.
      This symbol was originally intended to be put into slot~215,
      but ended up in this slot by mistake, at which time it was
      considered too late to change it.}
\endsetslot

\nextslot{246}
\setslot{divide.inferior}
   \comment{The divison sign `\textdiv'.
      This symbol was originally intended to be put into slot~247,
      but ended up in this slot by mistake, at which time it was
      onsidered too late to change it.}
\endsetslot

\endencoding

%    \end{macrocode}
% \end{encoding}
% \iffalse
%</ts1-dotinf>
% \fi
% 
% 
% \subsubsection{fontscripts-ts1-dotsup.etx}\label{subsubsec:ts1-dotsup}
% 
% \iffalse
%<*ts1-dotsup>
% \fi
% \begin{encoding}{fontscripts-ts1-dotsup.etx}
% \changes{v0.0}{2025-02-10}{Filename prefix for Karl.}
%    \begin{macrocode}
%%
%% - The commentary in the original is deleted in this version. For 
%% information about the TS1 etc., typeset the original ts1.etx 
%% included with fontinst.
%% - The original notices at the top of that file concerning authors,
%% maintenance etc. are replaced by this notice.
%% - The file is renamed.
%% - The encoding name is modified.
%% - The file is modified to accommodate differences in glyph names.
%% - The file may be modified for use in encoding other characters.
%%
%%%%%%%%%%%%%%%%%%%%%%%%%%%%%%%%%%%%%%%%%%%%%%%%%
\relax

\encoding

\setstr{codingscheme}{TEX TEXT COMPANION 1---TS1 DOTSUP}

\ifisglyph{x}\then
   \setint{xheight}{\height{x}}
\else
   \setint{xheight}{500}
\fi

\ifisglyph{space}\then
   \setint{interword}{\width{space}}
\else\ifisglyph{i}\then
   \setint{interword}{\width{i}}
\else
   \setint{interword}{333}
\fi\fi

\setint{italicslant}{0}


\setint{fontdimen(1)}{\int{italicslant}}              % italic slant
\setint{fontdimen(2)}{\int{interword}}                % interword space
\setint{fontdimen(3)}{0}                              % interword stretch
\setint{fontdimen(4)}{0}                              % interword shrink
\setint{fontdimen(5)}{\int{xheight}}                  % x-height
\setint{fontdimen(6)}{1000}                           % quad
\setint{fontdimen(7)}{\int{interword}}                % extra space after .

\nextslot{0}
\setslot{capitalgrave.superior}
   \comment{The grave accent `\capitalgrave{}', intended for use with
      capital letters.}
\endsetslot

\setslot{capitalacute.superior}
   \comment{The acute accent `\capitalacute{}', intended for use with
      capital letters.}
\endsetslot

\setslot{capitalcircumflex.superior}
   \comment{The circumflex accent `\capitalcircumflex{}', intended for
      use with capital letters.}
\endsetslot

\setslot{capitaltilde.superior}
   \comment{The tilde accent `\capitaltilde{}', intended for use with
      capital letters.}
\endsetslot

\setslot{capitaldieresis.superior}
   \comment{The umlaut or dieresis accent `\capitaldieresis{}',
      intended for use with capital letters.}
\endsetslot

\setslot{capitalhungarumlaut.superior}
   \comment{The long Hungarian umlaut `\capitalhungarumlaut{}',
      intended for use with capital letters.}
\endsetslot

\setslot{capitalring.superior}
   \comment{The ring accent `\capitalring{}', intended for use with
      capital letters.}
\endsetslot

\setslot{capitalcaron.superior}
   \comment{The caron or h\'a\v cek accent `\capitalcaron{}', intended
      for use with capital letters.}
\endsetslot

\setslot{capitalbreve.superior}
   \comment{The breve accent `\capitalbreve{}', intended for use with
      capital letters.}
\endsetslot

\setslot{capitalmacron.superior}
   \comment{The macron accent `\capitalmacron{}', intended for use with
      capital letters.}
\endsetslot

\setslot{capitaldotaccent.superior}
   \comment{The dot accent `\capitaldotaccent{}', intended for use with
      capital letters.}
\endsetslot

\setslot{cedilla.superior}
   \comment{The cedilla accent `\capitalcedilla{}', intended for use
      with capital letters.}
\endsetslot

\setslot{ogonek.superior}
   \comment{The ogonek accent `\capitalogonek{}', intended for use with
      capital letters.}
\endsetslot

\nextslot{13}
\setslot{quotesinglbase.superior}
   \comment{A straight single quote mark on the baseline,
      `\textquotestraightbase'.}
\endsetslot

\nextslot{18}
\setslot{quotedblbase.superior}
   \comment{A straight double quote mark on the baseline,
      `\textquotestraightdblbase'.}
\endsetslot

\nextslot{21}
\setslot{twelveudash.superior}
   \comment{A 2/3~em dash, `\texttwelveudash'.}
\endsetslot

\setslot{threequartersemdash.superior}
   \comment{A 3/4~em dash, `\textthreequartersemdash'.}
\endsetslot

\nextslot{23}
\setslot{capitalcompwordmark.superior}
    \comment{An invisible glyph, with zero width and depth, but the
      height of capital letters.
      It is used to stop ligaturing in words like `shelf{}ful'.}
\endsetslot

\nextslot{24}
\setslot{arrowleft.superior}
   \comment{A left pointing arrow, `\textleftarrow', unavailable in
      most PostScript fonts.}
\endsetslot

\setslot{arrowright.superior}
   \comment{A right pointing arrow, `\textrightarrow', unavailable in
      most PostScript fonts.}
\endsetslot

\nextslot{26}
\setslot{tieaccentlowercase.superior}
   \comment{The original tie accent `\t{}', intended for use with
      lowercase letters.}
\endsetslot

\setslot{tieaccentcapital.superior}
   \comment{The tie accent `\capitaltie{}', intended for use with
      capital letters.}
\endsetslot

\setslot{newtieaccentlowercase.superior}
   \comment{A new tie accent `\newtie{}', intended for use with
      lowercase letters.}
\endsetslot

\setslot{newtieaccentcapital.superior}
   \comment{A new tie accent `\capitalnewtie{}', intended for use
      with capital letters.}
\endsetslot

\nextslot{31}
\setslot{ascendercompwordmark.superior}
    \comment{An invisible glyph, with zero width and depth, but the
      height of lowercase letters with ascenders.
      It is used to stop ligaturing in words like `shelf{}ful'.}
\endsetslot

\nextslot{32}
\setslot{blank.superior}
   \comment{The blank indicator `\textblank', similar to the letter `b'
      with an oblique bar throgh the stem.}
\endsetslot

\nextslot{36}
\setslot{dollar.superior}
   \comment{The dollar sign `\textdollar'.}
\endsetslot

\nextslot{39}
\setslot{quotesingle.superior}
   \comment{A straight single quote mark, `\textquotesingle'.}
\endsetslot

\nextslot{42}
\setslot{asteriskcentered.superior}
   \comment{The centered asterisk, `\textasteriskcentered'.}
\endsetslot

\nextslot{44}
\setslot{comma.superior}
   \comment{The decimal comma `,'.}
\endsetslot

\nextslot{45}
\setslot{hyphendbl.superior}
   \comment{An alternate double hyphen, `\textdblhyphen'.}
\endsetslot

\nextslot{46}
\setslot{period.superior}
   \comment{The decimal point `.'.}
\endsetslot

\nextslot{47}
\setslot{fraction.superior}
   \comment{The fraction slash `\textfractionsolidus'.}
\endsetslot

\nextslot{48}
\setslot{zerooldstyle.superior}
   \comment{The oldstyle number `\oldstylenums{0}'.}
\endsetslot

\setslot{oneoldstyle.superior}
   \comment{The oldstyle number `\oldstylenums{1}'.}
\endsetslot

\setslot{twooldstyle.superior}
   \comment{The oldstyle number `\oldstylenums{2}'.}
\endsetslot

\setslot{threeoldstyle.superior}
   \comment{The oldstyle number `\oldstylenums{3}'.}
\endsetslot

\setslot{fouroldstyle.superior}
   \comment{The oldstyle number `\oldstylenums{4}'.}
\endsetslot

\setslot{fiveoldstyle.superior}
   \comment{The oldstyle number `\oldstylenums{5}'.}
\endsetslot

\setslot{sixoldstyle.superior}
   \comment{The oldstyle number `\oldstylenums{6}'.}
\endsetslot

\setslot{sevenoldstyle.superior}
   \comment{The oldstyle number `\oldstylenums{7}'.}
\endsetslot

\setslot{eightoldstyle.superior}
   \comment{The oldstyle number `\oldstylenums{8}'.}
\endsetslot

\setslot{nineoldstyle.superior}
   \comment{The oldstyle number `\oldstylenums{9}'.}
\endsetslot

\nextslot{60}
\setslot{angbracketleft.superior}
   \comment{The opening angle bracket `\textlangle', unavailable in
      most PostScript fonts.}
\endsetslot

\nextslot{61}
\setslot{minus.superior}
   \comment{The subtraction sign `\textminus'.}
\endsetslot

\nextslot{62}
\setslot{angbracketright.superior}
   \comment{The closing angle bracket `\textrangle', unavailable in
      most PostScript fonts.}
\endsetslot

\nextslot{77}
\setslot{Omegainv.superior}
   \comment{The inverted Ohm sign `\textmho', unavailable in most fonts.}
\endsetslot

\nextslot{79}
   \comment{A circle `\textbigcircle', big enough to enclose a letter
      as in `\textcopyright' or `\textregistered'.}
\setslot{bigcircle.superior}
\endsetslot

\nextslot{87}
\setslot{Omega.superior}
   \comment{The upright Ohm sign `\textohm', unavailable in most fonts.
      Even if it is available in Mac-encoded fonts, it isn't directly
      accessible in the 8r or 8y encodings.}
\endsetslot

\nextslot{91}
\setslot{openbracketleft.superior}
   \comment{The opening double square bracket `\textlbrackdbl',
      unavailable in most PostScript fonts.}
\endsetslot

\nextslot{93}
\setslot{openbracketright.superior}
   \comment{The closing double square bracket `\textrbrackdbl',
      unavailable in most PostScript fonts.}
\endsetslot

\nextslot{94}
\setslot{arrowup.superior}
   \comment{An upwards pointing arrow `\textuparrow', unavailable in
      most PostScript fonts.}
\endsetslot

\nextslot{95}
\setslot{arrowdown.superior}
   \comment{An downwards pointing arrow `\textdownarrow', unavailable
      in most PostScript fonts.}
\endsetslot

\nextslot{96}
\setslot{asciigrave.superior}
   \comment{An ASCII-style grave `\textasciigrave'. This is supposed
      to be a character by itself rather than a combining accents.}
\endsetslot

\nextslot{98}
\setslot{born.superior}
   \comment{The born symbol `\textborn', unavailable in most PostScript
      fonts.}
\endsetslot

\nextslot{99}
\setslot{divorced.superior}
   \comment{The divorced symbol `\textdivorced', unavailable in most
      PostScript fonts.}
\endsetslot

\nextslot{100}
\setslot{died.superior}
   \comment{The died symbol `\textdied', unavailable in most PostScript
      fonts.}
\endsetslot

\nextslot{108}
\setslot{leaf.superior}
   \comment{The leaf symbol `\textleaf', unavailable in most PostScript
      fonts.}
\endsetslot

\nextslot{109}
\setslot{married.superior}
   \comment{The married symbol `\textmarried', unavailable in most
      PostScript  fonts.}
\endsetslot

\nextslot{110}
\setslot{musicalnote.superior}
   \comment{A musical note symbol `\textmusicalnote', unavailable in
      most PostScript fonts.}
\endsetslot

\nextslot{126}
\setslot{tildelow.superior}
   \comment{A lowered tilde `\texttildelow'.  In most PostScript fonts
      it can be substituted by `asciitilde', while `\textasciitilde'
      is supposed to be a raised `tilde'.}
\endsetslot

\nextslot{127}
\setslot{hyphendblchar.superior}
    \comment{The glyph used for hyphenation in this font, which will
      almost always be the same as `hyphendbl'.}
\endsetslot

\nextslot{128}
\setslot{asciibreve.superior}
   \comment{An ASCII-style breve `\textasciibreve'. This is supposed
      to be a character by itself rather than a combining accents.}
\endsetslot

\setslot{asciicaron.superior}
   \comment{An ASCII-style caron `\textasciicaron'. This is supposed
      to be a character by itself rather than a combining accents.}
\endsetslot

\setslot{asciiacutedbl.superior}
   \comment{An ASCII-style double tick mark, `\textacutedbl'.}
\endsetslot

\setslot{asciigravedbl.superior}
   \comment{An ASCII-style double backtick mark, `\textgravedbl'.}
\endsetslot

\setslot{dagger.superior}
   \comment{The single dagger `\textdagger'.}
\endsetslot

\setslot{daggerdbl.superior}
   \comment{The double dagger `\textdaggerdbl'.}
\endsetslot

\setslot{bardbl.superior}
   \comment{The double vertical bar `\textbardbl'.}
\endsetslot

\setslot{perthousand.superior}
   \comment{The perthousand sign `\textperthousand'.}
\endsetslot

\setslot{bullet.superior}
   \comment{The centered bullet `\textbullet'.}
\endsetslot

\setslot{centigrade.superior}
   \comment{The degree centigrade symbol `\textcelsius'.}
\endsetslot

\setslot{dollaroldstyle.superior}
   \comment{An oldstyle dollar sign `\textdollaroldstyle'.}
\endsetslot

\setslot{centoldstyle.superior}
   \comment{An oldstyle cent sign `\textcentoldstyle'.}
\endsetslot

\setslot{florin.superior}
   \comment{The florin sign `\textflorin'.}
\endsetslot

\setslot{colonmonetary.superior}
   \comment{The Colon currency sign `\textcolonmonetary', similar to
      a capital `C' with a vertical bar through the middle.}
\endsetslot

\setslot{won.superior}
   \comment{The Won currency sign `\textwon', similar to a capital `W'
      with two horizontal bars.}
\endsetslot

\setslot{naira.superior}
   \comment{The Naira currency sign `\textnaira', similar to a
      capital `N' with two horizontal bars.}
\endsetslot

\setslot{guarani.superior}
   \comment{The Guarani currency sign `\textguarani',  similar to
      a capital `G' with a vertical bar through the middle.}
\endsetslot

\setslot{peso.superior}
   \comment{The Peso currency sign `\textpeso', similar to a capital `P'
      with a horizontal bar through the bowl or below the bowl.}
\endsetslot

\setslot{lira.superior}
   \comment{The Lira currency sign `\textlira', similar to a sterling
      sign `\textsterling' with two horizontal bars.}
\endsetslot

\setslot{recipe.superior}
   \comment{The recipe symbol `\textrecipe', similar to a capital `R'
      with an oblique bar through the tail.}
\endsetslot

\setslot{interrobang.superior}
   \comment{The interrobang symbol `\textinterrobang', similar to
      a combination of an exclamation mark and a question mark.}
\endsetslot

\setslot{interrobangdown.superior}
   \comment{The inverted interrobang symbol `\textinterrobangdown',
      similar to a combination of an inverted exclamation mark
      and an inverted question mark.}
\endsetslot

\setslot{dong.superior}
   \comment{The Dong currency sign `\textdong', similar to a lowercase
      `d'  with a horizontal bar through the stem and another bar below
      the letter.}
\endsetslot

\setslot{trademark.superior}
   \comment{The trademark sign `\texttrademark', similar to the raised
     letters `TM'.}
\endsetslot

\setslot{pertenthousand.superior}
   \comment{The pertenthousand sign `\textpertenthousand', unavailable
     in most PostScript fonts.}
\endsetslot

\setslot{pilcrow.superior}
   \comment{The pilcrow mark `\textpilcrow', similar to a paragraph
      mark `\textparagraph' with a single stem.}
\endsetslot

\setslot{baht.superior}
   \comment{The Baht currency sign `\textbaht', similar to a capital `B'
      with a vertical bar through the middle.}
\endsetslot

\setslot{numero.superior}
   \comment{The numero sign `\textnumero', similar to the letter `N'
      with a raised `o', unavailable in most PostScript fonts.}
\endsetslot

\setslot{discount.superior}
   \comment{The discount sign `\textdiscount', similar to a stylized
      percent sign, unavailable in most PostScript fonts.}
\endsetslot

\setslot{estimated.superior}
   \comment{The estimated sign `\textestimated', similar to an enlarged
      lowercase `e', unavailable in most PostScript fonts.}
\endsetslot

\setslot{openbullet.superior}
   \comment{The centered open bullet `\textopenbullet'', unavailable
      in most PostScript fonts.}
\endsetslot

\setslot{servicemark.superior}
   \comment{The service mark sign `\textservicemark', similar to the
      raised letters `SM', unavailable in most PostScript fonts.}
\endsetslot

\nextslot{160}
\setslot{quillbracketleft.superior}
   \comment{The opening quill bracket `\textlquill', unavailable in
      most PostScript fonts.}
\endsetslot

\setslot{quillbracketright.superior}
   \comment{The closing quill bracket `\textrquill', unavailable in
      most PostScript fonts.}
\endsetslot

\setslot{cent.superior}
   \comment{The cent sign `\textcent'.}
\endsetslot

\setslot{sterling.superior}
   \comment{The British currency sign, `\textsterling'.}
\endsetslot

\setslot{currency.superior}
   \comment{The international currency sign, `\textcurrency'.}
\endsetslot

\setslot{yen.superior}
   \comment{The Japanese currency sign, `\textyen'.}
\endsetslot

\setslot{brokenbar.superior}
   \comment{A broken vertical bar, `\textbrokenbar', similar to
      `\textbar' with a gap through the middle.}
\endsetslot

\setslot{section.superior}
   \comment{The section mark `\textsection'.}
\endsetslot

\setslot{asciidieresis.superior}
   \comment{An ASCII-style dieresis `\textasciidieresis'. This is
       supposed to be character by itself  rather than an accents.}
\endsetslot

\setslot{copyright.superior}
   \comment{The copyright sign `\textcopyright',  similar to a small
       letter `C' enclosed by a circle.}
\endsetslot

\setslot{ordfeminine.superior}
   \comment{The raised letter `\textordfeminine'.}
\endsetslot

\setslot{copyleft.superior}
   \comment{The reversed copyright sign `\textcopyleft', similar to
      a small reversed `C' enclosed by a circle.}
\endsetslot

\setslot{logicalnot.superior}
   \comment{The logical not sign `\textlnot'.}
\endsetslot

\setslot{circledP.superior}
   \comment{A small letter `P' enclosed by a circle, `\textcircledP',
      unavailable in most fonts.}
\endsetslot

\setslot{registered.superior}
   \comment{The registered trademark sign `\textregistered', similar to
      a small letter `R' enclosed by a circle.}
\endsetslot

\setslot{asciimacron.superior}
   \comment{An ASCII-style macron `\textasciimacron'. This is supposed
      to be a character by itself rather than a combining accents.}
\endsetslot

\setslot{degree.superior}
   \comment{The degree sign `\textdegree'.}
\endsetslot

\setslot{plusminus.superior}
   \comment{The plus or minus sign `\textpm'.}
\endsetslot

\setslot{two.superior}
   \comment{The raised digit `\texttwosuperior'.}
\endsetslot

\setslot{three.superior}
   \comment{The raised digit `\textthreesuperior'.}
\endsetslot

\setslot{asciiacute.superior}
   \comment{An ASCII-style acute `\textasciiacute'. This is supposed
      to be a character by itself rather than a combining accents.}
\endsetslot

\setslot{mu.superior}
   \comment{The lowercase Greek letter `\textmu', intended  for use as
      a prefix `micro' in physical units.}
\endsetslot

\setslot{paragraph.superior}
   \comment{The paragraph mark `\textparagraph'.}
\endsetslot

\setslot{periodcentered.superior}
   \comment{The centered period `\textperiodcentered'.}
\endsetslot

\setslot{referencemark.superior}
   \comment{The reference mark `\textreferencemark', similar to
      a combination of the `multiply' and `divide' symbols.}
\endsetslot

\setslot{one.superior}
   \comment{The raised digit `\textonesuperior'.}
\endsetslot

\setslot{ordmasculine.superior}
   \comment{The raised letter `\textordmasculine'.}
\endsetslot

\setslot{radical.superior}
   \comment{The radical sign `\textsurd', unavailable in most fonts.
      Even if it is available in Mac-encoded fonts, it isn't directly
      accessible in the 8r or 8y encodings.}
\endsetslot

\setslot{onequarter.superior}
   \comment{The fraction `\textonequarter'.}
\endsetslot

\setslot{onehalf.superior}
   \comment{The fraction `\textonehalf'.}
\endsetslot

\setslot{threequarters.superior}
   \comment{The fraction `\textthreequarters'.}
\endsetslot

\setslot{Euro.superior}
   \comment{The European currency sign, similar to `\texteuro'.}
\endsetslot


\nextslot{214}
\setslot{multiply.superior}
   \comment{The multiplication sign `\texttimes'.
      This symbol was originally intended to be put into slot~215,
      but ended up in this slot by mistake, at which time it was
      considered too late to change it.}
\endsetslot

\nextslot{246}
\setslot{divide.superior}
   \comment{The divison sign `\textdiv'.
      This symbol was originally intended to be put into slot~247,
      but ended up in this slot by mistake, at which time it was
      onsidered too late to change it.}
\endsetslot

\endencoding

%    \end{macrocode}
% \end{encoding}
% \iffalse
%</ts1-dotsup>
% \fi
% 
% 
% \subsubsection{fontscripts-ts1-euro.etx}\label{subsubsec:ts1-euro}
% 
% \iffalse
%<*ts1-euro>
% \fi
% \begin{encoding}{fontscripts-ts1-euro.etx}
% \changes{v0.0}{2025-02-10}{Filename prefix for Karl.}
%    \begin{macrocode}
%%
%% - The original notices at the top of that file concerning authors,
%% maintenance etc. are replaced by this notice.
%% - The file is renamed.
%% - The encoding is modified to accommodate euro/Euro.%
%%
%%%%%%%%%%%%%%%%%%%%%%%%%%%%%%%%%%%%%%%%%%%%%%%%%
\relax
\encoding

\setstr{codingscheme}{TEX TEXT COMPANION 1---TS1 - EURO}

\ifisglyph{x}\then
   \setint{xheight}{\height{x}}
\else
   \setint{xheight}{500}
\fi

\ifisglyph{space}\then
   \setint{interword}{\width{space}}
\else\ifisglyph{i}\then
   \setint{interword}{\width{i}}
\else
   \setint{interword}{333}
\fi\fi


\setint{italicslant}{0}


\setint{fontdimen(1)}{\int{italicslant}}              % italic slant
\setint{fontdimen(2)}{\int{interword}}                % interword space
\setint{fontdimen(3)}{0}                              % interword stretch
\setint{fontdimen(4)}{0}                              % interword shrink
\setint{fontdimen(5)}{\int{xheight}}                  % x-height
\setint{fontdimen(6)}{1000}                           % quad
\setint{fontdimen(7)}{\int{interword}}                % extra space after .


\nextslot{0}
\setslot{capitalgrave}
   \comment{The grave accent `\capitalgrave{}', intended for use with
      capital letters.}
\endsetslot

\setslot{capitalacute}
   \comment{The acute accent `\capitalacute{}', intended for use with
      capital letters.}
\endsetslot

\setslot{capitalcircumflex}
   \comment{The circumflex accent `\capitalcircumflex{}', intended for
      use with capital letters.}
\endsetslot

\setslot{capitaltilde}
   \comment{The tilde accent `\capitaltilde{}', intended for use with
      capital letters.}
\endsetslot

\setslot{capitaldieresis}
   \comment{The umlaut or dieresis accent `\capitaldieresis{}',
      intended for use with capital letters.}
\endsetslot

\setslot{capitalhungarumlaut}
   \comment{The long Hungarian umlaut `\capitalhungarumlaut{}',
      intended for use with capital letters.}
\endsetslot

\setslot{capitalring}
   \comment{The ring accent `\capitalring{}', intended for use with
      capital letters.}
\endsetslot

\setslot{capitalcaron}
   \comment{The caron or h\'a\v cek accent `\capitalcaron{}', intended
      for use with capital letters.}
\endsetslot

\setslot{capitalbreve}
   \comment{The breve accent `\capitalbreve{}', intended for use with
      capital letters.}
\endsetslot

\setslot{capitalmacron}
   \comment{The macron accent `\capitalmacron{}', intended for use with
      capital letters.}
\endsetslot

\setslot{capitaldotaccent}
   \comment{The dot accent `\capitaldotaccent{}', intended for use with
      capital letters.}
\endsetslot

\setslot{cedilla}
   \comment{The cedilla accent `\capitalcedilla{}', intended for use
      with capital letters.}
\endsetslot

\setslot{ogonek}
   \comment{The ogonek accent `\capitalogonek{}', intended for use with
      capital letters.}
\endsetslot

\nextslot{13}
\setslot{quotesinglbase}
   \comment{A straight single quote mark on the baseline,
      `\textquotestraightbase'.}
\endsetslot

\nextslot{18}
\setslot{quotedblbase}
   \comment{A straight double quote mark on the baseline,
      `\textquotestraightdblbase'.}
\endsetslot

\nextslot{21}
\setslot{twelveudash}
   \comment{A 2/3~em dash, `\texttwelveudash'.}
\endsetslot

\setslot{threequartersemdash}
   \comment{A 3/4~em dash, `\textthreequartersemdash'.}
\endsetslot

\nextslot{23}
\setslot{capitalcompwordmark}
    \comment{An invisible glyph, with zero width and depth, but the
      height of capital letters.
      It is used to stop ligaturing in words like `shelf{}ful'.}
\endsetslot

\nextslot{24}
\setslot{arrowleft}
   \comment{A left pointing arrow, `\textleftarrow', unavailable in
      most PostScript fonts.}
\endsetslot

\setslot{arrowright}
   \comment{A right pointing arrow, `\textrightarrow', unavailable in
      most PostScript fonts.}
\endsetslot

\nextslot{26}
\setslot{tieaccentlowercase}
   \comment{The original tie accent `\t{}', intended for use with
      lowercase letters.}
\endsetslot

\setslot{tieaccentcapital}
   \comment{The tie accent `\capitaltie{}', intended for use with
      capital letters.}
\endsetslot

\setslot{newtieaccentlowercase}
   \comment{A new tie accent `\newtie{}', intended for use with
      lowercase letters.}
\endsetslot

\setslot{newtieaccentcapital}
   \comment{A new tie accent `\capitalnewtie{}', intended for use
      with capital letters.}
\endsetslot

\nextslot{31}
\setslot{ascendercompwordmark}
    \comment{An invisible glyph, with zero width and depth, but the
      height of lowercase letters with ascenders.
      It is used to stop ligaturing in words like `shelf{}ful'.}
\endsetslot

\nextslot{32}
\setslot{blank}
   \comment{The blank indicator `\textblank', similar to the letter `b'
      with an oblique bar throgh the stem.}
\endsetslot

\nextslot{36}
\setslot{dollar}
   \comment{The dollar sign `\textdollar'.}
\endsetslot

\nextslot{39}
\setslot{quotesingle}
   \comment{A straight single quote mark, `\textquotesingle'.}
\endsetslot

\nextslot{42}
\setslot{asteriskcentered}
   \comment{The centered asterisk, `\textasteriskcentered'.}
\endsetslot

\nextslot{44}
\setslot{comma}
   \comment{The decimal comma `,'.}
\endsetslot

\nextslot{45}
\setslot{hyphendbl}
   \comment{An alternate double hyphen, `\textdblhyphen'.}
\endsetslot

\nextslot{46}
\setslot{period}
   \comment{The decimal point `.'.}
\endsetslot

\nextslot{47}
\setslot{fraction}
   \comment{The fraction slash `\textfractionsolidus'.}
\endsetslot

\nextslot{48}
\setslot{zerooldstyle}
   \comment{The oldstyle number `\oldstylenums{0}'.}
\endsetslot

\setslot{oneoldstyle}
   \comment{The oldstyle number `\oldstylenums{1}'.}
\endsetslot

\setslot{twooldstyle}
   \comment{The oldstyle number `\oldstylenums{2}'.}
\endsetslot

\setslot{threeoldstyle}
   \comment{The oldstyle number `\oldstylenums{3}'.}
\endsetslot

\setslot{fouroldstyle}
   \comment{The oldstyle number `\oldstylenums{4}'.}
\endsetslot

\setslot{fiveoldstyle}
   \comment{The oldstyle number `\oldstylenums{5}'.}
\endsetslot

\setslot{sixoldstyle}
   \comment{The oldstyle number `\oldstylenums{6}'.}
\endsetslot

\setslot{sevenoldstyle}
   \comment{The oldstyle number `\oldstylenums{7}'.}
\endsetslot

\setslot{eightoldstyle}
   \comment{The oldstyle number `\oldstylenums{8}'.}
\endsetslot

\setslot{nineoldstyle}
   \comment{The oldstyle number `\oldstylenums{9}'.}
\endsetslot

\nextslot{60}
\setslot{angbracketleft}
   \comment{The opening angle bracket `\textlangle', unavailable in
      most PostScript fonts.}
\endsetslot

\nextslot{61}
\setslot{minus}
   \comment{The subtraction sign `\textminus'.}
\endsetslot

\nextslot{62}
\setslot{angbracketright}
   \comment{The closing angle bracket `\textrangle', unavailable in
      most PostScript fonts.}
\endsetslot

\nextslot{77}
\setslot{Omegainv}
   \comment{The inverted Ohm sign `\textmho', unavailable in most fonts.}
\endsetslot

\nextslot{79}
   \comment{A circle `\textbigcircle', big enough to enclose a letter
      as in `\textcopyright' or `\textregistered'.}
\setslot{bigcircle}
\endsetslot

\nextslot{87}
\setslot{Omega}
   \comment{The upright Ohm sign `\textohm', unavailable in most fonts.
      Even if it is available in Mac-encoded fonts, it isn't directly
      accessible in the 8r or 8y encodings.}
\endsetslot

\nextslot{91}
\setslot{openbracketleft}
   \comment{The opening double square bracket `\textlbrackdbl',
      unavailable in most PostScript fonts.}
\endsetslot

\nextslot{93}
\setslot{openbracketright}
   \comment{The closing double square bracket `\textrbrackdbl',
      unavailable in most PostScript fonts.}
\endsetslot

\nextslot{94}
\setslot{arrowup}
   \comment{An upwards pointing arrow `\textuparrow', unavailable in
      most PostScript fonts.}
\endsetslot

\nextslot{95}
\setslot{arrowdown}
   \comment{An downwards pointing arrow `\textdownarrow', unavailable
      in most PostScript fonts.}
\endsetslot

\nextslot{96}
\setslot{asciigrave}
   \comment{An ASCII-style grave `\textasciigrave'. This is supposed
      to be a character by itself rather than a combining accents.}
\endsetslot

\nextslot{98}
\setslot{born}
   \comment{The born symbol `\textborn', unavailable in most PostScript
      fonts.}
\endsetslot

\nextslot{99}
\setslot{divorced}
   \comment{The divorced symbol `\textdivorced', unavailable in most
      PostScript fonts.}
\endsetslot

\nextslot{100}
\setslot{died}
   \comment{The died symbol `\textdied', unavailable in most PostScript
      fonts.}
\endsetslot

\nextslot{108}
\setslot{leaf}
   \comment{The leaf symbol `\textleaf', unavailable in most PostScript
      fonts.}
\endsetslot

\nextslot{109}
\setslot{married}
   \comment{The married symbol `\textmarried', unavailable in most
      PostScript  fonts.}
\endsetslot

\nextslot{110}
\setslot{musicalnote}
   \comment{A musical note symbol `\textmusicalnote', unavailable in
      most PostScript fonts.}
\endsetslot

\nextslot{126}
\setslot{tildelow}
   \comment{A lowered tilde `\texttildelow'.  In most PostScript fonts
      it can be substituted by `asciitilde', while `\textasciitilde'
      is supposed to be a raised `tilde'.}
\endsetslot

\nextslot{127}
\setslot{hyphendblchar}
    \comment{The glyph used for hyphenation in this font, which will
      almost always be the same as `hyphendbl'.}
\endsetslot

\nextslot{128}
\setslot{asciibreve}
   \comment{An ASCII-style breve `\textasciibreve'. This is supposed
      to be a character by itself rather than a combining accents.}
\endsetslot

\setslot{asciicaron}
   \comment{An ASCII-style caron `\textasciicaron'. This is supposed
      to be a character by itself rather than a combining accents.}
\endsetslot

\setslot{asciiacutedbl}
   \comment{An ASCII-style double tick mark, `\textacutedbl'.}
\endsetslot

\setslot{asciigravedbl}
   \comment{An ASCII-style double backtick mark, `\textgravedbl'.}
\endsetslot

\setslot{dagger}
   \comment{The single dagger `\textdagger'.}
\endsetslot

\setslot{daggerdbl}
   \comment{The double dagger `\textdaggerdbl'.}
\endsetslot

\setslot{bardbl}
   \comment{The double vertical bar `\textbardbl'.}
\endsetslot

\setslot{perthousand}
   \comment{The perthousand sign `\textperthousand'.}
\endsetslot

\setslot{bullet}
   \comment{The centered bullet `\textbullet'.}
\endsetslot

\setslot{centigrade}
   \comment{The degree centigrade symbol `\textcelsius'.}
\endsetslot

\setslot{dollaroldstyle}
   \comment{An oldstyle dollar sign `\textdollaroldstyle'.}
\endsetslot

\setslot{centoldstyle}
   \comment{An oldstyle cent sign `\textcentoldstyle'.}
\endsetslot

\setslot{florin}
   \comment{The florin sign `\textflorin'.}
\endsetslot

\setslot{colonmonetary}
   \comment{The Colon currency sign `\textcolonmonetary', similar to
      a capital `C' with a vertical bar through the middle.}
\endsetslot

\setslot{won}
   \comment{The Won currency sign `\textwon', similar to a capital `W'
      with two horizontal bars.}
\endsetslot

\setslot{naira}
   \comment{The Naira currency sign `\textnaira', similar to a
      capital `N' with two horizontal bars.}
\endsetslot

\setslot{guarani}
   \comment{The Guarani currency sign `\textguarani',  similar to
      a capital `G' with a vertical bar through the middle.}
\endsetslot

\setslot{peso}
   \comment{The Peso currency sign `\textpeso', similar to a capital `P'
      with a horizontal bar through the bowl or below the bowl.}
\endsetslot

\setslot{lira}
   \comment{The Lira currency sign `\textlira', similar to a sterling
      sign `\textsterling' with two horizontal bars.}
\endsetslot

\setslot{recipe}
   \comment{The recipe symbol `\textrecipe', similar to a capital `R'
      with an oblique bar through the tail.}
\endsetslot

\setslot{interrobang}
   \comment{The interrobang symbol `\textinterrobang', similar to
      a combination of an exclamation mark and a question mark.}
\endsetslot

\setslot{interrobangdown}
   \comment{The inverted interrobang symbol `\textinterrobangdown',
      similar to a combination of an inverted exclamation mark
      and an inverted question mark.}
\endsetslot

\setslot{dong}
   \comment{The Dong currency sign `\textdong', similar to a lowercase
      `d'  with a horizontal bar through the stem and another bar below
      the letter.}
\endsetslot

\setslot{trademark}
   \comment{The trademark sign `\texttrademark', similar to the raised
     letters `TM'.}
\endsetslot

\setslot{pertenthousand}
   \comment{The pertenthousand sign `\textpertenthousand', unavailable
     in most PostScript fonts.}
\endsetslot

\setslot{pilcrow}
   \comment{The pilcrow mark `\textpilcrow', similar to a paragraph
      mark `\textparagraph' with a single stem.}
\endsetslot

\setslot{baht}
   \comment{The Baht currency sign `\textbaht', similar to a capital `B'
      with a vertical bar through the middle.}
\endsetslot

\setslot{numero}
   \comment{The numero sign `\textnumero', similar to the letter `N'
      with a raised `o', unavailable in most PostScript fonts.}
\endsetslot

\setslot{discount}
   \comment{The discount sign `\textdiscount', similar to a stylized
      percent sign, unavailable in most PostScript fonts.}
\endsetslot

\setslot{estimated}
   \comment{The estimated sign `\textestimated', similar to an enlarged
      lowercase `e', unavailable in most PostScript fonts.}
\endsetslot

\setslot{openbullet}
   \comment{The centered open bullet `\textopenbullet'', unavailable
      in most PostScript fonts.}
\endsetslot

\setslot{servicemark}
   \comment{The service mark sign `\textservicemark', similar to the
      raised letters `SM', unavailable in most PostScript fonts.}
\endsetslot

\nextslot{160}
\setslot{quillbracketleft}
   \comment{The opening quill bracket `\textlquill', unavailable in
      most PostScript fonts.}
\endsetslot

\setslot{quillbracketright}
   \comment{The closing quill bracket `\textrquill', unavailable in
      most PostScript fonts.}
\endsetslot

\setslot{cent}
   \comment{The cent sign `\textcent'.}
\endsetslot

\setslot{sterling}
   \comment{The British currency sign, `\textsterling'.}
\endsetslot

\setslot{currency}
   \comment{The international currency sign, `\textcurrency'.}
\endsetslot

\setslot{yen}
   \comment{The Japanese currency sign, `\textyen'.}
\endsetslot

\setslot{brokenbar}
   \comment{A broken vertical bar, `\textbrokenbar', similar to
      `\textbar' with a gap through the middle.}
\endsetslot

\setslot{section}
   \comment{The section mark `\textsection'.}
\endsetslot

\setslot{asciidieresis}
   \comment{An ASCII-style dieresis `\textasciidieresis'. This is
       supposed to be character by itself  rather than an accents.}
\endsetslot

\setslot{copyright}
   \comment{The copyright sign `\textcopyright',  similar to a small
       letter `C' enclosed by a circle.}
\endsetslot

\setslot{ordfeminine}
   \comment{The raised letter `\textordfeminine'.}
\endsetslot

\setslot{copyleft}
   \comment{The reversed copyright sign `\textcopyleft', similar to
      a small reversed `C' enclosed by a circle.}
\endsetslot

\setslot{logicalnot}
   \comment{The logical not sign `\textlnot'.}
\endsetslot

\setslot{circledP}
   \comment{A small letter `P' enclosed by a circle, `\textcircledP',
      unavailable in most fonts.}
\endsetslot

\setslot{registered}
   \comment{The registered trademark sign `\textregistered', similar to
      a small letter `R' enclosed by a circle.}
\endsetslot

\setslot{asciimacron}
   \comment{An ASCII-style macron `\textasciimacron'. This is supposed
      to be a character by itself rather than a combining accents.}
\endsetslot

\setslot{degree}
   \comment{The degree sign `\textdegree'.}
\endsetslot

\setslot{plusminus}
   \comment{The plus or minus sign `\textpm'.}
\endsetslot

\setslot{twosuperior}
   \comment{The raised digit `\texttwosuperior'.}
\endsetslot

\setslot{threesuperior}
   \comment{The raised digit `\textthreesuperior'.}
\endsetslot

\setslot{asciiacute}
   \comment{An ASCII-style acute `\textasciiacute'. This is supposed
      to be a character by itself rather than a combining accents.}
\endsetslot

\setslot{mu}
   \comment{The lowercase Greek letter `\textmu', intended  for use as
      a prefix `micro' in physical units.}
\endsetslot

\setslot{paragraph}
   \comment{The paragraph mark `\textparagraph'.}
\endsetslot

\setslot{periodcentered}
   \comment{The centered period `\textperiodcentered'.}
\endsetslot

\setslot{referencemark}
   \comment{The reference mark `\textreferencemark', similar to
      a combination of the `multiply' and `divide' symbols.}
\endsetslot

\setslot{onesuperior}
   \comment{The raised digit `\textonesuperior'.}
\endsetslot

\setslot{ordmasculine}
   \comment{The raised letter `\textordmasculine'.}
\endsetslot

\setslot{radical}
   \comment{The radical sign `\textsurd', unavailable in most fonts.
      Even if it is available in Mac-encoded fonts, it isn't directly
      accessible in the 8r or 8y encodings.}
\endsetslot

\setslot{onequarter}
   \comment{The fraction `\textonequarter'.}
\endsetslot

\setslot{onehalf}
   \comment{The fraction `\textonehalf'.}
\endsetslot

\setslot{threequarters}
   \comment{The fraction `\textthreequarters'.}
\endsetslot

\ifisglyph{euro}\then
	\setslot{euro}
		\comment{The European currency sign, similar to `\texteuro'.}
	\endsetslot
\Else
	\setslot{Euro}
		\comment{The European currency sign, similar to `\texteuro'.}
	\endsetslot
\Fi

\setslot{Euro}
	\comment{This just makes sure that any glyph labelled `Euro' in the font gets encoded. 
  The TS1 encoding will use the previous slot when the font is actually used by tex. 
  At least, I think so. 
  That is, since we've got spare slots in this encoding, we can use them to enable `either... or...' encoding options both for reencoding the fonts for fontinst and for the tex encodings. (?!)}
\endsetslot
	
\setslot{euro}
	\comment{This just makes sure that any glyph labelled `euro' in the font gets encoded. 
  The TS1 encoding will use the previous slot when the font is actually used by tex. 
  At least, I think so. 
  That is, since we've got spare slots in this encoding, we can use them to enable `either... or...' encoding options both for reencoding the fonts for fontinst and for the tex encodings. (?!)}
\endsetslot

\nextslot{214}
\setslot{multiply}
   \comment{The multiplication sign `\texttimes'.
      This symbol was originally intended to be put into slot~215,
      but ended up in this slot by mistake, at which time it was
      considered too late to change it.}
\endsetslot

\nextslot{246}
\setslot{divide}
   \comment{The divison sign `\textdiv'.
      This symbol was originally intended to be put into slot~247,
      but ended up in this slot by mistake, at which time it was
      onsidered too late to change it.}
\endsetslot

\endencoding


%    \end{macrocode}
% \end{encoding}
% \iffalse
%</ts1-euro>
% \fi
% 
% 
% \subsubsection{fontscripts-ucdotalt.etx}\label{subsubsec:ucdotalt}
% 
% \iffalse
%<*ucdotalt>
% \fi
% \begin{encoding}{fontscripts-ucdotalt.etx}
% \changes{v0.0}{2025-02-10}{Filename prefix for Karl.}
%    \begin{macrocode}
\relax
\encoding

\setcommand\uc#1#2{#1.alt}
\setcommand\uctop#1#2{#1.alt}
\ifisint{letterspacing}\then
   \ifnumber{\int{letterspacing}}={0}\then \Else
      \setcommand\uclig#1#2{#1.altspaced}
      \comment{Here we set \verb|\uclig#1#2| to \verb|#1.altspaced|, but 
      you can't see it as \verb|\setcommand| commands are invisible in 
      the typeset output.}
   \Fi
\Fi
\setcommand\uclig#1#2{#1.alt}

\endencoding
%    \end{macrocode}
% \end{encoding}
% \iffalse
%</ucdotalt>
% \fi
% 
%\Finale
