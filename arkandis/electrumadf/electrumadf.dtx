% \iffalse meta-comment
%%%%%%%%%%%%%%%%%%%%%%%%%%%%%%%%%%%%%%%%%%%%%%%%%
% electrumadf.dtx
% Additions and changes Copyright (C) 2010-2024 Clea F. Rees.
% Code from skeleton.dtx Copyright (C) 2015-2024 Scott Pakin (see below).
%
% This work may be distributed and/or modified under the
% conditions of the LaTeX Project Public License, either version 1.3c
% of this license or (at your option) any later version.
% The latest version of this license is in
%   https://www.latex-project.org/lppl.txt
% and version 1.3c or later is part of all distributions of LaTeX
% version 2008-05-04 or later.
%
% This work has the LPPL maintenance status `maintained'.
%
% The Current Maintainer of this work is Clea F. Rees.
%
% This work consists of all files listed in manifest.txt.
%
% The file electrum.dtx is a derived work under the terms of the
% LPPL. It is based on version 2.4 of skeleton.dtx which is part of 
% dtxtut by Scott Pakin. A copy of dtxtut, including the 
% unmodified version of skeleton.dtx is available from
% https://www.ctan.org/pkg/dtxtut and released under the LPPL.
%%%%%%%%%%%%%%%%%%%%%%%%%%%%%%%%%%%%%%%%%%%%%%%%%
% \fi
%
% \iffalse
%<*driver>
\RequirePackage{svn-prov}
\ProvidesFileSVN{$Id: electrumadf.dtx 10164 2024-07-16 21:04:29Z cfrees $}[v1.1 \revinfo][electrumadf DTX: electrum for 8-bit engines]
\DefineFileInfoSVN
\GetFileInfoSVN*
\documentclass[11pt,british]{ltxdoc}
\EnableCrossrefs
\CodelineIndex
\RecordChanges
\OnlyDescription
\DoNotIndex{\verb,\ProvidesPackageSVN,\NeedsTeXFormat,\ProcessKeyOptions}
\usepackage{babel}
\usepackage{lmodern}
\renewcommand{\ttdefault}{lmvtt}
\let\origrmdefault\rmdefault
\DeclareRobustCommand{\origrmfamily}{%
  \fontencoding{T1}%
  \fontfamily{\origrmdefault}%
  \selectfont}
\DeclareTextFontCommand{\textorigrm}{\origrmfamily}
\usepackage[lf]{electrum}
\pdfmapfile{yes.map}	% not necessary for installed package
\pdfmapfile{+pdftex.map}	% not necessary for installed package
\usepackage{fancyhdr}
\usepackage{fixfoot}
\usepackage{array,verbatim,tabularx}
\usepackage[referable]{threeparttablex}
\usepackage{enumitem}
\makeatletter
\def\TPT@measurement{% ateb David Carlisle: https://tex.stackexchange.com/a/370691/
  \ifdim\wd\@tempboxb<\TPTminimum
    \hsize \TPTminimum
  \else
    \hsize\wd\@tempboxb
  \fi
  \xdef\TPT@hsize{\hsize\the\hsize \noexpand\@parboxrestore}\TPT@hsize
  \ifx\TPT@docapt\@undefined\else
    \TPT@docapt \vskip.2\baselineskip
  \fi \par
  \dimen@\dp\@tempboxb % new
  \box\@tempboxb
  \ifvmode \prevdepth\dimen@ \fi% was \z@ not \dimen@
}
\renewlist{tablenotes}{enumerate}{1}
\setlist[tablenotes]{label=\tnote{\alph*},ref=\alph*,itemsep=\z@,topsep=\z@skip,partopsep=\z@skip,parsep=\z@,itemindent=\z@,labelindent=\tabcolsep,labelsep=.2em,leftmargin=*,align=left,before={\unskip\medskip\footnotesize}}
\makeatother
\usepackage{booktabs}
\usepackage{multirow}
\usepackage{xcolor}
\usepackage{xurl}
\urlstyle{tt}
\usepackage{microtype}
\usepackage[a4paper,headheight=14pt]{geometry}	% use 14pt for 11pt text, 15pt for 12pt text
\usepackage{csquotes}
\MakeAutoQuote{‘}{’}
\MakeAutoQuote*{“}{”}
\usepackage{caption}
\DeclareCaptionFont{lf}{\lstyle}
\captionsetup[table]{labelfont=lf}
% sicrhau hyperindex=false: llwytho CYN bookmark
\usepackage{hypdoc}% ateb Ulrike Fischer: https://tex.stackexchange.com/a/695555/
\usepackage{bookmark}
\hypersetup{%
  colorlinks=true,
  citecolor={moss},
  extension=pdf,
  linkcolor={strawberry},
  linktocpage=true,
  pdfcreator={TeX},
  pdfproducer={pdfeTeX},
  urlcolor={blueberry}%
}
\usepackage{cleveref}
\title{\filebase}
\author{Clea F. Rees}
\date{\filedate}
\pagestyle{fancy}
\fancyhf[lh]{\filebase~\fileversion}
\fancyhf[rh]{\filedate}
\fancyhf[ch]{}
\fancyhf[lf]{}
\fancyhf[rf]{}
\fancyhf[cf]{--- \thepage~/~\lastpage{} ---}
\ExplSyntaxOn
\hook_gput_code:nnn {shipout/lastpage} {.}
{
  \property_record:nn {t:lastpage}{abspage,page,pagenum}
}
\cs_new_protected_nopar:Npn \lastpage 
{
  \property_ref:nn {t:lastpage}{page}
}
\ExplSyntaxOff
\definecolor{strawberry}{rgb}{1.000,0.000,0.502}
\definecolor{blueberry}{rgb}{0.000,0.000,1.000}
\definecolor{moss}{rgb}{0.000,0.502,0.251}
\begin{document}
  \DocInput{\filename}
\end{document}
%</driver>
% \fi
%
% \changes{v1.0}{2010/07/17}{First public release.}
% \changes{\fileversion}{\filedate}{Belated update for (New) NFSS and revised nfssext-cfr.
% Try switching to DTX/INS.}
% \maketitle\thispagestyle{empty}
% \pdfinfo{%
%   /Creator	(TeX)
%   /Producer	(pdfTeX)
%   /Author		(Clea F. Rees)
%   /Title		(electrumadf)
%   /Subject	(TeX)
%   /Keywords		(TeX,LaTeX,font,fonts,tex,latex,Electrum,electrum,electrumadf,ElectrumADF,ADF,adf,Arkandis,Digital,Foundry,arkandis,digital,foundry,Hirwen,Harendal,Clea,Rees)}
% \setlength{\parindent}{0pt}
% \setlength{\parskip}{0.5em}
% 
% \newcommand*{\adf}{\textsc{adf}}
% \newcommand*{\lpack}[1]{\textsf{#1}}
% \newcommand*{\fgroup}[1]{\textsf{#1}}
% \newcommand*{\fname}[1]{\textsf{#1}}
% 
% \begin{abstract}
%   \hspace*{-\parindent}Hirwen Harendal, Arkandis Digital Foundry (\adf) has produced the Electrum \adf\ font collection. This guide outlines the \TeX/\LaTeX\ support provided by \lpack{electrumadf} for version 1.005 of the fonts.
% \end{abstract}
% 
% \tableofcontents
% 
% \section{Introduction}
% 
% This document explains how to use the \TeX/\LaTeX\ support provided for version 1.005 of the Electrum \adf\ font collection developed by Hirwen Harendal of the Arkandis Digital Foundry (\adf).
% \lpack{electrumadf} includes copies of the fonts in postscript type 1 format.
% Further  information about the fonts themselves and alternative font formats for use with other programmes can be found at \url{http://arkandis.tuxfamily.org/adffonts.html}. 
% The fonts are released under the \textsc{gnu} General Public License as published by the Free Software Foundation; either version 2 of the License, or any later version, with font exception. 
% For details, see \textsc{notice}.txt and \textsc{copying}.
% 
% The \TeX/\LaTeX\ support package consists of all files listed in \path{manifest.txt}\ and these files are released under the \LaTeX\ Project Public Licence as explained in the included licensing notices.
% Please let me know of any problems so that I can solve them if I can.
% If you can correct the problems and send me the fix, that would be even better.
% 
% \section{The collection}
% 
% Electrum \adf\ is a slab serif family designed as a substitute for Eurostyle or URWCity.
% The family currently includes upright, oblique, small-caps and oblique small-caps shapes in each of light, regular, semi-bold and bold weights (\cref{tab:fams}). 
% Four sets of digits are provided: oldstyle, lining, inferior and superior\footnote{In fact, the fonts also include denominator and numerator figures.
%   Since there is currently no use for these in TeX, however, the support package ignores them.}.
% The support package renames the fonts according to the Karl Berry fontname scheme and defines six families.
% Two of these primarily provide access to the ‘standard’ or default characters while the two ‘ligature’ families support additional non-standard ligatures\footnote{\Cref{sec:encs} describes the encodings used to create these families. 
%   The fifth and sixth families include the inferior and superior figures, together with any other complementary characters included in the fonts.
%   For further details see the encoding (\path{.etx}) files.}. 
% The included package files provide access to these features in \LaTeX\ as explained in \cref{sec:support} and \cref{sec:commands}.
% 
% \begin{table}
% \centering
% \caption[LaTeX families]{\LaTeX{} families}\label{tab:fams}
% \begin{tabularx}\linewidth{lXll}
%   \toprule
%   \textbf{\TeX\ directory}	&	\textbf{font families}	&	\textbf{Original name}	& \textbf{\TeX\ name}\\\midrule
%   electrum	& yes, yesj, yesjw,		&	ElectrumADFExp-Light	&	yesl8a\\
%   &  yesw, \textl{yes0, yes1}	&	ElectrumADFExp-LightOblique		&	yeslo8a\\
%   &		&	ElectrumADFExp-Regular	&	yesr8a\\
%   &		&	ElectrumADFExp-Oblique	&	yesro8a\\
%   &		&	ElectrumADFExp-SemiBold		&	yess8a\\						
%   & 	&	ElectrumADFExp-SemiBoldOblique	&	yesso8a\\
%   &		&	ElectrumADFExp-Bold		&	yesb8a\\
%   &		&	ElectrumADFExp-BoldOblique	&	yesbo8a\\
%   \bottomrule
% \end{tabularx}
% \end{table}
% 
% \section{Requirements}
% 
% Apart from such obvious requirements as \LaTeXe, the \LaTeX\ support provided by \path{electrum.sty} requires \lpack{nfssext-cfr}.
% Without this, you will get errors complaining that the package cannot be found and you will not be able to use any of the additional font commands described in \cref{sec:commands}.
% 
% The documentation requires in additional packages.
% These are all standard and available from \textsc{ctan} but you can always comment out the relevant lines in \path{electrumadf.tex} if you wish.
% 
% \section{The support package}\label{sec:support}
% 
% \subsection{Encodings}\label{sec:encs}
% 
% The package supports modified \textsc{ec}/\textsc{t1} and Text Companion (\textsc{ts1}) encodings.
% Most characters in the \textsc{ec} encoding are available and the fonts provide a small number of characters from the \textsc{ts1} encoding as well, including the \texteuro.
% The regular versions of the \textsc{ec}/\textsc{t1} encoding (\path{t1-yes.etx}/\path{t1j-yes.etx}) reassign three slots which would otherwise be empty due to missing glyphs which \path{fontinst} cannot fake. 
% In the \textsc{t1} encoding, these slot are standardly used for the per thousand zero and the unfakable Sami Eng/eng characters (\textorigrm{\NG}/\textorigrm{\ng}). 
% The modified encodings use \verb|zero.denominator| in place of the per thousand zero which should provide a reasonable substitute (\textl{\textperthousand}/\textl{\textpertenthousand}) when lining figures are in use and a substitute which is at least intelligible (\textperthousand/\textpertenthousand) for oldstyle digits.
% The two further slots are used for the alternate ‘Q’ (Q*) and the \verb|t_t| ligature (tt).
% 
% The ‘ligature’ versions of the \textsc{ec}/\textsc{t1} encoding (\path{t1-yesw.etx}/\path{t1j-yesw.etx}/\path{t1-yesw-sc.etx}/\path{t1j-yesw-sc.etx}) provide access to the full range of ligatures available --- including ‘\textswash{ct}’, ‘\textswash{it}’, ‘\textswash{sp}’ and ‘\textswash{st}’. 
% In addition the alternate ‘Q’ (\textswash{Q}) becomes the default ‘Q’ and the standard ‘Q’ is installed as an alternate (\textswash{Q*}).
% Because further slots are required to accommodate the additional ligatures, a number of characters normally available in the \textsc{ec} encoding are unavailable in upright and oblique shapes. 
% These are the \textsc{ascii} upward-pointing arrowhead (\textasciicircum), the \textsc{ascii} tilde (\textasciitilde) and the \verb|dbar| (\dj). 
% Attempting to access these characters while using the ligature versions of the fonts may result in errors of various kinds and will certainly produce unexpected output even though the characters are provided by the fonts, as the previous sentence demonstrates.
% To access these glyphs, ensure that the regular version of the fonts is active.
% 
% The difference between the \verb|t1-| and \verb|t1j-| encodings is that the latter use oldstyle rather than lining figures and the corresponding symbols designed to complement them. 
% For example, \verb|0123456789 \& \$ \pounds\ \%| produces \textl{0123456789 \& \$ \pounds\ \%} when an encoding of the former kind is active, but 0123456789 \& \$ \pounds\ \% when an encoding of the latter sort is used.
% 
% Finally, \path{t1-dotinfs.etx} and \path{t1-dotsups.etx} support the inferiors and superiors provided by the fonts.
% This amounts to the digits (\textinf{0123456789}/\textsu{0123456789}), some basic punctuation (\textinf{(,)-.}/\textsu{(,)-.}) and symbols (\textinf{+\pounds\$}/\textsu{+\pounds\$}) and, in the case of superiors, a selection of lowercase letters (\textsu{abcdefghijklmnopqrstuvwxyz}).
% 
% The \verb|ts1-| encodings complement the corresponding \verb|t1-| encodings as usual. 
% \path{ts1-yes.etx} simply adapts the names appropriately for Electrum. 
% \verb|ts1-dotoldstyle-yes.etx| also replaces standard symbols with oldstyle variants where these are available.
% This means that
% \iffalse
%<*verb>
%\fi
\begin{verbatim}
  \oldstylenums{0} \oldstylenums{1} \oldstylenums{2} \oldstylenums{3} 
  \oldstylenums{4} \oldstylenums{5} \oldstylenums{6} \oldstylenums{7} 
  \oldstylenums{8} \oldstylenums{9} \textdollaroldstyle\ \textcentoldstyle
\end{verbatim}
% \iffalse
%</verb>
% \fi
% should produce \oldstylenums{0} \oldstylenums{1} \oldstylenums{2} \oldstylenums{3} \oldstylenums{4} \oldstylenums{5} \oldstylenums{6} \oldstylenums{7} \oldstylenums{8} \oldstylenums{9} \textdollaroldstyle\ \textcentoldstyle\ for both encodings, but
% \iffalse
%<*verb>
%\fi
\begin{verbatim}
  \textdollar\ \textcent\ \textsterling\ \texteuro\ \textyen\ \textperthousand
\end{verbatim}
% \iffalse
%</verb>
% \fi
% for example, will produce \textl{\textdollar\ \textcent\ \textsterling\ \texteuro\ \textyen\ \textperthousand\ }if lining figures are active but \texto{\textdollar\ \textcent\ \textsterling\ \texteuro\ \textyen\ \textperthousand\ }when oldstyle digits are in use.
% Similarly, \path{ts1-dotinf.etx} and \path{ts1-dotsup.etx} contain subscript and superscript symbols where available (\textinf{\texteuro \textsterling \textyen \textdollar \textcent}/\textsu{\texteuro \textsterling \textyen \textdollar \textcent}). 
% Unlike \path{ts1-dotoldstyle-yes.etx}, however, the ‘standard’ symbols make no sense here so when inferiors or superiors are in use \emph{only} those symbols available in subscript or superscript form are provided.
% 
% \subsection{\LaTeX\ package}
% 
% To use the fonts in a \LaTeX\ document, add \verb|\usepackage{electrum}| to your document preamble.
% This will set the default serif/roman family to \fname{yes} (\fgroup{electrum}) and enable access to the various alternates, styles and ligatures.
% Three optional arguments are available to tailor the behaviour of the package: \verb|lf|, \verb|osf|, and \verb|lig|.
% By default, oldstyle figures are used as standard and lining digits are available using the commands explained in \cref{sec:commands}.
% To make lining figures the default instead, use one of the following when loading the package:
% \iffalse
%<*verb>
%\fi
\begin{verbatim}
  \usepackage[lf]{electrum}
  \usepackage[lf=true]{electrum}
  \usepackage[osf=false]{electrum}	
\end{verbatim}
% \iffalse
%</verb>
% \fi
% Similarly, to explicitly request oldstyle figures:
% \iffalse
%<*verb>
%\fi
\begin{verbatim}
  \usepackage[osf]{electrum}
  \usepackage[osf=true]{electrum}
  \usepackage[lf=flase]{electrum}	
\end{verbatim}
% \iffalse
%</verb>
% \fi
% 
% Loading \lpack{electrum} with \verb|lig| or \verb|lig=true| will select the versions which enable the additional ligatures and the alternate Q as default (\cref{tab:electrumadf}).
% \textbf{\emph{This option is not recommended unless you are \emph{certain} you do not wish to access any of the characters described in \cref{sec:encs}.}} 
% You should also note that this option will mean all of the additional ligatures will be active, which may not be what you want.
% Again, passing \verb|lig=false| will explicitly request the default --- and strongly recommended --- behaviour which is to \emph{not} enable the additional ligatures by default.
% 
% \begin{table}
% \centering
% \caption{ElectrumADF}\label{tab:electrumadf}
% \begin{tabularx}\textwidth{lllllXX}
%   \toprule
%   \textbf{weights}           &        \textbf{shapes}        &       \textbf{ligatures}      &       \textbf{\texttt{Q}}     &       \textbf{\texttt{Q*}}    & \textbf{figures}      &       \textbf{family}\tabularnewline\midrule
%   \multirow[t]{7}{.15\textwidth}{light,\\regular,\\semi-bold,\\bold}
%   &  \multirow[t]{5}{.26\textwidth}{upright,\\oblique,\\small-caps,\\oblique small-caps}%
%   &  \multirow[t]{2}{.15\textwidth}{standard, tt}%
%   &  \multirow[t]{2}*{Q}        &       \multirow[t]{2}*{Q*}               &       lining  &       yes\tabularnewline\cmidrule{6-7}
%   &  &       &       &       & oldstyle              &       yesj\tabularnewline\cmidrule{3-7}
%   &  &       \multirow[t]{2}{.15\textwidth}{standard, ct, it, sp, st, it}%
%   &  \multirow[t]{2}*{\textswash{Q}}    &       \multirow[t]{2}*{\textswash{Q*}}   &       lining  &       yesw\tabularnewline\cmidrule{6-7}
%   &  &       &       &       &       oldstyle &      yesjw\tabularnewline\cmidrule{2-7}
%   &  \multirow[t]{2}{.26\textwidth}{upright (very incomplete)}%
%   &  \multirow[t]{2}*{---}%
%   &  \multirow[t]{2}*{---}      &       \multirow[t]{2}*{---}      &       inferior &      \textl{yes0}\tabularnewline\cmidrule{6-7}
%   &  & &     & & superior &  \textl{yes1}\tabularnewline
%   \bottomrule
% \end{tabularx}
% \end{table}
% 
% Note that loading \path{electrum.sty} will not affect the default sans-serif or typewriter families.
% 
% \section{Additional font selection commands}\label{sec:commands}
% 
% The \LaTeX\ package \lpack{electrum}\ loads \lpack{nfssext-cfr}\ which is an extension of the package \lpack{nfssext}\ supplied by Philipp Lehman as part of The Font Installation Guide. 
% The file extends the font selection commands to facilitate access to various font features. 
% Both the original and the extension are designed for use with a wide range of fonts.
% For this reason, only a subset of the additional commands are relevant to any particular font support package. 
% Those relevant to \lpack{electrumadf}\ are described below.
% 
% \subsection{nfssext-cfr}
% 
% These commands are available when \lpack{electrum} is loaded.
% If for some reason you wish to make them available when no relevant package is loaded, use \verb|\usepackage{nfssext-cfr}| in your document preamble.
% 
% \subsubsection{Weights}
% 
% \begin{table}
% \centering
% \caption{Weights}\label{tab:weights}
% \begin{tabular}{lll}
%   \toprule
%   \textbf{weight}	  &	\textbf{weight command}	&	\textbf{text command}\\\midrule
%   light				  &	\verb|\lgweight|		&	\verb|\textlg{}|\\
%   semibold			  &	\verb|\sbweight|		&	\verb|\textsb{}|\\
%   \bottomrule
% \end{tabular}
% \end{table}
% 
% The commands in \cref{tab:weights} work in the same way as the standard \LaTeX\ commands for switching to bold text, \verb|\bfseries| and \verb|\textbf{}| except that since these commands affect only the weight and not the width, \verb|weight| replaces \verb|series|.
% \iffalse
%<*verb>
%\fi
\begin{verbatim}
  \textlg{From Light} through regular	\textsb{and semibold} \textbf{to bold.}
\end{verbatim}
% \iffalse
%</verb>
% \fi
% produces:
% \begin{center}
%   \textlg{From Light} through regular	\textsb{and semibold} \textbf{to bold.}
% \end{center}
% 
% \subsubsection{Shapes}
% 
% \begin{table}
% \centering
% \begin{threeparttable}
% \caption{Shapes}\label{tab:shapes}
% \begin{tabular}{lll}
%   \toprule
%   \textbf{shape}			&	\textbf{shape command}	&	\textbf{text command}\\\midrule
%   oblique small-caps		&	\verb|\scshape\slshape|\tnotex{tn:rec} &   \verb|\textsc{\textsl{}}|\tnotex{tn:rec} \\
%   		&	\verb|\slshape\scshape|\tnotex{tn:rec} &   \verb|\textsl{\textsc{}}|\tnotex{tn:rec} \\
%   		&	\verb|\itshape\scshape|\tnotex{tn:all} &   \verb|\textit{\textsc{}}|\tnotex{tn:all} \\
%   		&	\verb|\scshape\itshape|\tnotex{tn:all} &   \verb|\textsc{\textit{}}|\tnotex{tn:all} \\
%        & \color{gray}\verb|\sishape|\tnotex{tn:dep}			&	\color{gray}\verb|\textsi{}|\tnotex{tn:dep} \\
%   \bottomrule
% \end{tabular}
% \begin{tablenotes}
%   \item \label{tn:rec}Only supported on post-2020 \LaTeXe{}.
%   \item \label{tn:all}Supported for all versions of \LaTeXe{}.
%   \item \label{tn:dep}Deprecated.
% \end{tablenotes}
% \end{threeparttable}
% \end{table}
% 
% The commands in \cref{tab:shapes} provide access to oblique small-caps, which also serves as italic small-caps.
% On post-2020 \LaTeX{}, support for these sequences is built into the kernel.
% On pre-2020 \LaTeX{}, support is provided by \lpack{nfssext-cfr}.
% For example, \texttt{\cs{textsc}\marg{\cs{textit}\marg{Lewis Carroll \cs{textulc}\marg{wrote}, ‘I \cs{emph}\marg{always \cs{textup}\marg{avoid}} a \cs{textup}\marg{kangaroo}’.}}} produces:
% \begin{center}
%   \textsc{\textit{Lewis Carroll \textulc{wrote}, ‘I \emph{always \textup{avoid}} a \textup{kangaroo}’.}}
% \end{center}	
% Note the somewhat unexpected behaviour of the \LaTeXe{} kernel's \cs{textup} here.
% Intuitively, one might have expected the command to simply reverse the effect of \cs{emph} as well as \cs{textit}, but, within the scope of \cs{emph}, it also reverses the effects of \cs{textsc}.
% This reflects the fact that, unlike most \LaTeXe{} font commands, \cs{emph} is a simple wrapper around \TeX{}'s \cs{em}, so it does not engage directly with the machinery of \textsc{nfss}.
%
% 
% \subsubsection{Styles}
% 
% \begin{table}
% \centering
% \caption{Styles}\label{tab:styles}
% \begin{tabular}{llll}
%   \toprule
%   \textbf{style}			&	\textbf{style command}	&	\textbf{text command}	&	\textbf{effect}\\\midrule
%   ligature/swash			&	\verb|\swashstyle|		&	\verb|\textswash{}|		&	italic, regular script\\
%   \bottomrule
% \end{tabular}
% \end{table}
% 
% \verb|\swashstyle| and \verb|\textswash{}| (\cref{tab:styles}) switch to the ‘ligature’ families (\fgroup{yesw}/\fgroup{yesjw}).
% Within the scope of these commands:
% \begin{itemize}
%   \item \verb|Q| will typeset the alternate ‘Q’ (\textswash{Q});
%   \item \verb|Q*| will typeset the default ‘Q’ (\textswash{Q*});
%   \item in upright and oblique text, \verb|ct|, \verb|it|, \verb|sp| and \verb|st|  will typeset the corresponding ligature (\textswash{ct}/\textswash{it}/\textswash{sp}/\textswash{st});
%   \item attempting to typeset certain standard characters will produce unexpected results (see \cref{sec:encs}).
% \end{itemize}
% 
% Outside the scope of these commands:
% \begin{itemize}
%   \item \verb|Q| will typeset the default ‘Q’ (Q);
%   \item \verb|Q*| will typeset the alternate ‘Q’ (Q*);
%   \item \verb|ct|, \verb|it|, \verb|sp| and \verb|st|  will not produce ligatures (ct/it/sp/st);
%   \item except as explained in \cref{sec:encs}, typesetting standard characters should produce the expected results.
% \end{itemize}	
% 
% For example, suppose that \lpack{electrum}\ was loaded and the following commands set up:
% \iffalse
%<*verb>
%\fi
\begin{verbatim}
  \newcommand{\fytext}{%
    Q*ueenie, actor-spy and Queen of AQ*UA as Acting Erector Aesthete,\\
    deactivated the sporadically impacted TORQUE despite aspirating stridently\\
    amidst the hysteria of wispy, wasted wasps wistfully whistling.}
  \newcommand{\fytest}{%
    \fytext\\[1em]
    \textswash{\fytext}\\[1em]
    \textsl{\fytext}\\[1em]
    \textswash{\textsl{\fytext}}\\[1em]
    \textsc{\fytext}\\[1em]
    \textsc{\textswash{\fytext}}\\[1em]			
    \textsc{\itshape\fytext}\\[1em]
    \textswash{\textsc{\itshape\fytext}}}
\end{verbatim}
% \iffalse
%</verb>
% \fi
% \newcommand{\fytext}{%
%   Q*ueenie, actor-spy and Queen of AQ*UA as Acting Erector Aesthete,\\
%   deactivated the sporadically impacted TORQUE despite aspirating stridently\\
%   amidst the hysteria of wispy, wasted wasps wistfully whistling.}
% \newcommand{\fytest}{%
%   \fytext\\[1em]
%   \textswash{\fytext}\\[1em]
%   \textsl{\fytext}\\[1em]
%   \textswash{\textsl{\fytext}}\\[1em]
%   \textsc{\fytext}\\[1em]
%   \textsc{\textswash{\fytext}}\\[1em]			
%   \textsi{\fytext}\\[1em]
%   \textswash{\textsi{\fytext}}}
% Then:
% \iffalse
%<*verb>
%\fi
\begin{verbatim}
  \fytest
\end{verbatim}
% \iffalse
%</verb>
% \fi
% produces:
% \begin{center}
%   \fytest
% \end{center}
% 
% \subsubsection{Figures}
% 
% \begin{table}
% \centering
% \begin{threeparttable}
% \caption{Figures}\label{tab:figs}
% \begin{tabular}{lll}
%   \toprule
%   \textbf{figure style}		&	\textbf{style command}	&	\textbf{text command}\\\midrule
%   lining					&	\verb|\lstyle|			&	\verb|\textl{}|\\
%   oldstyle					&	\verb|\ostyle|			&	\verb|\texto{}|\\
%   inferior/subscript		&	\verb|\infstyle|\tnotex{tn:in}	&	\verb|\textinf{}|\tnotex{tn:in}\\
%   superior/superscript		&	\verb|\sustyle|			&	\verb|\textsu{}|\\
%   \bottomrule
% \end{tabular}
% \begin{tablenotes}
%   \item \label{tn:in}Previous versions of \lpack{nfssext-cfr} provided \cs{instyle} and \cs{textin}.
%   Unfortunately, \lpack{hyperref} now breaks \cs{textin} in a particularly confusing way: it causes an error because the command is not defined for the current font encoding.
%   Although \cs{instyle} is still available for compatibility reasons and \cs{textin} is provided if \lpack{hyperref} is not loaded, \cs{infstyle} and \cs{textinf} should be used in all new documents and older documents may need to be updated if they use \lpack{hyperref}.
% \end{tablenotes}
% \end{threeparttable}
% \end{table}
% 
% In this document, lining figures are used when available by default:
% \begin{center}
%   0123456789
% \end{center}
% but oldstyle figures are also accessible.
% For example, \verb|\texto{0123456789}| produces:
% \begin{center}
%   \texto{0123456789}
% \end{center}
% In addition to modifying the figure style, the commands in \cref{tab:figs} affect the style of certain complementary characters in the \textsc{t1} and \textsc{ts1} encodings as explained in \cref{sec:encs}. 
% This means that:
% \iffalse
%<*verb>
%\fi
\begin{verbatim}
  50\%\ off! That's just \texteuro 2.95, \pounds 3.41, \textyen 5.28 
  \& \$8.67\textcent\ \textsl{or} less than \textdollar 1 \& \textsterling 0.99!!
\end{verbatim}
% \iffalse
%</verb>
% \fi
% produces:
% \begin{center}
%   50\%\ off! That's just \texteuro 2.95, \pounds 3.41, \textyen 5.28
%   \& \$8.67\textcent\ \textsl{or} less than \textdollar 1 \& \textsterling 0.99!!
% \end{center}
% when lining digits are in use, but:
% \begin{center}
%   \texto{50\%\ off! That's just \texteuro 2.95, \pounds 3.41, \textyen 5.28 
%     \& \$8.67\textcent\ \textsl{or} less than \textdollar 1 \& \textsterling 0.99!!}
% \end{center}
% after switching to oldstyle figures.
% 
% Note that the commands for inferior and superior figures make further changes.
% \textbf{\emph{Normal text cannot be typeset within the scope of the commands for inferiors or superiors.}}
% The commands for subscript activate basic symbols and punctuation to complement the digits. 
% So \verb|Llundain\textinf{(1,4+\$5)}| produces Llundain\textinf{(1,4+\$5)}.
% The commands for superscript activate a partial lowercase in as well. 
% For example, \verb|postbox\textsu{9(iii)}| produces postbox\textsu{9(iii)}.
% 
% \appendix
% 
% \DeclareFixedFootnote \fnupdmap {%
%   See, for example, \href{https://tex.stackexchange.com/q/255709/}{Why shouldn't I use \texttt{getnonfreefonts} to install additional fonts? Why shouldn't I use \texttt{updmap} when installing or removing fonts?}.}
% 
% \section{Installation}
% 
% \textbf{The vast majority of users should IGNORE this section entirely.}
% \lpack{electrumadf} is included in all major \TeX{} distributions and should be installed as part of your \TeX{} installation.
% Installing the package yourself should be done only as a last resort or an educational exercise.
% 
% Note, in particular, that this version of \lpack{electrumadf} should \textbf{not} be installed on older \LaTeX{} kernels as it is designed to work with the (New) New Font Selection Scheme, as updated around 2020\footnote{%
%   The package should\texttrademark{} work fine on older kernels, but the new version is bound to have some bugs and there is no reason to use it on these systems.
%   The sole purpose of the update is to accommodate the breaking changes made to font selection.
%   If you don't have those changes installed locally, nothing should be broken and the newer version of \lpack{electrumadf} offers no advantage at all.%
% }.
% Use the initial release of \lpack{electrumadf} if your installation of \LaTeX{} predates those changes.
% 
% Installation varies with \TeX\ distribution so you should consult the documentation which came with your system for details.
% In most cases, you will need to perform three steps:
% \begin{enumerate}
%   \item move or copy the package files to appropriate locations on your system;
%   \item refresh the \TeX\ database;
%   \item incorporate the included map file fragments for the different engines your distribution supports.
% \end{enumerate}
% 
% The following instructions assume you are using \TeX~Live\footnote{This includes Mac\TeX\ for OS X users.}. 
% They should not be too difficult to adapt if you are using a different distribution.
% 
% \subsection{Install the files}
% 
% The files should be installed in one of two locations: \emph{either} the local system-wide \TeX\ tree \emph{or} your personal tree.
% If the package is installed system-wide, all users will have access to it.
% On the other hand, you may need privileges you do not have to do this in which case you must use your personal tree.
% 
% \textbf{There are serious disadvantages to installing the package into your personal tree.
% In particular, these pertain to use of \verb|updmap --user| rather than \verb|updmap --sys|.
% If you are not aware of these disadvantages, please ensure you are fully cognisant of them before proceeding\fnupdmap.
% Merely removing the package from your personal tree at a later point will \emph{not} undo the effects.}
% 
% For \TeX~Live, \verb|kpsewhich -var-value TEXMFLOCAL| will return the path to the local tree and \verb|kpsewhich -var-value TEXMFHOME| the path to your personal tree. 
% The package already includes a hierarchy of files to help you install them correctly.
% Ignoring any symbolic link in the top directory, move or copy the files in \path{doc}, \path{fonts} and \path{tex} into the appropriate locations.
% If the tree is initially empty, you can simply move or copy the directories in as they are.
% If the tree already contains other packages, you may need to merge the package hierarchy with the pre-existing one.
% For example, if you already have a \path{doc/fonts} directory, move or copy \path{doc/fonts/electrum} into \path{doc/fonts/}.
% If you have a \path{doc} directory but not a \path{doc/fonts}, move \path{doc/fonts} into \path{doc/}.
% 
% \subsection{Refresh the database}
% 
% Again, this depends on your distribution. For \TeX~Live, \verb|mktexlsr <path to directory>| for the directory you used in the first step should do the trick.
% Note that you \emph{may} be able to skip this step if you install into your personal tree.
% Whether this is so depends on the details of your set-up.
% As a test, move to a directory containing none of the package files and try \verb|kpsewhich electrum.sty|.
% If the file is found, you don't need to refresh the database; otherwise use \verb|mktexlsr| and then try again.
% 
% \subsection{Install the map fragments}
% 
% For \TeX~Live, there are at least two ways of doing this. 
% The second method varies according to the version of \TeX~Live and instructions are provided accordingly.
% Both methods depend on whether you installed into \verb|TEXMFLOCAL| or \verb|TEXMFHOME|.
% If you installed system-wide, the choice is relatively straightforward --- it obviously makes sense in that case to update the font maps system-wide as well.
% 
% If, on the other hand, you installed into your personal tree, the matter is more complex. 
% On the one hand, updating the system-wide maps may create difficulties or confusion for other users because while the map files will list the fonts as available, they will not be able to access them.
% On the other hand, maintaining personal font map files can produce difficulties and confusions of its own\fnupdmap.
% Whether it is to be preferred or not is a complex issue and depends on the details of your \TeX\ distribution, local configuration and personal preference. 
% The one clear case is that in which you install into your personal tree because you lack the privileges needed to install system-wide. 
% In that case, you have no choice but to maintain personal font map files or forgo the use of all fonts not provided by your administrator.
% Other cases are thankfully beyond the scope of this document.
% 
% \subsubsection{Method 1}
% 
% If you installed the package system-wide, use the command:
% \iffalse
%% ateb Heiko Oberdiek: https://tex.stackexchange.com/a/172896/
%% dyw hwn ddim yn gweithio tu mewn i arg macro! 
%% (*wrth gwrs* dyw e ddim yn weithio! wyt ti'n *hollol* dwp?)
%<*verb>
% \fi
\begin{verbatim}
  updmap-sys --enable Map=yes.map
\end{verbatim}
% \iffalse
%</verb>
% \fi
%
% If you installed the package in your personal tree, you \emph{may} prefer\fnupdmap:
% \iffalse
%<*verb>
% \fi
\begin{verbatim}
  updmap --enable Map=yes.map
\end{verbatim}
% \iffalse
%</verb>
% \fi
% Either way, \verb|updmap| will output a good deal of information after each incantation.
% This is normal.
% Just check that it does not end with an error and that it found the new map file.
% 
% \subsubsection{Method 2: \TeX~Live 2008 (and probably earlier)}
% 
% If you installed the package system-wide, use \verb|updmap-sys --edit|.
% 
% If you installed into your personal tree, you \emph{may} prefer to use	\verb|updmap --edit|\fnupdmap.
% 
% Either way, a configuration file will be opened which you can edit.
% Move to the end of the file and add the following line:
% \iffalse
%<*verb>
% \fi
\begin{verbatim}
  Map yes.map
\end{verbatim}
% \iffalse
%</verb>
% \fi
% When you are done, save the file.
% \verb|updmap| or \verb|updmap-sys| will produce a great deal of output if all is well. 
% Just check that it does not end with an error and that \path{yes.map} is found.
% 
% \subsubsection{Method 2: \TeX~Live 2009 (and possibly later)}
% 
% If you installed the package system-wide, edit or or create \path{TEXMFLOCAL/web2c/updmap-local.cfg} and add the following line to the end of the file:
% \iffalse
%<*verb>
% \fi
\begin{verbatim}
  Map yes.map
\end{verbatim}
% \iffalse
%</verb>
% \fi
% Save the file and tell \verb|tlmgr| to merge in your addition using the command:
% \iffalse
%<*verb>
% \fi
\begin{verbatim}
  tlmgr generate updmap
\end{verbatim}
% \iffalse
%</verb>
% \fi
% \verb|tlmgr| will then tell you that you need to ensure the changes are propagated correctly by calling \verb|updmap-sys|. 
% This should produce a great deal of output.
% Check that it finds the new map file and does not end with an error.
% 
% If you installed into your personal tree, you \emph{may} prefer to use \verb|updmap --edit| as described above for \TeX~Live 2008\fnupdmap.
% 
% \subsubsection{Method 3: Current/Recent \TeX~Live}
% 
% If you installed the package system-wide, tell \cs{updmap} to enable the map file:
% \iffalse
%<*verb>
% \fi
\begin{verbatim}
  updmap --sys --enable Map=yes.map
\end{verbatim}
% \iffalse
%</verb>
% \fi
% This should produce a great deal of output.
% Check that it finds the new map file and does not end with an error.
% 
% If you installed into your personal tree, you \emph{could} use \verb|updmap --user| in place of \verb|updmap --sys| as described above for \TeX~Live 2008, but this is \textbf{not} recommended\fnupdmap.
% 
% To test your installation and that the package works on your system, latex this file (\path{electrumadf.tex}).
% The console output and/or log should tell you whether any fonts were not found. 
% If you are careful not to overwrite it, you may also compare your output with \path{electrumadf.pdf}.%
% 
% \MaybeStop{%
% \PrintChanges
% \PrintIndex
% }
% 
% \section{Implementation}
%
% You do not need to read the remainder of this document in order to install
% or use the fonts.
%
% \iffalse
%<*sty>
% \fi
%    \begin{macrocode}
\NeedsTeXFormat{LaTeX2e}
\RequirePackage{svn-prov}
\ProvidesPackageSVN[electrum.sty]{$Id: electrumadf.dtx 10164 2024-07-16 21:04:29Z cfrees $}[\filebase v1.1 \revinfo]
\DefineFileInfoSVN[electrum]
\RequirePackage[T1]{fontenc}
\RequirePackage{nfssext-cfr}
%    \end{macrocode}
% \lpack{nfssext-cfr} provides \cs{ProcessKeyOptions}, \cs{IfFormatAtLeastTF} on older kernels.
%    \begin{macrocode}
\IfFormatAtLeastTF {2020-02-02}{%
%    \end{macrocode}
% To get the oldstyle numbers etc.\ used from TS1, we need to set the subset to 0 or 1.
% Unfortunately, this means characters missing from the fonts will not use default symbols as fallback, but this seems to be unavoidable.
%    \begin{macrocode}
  \DeclareEncodingSubset{TS1}{yes}{1}%
  \DeclareEncodingSubset{TS1}{yesj}{1}%
  \DeclareEncodingSubset{TS1}{yesw}{1}%
  \DeclareEncodingSubset{TS1}{yesjw}{1}%
  \DeclareEncodingSubset{TS1}{yes0}{1}%
  \DeclareEncodingSubset{TS1}{yes1}{1}%
}{%
  \RequirePackage{textcomp}}
\UndeclareTextCommand{\textperthousand}{T1}
\ExplSyntaxOn
%    \end{macrocode}
% The actual \verb|sty| is ultra simple.
% Three options of which only two are actually needed.
% \texttt{lig} sets additional ligatures (swash) or not.
% \texttt{osf}/\texttt{lf} set lining or oldstyle figures.
% Two booleans suffice.
%    \begin{macrocode}
\keys_define:nn { electrum }
{
  lig .bool_set:N = \g__electrum_lig_bool,
  lig .default:n = true,
  lig .initial:n = false,
  lf .bool_set_inverse:N = \g__electrum_osf_bool,
  lf .default:n = true,
  osf .bool_set:N = \g__electrum_osf_bool,
  osf .default:n = true,
  osf .initial:n = true,
}
%    \end{macrocode}
% Note the optional argument is mandatory in case we're on an older kernel.
% \begin{macrocode}
\ProcessKeyOptions[electrum]
%    \end{macrocode}
% Use a token list initialised with the bare Berry name.
% %% could I just use \rmdefault directly here?
% %% does it matter if it is a cs rather than a tl?
%    \begin{macrocode}
\tl_new:N \g__electrum_rm_tl
\tl_gset:Nn \g__electrum_rm_tl {yes}
%    \end{macrocode}
% Add indicator for style of figures.
%    \begin{macrocode}
\bool_if:NT \g__electrum_osf_bool
{
  \tl_gput_right:Nn \g__electrum_rm_tl {j}
}
%    \end{macrocode}
% Add indicator for swash or not.
%    \begin{macrocode}
\bool_if:NT \g__electrum_lig_bool
{
  \tl_gput_right:Nn \g__electrum_rm_tl {w}
}
%    \end{macrocode}
% Order is critical as we're matching on family names.
% Make ElectrumADF default roman font, using the assembled name to implement requested options.
%    \begin{macrocode}
\renewcommand{\rmdefault}{\g__electrum_rm_tl}
\cs_if_exist:NF \infstyle
{
  \hook_gput_code:nnn { begindocument } { . }
  {
    \PackageWarning{electrum}
    {electrum ~ requires ~ updated ~ nfssext-cfr ~ for ~ compatibility ~ with ~ hyperref}
    \@ifpackageloaded{hyperref}{
      \hook_gput_code:nnn { cmd/textin/before } { . }
      {
        \PackageWarning{electrum}{Note ~ that ~ '\protect\textin' ~ is ~ defined ~ by ~ hyperref.\MessageBreak
          Use ~ for ~ inferior ~ digits ~ will ~ yield ~ an\MessageBreak
          undefined ~ command ~ error ~ with ~ document ~ font ~ encodings.\MessageBreak
          Use ~ '\protect\textinf' ~ to ~ access ~ inferior ~ digits}
      }
      \let\infstyle\instyle
      \DeclareTextFontCommand{\textinf}{\infstyle}
    }{}
  }
}
\ExplSyntaxOff
%% paid â chynnwys \endinput - docstrip yn chwilio amddo fe yn arbennigol
%% & bydd doctrip yn ei ychwanegu fe beth bynnag
%% (Martin Scharrer: https://tex.stackexchange.com/a/28997/)
%% end electrum.sty
%    \end{macrocode}
% \iffalse
%</sty>
% \fi
% 
% The remaining files are not used directly, but are required to generate the
% files which allow \TeX{} and \LaTeX{} to use the fonts.
% The sources use \verb|fontinst| as explained in the (sparse) comments.
% While you can install these files into a \TeX{} tree, they are not required
% for typesetting.
% 
% \subsection{Driver}
%
% The file does all the initial setup of the fonts.
% It organises the fonts into families, defines shapes and reencodes as required.
%
% \iffalse
%<*drv>
% \fi
%    \begin{macrocode}
\input fontinst.sty
\needsfontinstversion{1.926}
%    \end{macrocode}
% Substitutions
% Bold for bold extended
%    \begin{macrocode}
\substitutesilent{bx}{b}
\substitutesilent{si}{scit}
\substitutesilent{scsl}{si}
\substitutesilent{sc}{n}
\substitutesilent{sl}{scsl}
\substitutesilent{it}{sl}
%    \end{macrocode}
% Rrecord transformations for later map file creation
%    \begin{macrocode}
\recordtransforms{yes-rec.tex}
%    \end{macrocode}
% Transformations : reencode fonts
%    \begin{macrocode}
  \transformfont{yesr8r}{\reencodefont{8r}{\fromafm{yesr8a}}}
  \transformfont{yesro8r}{\reencodefont{8r}{\fromafm{yesro8a}}}
  \transformfont{yesb8r}{\reencodefont{8r}{\fromafm{yesb8a}}}
  \transformfont{yesbo8r}{\reencodefont{8r}{\fromafm{yesbo8a}}}
  \transformfont{yesr8s}{\reencodefont{supp-yes}{\fromafm{yesr8a}}}
  \transformfont{yesro8s}{\reencodefont{supp-yes}{\fromafm{yesro8a}}}
  \transformfont{yesb8s}{\reencodefont{supp-yes}{\fromafm{yesb8a}}}
  \transformfont{yesbo8s}{\reencodefont{supp-yes}{\fromafm{yesbo8a}}}
  \transformfont{yesl8r}{\reencodefont{8r}{\fromafm{yesl8a}}}
  \transformfont{yeslo8r}{\reencodefont{8r}{\fromafm{yeslo8a}}}
  \transformfont{yess8r}{\reencodefont{8r}{\fromafm{yess8a}}}
  \transformfont{yesso8r}{\reencodefont{8r}{\fromafm{yesso8a}}}
  \transformfont{yesl8s}{\reencodefont{supp-yes}{\fromafm{yesl8a}}}
  \transformfont{yeslo8s}{\reencodefont{supp-yes}{\fromafm{yeslo8a}}}
  \transformfont{yess8s}{\reencodefont{supp-yes}{\fromafm{yess8a}}}
  \transformfont{yesso8s}{\reencodefont{supp-yes}{\fromafm{yesso8a}}}
  \input reglyph-yes.tex
%    \end{macrocode}
% Installation: creation of virtual fonts
%    \begin{macrocode}
  \installfonts
    \installfamily{T1}{yes}{}
    \installfont{yesr8t}{yesr8r,yesr8sr,newlatin option nosc}{t1-yes}{T1}{yes}{m}{n}{}
    \installfont{yesrc8t}{yesr8r,yesr8sr,build-ttsc,newlatin-dotsc,yesr8r suffix .sc,yesr8sr suffix .sc}{dotsc2,t1-yes}{T1}{yes}{m}{sc}{}
    \installfont{yesro8t}{yesro8r,yesro8sr,newlatin option nosc}{t1-yes}{T1}{yes}{m}{sl}{}
    \installfont{yesrco8t}{yesro8r,yesro8sr,build-ttsc,newlatin-dotsc,yesro8r suffix .sc,yesro8sr suffix .sc}{dotsc2,t1-yes}{T1}{yes}{m}{scit}{}
    \installfont{yesb8t}{yesb8r,yesb8sr,newlatin option nosc}{t1-yes}{T1}{yes}{b}{n}{}
    \installfont{yesbc8t}{yesb8r,yesb8sr,build-ttsc,newlatin-dotsc,yesb8r suffix .sc,yesb8sr suffix .sc}{dotsc2,t1-yes}{T1}{yes}{b}{sc}{}
    \installfont{yesbo8t}{yesbo8r,yesbo8sr,newlatin option nosc}{t1-yes}{T1}{yes}{b}{sl}{}
    \installfont{yesbco8t}{yesbo8r,yesbo8sr,build-ttsc,newlatin-dotsc,yesbo8r suffix .sc,yesbo8sr suffix .sc}{dotsc2,t1-yes}{T1}{yes}{b}{scit}{}
    \installfont{yesl8t}{yesl8r,yesl8sr,newlatin option nosc}{t1-yes}{T1}{yes}{l}{n}{}
    \installfont{yeslc8t}{yesl8r,yesl8sr,build-ttsc,newlatin-dotsc,yesl8r suffix .sc,yesl8sr suffix .sc}{dotsc2,t1-yes}{T1}{yes}{l}{sc}{}
    \installfont{yeslo8t}{yeslo8r,yeslo8sr,newlatin option nosc}{t1-yes}{T1}{yes}{l}{sl}{}
    \installfont{yeslco8t}{yeslo8r,yeslo8sr,build-ttsc,newlatin-dotsc,yeslo8r suffix .sc,yeslo8sr suffix .sc}{dotsc2,t1-yes}{T1}{yes}{l}{scit}{}
    \installfont{yess8t}{yess8r,yess8sr,newlatin option nosc}{t1-yes}{T1}{yes}{sb}{n}{}
    \installfont{yessc8t}{yess8r,yess8sr,build-ttsc,newlatin-dotsc,yess8r suffix .sc,yess8sr suffix .sc}{dotsc2,t1-yes}{T1}{yes}{sb}{sc}{}
    \installfont{yesso8t}{yesso8r,yesso8sr,newlatin option nosc}{t1-yes}{T1}{yes}{sb}{sl}{}
    \installfont{yessco8t}{yesso8r,yesso8sr,build-ttsc,newlatin-dotsc,yesso8r suffix .sc,yesso8sr suffix .sc}{dotsc2,t1-yes}{T1}{yes}{sb}{scit}{}
    \installfamily{T1}{yesj}{}
    \installfont{yesrj8t}{yesr8r,yesr8sr,newlatin option nosc,yesr8r suffix .oldstyle}{t1j-yes}{T1}{yesj}{m}{n}{}
    \installfont{yesrcj8t}{yesr8r,yesr8sr,build-ttsc,newlatin-dotsc,yesr8r suffix .sc,yesr8sr suffix .sc,yesr8r suffix .oldstyle}{dotsc2,t1j-yes}{T1}{yesj}{m}{sc}{}
    \installfont{yesrjo8t}{yesro8r,yesro8sr,newlatin option nosc,yesro8r suffix .oldstyle}{t1j-yes}{T1}{yesj}{m}{sl}{}
    \installfont{yesrcjo8t}{yesro8r,yesro8sr,build-ttsc,newlatin-dotsc,yesro8r suffix .sc,yesro8sr suffix .sc,yesro8r suffix .oldstyle}{dotsc2,t1j-yes}{T1}{yesj}{m}{scit}{}
    \installfont{yesbj8t}{yesb8r,yesb8sr,newlatin option nosc,yesb8r suffix .oldstyle}{t1j-yes}{T1}{yesj}{b}{n}{}
    \installfont{yesbcj8t}{yesb8r,yesb8sr,build-ttsc,newlatin-dotsc,yesb8r suffix .sc,yesb8sr suffix .sc,yesb8r suffix .oldstyle}{dotsc2,t1j-yes}{T1}{yesj}{b}{sc}{}
    \installfont{yesbjo8t}{yesbo8r,yesbo8sr,newlatin option nosc,yesbo8r suffix .oldstyle}{t1j-yes}{T1}{yesj}{b}{sl}{}
    \installfont{yesbcjo8t}{yesbo8r,yesbo8sr,build-ttsc,newlatin-dotsc,yesbo8r suffix .sc,yesbo8sr suffix .sc,yesbo8r suffix .oldstyle}{dotsc2,t1j-yes}{T1}{yesj}{b}{scit}{}
    \installfont{yeslj8t}{yesl8r,yesl8sr,newlatin option nosc,yesl8r suffix .oldstyle}{t1j-yes}{T1}{yesj}{l}{n}{}
    \installfont{yeslcj8t}{yesl8r,yesl8sr,build-ttsc,newlatin-dotsc,yesl8r suffix .sc,yesl8sr suffix .sc,yesl8r suffix .oldstyle}{dotsc2,t1j-yes}{T1}{yesj}{l}{sc}{}
    \installfont{yesljo8t}{yeslo8r,yeslo8sr,newlatin option nosc,yeslo8r suffix .oldstyle}{t1j-yes}{T1}{yesj}{l}{sl}{}
    \installfont{yeslcjo8t}{yeslo8r,yeslo8sr,build-ttsc,newlatin-dotsc,yeslo8r suffix .sc,yeslo8sr suffix .sc,yeslo8r suffix .oldstyle}{dotsc2,t1j-yes}{T1}{yesj}{l}{scit}{}
    \installfont{yessj8t}{yess8r,yess8sr,newlatin option nosc,yess8r suffix .oldstyle}{t1j-yes}{T1}{yesj}{sb}{n}{}
    \installfont{yesscj8t}{yess8r,yess8sr,build-ttsc,newlatin-dotsc,yess8r suffix .sc,yess8sr suffix .sc,yess8r suffix .oldstyle}{dotsc2,t1j-yes}{T1}{yesj}{sb}{sc}{}
    \installfont{yessjo8t}{yesso8r,yesso8sr,newlatin option nosc,yesso8r suffix .oldstyle}{t1j-yes}{T1}{yesj}{sb}{sl}{}
    \installfont{yesscjo8t}{yesso8r,yesso8sr,build-ttsc,newlatin-dotsc,yesso8r suffix .sc,yesso8sr suffix .sc,yesso8r suffix .oldstyle}{dotsc2,t1j-yes}{T1}{yesj}{sb}{scit}{}
    \installfamily{T1}{yesw}{}
    \installfont{yesrw8t}{yesr8r,yesr8sr,newlatin option nosc}{t1-yesw}{T1}{yesw}{m}{n}{}
    \installfont{yesrcw8t}{yesr8r,yesr8sr,build-ttsc,newlatin-dotsc,yesr8r suffix .sc,yesr8sr suffix .sc}{dotsc2,t1-yesw-sc}{T1}{yesw}{m}{sc}{}
    \installfont{yesrow8t}{yesro8r,yesro8sr,newlatin option nosc}{t1-yesw}{T1}{yesw}{m}{sl}{}
    \installfont{yesrcow8t}{yesro8r,yesro8sr,build-ttsc,newlatin-dotsc,yesro8r suffix .sc,yesro8sr suffix .sc}{dotsc2,t1-yesw-sc}{T1}{yesw}{m}{scit}{}
    \installfont{yesbw8t}{yesb8r,yesb8sr,newlatin option nosc}{t1-yesw}{T1}{yesw}{b}{n}{}
    \installfont{yesbcw8t}{yesb8r,yesb8sr,build-ttsc,newlatin-dotsc,yesb8r suffix .sc,yesb8sr suffix .sc}{dotsc2,t1-yesw-sc}{T1}{yesw}{b}{sc}{}
    \installfont{yesbow8t}{yesbo8r,yesbo8sr,newlatin option nosc}{t1-yesw}{T1}{yesw}{b}{sl}{}
    \installfont{yesbcow8t}{yesbo8r,yesbo8sr,build-ttsc,newlatin-dotsc,yesbo8r suffix .sc,yesbo8sr suffix .sc}{dotsc2,t1-yesw-sc}{T1}{yesw}{b}{scit}{}
    \installfont{yeslw8t}{yesl8r,yesl8sr,newlatin option nosc}{t1-yesw}{T1}{yesw}{l}{n}{}
    \installfont{yeslcw8t}{yesl8r,yesl8sr,build-ttsc,newlatin-dotsc,yesl8r suffix .sc,yesl8sr suffix .sc}{dotsc2,t1-yesw-sc}{T1}{yesw}{l}{sc}{}
    \installfont{yeslow8t}{yeslo8r,yeslo8sr,newlatin option nosc}{t1-yesw}{T1}{yesw}{l}{sl}{}
    \installfont{yeslcow8t}{yeslo8r,yeslo8sr,build-ttsc,newlatin-dotsc,yeslo8r suffix .sc,yeslo8sr suffix .sc}{dotsc2,t1-yesw-sc}{T1}{yesw}{l}{scit}{}
    \installfont{yessw8t}{yess8r,yess8sr,newlatin option nosc}{t1-yesw}{T1}{yesw}{sb}{n}{}
    \installfont{yesscw8t}{yess8r,yess8sr,build-ttsc,newlatin-dotsc,yess8r suffix .sc,yess8sr suffix .sc}{dotsc2,t1-yesw-sc}{T1}{yesw}{sb}{sc}{}
    \installfont{yessow8t}{yesso8r,yesso8sr,newlatin option nosc}{t1-yesw}{T1}{yesw}{sb}{sl}{}
    \installfont{yesscow8t}{yesso8r,yesso8sr,build-ttsc,newlatin-dotsc,yesso8r suffix .sc,yesso8sr suffix .sc}{dotsc2,t1-yesw-sc}{T1}{yesw}{sb}{scit}{}
    \installfamily{T1}{yesjw}{}
    \installfont{yesrjw8t}{yesr8r,yesr8sr,newlatin option nosc,yesr8r suffix .oldstyle}{t1j-yesw}{T1}{yesjw}{m}{n}{}
    \installfont{yesrcjw8t}{yesr8r,yesr8sr,build-ttsc,newlatin-dotsc,yesr8r suffix .sc,yesr8sr suffix .sc,yesr8r suffix .oldstyle}{dotsc2,t1j-yesw-sc}{T1}{yesjw}{m}{sc}{}
    \installfont{yesrjow8t}{yesro8r,yesro8sr,newlatin option nosc,yesro8r suffix .oldstyle}{t1j-yesw}{T1}{yesjw}{m}{sl}{}
    \installfont{yesrcjow8t}{yesro8r,yesro8sr,build-ttsc,newlatin-dotsc,yesro8r suffix .sc,yesro8sr suffix .sc,yesro8r suffix .oldstyle}{dotsc2,t1j-yesw-sc}{T1}{yesjw}{m}{scit}{}
    \installfont{yesbjw8t}{yesb8r,yesb8sr,newlatin option nosc,yesb8r suffix .oldstyle}{t1j-yesw}{T1}{yesjw}{b}{n}{}
    \installfont{yesbcjw8t}{yesb8r,yesb8sr,build-ttsc,newlatin-dotsc,yesb8r suffix .sc,yesb8sr suffix .sc,yesb8r suffix .oldstyle}{dotsc2,t1j-yesw-sc}{T1}{yesjw}{b}{sc}{}
    \installfont{yesbjow8t}{yesbo8r,yesbo8sr,newlatin option nosc,yesbo8r suffix .oldstyle}{t1j-yesw}{T1}{yesjw}{b}{sl}{}
    \installfont{yesbcjow8t}{yesbo8r,yesbo8sr,build-ttsc,newlatin-dotsc,yesbo8r suffix .sc,yesbo8sr suffix .sc,yesbo8r suffix .oldstyle}{dotsc2,t1j-yesw-sc}{T1}{yesjw}{b}{scit}{}
    \installfont{yesljw8t}{yesl8r,yesl8sr,newlatin option nosc,yesl8r suffix .oldstyle}{t1j-yesw}{T1}{yesjw}{l}{n}{}
    \installfont{yeslcjw8t}{yesl8r,yesl8sr,build-ttsc,newlatin-dotsc,yesl8r suffix .sc,yesl8sr suffix .sc,yesl8r suffix .oldstyle}{dotsc2,t1j-yesw-sc}{T1}{yesjw}{l}{sc}{}
    \installfont{yesljow8t}{yeslo8r,yeslo8sr,newlatin option nosc,yeslo8r suffix .oldstyle}{t1j-yesw}{T1}{yesjw}{l}{sl}{}
    \installfont{yeslcjow8t}{yeslo8r,yeslo8sr,build-ttsc,newlatin-dotsc,yeslo8r suffix .sc,yeslo8sr suffix .sc,yeslo8r suffix .oldstyle}{dotsc2,t1j-yesw-sc}{T1}{yesjw}{l}{scit}{}
    \installfont{yessjw8t}{yess8r,yess8sr,newlatin option nosc,yess8r suffix .oldstyle}{t1j-yesw}{T1}{yesjw}{sb}{n}{}
    \installfont{yesscjw8t}{yess8r,yess8sr,build-ttsc,newlatin-dotsc,yess8r suffix .sc,yess8sr suffix .sc,yess8r suffix .oldstyle}{dotsc2,t1j-yesw-sc}{T1}{yesjw}{sb}{sc}{}
    \installfont{yessjow8t}{yesso8r,yesso8sr,newlatin option nosc,yesso8r suffix .oldstyle}{t1j-yesw}{T1}{yesjw}{sb}{sl}{}
    \installfont{yesscjow8t}{yesso8r,yesso8sr,build-ttsc,newlatin-dotsc,yesso8r suffix .sc,yesso8sr suffix .sc,yesso8r suffix .oldstyle}{dotsc2,t1j-yesw-sc}{T1}{yesjw}{sb}{scit}{}
    \installfamily{T1}{yes0}{}
    \installfont{yesr08t}{yesr8r,yesr8sr,newlatin option nosc}{t1-dotinf}{T1}{yes0}{m}{n}{}
    \installfont{yesr0o8t}{yesro8r,yesro8sr,newlatin option nosc}{t1-dotinf}{T1}{yes0}{m}{sl}{}
    \installfont{yesb08t}{yesb8r,yesb8sr,newlatin option nosc}{t1-dotinf}{T1}{yes0}{b}{n}{}
    \installfont{yesb0o8t}{yesbo8r,yesbo8sr,newlatin option nosc}{t1-dotinf}{T1}{yes0}{b}{sl}{}
    \installfont{yesl08t}{yesl8r,yesl8sr,newlatin option nosc}{t1-dotinf}{T1}{yes0}{l}{n}{}
    \installfont{yesl0o8t}{yeslo8r,yeslo8sr,newlatin option nosc}{t1-dotinf}{T1}{yes0}{l}{sl}{}
    \installfont{yess08t}{yess8r,yess8sr,newlatin option nosc}{t1-dotinf}{T1}{yes0}{sb}{n}{}
    \installfont{yess0o8t}{yesso8r,yesso8sr,newlatin option nosc}{t1-dotinf}{T1}{yes0}{sb}{sl}{}
    \installfamily{T1}{yes1}{}
    \installfont{yesr18t}{yesr8r,yesr8sr,newlatin option nosc}{t1-dotsup}{T1}{yes1}{m}{n}{}
    \installfont{yesr1o8t}{yesro8r,yesro8sr,newlatin option nosc}{t1-dotsup}{T1}{yes1}{m}{sl}{}
    \installfont{yesb18t}{yesb8r,yesb8sr,newlatin option nosc}{t1-dotsup}{T1}{yes1}{b}{n}{}
    \installfont{yesb1o8t}{yesbo8r,yesbo8sr,newlatin option nosc}{t1-dotsup}{T1}{yes1}{b}{sl}{}
    \installfont{yesl18t}{yesl8r,yesl8sr,newlatin option nosc}{t1-dotsup}{T1}{yes1}{l}{n}{}
    \installfont{yesl1o8t}{yeslo8r,yeslo8sr,newlatin option nosc}{t1-dotsup}{T1}{yes1}{l}{sl}{}
    \installfont{yess18t}{yess8r,yess8sr,newlatin option nosc}{t1-dotsup}{T1}{yes1}{sb}{n}{}
    \installfont{yess1o8t}{yesso8r,yesso8sr,newlatin option nosc}{t1-dotsup}{T1}{yes1}{sb}{sl}{}
    \installfamily{TS1}{yes}{}
    \installfont{yesr8c}{yesr8r,yesr8sr,yes-ts1-buildper,textcomp}{ts1-yes}{TS1}{yes}{m}{n}{}
    \installfont{yesro8c}{yesro8r,yesro8sr,yes-ts1-buildper,textcomp}{ts1-yes}{TS1}{yes}{m}{sl}{}
    \installfont{yesb8c}{yesb8r,yesb8sr,yes-ts1-buildper,textcomp}{ts1-yes}{TS1}{yes}{b}{n}{}
    \installfont{yesbo8c}{yesbo8r,yesbo8sr,yes-ts1-buildper,textcomp}{ts1-yes}{TS1}{yes}{b}{sl}{}
    \installfont{yesl8c}{yesl8r,yesl8sr,yes-ts1-buildper,textcomp}{ts1-yes}{TS1}{yes}{l}{n}{}
    \installfont{yeslo8c}{yeslo8r,yeslo8sr,yes-ts1-buildper,textcomp}{ts1-yes}{TS1}{yes}{l}{sl}{}
    \installfont{yess8c}{yess8r,yess8sr,yes-ts1-buildper,textcomp}{ts1-yes}{TS1}{yes}{sb}{n}{}
    \installfont{yesso8c}{yesso8r,yesso8sr,yes-ts1-buildper,textcomp}{ts1-yes}{TS1}{yes}{sb}{sl}{}
    \installfontas{yesr8c}{TS1}{yes}{m}{sc}{}
    \installfontas{yesro8c}{TS1}{yes}{m}{scit}{}
    \installfontas{yesb8c}{TS1}{yes}{b}{sc}{}
    \installfontas{yesbo8c}{TS1}{yes}{b}{scit}{}
    \installfontas{yesl8c}{TS1}{yes}{l}{sc}{}
    \installfontas{yeslo8c}{TS1}{yes}{l}{scit}{}
    \installfontas{yess8c}{TS1}{yes}{sb}{sc}{}
    \installfontas{yesso8c}{TS1}{yes}{sb}{scit}{}
    \installfamily{TS1}{yesj}{}
    \installfont{yesrj8c}{yesr8r,yesr8sr,yes-ts1-buildper,textcomp}{ts1-dotoldstyle-yes}{TS1}{yesj}{m}{n}{}
    \installfont{yesrjo8c}{yesro8r,yesro8sr,yes-ts1-buildper,textcomp}{ts1-dotoldstyle-yes}{TS1}{yesj}{m}{sl}{}
    \installfont{yesbj8c}{yesb8r,yesb8sr,yes-ts1-buildper,textcomp}{ts1-dotoldstyle-yes}{TS1}{yesj}{b}{n}{}
    \installfont{yesbjo8c}{yesbo8r,yesbo8sr,yes-ts1-buildper,textcomp}{ts1-dotoldstyle-yes}{TS1}{yesj}{b}{sl}{}
    \installfont{yeslj8c}{yesl8r,yesl8sr,yes-ts1-buildper,textcomp}{ts1-dotoldstyle-yes}{TS1}{yesj}{l}{n}{}
    \installfont{yesljo8c}{yeslo8r,yeslo8sr,yes-ts1-buildper,textcomp}{ts1-dotoldstyle-yes}{TS1}{yesj}{l}{sl}{}
    \installfont{yessj8c}{yess8r,yess8sr,yes-ts1-buildper,textcomp}{ts1-dotoldstyle-yes}{TS1}{yesj}{sb}{n}{}
    \installfont{yessjo8c}{yesso8r,yesso8sr,yes-ts1-buildper,textcomp}{ts1-dotoldstyle-yes}{TS1}{yesj}{sb}{sl}{}
    \installfontas{yesrj8c}{TS1}{yesj}{m}{sc}{}
    \installfontas{yesrjo8c}{TS1}{yesj}{m}{scit}{}
    \installfontas{yesbj8c}{TS1}{yesj}{b}{sc}{}
    \installfontas{yesbjo8c}{TS1}{yesj}{b}{scit}{}
    \installfontas{yeslj8c}{TS1}{yesj}{l}{sc}{}
    \installfontas{yesljo8c}{TS1}{yesj}{l}{scit}{}
    \installfontas{yessj8c}{TS1}{yesj}{sb}{sc}{}
    \installfontas{yessjo8c}{TS1}{yesj}{sb}{scit}{}
    \installfamily{TS1}{yesw}{}
    \installfontas{yesr8c}{TS1}{yesw}{m}{n}{}
    \installfontas{yesro8c}{TS1}{yesw}{m}{sl}{}
    \installfontas{yesb8c}{TS1}{yesw}{b}{n}{}
    \installfontas{yesbo8c}{TS1}{yesw}{b}{sl}{}
    \installfontas{yesl8c}{TS1}{yesw}{l}{n}{}
    \installfontas{yeslo8c}{TS1}{yesw}{l}{sl}{}
    \installfontas{yess8c}{TS1}{yesw}{sb}{n}{}
    \installfontas{yesso8c}{TS1}{yesw}{sb}{sl}{}
    \installfontas{yesr8c}{TS1}{yesw}{m}{sc}{}
    \installfontas{yesro8c}{TS1}{yesw}{m}{scit}{}
    \installfontas{yesb8c}{TS1}{yesw}{b}{sc}{}
    \installfontas{yesbo8c}{TS1}{yesw}{b}{scit}{}
    \installfontas{yesl8c}{TS1}{yesw}{l}{sc}{}
    \installfontas{yeslo8c}{TS1}{yesw}{l}{scit}{}
    \installfontas{yess8c}{TS1}{yesw}{sb}{sc}{}
    \installfontas{yesso8c}{TS1}{yesw}{sb}{scit}{}
    \installfamily{TS1}{yesjw}{}
    \installfontas{yesrj8c}{TS1}{yesjw}{m}{n}{}
    \installfontas{yesrjo8c}{TS1}{yesjw}{m}{sl}{}
    \installfontas{yesbj8c}{TS1}{yesjw}{b}{n}{}
    \installfontas{yesbjo8c}{TS1}{yesjw}{b}{sl}{}
    \installfontas{yeslj8c}{TS1}{yesjw}{l}{n}{}
    \installfontas{yesljo8c}{TS1}{yesjw}{l}{sl}{}
    \installfontas{yessj8c}{TS1}{yesjw}{sb}{n}{}
    \installfontas{yessjo8c}{TS1}{yesjw}{sb}{sl}{}
    \installfontas{yesrj8c}{TS1}{yesjw}{m}{sc}{}
    \installfontas{yesrjo8c}{TS1}{yesjw}{m}{scit}{}
    \installfontas{yesbj8c}{TS1}{yesjw}{b}{sc}{}
    \installfontas{yesbjo8c}{TS1}{yesjw}{b}{scit}{}
    \installfontas{yeslj8c}{TS1}{yesjw}{l}{sc}{}
    \installfontas{yesljo8c}{TS1}{yesjw}{l}{scit}{}
    \installfontas{yessj8c}{TS1}{yesjw}{sb}{sc}{}
    \installfontas{yessjo8c}{TS1}{yesjw}{sb}{scit}{}
    \installfamily{TS1}{yes0}{}
    \installfont{yesr08c}{yesr8r,yesr8sr,yes-ts1-buildper,textcomp}{ts1-dotinf}{TS1}{yes0}{m}{n}{}
    \installfont{yesr0o8c}{yesro8r,yesro8sr,yes-ts1-buildper,textcomp}{ts1-dotinf}{TS1}{yes0}{m}{sl}{}
    \installfont{yesb08c}{yesb8r,yesb8sr,yes-ts1-buildper,textcomp}{ts1-dotinf}{TS1}{yes0}{b}{n}{}
    \installfont{yesb0o8c}{yesbo8r,yesbo8sr,yes-ts1-buildper,textcomp}{ts1-dotinf}{TS1}{yes0}{b}{sl}{}
    \installfont{yesl08c}{yesl8r,yesl8sr,yes-ts1-buildper,textcomp}{ts1-dotinf}{TS1}{yes0}{l}{n}{}
    \installfont{yesl0o8c}{yeslo8r,yeslo8sr,yes-ts1-buildper,textcomp}{ts1-dotinf}{TS1}{yes0}{l}{sl}{}
    \installfont{yess08c}{yess8r,yess8sr,yes-ts1-buildper,textcomp}{ts1-dotinf}{TS1}{yes0}{sb}{n}{}
    \installfont{yess0o8c}{yesso8r,yesso8sr,yes-ts1-buildper,textcomp}{ts1-dotinf}{TS1}{yes0}{sb}{sl}{}
    \installfamily{TS1}{yes1}{}
    \installfont{yesr18c}{yesr8r,yesr8sr,yes-ts1-buildper,textcomp}{ts1-dotsup}{TS1}{yes1}{m}{n}{}
    \installfont{yesr1o8c}{yesro8r,yesro8sr,yes-ts1-buildper,textcomp}{ts1-dotsup}{TS1}{yes1}{m}{sl}{}
    \installfont{yesb18c}{yesb8r,yesb8sr,yes-ts1-buildper,textcomp}{ts1-dotsup}{TS1}{yes1}{b}{n}{}
    \installfont{yesb1o8c}{yesbo8r,yesbo8sr,yes-ts1-buildper,textcomp}{ts1-dotsup}{TS1}{yes1}{b}{sl}{}
    \installfont{yesl18c}{yesl8r,yesl8sr,yes-ts1-buildper,textcomp}{ts1-dotsup}{TS1}{yes1}{l}{n}{}
    \installfont{yesl1o8c}{yeslo8r,yeslo8sr,yes-ts1-buildper,textcomp}{ts1-dotsup}{TS1}{yes1}{l}{sl}{}
    \installfont{yess18c}{yess8r,yess8sr,yes-ts1-buildper,textcomp}{ts1-dotsup}{TS1}{yes1}{sb}{n}{}
    \installfont{yess1o8c}{yesso8r,yesso8sr,yes-ts1-buildper,textcomp}{ts1-dotsup}{TS1}{yes1}{sb}{sl}{}
  \endinstallfonts
\endrecordtransforms
\bye
%    \end{macrocode}
% \iffalse
%</drv>
% \fi
% \subsection{Map}
% 
% This file is compiled to produce the map file fragment \verb|updmap| 
% needs to install the fonts.
% It uses files recorded during compilation of the driver.
% \iffalse
%<*map>
% \fi
%    \begin{macrocode}
\input finstmsc.sty
\resetstr{PSfontsuffix}{.pfb}
\adddriver{dvips}{yes.map}
\adddriver{pltotf}{yes-pltotf.sh}
\input yes-rec.tex
\donedrivers
\bye
%    \end{macrocode}
% \iffalse
%</map>
% \fi
% \subsection{Supplementary (raw)}
% 
% We need an additional ‘raw’ encoding to pick up characters otherwise missed.
% Many of these are here just because they are named differently, but this 
% also covers fancy ligatures, alternate styles of digits etc.
%
% Note that \verb|etx| files may specify raw and/or output encodings.
% Those which are specific to Electrum are described below and included in
% this file.
% Those which are not are included as separate sources unless provided by
% \verb|fontinst|.
% \iffalse
%<*supp-yes>
% \fi
%    \begin{macrocode}
\relax
\encoding
  \setslot{f_f}\endsetslot
  \setslot{f_f_i}\endsetslot
  \setslot{f_f_l}\endsetslot
  \setslot{longs_t}\endsetslot
  \setslot{s_t}\endsetslot
  \setslot{i_t}\endsetslot
  \setslot{longs}\endsetslot
  \setslot{uni00B2}\endsetslot
  \setslot{uni00B3}\endsetslot
  \setslot{uni00B9}\endsetslot
  \setslot{uni2070}\endsetslot
  \setslot{uni2074}\endsetslot
  \setslot{uni2075}\endsetslot
  \setslot{uni2076}\endsetslot
  \setslot{uni2077}\endsetslot
  \setslot{uni2078}\endsetslot
  \setslot{uni2079}\endsetslot
  \setslot{uni207D}\endsetslot
  \setslot{uni207E}\endsetslot
  \setslot{uni207F}\endsetslot
  \setslot{uni2080}\endsetslot
  \setslot{uni2081}\endsetslot
  \setslot{uni2082}\endsetslot
  \setslot{uni2083}\endsetslot
  \setslot{uni2084}\endsetslot
  \setslot{uni2085}\endsetslot
  \setslot{uni2086}\endsetslot
  \setslot{uni2087}\endsetslot
  \setslot{uni2088}\endsetslot
  \setslot{uni2089}\endsetslot
  \setslot{uni208D}\endsetslot
  \setslot{uni208E}\endsetslot
  \setslot{afii61289}\endsetslot
  \setslot{afii61352}\endsetslot
  \setslot{uni2155}\endsetslot
  \setslot{uni2156}\endsetslot
  \setslot{uni2157}\endsetslot
  \setslot{uni2158}\endsetslot
  \setslot{uni2159}\endsetslot
  \setslot{uni215A}\endsetslot
  \setslot{uniE023}\endsetslot
  \setslot{uniE099}\endsetslot
  \setslot{uniE0A4}\endsetslot
  \setslot{uniE0D5}\endsetslot
  \setslot{uniE0DB}\endsetslot
  \setslot{uniE0DD}\endsetslot
  \setslot{uniE101}\endsetslot
  \setslot{uniE1A2}\endsetslot
  \setslot{uniE09C}\endsetslot
  \setslot{uniE087}\endsetslot
  \setslot{uniE0B3}\endsetslot
  \setslot{estimated}\endsetslot
  \setslot{Euro.inferior}\endsetslot
  \setslot{cent.inferior}\endsetslot
  \setslot{comma.inferior}\endsetslot
  \setslot{dollar.inferior}\endsetslot
  \setslot{eight.inferior}\endsetslot
  \setslot{five.inferior}\endsetslot
  \setslot{four.inferior}\endsetslot
  \setslot{hyphen.inferior}\endsetslot
  \setslot{nine.inferior}\endsetslot
  \setslot{one.inferior}\endsetslot
  \setslot{parenleft.inferior}\endsetslot
  \setslot{parenright.inferior}\endsetslot
  \setslot{period.inferior}\endsetslot
  \setslot{plus.inferior}\endsetslot
  \setslot{seven.inferior}\endsetslot
  \setslot{six.inferior}\endsetslot
  \setslot{sterling.inferior}\endsetslot
  \setslot{three.inferior}\endsetslot
  \setslot{two.inferior}\endsetslot
  \setslot{yen.inferior}\endsetslot
  \setslot{zero.inferior}\endsetslot
  \setslot{Euro.oldstyle}\endsetslot
  \setslot{cent.oldstyle}\endsetslot
  \setslot{dollar.oldstyle}\endsetslot
  \setslot{eight.oldstyle}\endsetslot
  \setslot{five.oldstyle}\endsetslot
  \setslot{four.oldstyle}\endsetslot
  \setslot{nine.oldstyle}\endsetslot
  \setslot{one.oldstyle}\endsetslot
  \setslot{percent.oldstyle}\endsetslot
  \setslot{perthousand.oldstyle}\endsetslot
  \setslot{seven.oldstyle}\endsetslot
  \setslot{six.oldstyle}\endsetslot
  \setslot{sterling.oldstyle}\endsetslot
  \setslot{three.oldstyle}\endsetslot
  \setslot{two.oldstyle}\endsetslot
  \setslot{yen.oldstyle}\endsetslot
  \setslot{zero.oldstyle}\endsetslot
  \setslot{a.sc}\endsetslot
  \setslot{aacute.sc}\endsetslot
  \setslot{acircumflex.sc}\endsetslot
  \setslot{adieresis.sc}\endsetslot
  \setslot{ae.sc}\endsetslot
  \setslot{agrave.sc}\endsetslot
  \setslot{ampersand.sc}\endsetslot
  \setslot{aring.sc}\endsetslot
  \setslot{atilde.sc}\endsetslot
  \setslot{b.sc}\endsetslot
  \setslot{c.sc}\endsetslot
  \setslot{ccedilla.sc}\endsetslot
  \setslot{d.sc}\endsetslot
  \setslot{e.sc}\endsetslot
  \setslot{eacute.sc}\endsetslot
  \setslot{ecircumflex.sc}\endsetslot
  \setslot{edieresis.sc}\endsetslot
  \setslot{egrave.sc}\endsetslot
  \setslot{eth.sc}\endsetslot
  \setslot{f.sc}\endsetslot
  \setslot{fi.sc}\endsetslot
  \setslot{fl.sc}\endsetslot
  \setslot{g.sc}\endsetslot
  \setslot{germandbls.sc}\endsetslot
  \setslot{h.sc}\endsetslot
  \setslot{i.sc}\endsetslot
  \setslot{iacute.sc}\endsetslot
  \setslot{icircumflex.sc}\endsetslot
  \setslot{idieresis.sc}\endsetslot
  \setslot{igrave.sc}\endsetslot
  \setslot{j.sc}\endsetslot
  \setslot{k.sc}\endsetslot
  \setslot{l.sc}\endsetslot
  \setslot{lslash.sc}\endsetslot
  \setslot{m.sc}\endsetslot
  \setslot{n.sc}\endsetslot
  \setslot{ntilde.sc}\endsetslot
  \setslot{o.sc}\endsetslot
  \setslot{oacute.sc}\endsetslot
  \setslot{ocircumflex.sc}\endsetslot
  \setslot{odieresis.sc}\endsetslot
  \setslot{oe.sc}\endsetslot
  \setslot{ograve.sc}\endsetslot
  \setslot{oslash.sc}\endsetslot
  \setslot{otilde.sc}\endsetslot
  \setslot{p.sc}\endsetslot
  \setslot{q.sc}\endsetslot
  \setslot{r.sc}\endsetslot
  \setslot{s.sc}\endsetslot
  \setslot{scaron.sc}\endsetslot
  \setslot{t.sc}\endsetslot
  \setslot{thorn.sc}\endsetslot
  \setslot{u.sc}\endsetslot
  \setslot{uacute.sc}\endsetslot
  \setslot{ucircumflex.sc}\endsetslot
  \setslot{udieresis.sc}\endsetslot
  \setslot{ugrave.sc}\endsetslot
  \setslot{v.sc}\endsetslot
  \setslot{w.sc}\endsetslot
  \setslot{x.sc}\endsetslot
  \setslot{y.sc}\endsetslot
  \setslot{yacute.sc}\endsetslot
  \setslot{ydieresis.sc}\endsetslot
  \setslot{z.sc}\endsetslot
  \setslot{zcaron.sc}\endsetslot
  \setslot{Euro.superior}\endsetslot
  \setslot{a.superior}\endsetslot
  \setslot{b.superior}\endsetslot
  \setslot{cent.superior}\endsetslot
  \setslot{comma.superior}\endsetslot
  \setslot{d.superior}\endsetslot
  \setslot{dollar.superior}\endsetslot
  \setslot{e.superior}\endsetslot
  \setslot{eight.superior}\endsetslot
  \setslot{five.superior}\endsetslot
  \setslot{four.superior}\endsetslot
  \setslot{hyphen.superior}\endsetslot
  \setslot{i.superior}\endsetslot
  \setslot{l.superior}\endsetslot
  \setslot{m.superior}\endsetslot
  \setslot{n.superior}\endsetslot
  \setslot{nine.superior}\endsetslot
  \setslot{o.superior}\endsetslot
  \setslot{one.superior}\endsetslot
  \setslot{parenleft.superior}\endsetslot
  \setslot{parenright.superior}\endsetslot
  \setslot{period.superior}\endsetslot
  \setslot{plus.superior}\endsetslot
  \setslot{r.superior}\endsetslot
  \setslot{s.superior}\endsetslot
  \setslot{seven.superior}\endsetslot
  \setslot{six.superior}\endsetslot
  \setslot{sterling.superior}\endsetslot
  \setslot{t.superior}\endsetslot
  \setslot{three.superior}\endsetslot
  \setslot{two.superior}\endsetslot
  \setslot{yen.superior}\endsetslot
  \setslot{zero.superior}\endsetslot
  \setslot{zero.denominator}\endsetslot
\endencoding
%    \end{macrocode}
% \iffalse
%</supp-yes>
% \fi
% \subsection{Reglyph}
% We need to rename characters whose names don't match our TeX font
% encodings.
% \iffalse
%<*reglyph>
% \fi
%    \begin{macrocode}
\relax
\reglyphfonts
  \renameglyph{two.superior}{uni00B2}
  \renameglyph{three.superior}{uni00B3}
  \renameglyph{one.superior}{uni00B9}
  \renameglyph{zero.superior}{uni2070}
  \renameglyph{four.superior}{uni2074}
  \renameglyph{five.superior}{uni2075}
  \renameglyph{six.superior}{uni2076}
  \renameglyph{seven.superior}{uni2077}
  \renameglyph{eight.superior}{uni2078}
  \renameglyph{nine.superior}{uni2079}
  \renameglyph{parenleft.superior}{uni207D}
  \renameglyph{parenright.superior}{uni207E}
  \renameglyph{n.superior}{uni207F}
  \renameglyph{zero.inferior}{uni2080}
  \renameglyph{one.inferior}{uni2081}
  \renameglyph{two.inferior}{uni2082}
  \renameglyph{three.inferior}{uni2083}
  \renameglyph{four.inferior}{uni2084}
  \renameglyph{five.inferior}{uni2085}
  \renameglyph{six.inferior}{uni2086}
  \renameglyph{seven.inferior}{uni2087}
  \renameglyph{eight.inferior}{uni2088}
  \renameglyph{nine.inferior}{uni2089}
  \renameglyph{parenleft.inferior}{uni208D}
  \renameglyph{parenright.inferior}{uni208E}
  \renameglyph{lscript}{afii61289}
  \renameglyph{numero}{afii61352}
  \renameglyph{onefifth}{uni2155}
  \renameglyph{twofifths}{uni2156}
  \renameglyph{threefifths}{uni2157}
  \renameglyph{fourfifths}{uni2158}
  \renameglyph{onesixth}{uni2159}
  \renameglyph{fivesixths}{uni215A}
  \renameglyph{Euro.oldstyle}{uniE023}
  \renameglyph{fi.sc}{uniE099}
  \renameglyph{germandbls.sc}{uniE0A4}
  \renameglyph{t_t}{uniE0D5}
  \renameglyph{s_p}{uniE0DB}
  \renameglyph{sterling.oldstyle}{uniE0DD}
  \renameglyph{yen.oldstyle}{uniE101}
  \renameglyph{Qalt}{uniE1A2}
  \renameglyph{fl.sc}{uniE09C}
  \renameglyph{c_t}{uniE087}
  \renameglyph{ampersand.oldstyle}{uniE0B3}
  \reglyphfont{yesb8sr}{yesb8s}
  \reglyphfont{yesbo8sr}{yesbo8s}
  \reglyphfont{yesl8sr}{yesl8s}
  \reglyphfont{yeslo8sr}{yeslo8s}
  \reglyphfont{yesr8sr}{yesr8s}
  \reglyphfont{yesro8sr}{yesro8s}
  \reglyphfont{yess8sr}{yess8s}
  \reglyphfont{yesso8sr}{yesso8s}
\endreglyphfonts
%    \end{macrocode}
% \iffalse
%</reglyph>
% \fi
% \subsection{Encodings (output)}
% These files define variant T1 and TS1 font encodings.
%
% In addition to these encodings, we use encoding files supplied by \verb|fontinst|
% and some custom files not included in this package's \verb|dtx| as they are not
% specific to Electrum.
% They are, however, part of the package:
% \begin{itemize}
%   \item dotsc2.etx
%   \item t1-dotinf.etx
%   \item t1-dotsup.etx
%   \item ts1-dotinf.etx
%   \item ts1-dotsup.etx
% \end{itemize}
% The \verb|etx| files are not used directly by \LaTeX{} or \TeX.
% Where needed, they are processed to produce \verb|enc| files.
% In some cases, however, they are not themselves standalone encodings.
% Instead, they change how some other encoding is interpreted.
% \iffalse
%<*t1-yes>
% \fi
%    \begin{macrocode}
\relax
\encoding

\needsfontinstversion{1.910}

\setcommand\lc#1#2{#2}
\setcommand\uc#1#2{#1}
\setcommand\lctop#1#2{#2}
\setcommand\uctop#1#2{#1}
\setcommand\lclig#1#2{#2}
\ifisint{letterspacing}\then
   \ifnumber{\int{letterspacing}}={0}\then \Else
      \setcommand\uclig#1#2{#1spaced}
      \comment{Here we set \verb|\uclig#1#2| to \verb|#1spaced|, but 
      you can't see it as \verb|\setcommand| commands are invisible in 
      the typeset output.}
   \Fi
\Fi
\setcommand\uclig#1#2{#1}
\setcommand\digit#1{#1}

\ifisint{monowidth}\then
   \setint{ligaturing}{0}
\Else
   % The following empty line is *important* to get the formatting
   % right here (sigh)! (Remember that it is a \par token.)
   
   \ifisint{letterspacing}\then
      \ifnumber{\int{letterspacing}}={0}\then \Else
         \setint{ligaturing}{0}
      \Fi
   \Fi
  \setint{ligaturing}{1}
\Fi

\setint{italicslant}{0}
\setint{quad}{1000}
\setint{baselineskip}{1200}

\ifisglyph{x}\then
   \setint{xheight}{\height{x}}
\Else
   \setint{xheight}{500}
\Fi

\ifisglyph{space}\then
   \setint{interword}{\width{space}}
\Else\ifisglyph{i}\then
   \setint{interword}{\width{i}}
\Else
   \setint{interword}{333}
\Fi\Fi

\ifisint{monowidth}\then
   \setint{stretchword}{0}
   \setint{shrinkword}{0}
   \setint{extraspace}{\int{interword}}
\Else
   \setint{stretchword}{\scale{\int{interword}}{600}}
   \setint{shrinkword}{\scale{\int{interword}}{240}}
   \setint{extraspace}{\scale{\int{interword}}{240}}
\Fi

\ifisglyph{X}\then
   \setint{capheight}{\height{X}}
\Else
   \setint{capheight}{750}
\Fi

\ifisglyph{d}\then
   \setint{ascender}{\height{d}}
\Else\ifisint{capheight}\then
   \setint{ascender}{\int{capheight}}
\Else
   \setint{ascender}{750}
\Fi\Fi

\ifisglyph{Aring}\then
   \setint{acccapheight}{\height{Aring}}
\Else
   \setint{acccapheight}{999}
\Fi

\ifisint{descender_neg}\then
   \setint{descender}{\neg{\int{descender_neg}}}
\Else\ifisglyph{p}\then
   \setint{descender}{\depth{p}}
\Else
   \setint{descender}{250}
\Fi\Fi

\ifisglyph{Aring}\then
   \setint{maxheight}{\height{Aring}}
\Else
   \setint{maxheight}{1000}
\Fi

\ifisint{maxdepth_neg}\then
   \setint{maxdepth}{\neg{\int{maxdepth_neg}}}
\Else\ifisglyph{j}\then
   \setint{maxdepth}{\depth{j}}
\Else
   \setint{maxdepth}{250}
\Fi\Fi

\ifisglyph{six}\then
   \setint{digitwidth}{\width{six}}
\Else
   \setint{digitwidth}{500}
\Fi

\setint{capstem}{0} % not in AFM files

\setfontdimen{1}{italicslant}    % italic slant
\setfontdimen{2}{interword}      % interword space
\setfontdimen{3}{stretchword}    % interword stretch
\setfontdimen{4}{shrinkword}     % interword shrink
\setfontdimen{5}{xheight}        % x-height
\setfontdimen{6}{quad}           % quad
\setfontdimen{7}{extraspace}     % extra space after .
\setfontdimen{8}{capheight}      % cap height
\setfontdimen{9}{ascender}       % ascender
\setfontdimen{10}{acccapheight}  % accented cap height
\setfontdimen{11}{descender}     % descender's depth
\setfontdimen{12}{maxheight}     % max height
\setfontdimen{13}{maxdepth}      % max depth
\setfontdimen{14}{digitwidth}    % digit width
\setfontdimen{15}{verticalstem}  % dominant width of verical stems
\setfontdimen{16}{baselineskip}  % baselineskip

\ifnumber{\int{ligaturing}}<{0}\then 
   \comment{In this case, the codingscheme can be different from the 
     default, and therefore we refrain from setting it.}
\Else
   \setstr{codingscheme}{EXTENDED TEX ENC - ELECTRUM}
\Fi

\setslot{\lc{Grave}{grave}}
   \comment{The grave accent `\`{}'.}
\endsetslot

\setslot{\lc{Acute}{acute}}
   \comment{The acute accent `\'{}'.}
\endsetslot

\setslot{\lc{Circumflex}{circumflex}}
   \comment{The circumflex accent `\^{}'.}
\endsetslot

\setslot{\lc{Tilde}{tilde}}
   \comment{The tilde accent `\~{}'.}
\endsetslot

\setslot{\lc{Dieresis}{dieresis}}
   \comment{The umlaut or dieresis accent `\"{}'.}
\endsetslot

\setslot{\lc{Hungarumlaut}{hungarumlaut}}
   \comment{The long Hungarian umlaut `\H{}'.}
\endsetslot

\setslot{\lc{Ring}{ring}}
   \comment{The ring accent `\r{}'.}
\endsetslot

\setslot{\lc{Caron}{caron}}
   \comment{The caron or h\'a\v cek accent `\v{}'.}
\endsetslot

\setslot{\lc{Breve}{breve}}
   \comment{The breve accent `\u{}'.}
\endsetslot

\setslot{\lc{Macron}{macron}}
   \comment{The macron accent `\={}'.}
\endsetslot

\setslot{\lc{Dotaccent}{dotaccent}}
   \comment{The dot accent `\.{}'.}
\endsetslot

\setslot{\lc{Cedilla}{cedilla}}
   \comment{The cedilla accent `\c {}'.}
\endsetslot

\setslot{\lc{Ogonek}{ogonek}}
   \comment{The ogonek accent `\k {}'.}
\endsetslot

\setslot{quotesinglbase}
  \comment{A German single quote mark `\quotesinglbase' similar to a comma,
      but with different sidebearings.}
\endsetslot

\setslot{guilsinglleft}
  \comment{A French single opening quote mark `\guilsinglleft',
      unavailable in \plain\ \TeX.}
\endsetslot

\setslot{guilsinglright}
  \comment{A French single closing quote mark `\guilsinglright',
      unavailable in \plain\ \TeX.}
\endsetslot

\setslot{quotedblleft}
  \comment{The English opening quote mark `\,\textquotedblleft\,'.}
\endsetslot

\setslot{quotedblright}
  \comment{The English closing quote mark `\,\textquotedblright\,'.}
\endsetslot

\setslot{quotedblbase}
  \comment{A German double quote mark `\quotedblbase' similar to two commas,
      but with tighter letterspacing and different sidebearings.}
\endsetslot

\setslot{guillemotleft}
  \comment{A French double opening quote mark `\guillemotleft',
      unavailable in \plain\ \TeX.}
\endsetslot

\setslot{guillemotright}
  \comment{A French closing opening quote mark `\guillemotright',
      unavailable in \plain\ \TeX.}
\endsetslot

\setslot{endash}
   \ligature{LIG}{hyphen}{emdash}
   \comment{The number range dash `1--9'. This is called `rangedash' by fontinst's t1.etx, but it needs to be called `endash' to work right. The `\textendash'.  In a monowidth font, this
      might be set as `\texttt{1{-}9}'.}
\endsetslot

\setslot{emdash}
   \comment{The punctuation dash `Oh---boy.' This is calle `punctdash' by fontinst's t1.etx, but needs to be called `emdash' to work right. The `\textemdash'.  In a monowidth font, this
      might be set as `\texttt{Oh{-}{-}boy.}'}
\endsetslot

\setslot{compwordmark}
   \comment{An invisible glyph, with zero width and depth, but the
      height of lowercase letters without ascenders.
      It is used to stop ligaturing in words like `shelf{}ful'.}
\endsetslot

\setslot{zero.denominator}
   \comment{A glyph which is placed after `\%' to produce a
      `per-thousand', or twice to produce `per-ten-thousand'.
      Your guess is as good as mine as to what this glyph should look
      like in a monowidth font.}
\endsetslot

\setslot{\lc{dotlessI}{dotlessi}}
   \comment{A dotless i `\i', used to produce accented letters such as
      `\=\i'.}
\endsetslot

\setslot{\lc{dotlessJ}{dotlessj}}
   \comment{A dotless j `\j', used to produce accented letters such as
      `\=\j'.  Most non-\TeX\ fonts do not have this glyph.}
\endsetslot

\ifnumber{\int{ligaturing}}<{0}\then \skipslots{5}\Else

\setslot{\lclig{FF}{f_f}}
   \ifnumber{\int{ligaturing}}>{0}\then
      \ligature{LIG}{\lc{I}{i}}{\lclig{FFI}{f_f_i}}
      \ligature{LIG}{\lc{L}{l}}{\lclig{FFL}{f_f_l}}
   \Fi
   \comment{The `ff' ligature.  It should be two characters wide in a
      monowidth font.}
\endsetslot

\setslot{\lclig{FI}{fi}}
   \comment{The `fi' ligature.  It should be two characters wide in a
      monowidth font.}
\endsetslot

\setslot{\lclig{FL}{fl}}
   \comment{The `fl' ligature.  It should be two characters wide in a
      monowidth font.}
\endsetslot

\setslot{\lclig{FFI}{f_f_i}}
   \comment{The `ffi' ligature.  It should be three characters wide in a
      monowidth font.}
\endsetslot

\setslot{\lclig{FFL}{f_f_l}}
   \comment{The `ffl' ligature.  It should be three characters wide in a
      monowidth font.}
\endsetslot

\Fi

\setslot{visiblespace}
   \comment{A visible space glyph `\textvisiblespace'.}
\endsetslot

\setslot{exclam}
   \ligature{LIG}{quoteleft}{exclamdown}
   \comment{The exclamation mark `!'.}
\endsetslot

\setslot{quotedbl}
  \comment{The `neutral' double quotation mark `\,\textquotedbl\,',
      included for use in monowidth fonts, or for setting computer
      programs.  Note that the inclusion of this glyph in this slot
      means that \TeX\ documents which used `{\tt\char`\"}' as an
      input character will no longer work.}
\endsetslot

\setslot{numbersign}
   \comment{The hash sign `\#'.}
\endsetslot

\setslot{dollar}
   \comment{The dollar sign `\$'.}
\endsetslot

\setslot{percent}
   \comment{The percent sign `\%'.}
\endsetslot

\setslot{ampersand}
   \comment{The ampersand sign `\&'.}
\endsetslot

\setslot{quoteright}
   \ligature{LIG}{quoteright}{quotedblright}
   \comment{The English closing single quote mark `\,\textquoteright\,'.}
\endsetslot

\setslot{parenleft}
   \comment{The opening parenthesis `('.}
\endsetslot

\setslot{parenright}
   \comment{The closing parenthesis `)'.}
\endsetslot

\setslot{asterisk}
   \comment{The raised asterisk `*'.}
\endsetslot

\setslot{plus}
   \comment{The addition sign `+'.}
\endsetslot

\setslot{comma}
   \ligature{LIG}{comma}{quotedblbase}
   \comment{The comma `,'.}
\endsetslot

\setslot{hyphen}
   \ligature{LIG}{hyphen}{endash}
   \ligature{LIG}{hyphenchar}{hyphenchar}
   \comment{The hyphen `-'.}
\endsetslot

\setslot{period}
   \comment{The period `.'.}
\endsetslot

\setslot{slash}
   \comment{The forward oblique `/'.}
\endsetslot

\setslot{\digit{zero}}
   \comment{The number `0'.  This (and all the other numerals) may be
      old style or ranging digits.}
\endsetslot

\setslot{\digit{one}}
   \comment{The number `1'.}
\endsetslot

\setslot{\digit{two}}
   \comment{The number `2'.}
\endsetslot

\setslot{\digit{three}}
   \comment{The number `3'.}
\endsetslot

\setslot{\digit{four}}
   \comment{The number `4'.}
\endsetslot

\setslot{\digit{five}}
   \comment{The number `5'.}
\endsetslot

\setslot{\digit{six}}
   \comment{The number `6'.}
\endsetslot

\setslot{\digit{seven}}
   \comment{The number `7'.}
\endsetslot

\setslot{\digit{eight}}
   \comment{The number `8'.}
\endsetslot

\setslot{\digit{nine}}
   \comment{The number `9'.}
\endsetslot

\setslot{colon}
   \comment{The colon punctuation mark `:'.}
\endsetslot

\setslot{semicolon}
   \comment{The semi-colon punctuation mark `;'.}
\endsetslot

\setslot{less}
   \ligature{LIG}{less}{guillemotleft}
   \comment{The less-than sign `\textless'.}
\endsetslot

\setslot{equal}
   \comment{The equals sign `='.}
\endsetslot

\setslot{greater}
   \ligature{LIG}{greater}{guillemotright}
   \comment{The greater-than sign `\textgreater'.}
\endsetslot

\setslot{question}
   \ligature{LIG}{quoteleft}{questiondown}
   \comment{The question mark `?'.}
\endsetslot

\setslot{at}
   \comment{The at sign `@'.}
\endsetslot

\setslot{\uc{A}{a}}
   \comment{The letter `{A}'.}
\endsetslot

\setslot{\uc{B}{b}}
   \comment{The letter `{B}'.}
\endsetslot

\setslot{\uc{C}{c}}
   \comment{The letter `{C}'.}
\endsetslot

\setslot{\uc{D}{d}}
   \comment{The letter `{D}'.}
\endsetslot

\setslot{\uc{E}{e}}
   \comment{The letter `{E}'.}
\endsetslot

\setslot{\uc{F}{f}}
   \comment{The letter `{F}'.}
\endsetslot

\setslot{\uc{G}{g}}
   \comment{The letter `{G}'.}
\endsetslot

\setslot{\uc{H}{h}}
   \comment{The letter `{H}'.}
\endsetslot

\ifnumber{\int{ligaturing}}<{-1}\then \skipslots{1}\Else

\setslot{\uc{I}{i}}
   \comment{The letter `{I}'.}
\endsetslot

\Fi

\setslot{\uc{J}{j}}
   \comment{The letter `{J}'.}
\endsetslot

\setslot{\uc{K}{k}}
   \comment{The letter `{K}'.}
\endsetslot

\setslot{\uc{L}{l}}
   \comment{The letter `{L}'.}
\endsetslot

\setslot{\uc{M}{m}}
   \comment{The letter `{M}'.}
\endsetslot

\setslot{\uc{N}{n}}
   \comment{The letter `{N}'.}
\endsetslot

\setslot{\uc{O}{o}}
   \comment{The letter `{O}'.}
\endsetslot

\setslot{\uc{P}{p}}
   \comment{The letter `{P}'.}
\endsetslot

\setslot{\uc{Q}{q}}
   \ifnumber{\int{ligaturing}}>{0}\then
      \ligature{LIG}{asterisk}{\uc{Qalt}{q}}
   \Fi
   \comment{The letter `{Q}'.}
\endsetslot

\setslot{\uc{R}{r}}
   \comment{The letter `{R}'.}
\endsetslot

\setslot{\uc{S}{s}}
   \comment{The letter `{S}'.}
\endsetslot

\setslot{\uc{T}{t}}
   \comment{The letter `{T}'.}
\endsetslot

\setslot{\uc{U}{u}}
   \comment{The letter `{U}'.}
\endsetslot

\setslot{\uc{V}{v}}
   \comment{The letter `{V}'.}
\endsetslot

\setslot{\uc{W}{w}}
   \comment{The letter `{W}'.}
\endsetslot

\setslot{\uc{X}{x}}
   \comment{The letter `{X}'.}
\endsetslot

\setslot{\uc{Y}{y}}
   \comment{The letter `{Y}'.}
\endsetslot

\setslot{\uc{Z}{z}}
   \comment{The letter `{Z}'.}
\endsetslot

\setslot{bracketleft}
   \comment{The opening square bracket `['.}
\endsetslot

\setslot{backslash}
   \comment{The backwards oblique `\textbackslash'.}
\endsetslot

\setslot{bracketright}
   \comment{The closing square bracket `]'.}
\endsetslot

\setslot{asciicircum}
   \comment{The ASCII upward-pointing arrow head `\textasciicircum'.
      This is included for compatibility with typewriter fonts used
      for computer listings.}
\endsetslot

\setslot{underscore}
   \comment{The ASCII underline character `\textunderscore', usually
      set on the baseline.
      This is included for compatibility with typewriter fonts used
      for computer listings.}
\endsetslot

\setslot{quoteleft}
   \ligature{LIG}{quoteleft}{quotedblleft}
   \comment{The English opening single quote mark `\,\textquoteleft\,'.}
\endsetslot

\setslot{\lc{A}{a}}
   \comment{The letter `{a}'.}
\endsetslot

\setslot{\lc{B}{b}}
   \comment{The letter `{b}'.}
\endsetslot

\ifnumber{\int{ligaturing}}<{-1}\then \skipslots{1}\Else

   \setslot{\lc{C}{c}}
      \comment{The letter `{c}'.}
   \endsetslot

\Fi

\setslot{\lc{D}{d}}
   \comment{The letter `{d}'.}
\endsetslot

\setslot{\lc{E}{e}}
   \comment{The letter `{e}'.}
\endsetslot

\ifnumber{\int{ligaturing}}<{-1}\then \skipslots{1}\Else

   \setslot{\lc{F}{f}}
      \ifnumber{\int{ligaturing}}>{0}\then
         \ligature{LIG}{\lc{I}{i}}{\lclig{FI}{fi}}
         \ligature{LIG}{\lc{F}{f}}{\lclig{FF}{f_f}}
         \ligature{LIG}{\lc{L}{l}}{\lclig{FL}{fl}}
      \Fi
      \comment{The letter `{f}'.}
   \endsetslot

\Fi

\setslot{\lc{G}{g}}
   \comment{The letter `{g}'.}
\endsetslot

\setslot{\lc{H}{h}}
   \comment{The letter `{h}'.}
\endsetslot

\ifnumber{\int{ligaturing}}<{-1}\then \skipslots{1}\Else

   \setslot{\lc{I}{i}}
      \comment{The letter `{i}'.}
   \endsetslot

\Fi

\setslot{\lc{J}{j}}
   \comment{The letter `{j}'.}
\endsetslot

\setslot{\lc{K}{k}}
   \comment{The letter `{k}'.}
\endsetslot

\setslot{\lc{L}{l}}
   \comment{The letter `{l}'.}
\endsetslot

\setslot{\lc{M}{m}}
   \comment{The letter `{m}'.}
\endsetslot

\setslot{\lc{N}{n}}
   \comment{The letter `{n}'.}
\endsetslot

\setslot{\lc{O}{o}}
   \comment{The letter `{o}'.}
\endsetslot

\setslot{\lc{P}{p}}
   \comment{The letter `{p}'.}
\endsetslot

\setslot{\lc{Q}{q}}
   \comment{The letter `{q}'.}
\endsetslot

\setslot{\lc{R}{r}}
   \comment{The letter `{r}'.}
\endsetslot

\ifnumber{\int{ligaturing}}<{-1}\then \skipslots{1}\Else

   \setslot{\lc{S}{s}}
      \comment{The letter `{s}'.}
   \endsetslot

\Fi

\setslot{\lc{T}{t}}
   \ifnumber{\int{ligaturing}}>{0}\then
      \ligature{LIG}{\lc{T}{t}}{\lclig{TT}{t_t}}
   \Fi
   \comment{The letter `{t}'.}
\endsetslot

\setslot{\lc{U}{u}}
   \comment{The letter `{u}'.}
\endsetslot

\setslot{\lc{V}{v}}
   \comment{The letter `{v}'.}
\endsetslot

\setslot{\lc{W}{w}}
   \comment{The letter `{w}'.}
\endsetslot

\setslot{\lc{X}{x}}
   \comment{The letter `{x}'.}
\endsetslot

\setslot{\lc{Y}{y}}
   \comment{The letter `{y}'.}
\endsetslot

\setslot{\lc{Z}{z}}
   \comment{The letter `{z}'.}
\endsetslot

\setslot{braceleft}
   \comment{The opening curly brace `\textbraceleft'.}
\endsetslot

\setslot{bar}
   \comment{The ASCII vertical bar `\textbar'.
      This is included for compatibility with typewriter fonts used
      for computer listings.}
\endsetslot

\setslot{braceright}
   \comment{The closing curly brace `\textbraceright'.}
\endsetslot

\setslot{asciitilde}
   \comment{The ASCII tilde `\textasciitilde'.
      This is included for compatibility with typewriter fonts used
      for computer listings.}
\endsetslot

\setslot{hyphenchar}
   \comment{The glyph used for hyphenation in this font, which will
      almost always be the same as `hyphen'.}
\endsetslot

\setslot{\uctop{Abreve}{abreve}}
   \comment{The letter `\u A'.}
\endsetslot

\setslot{\uc{Aogonek}{aogonek}}
   \comment{The letter `\k A'.}
\endsetslot

\setslot{\uctop{Cacute}{cacute}}
   \comment{The letter `\' C'.}
\endsetslot

\setslot{\uctop{Ccaron}{ccaron}}
   \comment{The letter `\v C'.}
\endsetslot

\setslot{\uctop{Dcaron}{dcaron}}
   \comment{The letter `\v D'.}
\endsetslot

\setslot{\uctop{Ecaron}{ecaron}}
   \comment{The letter `\v E'.}
\endsetslot

\setslot{\uc{Eogonek}{eogonek}}
   \comment{The letter `\k E'.}
\endsetslot

\setslot{\uctop{Gbreve}{gbreve}}
   \comment{The letter `\u G'.}
\endsetslot

\setslot{\uctop{Lacute}{lacute}}
   \comment{The letter `\' L'.}
\endsetslot

\setslot{\uc{Lcaron}{lcaron}}
   \comment{The letter `\v L'.}
\endsetslot

\setslot{\uc{Lslash}{lslash}}
   \comment{The letter `\L'.}
\endsetslot

\setslot{\uctop{Nacute}{nacute}}
   \comment{The letter `\' N'.}
\endsetslot

\setslot{\uctop{Ncaron}{ncaron}}
   \comment{The letter `\v N'.}
\endsetslot

\ifnumber{\int{ligaturing}}<{0}\then \skipslots{1}\Else

  \setslot{\lclig{TT}{t_t}}
  \endsetslot

\Fi

\setslot{\uctop{Ohungarumlaut}{ohungarumlaut}}
   \comment{The letter `\H O'.}
\endsetslot

\setslot{\uctop{Racute}{racute}}
   \comment{The letter `\' R'.}
\endsetslot

\setslot{\uctop{Rcaron}{rcaron}}
   \comment{The letter `\v R'.}
\endsetslot

\setslot{\uctop{Sacute}{sacute}}
   \comment{The letter `\' S'.}
\endsetslot

\setslot{\uctop{Scaron}{scaron}}
   \comment{The letter `\v S'.}
\endsetslot

\setslot{\uc{Scedilla}{scedilla}}
   \comment{The letter `\c S'.}
\endsetslot

\setslot{\uctop{Tcaron}{tcaron}}
   \comment{The letter `\v T'.}
\endsetslot

\setslot{\uc{Tcedilla}{tcedilla}}
   \comment{The letter `\c T'.}
\endsetslot

\setslot{\uctop{Uhungarumlaut}{uhungarumlaut}}
   \comment{The letter `\H U'.}
\endsetslot

\setslot{\uctop{Uring}{uring}}
   \comment{The letter `\r U'.}
\endsetslot

\setslot{\uctop{Ydieresis}{ydieresis}}
   \comment{The letter `\" Y'.}
\endsetslot

\setslot{\uctop{Zacute}{zacute}}
   \comment{The letter `\' Z'.}
\endsetslot

\setslot{\uctop{Zcaron}{zcaron}}
   \comment{The letter `\v Z'.}
\endsetslot

\setslot{\uctop{Zdotaccent}{zdotaccent}}
   \comment{The letter `\. Z'.}
\endsetslot

\ifnumber{\int{ligaturing}}<{0}\then \skipslots{1}\Else

   \setslot{\uclig{IJ}{ij}}
      \comment{The letter `IJ'.  This is a single letter, and in a 
        monowidth font should ideally be one letter wide.}
   \endsetslot

\Fi

\setslot{\uctop{Idotaccent}{idotaccent}}
   \comment{The letter `\. I'.}
\endsetslot

\setslot{\lc{Dbar}{dbar}}
   \comment{The letter `\dj'.}
\endsetslot

\setslot{section}
   \comment{The section mark `\textsection'.}
\endsetslot

\setslot{\lctop{Abreve}{abreve}}
   \comment{The letter `\u a'.}
\endsetslot

\setslot{\lc{Aogonek}{aogonek}}
   \comment{The letter `\k a'.}
\endsetslot

\setslot{\lctop{Cacute}{cacute}}
   \comment{The letter `\' c'.}
\endsetslot

\setslot{\lctop{Ccaron}{ccaron}}
   \comment{The letter `\v c'.}
\endsetslot

\setslot{\lctop{Dcaron}{dcaron}}
   \comment{The letter `\v d'.}
\endsetslot

\setslot{\lctop{Ecaron}{ecaron}}
   \comment{The letter `\v e'.}
\endsetslot

\setslot{\lc{Eogonek}{eogonek}}
   \comment{The letter `\k e'.}
\endsetslot

\setslot{\lctop{Gbreve}{gbreve}}
   \comment{The letter `\u g'.}
\endsetslot

\setslot{\lctop{Lacute}{lacute}}
   \comment{The letter `\' l'.}
\endsetslot

\setslot{\lc{Lcaron}{lcaron}}
   \comment{The letter `\v l'.}
\endsetslot

\setslot{\lc{Lslash}{lslash}}
   \comment{The letter `\l'.}
\endsetslot

\setslot{\lctop{Nacute}{nacute}}
   \comment{The letter `\' n'.}
\endsetslot

\setslot{\lctop{Ncaron}{ncaron}}
   \comment{The letter `\v n'.}
\endsetslot

\setslot{\uc{Qalt}{q}}
\endsetslot

\setslot{\lctop{Ohungarumlaut}{ohungarumlaut}}
   \comment{The letter `\H o'.}
\endsetslot

\setslot{\lctop{Racute}{racute}}
   \comment{The letter `\' r'.}
\endsetslot

\setslot{\lctop{Rcaron}{rcaron}}
   \comment{The letter `\v r'.}
\endsetslot

\setslot{\lctop{Sacute}{sacute}}
   \comment{The letter `\' s'.}
\endsetslot

\setslot{\lctop{Scaron}{scaron}}
   \comment{The letter `\v s'.}
\endsetslot

\setslot{\lc{Scedilla}{scedilla}}
   \comment{The letter `\c s'.}
\endsetslot

\setslot{\lctop{Tcaron}{tcaron}}
   \comment{The letter `\v t'.}
\endsetslot

\setslot{\lc{Tcedilla}{tcedilla}}
   \comment{The letter `\c t'.}
\endsetslot

\setslot{\lctop{Uhungarumlaut}{uhungarumlaut}}
   \comment{The letter `\H u'.}
\endsetslot

\setslot{\lctop{Uring}{uring}}
   \comment{The letter `\r u'.}
\endsetslot

\setslot{\lctop{Ydieresis}{ydieresis}}
   \comment{The letter `\" y'.}
\endsetslot

\setslot{\lctop{Zacute}{zacute}}
   \comment{The letter `\' z'.}
\endsetslot

\setslot{\lctop{Zcaron}{zcaron}}
   \comment{The letter `\v z'.}
\endsetslot

\setslot{\lctop{Zdotaccent}{zdotaccent}}
   \comment{The letter `\. z'.}
\endsetslot

\ifnumber{\int{ligaturing}}<{0}\then \skipslots{1}\Else

   \setslot{\lclig{IJ}{ij}}
      \comment{The letter `ij'.  This is a single letter, and in a 
        monowidth font should ideally be one letter wide.}
   \endsetslot

\Fi

\setslot{exclamdown}
   \comment{The Spanish punctuation mark `!`'.}
\endsetslot

\setslot{questiondown}
   \comment{The Spanish punctuation mark `?`'.}
\endsetslot

\setslot{sterling}
   \comment{The British currency mark `\textsterling'.}
\endsetslot

\setslot{\uctop{Agrave}{agrave}}
   \comment{The letter `\` A'.}
\endsetslot

\setslot{\uctop{Aacute}{aacute}}
   \comment{The letter `\' A'.}
\endsetslot

\setslot{\uctop{Acircumflex}{acircumflex}}
   \comment{The letter `\^ A'.}
\endsetslot

\setslot{\uctop{Atilde}{atilde}}
   \comment{The letter `\~ A'.}
\endsetslot

\setslot{\uctop{Adieresis}{adieresis}}
   \comment{The letter `\" A'.}
\endsetslot

\setslot{\uctop{Aring}{aring}}
   \comment{The letter `\r A'.}
\endsetslot

\setslot{\uc{AE}{ae}}
   \comment{The letter `\AE'.  This is a single letter, and should not be
      faked with `AE'.}
\endsetslot

\setslot{\uc{Ccedilla}{ccedilla}}
   \comment{The letter `\c C'.}
\endsetslot

\setslot{\uctop{Egrave}{egrave}}
   \comment{The letter `\` E'.}
\endsetslot

\setslot{\uctop{Eacute}{eacute}}
   \comment{The letter `\' E'.}
\endsetslot

\setslot{\uctop{Ecircumflex}{ecircumflex}}
   \comment{The letter `\^ E'.}
\endsetslot

\setslot{\uctop{Edieresis}{edieresis}}
   \comment{The letter `\" E'.}
\endsetslot

\setslot{\uctop{Igrave}{igrave}}
   \comment{The letter `\` I'.}
\endsetslot

\setslot{\uctop{Iacute}{iacute}}
   \comment{The letter `\' I'.}
\endsetslot

\setslot{\uctop{Icircumflex}{icircumflex}}
   \comment{The letter `\^ I'.}
\endsetslot

\setslot{\uctop{Idieresis}{idieresis}}
   \comment{The letter `\" I'.}
\endsetslot

\setslot{\uc{Eth}{eth}}
   \comment{The uppercase Icelandic letter `Eth' similar to a `D'
      with a horizontal bar through the stem.  It is unavailable
      in \plain\ \TeX.}
\endsetslot

\setslot{\uctop{Ntilde}{ntilde}}
   \comment{The letter `\~ N'.}
\endsetslot

\setslot{\uctop{Ograve}{ograve}}
   \comment{The letter `\` O'.}
\endsetslot

\setslot{\uctop{Oacute}{oacute}}
   \comment{The letter `\' O'.}
\endsetslot

\setslot{\uctop{Ocircumflex}{ocircumflex}}
   \comment{The letter `\^ O'.}
\endsetslot

\setslot{\uctop{Otilde}{otilde}}
   \comment{The letter `\~ O'.}
\endsetslot

\setslot{\uctop{Odieresis}{odieresis}}
   \comment{The letter `\" O'.}
\endsetslot

\setslot{\uc{OE}{oe}}
   \comment{The letter `\OE'.  This is a single letter, and should not be
      faked with `OE'.}
\endsetslot

\setslot{\uc{Oslash}{oslash}}
   \comment{The letter `\O'.}
\endsetslot

\setslot{\uctop{Ugrave}{ugrave}}
   \comment{The letter `\` U'.}
\endsetslot

\setslot{\uctop{Uacute}{uacute}}
   \comment{The letter `\' U'.}
\endsetslot

\setslot{\uctop{Ucircumflex}{ucircumflex}}
   \comment{The letter `\^ U'.}
\endsetslot

\setslot{\uctop{Udieresis}{udieresis}}
   \comment{The letter `\" U'.}
\endsetslot

\setslot{\uctop{Yacute}{yacute}}
   \comment{The letter `\' Y'.}
\endsetslot

\setslot{\uc{Thorn}{thorn}}
   \comment{The Icelandic capital letter Thorn, similar to a `P'
      with the bowl moved down.  It is unavailable in \plain\ \TeX.}
\endsetslot

\setslot{\uclig{SS}{germandbls}}
   \comment{The ligature `SS', used to give an upper case `\ss'.
      In a monowidth font it should be two letters wide.}
\endsetslot

\setslot{\lctop{Agrave}{agrave}}
   \comment{The letter `\` a'.}
\endsetslot

\setslot{\lctop{Aacute}{aacute}}
   \comment{The letter `\' a'.}
\endsetslot

\setslot{\lctop{Acircumflex}{acircumflex}}
   \comment{The letter `\^ a'.}
\endsetslot

\setslot{\lctop{Atilde}{atilde}}
   \comment{The letter `\~ a'.}
\endsetslot

\setslot{\lctop{Adieresis}{adieresis}}
   \comment{The letter `\" a'.}
\endsetslot

\setslot{\lctop{Aring}{aring}}
   \comment{The letter `\r a'.}
\endsetslot

\setslot{\lc{AE}{ae}}
   \comment{The letter `\ae'.  This is a single letter, and should not be
      faked with `ae'.}
\endsetslot

\setslot{\lc{Ccedilla}{ccedilla}}
   \comment{The letter `\c c'.}
\endsetslot

\setslot{\lctop{Egrave}{egrave}}
   \comment{The letter `\` e'.}
\endsetslot

\setslot{\lctop{Eacute}{eacute}}
   \comment{The letter `\' e'.}
\endsetslot

\setslot{\lctop{Ecircumflex}{ecircumflex}}
   \comment{The letter `\^ e'.}
\endsetslot

\setslot{\lctop{Edieresis}{edieresis}}
   \comment{The letter `\" e'.}
\endsetslot

\setslot{\lctop{Igrave}{igrave}}
   \comment{The letter `\`\i'.}
\endsetslot

\setslot{\lctop{Iacute}{iacute}}
   \comment{The letter `\'\i'.}
\endsetslot

\setslot{\lctop{Icircumflex}{icircumflex}}
   \comment{The letter `\^\i'.}
\endsetslot

\setslot{\lctop{Idieresis}{idieresis}}
   \comment{The letter `\"\i'.}
\endsetslot

\setslot{\lc{Eth}{eth}}
   \comment{The Icelandic lowercase letter `eth' similar to
     a `$\partial$' with an oblique bar through the stem.
     It is unavailable in \plain\ \TeX.}
\endsetslot

\setslot{\lctop{Ntilde}{ntilde}}
   \comment{The letter `\~ n'.}
\endsetslot

\setslot{\lctop{Ograve}{ograve}}
   \comment{The letter `\` o'.}
\endsetslot

\setslot{\lctop{Oacute}{oacute}}
   \comment{The letter `\' o'.}
\endsetslot

\setslot{\lctop{Ocircumflex}{ocircumflex}}
   \comment{The letter `\^ o'.}
\endsetslot

\setslot{\lctop{Otilde}{otilde}}
   \comment{The letter `\~ o'.}
\endsetslot

\setslot{\lctop{Odieresis}{odieresis}}
   \comment{The letter `\" o'.}
\endsetslot

\setslot{\lc{OE}{oe}}
   \comment{The letter `\oe'.  This is a single letter, and should not be
      faked with `oe'.}
\endsetslot

\setslot{\lc{Oslash}{oslash}}
   \comment{The letter `\o'.}
\endsetslot

\setslot{\lctop{Ugrave}{ugrave}}
   \comment{The letter `\` u'.}
\endsetslot

\setslot{\lctop{Uacute}{uacute}}
   \comment{The letter `\' u'.}
\endsetslot

\setslot{\lctop{Ucircumflex}{ucircumflex}}
   \comment{The letter `\^ u'.}
\endsetslot

\setslot{\lctop{Udieresis}{udieresis}}
   \comment{The letter `\" u'.}
\endsetslot

\setslot{\lctop{Yacute}{yacute}}
   \comment{The letter `\' y'.}
\endsetslot

\setslot{\lc{Thorn}{thorn}}
   \comment{The Icelandic lowercase letter `thorn', similar to a `p'
      with an ascender rising from the stem.  It is unavailable
      in \plain\ \TeX.}
\endsetslot

\setslot{\lc{SS}{germandbls}}
   \comment{The letter `\ss'.}
\endsetslot

\endencoding
%    \end{macrocode}
% \iffalse
%</t1-yes>
% \fi
% \iffalse
%<*t1-yesw>
% \fi
%    \begin{macrocode}
\relax
\encoding

\needsfontinstversion{1.910}

\setcommand\lc#1#2{#2}
\setcommand\uc#1#2{#1}
\setcommand\lctop#1#2{#2}
\setcommand\uctop#1#2{#1}
\setcommand\lclig#1#2{#2}
\ifisint{letterspacing}\then
\ifnumber{\int{letterspacing}}={0}\then \Else
\setcommand\uclig#1#2{#1spaced}
\comment{Here we set \verb|\uclig#1#2| to \verb|#1spaced|, but 
  you can't see it as \verb|\setcommand| commands are invisible in 
  the typeset output.}
\Fi
\Fi
\setcommand\uclig#1#2{#1}
\setcommand\digit#1{#1}

\ifisint{monowidth}\then
\setint{ligaturing}{0}
\Else
% The following empty line is *important* to get the formatting
% right here (sigh)! (Remember that it is a \par token.)

\ifisint{letterspacing}\then
\ifnumber{\int{letterspacing}}={0}\then \Else
\setint{ligaturing}{0}
\Fi
\Fi
\setint{ligaturing}{1}
\Fi

\setint{italicslant}{0}
\setint{quad}{1000}
\setint{baselineskip}{1200}

\ifisglyph{x}\then
\setint{xheight}{\height{x}}
\Else
\setint{xheight}{500}
\Fi

\ifisglyph{space}\then
\setint{interword}{\width{space}}
\Else\ifisglyph{i}\then
\setint{interword}{\width{i}}
\Else
\setint{interword}{333}
\Fi\Fi

\ifisint{monowidth}\then
\setint{stretchword}{0}
\setint{shrinkword}{0}
\setint{extraspace}{\int{interword}}
\Else
\setint{stretchword}{\scale{\int{interword}}{600}}
\setint{shrinkword}{\scale{\int{interword}}{240}}
\setint{extraspace}{\scale{\int{interword}}{240}}
\Fi

\ifisglyph{X}\then
\setint{capheight}{\height{X}}
\Else
\setint{capheight}{750}
\Fi

\ifisglyph{d}\then
\setint{ascender}{\height{d}}
\Else\ifisint{capheight}\then
\setint{ascender}{\int{capheight}}
\Else
\setint{ascender}{750}
\Fi\Fi

\ifisglyph{Aring}\then
\setint{acccapheight}{\height{Aring}}
\Else
\setint{acccapheight}{999}
\Fi

\ifisint{descender_neg}\then
\setint{descender}{\neg{\int{descender_neg}}}
\Else\ifisglyph{p}\then
\setint{descender}{\depth{p}}
\Else
\setint{descender}{250}
\Fi\Fi

\ifisglyph{Aring}\then
\setint{maxheight}{\height{Aring}}
\Else
\setint{maxheight}{1000}
\Fi

\ifisint{maxdepth_neg}\then
\setint{maxdepth}{\neg{\int{maxdepth_neg}}}
\Else\ifisglyph{j}\then
\setint{maxdepth}{\depth{j}}
\Else
\setint{maxdepth}{250}
\Fi\Fi

\ifisglyph{six}\then
\setint{digitwidth}{\width{six}}
\Else
\setint{digitwidth}{500}
\Fi

\setint{capstem}{0} % not in AFM files

\setfontdimen{1}{italicslant}    % italic slant
\setfontdimen{2}{interword}      % interword space
\setfontdimen{3}{stretchword}    % interword stretch
\setfontdimen{4}{shrinkword}     % interword shrink
\setfontdimen{5}{xheight}        % x-height
\setfontdimen{6}{quad}           % quad
\setfontdimen{7}{extraspace}     % extra space after .
\setfontdimen{8}{capheight}      % cap height
\setfontdimen{9}{ascender}       % ascender
\setfontdimen{10}{acccapheight}  % accented cap height
\setfontdimen{11}{descender}     % descender's depth
\setfontdimen{12}{maxheight}     % max height
\setfontdimen{13}{maxdepth}      % max depth
\setfontdimen{14}{digitwidth}    % digit width
\setfontdimen{15}{verticalstem}  % dominant width of verical stems
\setfontdimen{16}{baselineskip}  % baselineskip

\ifnumber{\int{ligaturing}}<{0}\then 
\comment{In this case, the codingscheme can be different from the 
  default, and therefore we refrain from setting it.}
\Else
\setstr{codingscheme}{EXTENDED TEX ENC - ELECTRUM LIG}
\Fi

\setslot{\lc{Grave}{grave}}
\comment{The grave accent `\`{}'.}
\endsetslot

\setslot{\lc{Acute}{acute}}
\comment{The acute accent `\'{}'.}
\endsetslot

\setslot{\lc{Circumflex}{circumflex}}
\comment{The circumflex accent `\^{}'.}
\endsetslot

\setslot{\lc{Tilde}{tilde}}
\comment{The tilde accent `\~{}'.}
\endsetslot

\setslot{\lc{Dieresis}{dieresis}}
\comment{The umlaut or dieresis accent `\"{}'.}
\endsetslot

\setslot{\lc{Hungarumlaut}{hungarumlaut}}
\comment{The long Hungarian umlaut `\H{}'.}
\endsetslot

\setslot{\lc{Ring}{ring}}
\comment{The ring accent `\r{}'.}
\endsetslot

\setslot{\lc{Caron}{caron}}
\comment{The caron or h\'a\v cek accent `\v{}'.}
\endsetslot

\setslot{\lc{Breve}{breve}}
\comment{The breve accent `\u{}'.}
\endsetslot

\setslot{\lc{Macron}{macron}}
\comment{The macron accent `\={}'.}
\endsetslot

\setslot{\lc{Dotaccent}{dotaccent}}
\comment{The dot accent `\.{}'.}
\endsetslot

\setslot{\lc{Cedilla}{cedilla}}
\comment{The cedilla accent `\c {}'.}
\endsetslot

\setslot{\lc{Ogonek}{ogonek}}
\comment{The ogonek accent `\k {}'.}
\endsetslot

\setslot{quotesinglbase}
\comment{A German single quote mark `\quotesinglbase' similar to a comma,
  but with different sidebearings.}
\endsetslot

\setslot{guilsinglleft}
\comment{A French single opening quote mark `\guilsinglleft',
  unavailable in \plain\ \TeX.}
\endsetslot

\setslot{guilsinglright}
\comment{A French single closing quote mark `\guilsinglright',
  unavailable in \plain\ \TeX.}
\endsetslot

\setslot{quotedblleft}
\comment{The English opening quote mark `\,\textquotedblleft\,'.}
\endsetslot

\setslot{quotedblright}
\comment{The English closing quote mark `\,\textquotedblright\,'.}
\endsetslot

\setslot{quotedblbase}
\comment{A German double quote mark `\quotedblbase' similar to two commas,
  but with tighter letterspacing and different sidebearings.}
\endsetslot

\setslot{guillemotleft}
\comment{A French double opening quote mark `\guillemotleft',
  unavailable in \plain\ \TeX.}
\endsetslot

\setslot{guillemotright}
\comment{A French closing opening quote mark `\guillemotright',
  unavailable in \plain\ \TeX.}
\endsetslot

\setslot{endash}
\ligature{LIG}{hyphen}{emdash}
\comment{The number range dash `1--9'. This is called `rangedash' by fontinst's t1.etx, but it needs to be called `endash' to work right. The `\textendash'.  In a monowidth font, this
  might be set as `\texttt{1{-}9}'.}
\endsetslot

\setslot{emdash}
\comment{The punctuation dash `Oh---boy.' This is calle `punctdash' by fontinst's t1.etx, but needs to be called `emdash' to work right. The `\textemdash'.  In a monowidth font, this
  might be set as `\texttt{Oh{-}{-}boy.}'}
\endsetslot

\setslot{compwordmark}
\comment{An invisible glyph, with zero width and depth, but the
  height of lowercase letters without ascenders.
  It is used to stop ligaturing in words like `shelf{}ful'.}
\endsetslot

\setslot{zero.denominator}
\comment{A glyph which is placed after `\%' to produce a
  `per-thousand', or twice to produce `per-ten-thousand'.
  Your guess is as good as mine as to what this glyph should look
  like in a monowidth font.}
\endsetslot

\setslot{\lc{dotlessI}{dotlessi}}
\comment{A dotless i `\i', used to produce accented letters such as
  `\=\i'.}
\endsetslot

\setslot{\lc{dotlessJ}{dotlessj}}
\comment{A dotless j `\j', used to produce accented letters such as
  `\=\j'.  Most non-\TeX\ fonts do not have this glyph.}
\endsetslot

\ifnumber{\int{ligaturing}}<{0}\then \skipslots{5}\Else

\setslot{\lclig{FF}{f_f}}
\ifnumber{\int{ligaturing}}>{0}\then
\ligature{LIG}{\lc{I}{i}}{\lclig{FFI}{f_f_i}}
\ligature{LIG}{\lc{L}{l}}{\lclig{FFL}{f_f_l}}
\Fi
\comment{The `ff' ligature.  It should be two characters wide in a
  monowidth font.}
\endsetslot

\setslot{\lclig{FI}{fi}}
\comment{The `fi' ligature.  It should be two characters wide in a
  monowidth font.}
\endsetslot

\setslot{\lclig{FL}{fl}}
\comment{The `fl' ligature.  It should be two characters wide in a
  monowidth font.}
\endsetslot

\setslot{\lclig{FFI}{f_f_i}}
\comment{The `ffi' ligature.  It should be three characters wide in a
  monowidth font.}
\endsetslot

\setslot{\lclig{FFL}{f_f_l}}
\comment{The `ffl' ligature.  It should be three characters wide in a
  monowidth font.}
\endsetslot

\Fi

\setslot{visiblespace}
\comment{A visible space glyph `\textvisiblespace'.}
\endsetslot

\setslot{exclam}
\ligature{LIG}{quoteleft}{exclamdown}
\comment{The exclamation mark `!'.}
\endsetslot

\ifnumber{\int{ligaturing}}<{0}\then \skipslots{1}\Else

\setslot{\lclig{CT}{c_t}}
\endsetslot

\Fi

\setslot{numbersign}
\comment{The hash sign `\#'.}
\endsetslot

\setslot{dollar}
\comment{The dollar sign `\$'.}
\endsetslot

\setslot{percent}
\comment{The percent sign `\%'.}
\endsetslot

\setslot{ampersand}
\comment{The ampersand sign `\&'.}
\endsetslot

\setslot{quoteright}
\ligature{LIG}{quoteright}{quotedblright}
\comment{The English closing single quote mark `\,\textquoteright\,'.}
\endsetslot

\setslot{parenleft}
\comment{The opening parenthesis `('.}
\endsetslot

\setslot{parenright}
\comment{The closing parenthesis `)'.}
\endsetslot

\setslot{asterisk}
\comment{The raised asterisk `*'.}
\endsetslot

\setslot{plus}
\comment{The addition sign `+'.}
\endsetslot

\setslot{comma}
\ligature{LIG}{comma}{quotedblbase}
\comment{The comma `,'.}
\endsetslot

\setslot{hyphen}
\ligature{LIG}{hyphen}{endash}
\ligature{LIG}{hyphenchar}{hyphenchar}
\comment{The hyphen `-'.}
\endsetslot

\setslot{period}
\comment{The period `.'.}
\endsetslot

\setslot{slash}
\comment{The forward oblique `/'.}
\endsetslot

\setslot{\digit{zero}}
\comment{The number `0'.  This (and all the other numerals) may be
  old style or ranging digits.}
\endsetslot

\setslot{\digit{one}}
\comment{The number `1'.}
\endsetslot

\setslot{\digit{two}}
\comment{The number `2'.}
\endsetslot

\setslot{\digit{three}}
\comment{The number `3'.}
\endsetslot

\setslot{\digit{four}}
\comment{The number `4'.}
\endsetslot

\setslot{\digit{five}}
\comment{The number `5'.}
\endsetslot

\setslot{\digit{six}}
\comment{The number `6'.}
\endsetslot

\setslot{\digit{seven}}
\comment{The number `7'.}
\endsetslot

\setslot{\digit{eight}}
\comment{The number `8'.}
\endsetslot

\setslot{\digit{nine}}
\comment{The number `9'.}
\endsetslot

\setslot{colon}
\comment{The colon punctuation mark `:'.}
\endsetslot

\setslot{semicolon}
\comment{The semi-colon punctuation mark `;'.}
\endsetslot

\setslot{less}
\ligature{LIG}{less}{guillemotleft}
\comment{The less-than sign `\textless'.}
\endsetslot

\setslot{equal}
\comment{The equals sign `='.}
\endsetslot

\setslot{greater}
\ligature{LIG}{greater}{guillemotright}
\comment{The greater-than sign `\textgreater'.}
\endsetslot

\setslot{question}
\ligature{LIG}{quoteleft}{questiondown}
\comment{The question mark `?'.}
\endsetslot

\setslot{at}
\comment{The at sign `@'.}
\endsetslot

\setslot{\uc{A}{a}}
\comment{The letter `{A}'.}
\endsetslot

\setslot{\uc{B}{b}}
\comment{The letter `{B}'.}
\endsetslot

\setslot{\uc{C}{c}}
\comment{The letter `{C}'.}
\endsetslot

\setslot{\uc{D}{d}}
\comment{The letter `{D}'.}
\endsetslot

\setslot{\uc{E}{e}}
\comment{The letter `{E}'.}
\endsetslot

\setslot{\uc{F}{f}}
\comment{The letter `{F}'.}
\endsetslot

\setslot{\uc{G}{g}}
\comment{The letter `{G}'.}
\endsetslot

\setslot{\uc{H}{h}}
\comment{The letter `{H}'.}
\endsetslot

\ifnumber{\int{ligaturing}}<{-1}\then \skipslots{1}\Else

\setslot{\uc{I}{i}}
\comment{The letter `{I}'.}
\endsetslot

\Fi

\setslot{\uc{J}{j}}
\comment{The letter `{J}'.}
\endsetslot

\setslot{\uc{K}{k}}
\comment{The letter `{K}'.}
\endsetslot

\setslot{\uc{L}{l}}
\comment{The letter `{L}'.}
\endsetslot

\setslot{\uc{M}{m}}
\comment{The letter `{M}'.}
\endsetslot

\setslot{\uc{N}{n}}
\comment{The letter `{N}'.}
\endsetslot

\setslot{\uc{O}{o}}
\comment{The letter `{O}'.}
\endsetslot

\setslot{\uc{P}{p}}
\comment{The letter `{P}'.}
\endsetslot

\setslot{\uc{Qalt}{q}}
\ifnumber{\int{ligaturing}}>{0}\then
\ligature{LIG}{asterisk}{\uc{Q}{q}}
\Fi
\comment{The letter `{Q}'.}
\endsetslot

\setslot{\uc{R}{r}}
\comment{The letter `{R}'.}
\endsetslot

\setslot{\uc{S}{s}}
\comment{The letter `{S}'.}
\endsetslot

\setslot{\uc{T}{t}}
\comment{The letter `{T}'.}
\endsetslot

\setslot{\uc{U}{u}}
\comment{The letter `{U}'.}
\endsetslot

\setslot{\uc{V}{v}}
\comment{The letter `{V}'.}
\endsetslot

\setslot{\uc{W}{w}}
\comment{The letter `{W}'.}
\endsetslot

\setslot{\uc{X}{x}}
\comment{The letter `{X}'.}
\endsetslot

\setslot{\uc{Y}{y}}
\comment{The letter `{Y}'.}
\endsetslot

\setslot{\uc{Z}{z}}
\comment{The letter `{Z}'.}
\endsetslot

\setslot{bracketleft}
\comment{The opening square bracket `['.}
\endsetslot

\setslot{backslash}
\comment{The backwards oblique `\textbackslash'.}
\endsetslot

\setslot{bracketright}
\comment{The closing square bracket `]'.}
\endsetslot

\ifnumber{\int{ligaturing}}<{0}\then \skipslots{1}\Else

\setslot{\lclig{SP}{s_p}}
\endsetslot

\Fi

\setslot{underscore}
\comment{The ASCII underline character `\textunderscore', usually
  set on the baseline.
  This is included for compatibility with typewriter fonts used
  for computer listings.}
\endsetslot

\setslot{quoteleft}
\ligature{LIG}{quoteleft}{quotedblleft}
\comment{The English opening single quote mark `\,\textquoteleft\,'.}
\endsetslot

\setslot{\lc{A}{a}}
\comment{The letter `{a}'.}
\endsetslot

\setslot{\lc{B}{b}}
\comment{The letter `{b}'.}
\endsetslot

\ifnumber{\int{ligaturing}}<{-1}\then \skipslots{1}\Else

\setslot{\lc{C}{c}}
\ifnumber{\int{ligaturing}}>{0}\then
\ligature{LIG}{\lc{T}{t}}{\lclig{CT}{c_t}}
\Fi
\comment{The letter `{c}'.}
\endsetslot

\Fi

\setslot{\lc{D}{d}}
\comment{The letter `{d}'.}
\endsetslot

\setslot{\lc{E}{e}}
\comment{The letter `{e}'.}
\endsetslot

\ifnumber{\int{ligaturing}}<{-1}\then \skipslots{1}\Else

\setslot{\lc{F}{f}}
\ifnumber{\int{ligaturing}}>{0}\then
\ligature{LIG}{\lc{I}{i}}{\lclig{FI}{fi}}
\ligature{LIG}{\lc{F}{f}}{\lclig{FF}{f_f}}
\ligature{LIG}{\lc{L}{l}}{\lclig{FL}{fl}}
\Fi
\comment{The letter `{f}'.}
\endsetslot

\Fi

\setslot{\lc{G}{g}}
\comment{The letter `{g}'.}
\endsetslot

\setslot{\lc{H}{h}}
\comment{The letter `{h}'.}
\endsetslot

\ifnumber{\int{ligaturing}}<{-1}\then \skipslots{1}\Else

\setslot{\lc{I}{i}}
\ifnumber{\int{ligaturing}}>{0}\then
\ligature{LIG}{\lc{T}{t}}{\lclig{IT}{i_t}}
\Fi
\comment{The letter `{i}'.}
\endsetslot

\Fi

\setslot{\lc{J}{j}}
\comment{The letter `{j}'.}
\endsetslot

\setslot{\lc{K}{k}}
\comment{The letter `{k}'.}
\endsetslot

\setslot{\lc{L}{l}}
\comment{The letter `{l}'.}
\endsetslot

\setslot{\lc{M}{m}}
\comment{The letter `{m}'.}
\endsetslot

\setslot{\lc{N}{n}}
\comment{The letter `{n}'.}
\endsetslot

\setslot{\lc{O}{o}}
\comment{The letter `{o}'.}
\endsetslot

\setslot{\lc{P}{p}}
\comment{The letter `{p}'.}
\endsetslot

\setslot{\lc{Qalt}{q}}
\comment{The letter `{q}'.}
\endsetslot

\setslot{\lc{R}{r}}
\comment{The letter `{r}'.}
\endsetslot

\ifnumber{\int{ligaturing}}<{-1}\then \skipslots{1}\Else

\setslot{\lc{S}{s}}
\ifnumber{\int{ligaturing}}>{0}\then
\ligature{LIG}{\lc{P}{p}}{\lclig{SP}{s_p}}
\ligature{LIG}{\lc{T}{t}}{\lclig{ST}{s_t}}
\Fi
\comment{The letter `{s}'.}
\endsetslot

\Fi

\setslot{\lc{T}{t}}
\ifnumber{\int{ligaturing}}>{0}\then
\ligature{LIG}{\lc{T}{t}}{\lclig{TT}{t_t}}
\Fi
\comment{The letter `{t}'.}
\endsetslot

\setslot{\lc{U}{u}}
\comment{The letter `{u}'.}
\endsetslot

\setslot{\lc{V}{v}}
\comment{The letter `{v}'.}
\endsetslot

\setslot{\lc{W}{w}}
\comment{The letter `{w}'.}
\endsetslot

\setslot{\lc{X}{x}}
\comment{The letter `{x}'.}
\endsetslot

\setslot{\lc{Y}{y}}
\comment{The letter `{y}'.}
\endsetslot

\setslot{\lc{Z}{z}}
\comment{The letter `{z}'.}
\endsetslot

\setslot{braceleft}
\comment{The opening curly brace `\textbraceleft'.}
\endsetslot

\setslot{bar}
\comment{The ASCII vertical bar `\textbar'.
  This is included for compatibility with typewriter fonts used
  for computer listings.}
\endsetslot

\setslot{braceright}
\comment{The closing curly brace `\textbraceright'.}
\endsetslot

\ifnumber{\int{ligaturing}}<{0}\then \skipslots{1}\Else

\setslot{\lclig{IT}{i_t}}
\endsetslot

\Fi

\setslot{hyphenchar}
\comment{The glyph used for hyphenation in this font, which will
  almost always be the same as `hyphen'.}
\endsetslot

\setslot{\uctop{Abreve}{abreve}}
\comment{The letter `\u A'.}
\endsetslot

\setslot{\uc{Aogonek}{aogonek}}
\comment{The letter `\k A'.}
\endsetslot

\setslot{\uctop{Cacute}{cacute}}
\comment{The letter `\' C'.}
\endsetslot

\setslot{\uctop{Ccaron}{ccaron}}
\comment{The letter `\v C'.}
\endsetslot

\setslot{\uctop{Dcaron}{dcaron}}
\comment{The letter `\v D'.}
\endsetslot

\setslot{\uctop{Ecaron}{ecaron}}
\comment{The letter `\v E'.}
\endsetslot

\setslot{\uc{Eogonek}{eogonek}}
\comment{The letter `\k E'.}
\endsetslot

\setslot{\uctop{Gbreve}{gbreve}}
\comment{The letter `\u G'.}
\endsetslot

\setslot{\uctop{Lacute}{lacute}}
\comment{The letter `\' L'.}
\endsetslot

\setslot{\uc{Lcaron}{lcaron}}
\comment{The letter `\v L'.}
\endsetslot

\setslot{\uc{Lslash}{lslash}}
\comment{The letter `\L'.}
\endsetslot

\setslot{\uctop{Nacute}{nacute}}
\comment{The letter `\' N'.}
\endsetslot

\setslot{\uctop{Ncaron}{ncaron}}
\comment{The letter `\v N'.}
\endsetslot

\ifnumber{\int{ligaturing}}<{0}\then \skipslots{1}\Else

\setslot{\lclig{TT}{t_t}}
\endsetslot

\Fi

\setslot{\uctop{Ohungarumlaut}{ohungarumlaut}}
\comment{The letter `\H O'.}
\endsetslot

\setslot{\uctop{Racute}{racute}}
\comment{The letter `\' R'.}
\endsetslot

\setslot{\uctop{Rcaron}{rcaron}}
\comment{The letter `\v R'.}
\endsetslot

\setslot{\uctop{Sacute}{sacute}}
\comment{The letter `\' S'.}
\endsetslot

\setslot{\uctop{Scaron}{scaron}}
\comment{The letter `\v S'.}
\endsetslot

\setslot{\uc{Scedilla}{scedilla}}
\comment{The letter `\c S'.}
\endsetslot

\setslot{\uctop{Tcaron}{tcaron}}
\comment{The letter `\v T'.}
\endsetslot

\setslot{\uc{Tcedilla}{tcedilla}}
\comment{The letter `\c T'.}
\endsetslot

\setslot{\uctop{Uhungarumlaut}{uhungarumlaut}}
\comment{The letter `\H U'.}
\endsetslot

\setslot{\uctop{Uring}{uring}}
\comment{The letter `\r U'.}
\endsetslot

\setslot{\uctop{Ydieresis}{ydieresis}}
\comment{The letter `\" Y'.}
\endsetslot

\setslot{\uctop{Zacute}{zacute}}
\comment{The letter `\' Z'.}
\endsetslot

\setslot{\uctop{Zcaron}{zcaron}}
\comment{The letter `\v Z'.}
\endsetslot

\setslot{\uctop{Zdotaccent}{zdotaccent}}
\comment{The letter `\. Z'.}
\endsetslot

\ifnumber{\int{ligaturing}}<{0}\then \skipslots{1}\Else

\setslot{\uclig{IJ}{ij}}
\comment{The letter `IJ'.  This is a single letter, and in a 
  monowidth font should ideally be one letter wide.}
\endsetslot

\Fi

\setslot{\uctop{Idotaccent}{idotaccent}}
\comment{The letter `\. I'.}
\endsetslot

\ifnumber{\int{ligaturing}}<{0}\then \skipslots{1}\Else

\setslot{\lclig{ST}{s_t}}
\endsetslot

\Fi

\setslot{section}
\comment{The section mark `\textsection'.}
\endsetslot

\setslot{\lctop{Abreve}{abreve}}
\comment{The letter `\u a'.}
\endsetslot

\setslot{\lc{Aogonek}{aogonek}}
\comment{The letter `\k a'.}
\endsetslot

\setslot{\lctop{Cacute}{cacute}}
\comment{The letter `\' c'.}
\endsetslot

\setslot{\lctop{Ccaron}{ccaron}}
\comment{The letter `\v c'.}
\endsetslot

\setslot{\lctop{Dcaron}{dcaron}}
\comment{The letter `\v d'.}
\endsetslot

\setslot{\lctop{Ecaron}{ecaron}}
\comment{The letter `\v e'.}
\endsetslot

\setslot{\lc{Eogonek}{eogonek}}
\comment{The letter `\k e'.}
\endsetslot

\setslot{\lctop{Gbreve}{gbreve}}
\comment{The letter `\u g'.}
\endsetslot

\setslot{\lctop{Lacute}{lacute}}
\comment{The letter `\' l'.}
\endsetslot

\setslot{\lc{Lcaron}{lcaron}}
\comment{The letter `\v l'.}
\endsetslot

\setslot{\lc{Lslash}{lslash}}
\comment{The letter `\l'.}
\endsetslot

\setslot{\lctop{Nacute}{nacute}}
\comment{The letter `\' n'.}
\endsetslot

\setslot{\lctop{Ncaron}{ncaron}}
\comment{The letter `\v n'.}
\endsetslot

\setslot{\uc{Q}{q}}
\endsetslot

\setslot{\lctop{Ohungarumlaut}{ohungarumlaut}}
\comment{The letter `\H o'.}
\endsetslot

\setslot{\lctop{Racute}{racute}}
\comment{The letter `\' r'.}
\endsetslot

\setslot{\lctop{Rcaron}{rcaron}}
\comment{The letter `\v r'.}
\endsetslot

\setslot{\lctop{Sacute}{sacute}}
\comment{The letter `\' s'.}
\endsetslot

\setslot{\lctop{Scaron}{scaron}}
\comment{The letter `\v s'.}
\endsetslot

\setslot{\lc{Scedilla}{scedilla}}
\comment{The letter `\c s'.}
\endsetslot

\setslot{\lctop{Tcaron}{tcaron}}
\comment{The letter `\v t'.}
\endsetslot

\setslot{\lc{Tcedilla}{tcedilla}}
\comment{The letter `\c t'.}
\endsetslot

\setslot{\lctop{Uhungarumlaut}{uhungarumlaut}}
\comment{The letter `\H u'.}
\endsetslot

\setslot{\lctop{Uring}{uring}}
\comment{The letter `\r u'.}
\endsetslot

\setslot{\lctop{Ydieresis}{ydieresis}}
\comment{The letter `\" y'.}
\endsetslot

\setslot{\lctop{Zacute}{zacute}}
\comment{The letter `\' z'.}
\endsetslot

\setslot{\lctop{Zcaron}{zcaron}}
\comment{The letter `\v z'.}
\endsetslot

\setslot{\lctop{Zdotaccent}{zdotaccent}}
\comment{The letter `\. z'.}
\endsetslot

\ifnumber{\int{ligaturing}}<{0}\then \skipslots{1}\Else

\setslot{\lclig{IJ}{ij}}
\comment{The letter `ij'.  This is a single letter, and in a 
  monowidth font should ideally be one letter wide.}
\endsetslot

\Fi

\setslot{exclamdown}
\comment{The Spanish punctuation mark `!`'.}
\endsetslot

\setslot{questiondown}
\comment{The Spanish punctuation mark `?`'.}
\endsetslot

\setslot{sterling}
\comment{The British currency mark `\textsterling'.}
\endsetslot

\setslot{\uctop{Agrave}{agrave}}
\comment{The letter `\` A'.}
\endsetslot

\setslot{\uctop{Aacute}{aacute}}
\comment{The letter `\' A'.}
\endsetslot

\setslot{\uctop{Acircumflex}{acircumflex}}
\comment{The letter `\^ A'.}
\endsetslot

\setslot{\uctop{Atilde}{atilde}}
\comment{The letter `\~ A'.}
\endsetslot

\setslot{\uctop{Adieresis}{adieresis}}
\comment{The letter `\" A'.}
\endsetslot

\setslot{\uctop{Aring}{aring}}
\comment{The letter `\r A'.}
\endsetslot

\setslot{\uc{AE}{ae}}
\comment{The letter `\AE'.  This is a single letter, and should not be
  faked with `AE'.}
\endsetslot

\setslot{\uc{Ccedilla}{ccedilla}}
\comment{The letter `\c C'.}
\endsetslot

\setslot{\uctop{Egrave}{egrave}}
\comment{The letter `\` E'.}
\endsetslot

\setslot{\uctop{Eacute}{eacute}}
\comment{The letter `\' E'.}
\endsetslot

\setslot{\uctop{Ecircumflex}{ecircumflex}}
\comment{The letter `\^ E'.}
\endsetslot

\setslot{\uctop{Edieresis}{edieresis}}
\comment{The letter `\" E'.}
\endsetslot

\setslot{\uctop{Igrave}{igrave}}
\comment{The letter `\` I'.}
\endsetslot

\setslot{\uctop{Iacute}{iacute}}
\comment{The letter `\' I'.}
\endsetslot

\setslot{\uctop{Icircumflex}{icircumflex}}
\comment{The letter `\^ I'.}
\endsetslot

\setslot{\uctop{Idieresis}{idieresis}}
\comment{The letter `\" I'.}
\endsetslot

\setslot{\uc{Eth}{eth}}
\comment{The uppercase Icelandic letter `Eth' similar to a `D'
  with a horizontal bar through the stem.  It is unavailable
  in \plain\ \TeX.}
\endsetslot

\setslot{\uctop{Ntilde}{ntilde}}
\comment{The letter `\~ N'.}
\endsetslot

\setslot{\uctop{Ograve}{ograve}}
\comment{The letter `\` O'.}
\endsetslot

\setslot{\uctop{Oacute}{oacute}}
\comment{The letter `\' O'.}
\endsetslot

\setslot{\uctop{Ocircumflex}{ocircumflex}}
\comment{The letter `\^ O'.}
\endsetslot

\setslot{\uctop{Otilde}{otilde}}
\comment{The letter `\~ O'.}
\endsetslot

\setslot{\uctop{Odieresis}{odieresis}}
\comment{The letter `\" O'.}
\endsetslot

\setslot{\uc{OE}{oe}}
\comment{The letter `\OE'.  This is a single letter, and should not be
  faked with `OE'.}
\endsetslot

\setslot{\uc{Oslash}{oslash}}
\comment{The letter `\O'.}
\endsetslot

\setslot{\uctop{Ugrave}{ugrave}}
\comment{The letter `\` U'.}
\endsetslot

\setslot{\uctop{Uacute}{uacute}}
\comment{The letter `\' U'.}
\endsetslot

\setslot{\uctop{Ucircumflex}{ucircumflex}}
\comment{The letter `\^ U'.}
\endsetslot

\setslot{\uctop{Udieresis}{udieresis}}
\comment{The letter `\" U'.}
\endsetslot

\setslot{\uctop{Yacute}{yacute}}
\comment{The letter `\' Y'.}
\endsetslot

\setslot{\uc{Thorn}{thorn}}
\comment{The Icelandic capital letter Thorn, similar to a `P'
  with the bowl moved down.  It is unavailable in \plain\ \TeX.}
\endsetslot

\setslot{\uclig{SS}{germandbls}}
\comment{The ligature `SS', used to give an upper case `\ss'.
  In a monowidth font it should be two letters wide.}
\endsetslot

\setslot{\lctop{Agrave}{agrave}}
\comment{The letter `\` a'.}
\endsetslot

\setslot{\lctop{Aacute}{aacute}}
\comment{The letter `\' a'.}
\endsetslot

\setslot{\lctop{Acircumflex}{acircumflex}}
\comment{The letter `\^ a'.}
\endsetslot

\setslot{\lctop{Atilde}{atilde}}
\comment{The letter `\~ a'.}
\endsetslot

\setslot{\lctop{Adieresis}{adieresis}}
\comment{The letter `\" a'.}
\endsetslot

\setslot{\lctop{Aring}{aring}}
\comment{The letter `\r a'.}
\endsetslot

\setslot{\lc{AE}{ae}}
\comment{The letter `\ae'.  This is a single letter, and should not be
  faked with `ae'.}
\endsetslot

\setslot{\lc{Ccedilla}{ccedilla}}
\comment{The letter `\c c'.}
\endsetslot

\setslot{\lctop{Egrave}{egrave}}
\comment{The letter `\` e'.}
\endsetslot

\setslot{\lctop{Eacute}{eacute}}
\comment{The letter `\' e'.}
\endsetslot

\setslot{\lctop{Ecircumflex}{ecircumflex}}
\comment{The letter `\^ e'.}
\endsetslot

\setslot{\lctop{Edieresis}{edieresis}}
\comment{The letter `\" e'.}
\endsetslot

\setslot{\lctop{Igrave}{igrave}}
\comment{The letter `\`\i'.}
\endsetslot

\setslot{\lctop{Iacute}{iacute}}
\comment{The letter `\'\i'.}
\endsetslot

\setslot{\lctop{Icircumflex}{icircumflex}}
\comment{The letter `\^\i'.}
\endsetslot

\setslot{\lctop{Idieresis}{idieresis}}
\comment{The letter `\"\i'.}
\endsetslot

\setslot{\lc{Eth}{eth}}
\comment{The Icelandic lowercase letter `eth' similar to
  a `$\partial$' with an oblique bar through the stem.
  It is unavailable in \plain\ \TeX.}
\endsetslot

\setslot{\lctop{Ntilde}{ntilde}}
\comment{The letter `\~ n'.}
\endsetslot

\setslot{\lctop{Ograve}{ograve}}
\comment{The letter `\` o'.}
\endsetslot

\setslot{\lctop{Oacute}{oacute}}
\comment{The letter `\' o'.}
\endsetslot

\setslot{\lctop{Ocircumflex}{ocircumflex}}
\comment{The letter `\^ o'.}
\endsetslot

\setslot{\lctop{Otilde}{otilde}}
\comment{The letter `\~ o'.}
\endsetslot

\setslot{\lctop{Odieresis}{odieresis}}
\comment{The letter `\" o'.}
\endsetslot

\setslot{\lc{OE}{oe}}
\comment{The letter `\oe'.  This is a single letter, and should not be
  faked with `oe'.}
\endsetslot

\setslot{\lc{Oslash}{oslash}}
\comment{The letter `\o'.}
\endsetslot

\setslot{\lctop{Ugrave}{ugrave}}
\comment{The letter `\` u'.}
\endsetslot

\setslot{\lctop{Uacute}{uacute}}
\comment{The letter `\' u'.}
\endsetslot

\setslot{\lctop{Ucircumflex}{ucircumflex}}
\comment{The letter `\^ u'.}
\endsetslot

\setslot{\lctop{Udieresis}{udieresis}}
\comment{The letter `\" u'.}
\endsetslot

\setslot{\lctop{Yacute}{yacute}}
\comment{The letter `\' y'.}
\endsetslot

\setslot{\lc{Thorn}{thorn}}
\comment{The Icelandic lowercase letter `thorn', similar to a `p'
  with an ascender rising from the stem.  It is unavailable
  in \plain\ \TeX.}
\endsetslot

\setslot{\lc{SS}{germandbls}}
\comment{The letter `\ss'.}
\endsetslot

\endencoding
%    \end{macrocode}
% \iffalse
%</t1-yesw>
% \fi
% \iffalse
%<*t1j-yes>
% \fi
%    \begin{macrocode}
\relax
\encoding

\needsfontinstversion{1.910}

\setcommand\lc#1#2{#2}
\setcommand\uc#1#2{#1}
\setcommand\lctop#1#2{#2}
\setcommand\uctop#1#2{#1}
\setcommand\lclig#1#2{#2}
\ifisint{letterspacing}\then
\ifnumber{\int{letterspacing}}={0}\then \Else
\setcommand\uclig#1#2{#1spaced}
\comment{Here we set \verb|\uclig#1#2| to \verb|#1spaced|, but 
  you can't see it as \verb|\setcommand| commands are invisible in 
  the typeset output.}
\Fi
\Fi
\setcommand\uclig#1#2{#1}
\setcommand\digit#1{#1.oldstyle}

\ifisint{monowidth}\then
\setint{ligaturing}{0}
\Else
% The following empty line is *important* to get the formatting
% right here (sigh)! (Remember that it is a \par token.)

\ifisint{letterspacing}\then
\ifnumber{\int{letterspacing}}={0}\then \Else
\setint{ligaturing}{0}
\Fi
\Fi
\setint{ligaturing}{1}
\Fi

\setint{italicslant}{0}
\setint{quad}{1000}
\setint{baselineskip}{1200}

\ifisglyph{x}\then
\setint{xheight}{\height{x}}
\Else
\setint{xheight}{500}
\Fi

\ifisglyph{space}\then
\setint{interword}{\width{space}}
\Else\ifisglyph{i}\then
\setint{interword}{\width{i}}
\Else
\setint{interword}{333}
\Fi\Fi

\ifisint{monowidth}\then
\setint{stretchword}{0}
\setint{shrinkword}{0}
\setint{extraspace}{\int{interword}}
\Else
\setint{stretchword}{\scale{\int{interword}}{600}}
\setint{shrinkword}{\scale{\int{interword}}{240}}
\setint{extraspace}{\scale{\int{interword}}{240}}
\Fi

\ifisglyph{X}\then
\setint{capheight}{\height{X}}
\Else
\setint{capheight}{750}
\Fi

\ifisglyph{d}\then
\setint{ascender}{\height{d}}
\Else\ifisint{capheight}\then
\setint{ascender}{\int{capheight}}
\Else
\setint{ascender}{750}
\Fi\Fi

\ifisglyph{Aring}\then
\setint{acccapheight}{\height{Aring}}
\Else
\setint{acccapheight}{999}
\Fi

\ifisint{descender_neg}\then
\setint{descender}{\neg{\int{descender_neg}}}
\Else\ifisglyph{p}\then
\setint{descender}{\depth{p}}
\Else
\setint{descender}{250}
\Fi\Fi

\ifisglyph{Aring}\then
\setint{maxheight}{\height{Aring}}
\Else
\setint{maxheight}{1000}
\Fi

\ifisint{maxdepth_neg}\then
\setint{maxdepth}{\neg{\int{maxdepth_neg}}}
\Else\ifisglyph{j}\then
\setint{maxdepth}{\depth{j}}
\Else
\setint{maxdepth}{250}
\Fi\Fi

\ifisglyph{six}\then
\setint{digitwidth}{\width{six}}
\Else
\setint{digitwidth}{500}
\Fi

\setint{capstem}{0} % not in AFM files

\setfontdimen{1}{italicslant}    % italic slant
\setfontdimen{2}{interword}      % interword space
\setfontdimen{3}{stretchword}    % interword stretch
\setfontdimen{4}{shrinkword}     % interword shrink
\setfontdimen{5}{xheight}        % x-height
\setfontdimen{6}{quad}           % quad
\setfontdimen{7}{extraspace}     % extra space after .
\setfontdimen{8}{capheight}      % cap height
\setfontdimen{9}{ascender}       % ascender
\setfontdimen{10}{acccapheight}  % accented cap height
\setfontdimen{11}{descender}     % descender's depth
\setfontdimen{12}{maxheight}     % max height
\setfontdimen{13}{maxdepth}      % max depth
\setfontdimen{14}{digitwidth}    % digit width
\setfontdimen{15}{verticalstem}  % dominant width of verical stems
\setfontdimen{16}{baselineskip}  % baselineskip

\ifnumber{\int{ligaturing}}<{0}\then 
\comment{In this case, the codingscheme can be different from the 
  default, and therefore we refrain from setting it.}
\Else
\setstr{codingscheme}{EXTENDED TEX ENC - ELECTRUM OSF}
\Fi

\setslot{\lc{Grave}{grave}}
\comment{The grave accent `\`{}'.}
\endsetslot

\setslot{\lc{Acute}{acute}}
\comment{The acute accent `\'{}'.}
\endsetslot

\setslot{\lc{Circumflex}{circumflex}}
\comment{The circumflex accent `\^{}'.}
\endsetslot

\setslot{\lc{Tilde}{tilde}}
\comment{The tilde accent `\~{}'.}
\endsetslot

\setslot{\lc{Dieresis}{dieresis}}
\comment{The umlaut or dieresis accent `\"{}'.}
\endsetslot

\setslot{\lc{Hungarumlaut}{hungarumlaut}}
\comment{The long Hungarian umlaut `\H{}'.}
\endsetslot

\setslot{\lc{Ring}{ring}}
\comment{The ring accent `\r{}'.}
\endsetslot

\setslot{\lc{Caron}{caron}}
\comment{The caron or h\'a\v cek accent `\v{}'.}
\endsetslot

\setslot{\lc{Breve}{breve}}
\comment{The breve accent `\u{}'.}
\endsetslot

\setslot{\lc{Macron}{macron}}
\comment{The macron accent `\={}'.}
\endsetslot

\setslot{\lc{Dotaccent}{dotaccent}}
\comment{The dot accent `\.{}'.}
\endsetslot

\setslot{\lc{Cedilla}{cedilla}}
\comment{The cedilla accent `\c {}'.}
\endsetslot

\setslot{\lc{Ogonek}{ogonek}}
\comment{The ogonek accent `\k {}'.}
\endsetslot

\setslot{quotesinglbase}
\comment{A German single quote mark `\quotesinglbase' similar to a comma,
  but with different sidebearings.}
\endsetslot

\setslot{guilsinglleft}
\comment{A French single opening quote mark `\guilsinglleft',
  unavailable in \plain\ \TeX.}
\endsetslot

\setslot{guilsinglright}
\comment{A French single closing quote mark `\guilsinglright',
  unavailable in \plain\ \TeX.}
\endsetslot

\setslot{quotedblleft}
\comment{The English opening quote mark `\,\textquotedblleft\,'.}
\endsetslot

\setslot{quotedblright}
\comment{The English closing quote mark `\,\textquotedblright\,'.}
\endsetslot

\setslot{quotedblbase}
\comment{A German double quote mark `\quotedblbase' similar to two commas,
  but with tighter letterspacing and different sidebearings.}
\endsetslot

\setslot{guillemotleft}
\comment{A French double opening quote mark `\guillemotleft',
  unavailable in \plain\ \TeX.}
\endsetslot

\setslot{guillemotright}
\comment{A French closing opening quote mark `\guillemotright',
  unavailable in \plain\ \TeX.}
\endsetslot

\setslot{endash}
\ligature{LIG}{hyphen}{emdash}
\comment{The number range dash `1--9'. This is called `rangedash' by fontinst's t1.etx, but it needs to be called `endash' to work right. The `\textendash'.  In a monowidth font, this
  might be set as `\texttt{1{-}9}'.}
\endsetslot

\setslot{emdash}
\comment{The punctuation dash `Oh---boy.' This is calle `punctdash' by fontinst's t1.etx, but needs to be called `emdash' to work right. The `\textemdash'.  In a monowidth font, this
  might be set as `\texttt{Oh{-}{-}boy.}'}
\endsetslot

\setslot{compwordmark}
\comment{An invisible glyph, with zero width and depth, but the
  height of lowercase letters without ascenders.
  It is used to stop ligaturing in words like `shelf{}ful'.}
\endsetslot

\setslot{zero.denominator}
\comment{A glyph which is placed after `\%' to produce a
  `per-thousand', or twice to produce `per-ten-thousand'.
  Your guess is as good as mine as to what this glyph should look
  like in a monowidth font.}
\endsetslot

\setslot{\lc{dotlessI}{dotlessi}}
\comment{A dotless i `\i', used to produce accented letters such as
  `\=\i'.}
\endsetslot

\setslot{\lc{dotlessJ}{dotlessj}}
\comment{A dotless j `\j', used to produce accented letters such as
  `\=\j'.  Most non-\TeX\ fonts do not have this glyph.}
\endsetslot

\ifnumber{\int{ligaturing}}<{0}\then \skipslots{5}\Else

\setslot{\lclig{FF}{f_f}}
\ifnumber{\int{ligaturing}}>{0}\then
\ligature{LIG}{\lc{I}{i}}{\lclig{FFI}{f_f_i}}
\ligature{LIG}{\lc{L}{l}}{\lclig{FFL}{f_f_l}}
\Fi
\comment{The `ff' ligature.  It should be two characters wide in a
  monowidth font.}
\endsetslot

\setslot{\lclig{FI}{fi}}
\comment{The `fi' ligature.  It should be two characters wide in a
  monowidth font.}
\endsetslot

\setslot{\lclig{FL}{fl}}
\comment{The `fl' ligature.  It should be two characters wide in a
  monowidth font.}
\endsetslot

\setslot{\lclig{FFI}{f_f_i}}
\comment{The `ffi' ligature.  It should be three characters wide in a
  monowidth font.}
\endsetslot

\setslot{\lclig{FFL}{f_f_l}}
\comment{The `ffl' ligature.  It should be three characters wide in a
  monowidth font.}
\endsetslot

\Fi

\setslot{visiblespace}
\comment{A visible space glyph `\textvisiblespace'.}
\endsetslot

\setslot{exclam}
\ligature{LIG}{quoteleft}{exclamdown}
\comment{The exclamation mark `!'.}
\endsetslot

\setslot{quotedbl}
\comment{The `neutral' double quotation mark `\,\textquotedbl\,',
  included for use in monowidth fonts, or for setting computer
  programs.  Note that the inclusion of this glyph in this slot
  means that \TeX\ documents which used `{\tt\char`\"}' as an
  input character will no longer work.}
\endsetslot

\setslot{numbersign}
\comment{The hash sign `\#'.}
\endsetslot

\setslot{dollar.oldstyle}
\comment{The dollar sign `\$'.}
\endsetslot

\setslot{percent.oldstyle}
\comment{The percent sign `\%'.}
\endsetslot

\setslot{ampersand.oldstyle}
\comment{The ampersand sign `\&'.}
\endsetslot

\setslot{quoteright}
\ligature{LIG}{quoteright}{quotedblright}
\comment{The English closing single quote mark `\,\textquoteright\,'.}
\endsetslot

\setslot{parenleft}
\comment{The opening parenthesis `('.}
\endsetslot

\setslot{parenright}
\comment{The closing parenthesis `)'.}
\endsetslot

\setslot{asterisk}
\comment{The raised asterisk `*'.}
\endsetslot

\setslot{plus}
\comment{The addition sign `+'.}
\endsetslot

\setslot{comma}
\ligature{LIG}{comma}{quotedblbase}
\comment{The comma `,'.}
\endsetslot

\setslot{hyphen}
\ligature{LIG}{hyphen}{endash}
\ligature{LIG}{hyphenchar}{hyphenchar}
\comment{The hyphen `-'.}
\endsetslot

\setslot{period}
\comment{The period `.'.}
\endsetslot

\setslot{slash}
\comment{The forward oblique `/'.}
\endsetslot

\setslot{\digit{zero}}
\comment{The number `0'.  This (and all the other numerals) may be
  old style or ranging digits.}
\endsetslot

\setslot{\digit{one}}
\comment{The number `1'.}
\endsetslot

\setslot{\digit{two}}
\comment{The number `2'.}
\endsetslot

\setslot{\digit{three}}
\comment{The number `3'.}
\endsetslot

\setslot{\digit{four}}
\comment{The number `4'.}
\endsetslot

\setslot{\digit{five}}
\comment{The number `5'.}
\endsetslot

\setslot{\digit{six}}
\comment{The number `6'.}
\endsetslot

\setslot{\digit{seven}}
\comment{The number `7'.}
\endsetslot

\setslot{\digit{eight}}
\comment{The number `8'.}
\endsetslot

\setslot{\digit{nine}}
\comment{The number `9'.}
\endsetslot

\setslot{colon}
\comment{The colon punctuation mark `:'.}
\endsetslot

\setslot{semicolon}
\comment{The semi-colon punctuation mark `;'.}
\endsetslot

\setslot{less}
\ligature{LIG}{less}{guillemotleft}
\comment{The less-than sign `\textless'.}
\endsetslot

\setslot{equal}
\comment{The equals sign `='.}
\endsetslot

\setslot{greater}
\ligature{LIG}{greater}{guillemotright}
\comment{The greater-than sign `\textgreater'.}
\endsetslot

\setslot{question}
\ligature{LIG}{quoteleft}{questiondown}
\comment{The question mark `?'.}
\endsetslot

\setslot{at}
\comment{The at sign `@'.}
\endsetslot

\setslot{\uc{A}{a}}
\comment{The letter `{A}'.}
\endsetslot

\setslot{\uc{B}{b}}
\comment{The letter `{B}'.}
\endsetslot

\setslot{\uc{C}{c}}
\comment{The letter `{C}'.}
\endsetslot

\setslot{\uc{D}{d}}
\comment{The letter `{D}'.}
\endsetslot

\setslot{\uc{E}{e}}
\comment{The letter `{E}'.}
\endsetslot

\setslot{\uc{F}{f}}
\comment{The letter `{F}'.}
\endsetslot

\setslot{\uc{G}{g}}
\comment{The letter `{G}'.}
\endsetslot

\setslot{\uc{H}{h}}
\comment{The letter `{H}'.}
\endsetslot

\ifnumber{\int{ligaturing}}<{-1}\then \skipslots{1}\Else

\setslot{\uc{I}{i}}
\comment{The letter `{I}'.}
\endsetslot

\Fi

\setslot{\uc{J}{j}}
\comment{The letter `{J}'.}
\endsetslot

\setslot{\uc{K}{k}}
\comment{The letter `{K}'.}
\endsetslot

\setslot{\uc{L}{l}}
\comment{The letter `{L}'.}
\endsetslot

\setslot{\uc{M}{m}}
\comment{The letter `{M}'.}
\endsetslot

\setslot{\uc{N}{n}}
\comment{The letter `{N}'.}
\endsetslot

\setslot{\uc{O}{o}}
\comment{The letter `{O}'.}
\endsetslot

\setslot{\uc{P}{p}}
\comment{The letter `{P}'.}
\endsetslot

\setslot{\uc{Q}{q}}
\ifnumber{\int{ligaturing}}>{0}\then
\ligature{LIG}{asterisk}{\uc{Qalt}{q}}
\Fi
\comment{The letter `{Q}'.}
\endsetslot

\setslot{\uc{R}{r}}
\comment{The letter `{R}'.}
\endsetslot

\setslot{\uc{S}{s}}
\comment{The letter `{S}'.}
\endsetslot

\setslot{\uc{T}{t}}
\comment{The letter `{T}'.}
\endsetslot

\setslot{\uc{U}{u}}
\comment{The letter `{U}'.}
\endsetslot

\setslot{\uc{V}{v}}
\comment{The letter `{V}'.}
\endsetslot

\setslot{\uc{W}{w}}
\comment{The letter `{W}'.}
\endsetslot

\setslot{\uc{X}{x}}
\comment{The letter `{X}'.}
\endsetslot

\setslot{\uc{Y}{y}}
\comment{The letter `{Y}'.}
\endsetslot

\setslot{\uc{Z}{z}}
\comment{The letter `{Z}'.}
\endsetslot

\setslot{bracketleft}
\comment{The opening square bracket `['.}
\endsetslot

\setslot{backslash}
\comment{The backwards oblique `\textbackslash'.}
\endsetslot

\setslot{bracketright}
\comment{The closing square bracket `]'.}
\endsetslot

\setslot{asciicircum}
\comment{The ASCII upward-pointing arrow head `\textasciicircum'.
  This is included for compatibility with typewriter fonts used
  for computer listings.}
\endsetslot

\setslot{underscore}
\comment{The ASCII underline character `\textunderscore', usually
  set on the baseline.
  This is included for compatibility with typewriter fonts used
  for computer listings.}
\endsetslot

\setslot{quoteleft}
\ligature{LIG}{quoteleft}{quotedblleft}
\comment{The English opening single quote mark `\,\textquoteleft\,'.}
\endsetslot

\setslot{\lc{A}{a}}
\comment{The letter `{a}'.}
\endsetslot

\setslot{\lc{B}{b}}
\comment{The letter `{b}'.}
\endsetslot

\ifnumber{\int{ligaturing}}<{-1}\then \skipslots{1}\Else

\setslot{\lc{C}{c}}
\comment{The letter `{c}'.}
\endsetslot

\Fi

\setslot{\lc{D}{d}}
\comment{The letter `{d}'.}
\endsetslot

\setslot{\lc{E}{e}}
\comment{The letter `{e}'.}
\endsetslot

\ifnumber{\int{ligaturing}}<{-1}\then \skipslots{1}\Else

\setslot{\lc{F}{f}}
\ifnumber{\int{ligaturing}}>{0}\then
\ligature{LIG}{\lc{I}{i}}{\lclig{FI}{fi}}
\ligature{LIG}{\lc{F}{f}}{\lclig{FF}{f_f}}
\ligature{LIG}{\lc{L}{l}}{\lclig{FL}{fl}}
\Fi
\comment{The letter `{f}'.}
\endsetslot

\Fi

\setslot{\lc{G}{g}}
\comment{The letter `{g}'.}
\endsetslot

\setslot{\lc{H}{h}}
\comment{The letter `{h}'.}
\endsetslot

\ifnumber{\int{ligaturing}}<{-1}\then \skipslots{1}\Else

\setslot{\lc{I}{i}}
\comment{The letter `{i}'.}
\endsetslot

\Fi

\setslot{\lc{J}{j}}
\comment{The letter `{j}'.}
\endsetslot

\setslot{\lc{K}{k}}
\comment{The letter `{k}'.}
\endsetslot

\setslot{\lc{L}{l}}
\comment{The letter `{l}'.}
\endsetslot

\setslot{\lc{M}{m}}
\comment{The letter `{m}'.}
\endsetslot

\setslot{\lc{N}{n}}
\comment{The letter `{n}'.}
\endsetslot

\setslot{\lc{O}{o}}
\comment{The letter `{o}'.}
\endsetslot

\setslot{\lc{P}{p}}
\comment{The letter `{p}'.}
\endsetslot

\setslot{\lc{Q}{q}}
\comment{The letter `{q}'.}
\endsetslot

\setslot{\lc{R}{r}}
\comment{The letter `{r}'.}
\endsetslot

\ifnumber{\int{ligaturing}}<{-1}\then \skipslots{1}\Else

\setslot{\lc{S}{s}}
\comment{The letter `{s}'.}
\endsetslot

\Fi

\setslot{\lc{T}{t}}
\ifnumber{\int{ligaturing}}>{0}\then
\ligature{LIG}{\lc{T}{t}}{\lclig{TT}{t_t}}
\Fi
\comment{The letter `{t}'.}
\endsetslot

\setslot{\lc{U}{u}}
\comment{The letter `{u}'.}
\endsetslot

\setslot{\lc{V}{v}}
\comment{The letter `{v}'.}
\endsetslot

\setslot{\lc{W}{w}}
\comment{The letter `{w}'.}
\endsetslot

\setslot{\lc{X}{x}}
\comment{The letter `{x}'.}
\endsetslot

\setslot{\lc{Y}{y}}
\comment{The letter `{y}'.}
\endsetslot

\setslot{\lc{Z}{z}}
\comment{The letter `{z}'.}
\endsetslot

\setslot{braceleft}
\comment{The opening curly brace `\textbraceleft'.}
\endsetslot

\setslot{bar}
\comment{The ASCII vertical bar `\textbar'.
  This is included for compatibility with typewriter fonts used
  for computer listings.}
\endsetslot

\setslot{braceright}
\comment{The closing curly brace `\textbraceright'.}
\endsetslot

\setslot{asciitilde}
\comment{The ASCII tilde `\textasciitilde'.
  This is included for compatibility with typewriter fonts used
  for computer listings.}
\endsetslot

\setslot{hyphenchar}
\comment{The glyph used for hyphenation in this font, which will
  almost always be the same as `hyphen'.}
\endsetslot

\setslot{\uctop{Abreve}{abreve}}
\comment{The letter `\u A'.}
\endsetslot

\setslot{\uc{Aogonek}{aogonek}}
\comment{The letter `\k A'.}
\endsetslot

\setslot{\uctop{Cacute}{cacute}}
\comment{The letter `\' C'.}
\endsetslot

\setslot{\uctop{Ccaron}{ccaron}}
\comment{The letter `\v C'.}
\endsetslot

\setslot{\uctop{Dcaron}{dcaron}}
\comment{The letter `\v D'.}
\endsetslot

\setslot{\uctop{Ecaron}{ecaron}}
\comment{The letter `\v E'.}
\endsetslot

\setslot{\uc{Eogonek}{eogonek}}
\comment{The letter `\k E'.}
\endsetslot

\setslot{\uctop{Gbreve}{gbreve}}
\comment{The letter `\u G'.}
\endsetslot

\setslot{\uctop{Lacute}{lacute}}
\comment{The letter `\' L'.}
\endsetslot

\setslot{\uc{Lcaron}{lcaron}}
\comment{The letter `\v L'.}
\endsetslot

\setslot{\uc{Lslash}{lslash}}
\comment{The letter `\L'.}
\endsetslot

\setslot{\uctop{Nacute}{nacute}}
\comment{The letter `\' N'.}
\endsetslot

\setslot{\uctop{Ncaron}{ncaron}}
\comment{The letter `\v N'.}
\endsetslot

\ifnumber{\int{ligaturing}}<{0}\then \skipslots{1}\Else

\setslot{\lclig{TT}{t_t}}
\endsetslot

\Fi

\setslot{\uctop{Ohungarumlaut}{ohungarumlaut}}
\comment{The letter `\H O'.}
\endsetslot

\setslot{\uctop{Racute}{racute}}
\comment{The letter `\' R'.}
\endsetslot

\setslot{\uctop{Rcaron}{rcaron}}
\comment{The letter `\v R'.}
\endsetslot

\setslot{\uctop{Sacute}{sacute}}
\comment{The letter `\' S'.}
\endsetslot

\setslot{\uctop{Scaron}{scaron}}
\comment{The letter `\v S'.}
\endsetslot

\setslot{\uc{Scedilla}{scedilla}}
\comment{The letter `\c S'.}
\endsetslot

\setslot{\uctop{Tcaron}{tcaron}}
\comment{The letter `\v T'.}
\endsetslot

\setslot{\uc{Tcedilla}{tcedilla}}
\comment{The letter `\c T'.}
\endsetslot

\setslot{\uctop{Uhungarumlaut}{uhungarumlaut}}
\comment{The letter `\H U'.}
\endsetslot

\setslot{\uctop{Uring}{uring}}
\comment{The letter `\r U'.}
\endsetslot

\setslot{\uctop{Ydieresis}{ydieresis}}
\comment{The letter `\" Y'.}
\endsetslot

\setslot{\uctop{Zacute}{zacute}}
\comment{The letter `\' Z'.}
\endsetslot

\setslot{\uctop{Zcaron}{zcaron}}
\comment{The letter `\v Z'.}
\endsetslot

\setslot{\uctop{Zdotaccent}{zdotaccent}}
\comment{The letter `\. Z'.}
\endsetslot

\ifnumber{\int{ligaturing}}<{0}\then \skipslots{1}\Else

\setslot{\uclig{IJ}{ij}}
\comment{The letter `IJ'.  This is a single letter, and in a 
  monowidth font should ideally be one letter wide.}
\endsetslot

\Fi

\setslot{\uctop{Idotaccent}{idotaccent}}
\comment{The letter `\. I'.}
\endsetslot

\setslot{\lc{Dbar}{dbar}}
\comment{The letter `\dj'.}
\endsetslot

\setslot{section}
\comment{The section mark `\textsection'.}
\endsetslot

\setslot{\lctop{Abreve}{abreve}}
\comment{The letter `\u a'.}
\endsetslot

\setslot{\lc{Aogonek}{aogonek}}
\comment{The letter `\k a'.}
\endsetslot

\setslot{\lctop{Cacute}{cacute}}
\comment{The letter `\' c'.}
\endsetslot

\setslot{\lctop{Ccaron}{ccaron}}
\comment{The letter `\v c'.}
\endsetslot

\setslot{\lctop{Dcaron}{dcaron}}
\comment{The letter `\v d'.}
\endsetslot

\setslot{\lctop{Ecaron}{ecaron}}
\comment{The letter `\v e'.}
\endsetslot

\setslot{\lc{Eogonek}{eogonek}}
\comment{The letter `\k e'.}
\endsetslot

\setslot{\lctop{Gbreve}{gbreve}}
\comment{The letter `\u g'.}
\endsetslot

\setslot{\lctop{Lacute}{lacute}}
\comment{The letter `\' l'.}
\endsetslot

\setslot{\lc{Lcaron}{lcaron}}
\comment{The letter `\v l'.}
\endsetslot

\setslot{\lc{Lslash}{lslash}}
\comment{The letter `\l'.}
\endsetslot

\setslot{\lctop{Nacute}{nacute}}
\comment{The letter `\' n'.}
\endsetslot

\setslot{\lctop{Ncaron}{ncaron}}
\comment{The letter `\v n'.}
\endsetslot

\setslot{\uc{Qalt}{q}}
\endsetslot

\setslot{\lctop{Ohungarumlaut}{ohungarumlaut}}
\comment{The letter `\H o'.}
\endsetslot

\setslot{\lctop{Racute}{racute}}
\comment{The letter `\' r'.}
\endsetslot

\setslot{\lctop{Rcaron}{rcaron}}
\comment{The letter `\v r'.}
\endsetslot

\setslot{\lctop{Sacute}{sacute}}
\comment{The letter `\' s'.}
\endsetslot

\setslot{\lctop{Scaron}{scaron}}
\comment{The letter `\v s'.}
\endsetslot

\setslot{\lc{Scedilla}{scedilla}}
\comment{The letter `\c s'.}
\endsetslot

\setslot{\lctop{Tcaron}{tcaron}}
\comment{The letter `\v t'.}
\endsetslot

\setslot{\lc{Tcedilla}{tcedilla}}
\comment{The letter `\c t'.}
\endsetslot

\setslot{\lctop{Uhungarumlaut}{uhungarumlaut}}
\comment{The letter `\H u'.}
\endsetslot

\setslot{\lctop{Uring}{uring}}
\comment{The letter `\r u'.}
\endsetslot

\setslot{\lctop{Ydieresis}{ydieresis}}
\comment{The letter `\" y'.}
\endsetslot

\setslot{\lctop{Zacute}{zacute}}
\comment{The letter `\' z'.}
\endsetslot

\setslot{\lctop{Zcaron}{zcaron}}
\comment{The letter `\v z'.}
\endsetslot

\setslot{\lctop{Zdotaccent}{zdotaccent}}
\comment{The letter `\. z'.}
\endsetslot

\ifnumber{\int{ligaturing}}<{0}\then \skipslots{1}\Else

\setslot{\lclig{IJ}{ij}}
\comment{The letter `ij'.  This is a single letter, and in a 
  monowidth font should ideally be one letter wide.}
\endsetslot

\Fi

\setslot{exclamdown}
\comment{The Spanish punctuation mark `!`'.}
\endsetslot

\setslot{questiondown}
\comment{The Spanish punctuation mark `?`'.}
\endsetslot

\setslot{sterling.oldstyle}
\comment{The British currency mark `\textsterling'.}
\endsetslot

\setslot{\uctop{Agrave}{agrave}}
\comment{The letter `\` A'.}
\endsetslot

\setslot{\uctop{Aacute}{aacute}}
\comment{The letter `\' A'.}
\endsetslot

\setslot{\uctop{Acircumflex}{acircumflex}}
\comment{The letter `\^ A'.}
\endsetslot

\setslot{\uctop{Atilde}{atilde}}
\comment{The letter `\~ A'.}
\endsetslot

\setslot{\uctop{Adieresis}{adieresis}}
\comment{The letter `\" A'.}
\endsetslot

\setslot{\uctop{Aring}{aring}}
\comment{The letter `\r A'.}
\endsetslot

\setslot{\uc{AE}{ae}}
\comment{The letter `\AE'.  This is a single letter, and should not be
  faked with `AE'.}
\endsetslot

\setslot{\uc{Ccedilla}{ccedilla}}
\comment{The letter `\c C'.}
\endsetslot

\setslot{\uctop{Egrave}{egrave}}
\comment{The letter `\` E'.}
\endsetslot

\setslot{\uctop{Eacute}{eacute}}
\comment{The letter `\' E'.}
\endsetslot

\setslot{\uctop{Ecircumflex}{ecircumflex}}
\comment{The letter `\^ E'.}
\endsetslot

\setslot{\uctop{Edieresis}{edieresis}}
\comment{The letter `\" E'.}
\endsetslot

\setslot{\uctop{Igrave}{igrave}}
\comment{The letter `\` I'.}
\endsetslot

\setslot{\uctop{Iacute}{iacute}}
\comment{The letter `\' I'.}
\endsetslot

\setslot{\uctop{Icircumflex}{icircumflex}}
\comment{The letter `\^ I'.}
\endsetslot

\setslot{\uctop{Idieresis}{idieresis}}
\comment{The letter `\" I'.}
\endsetslot

\setslot{\uc{Eth}{eth}}
\comment{The uppercase Icelandic letter `Eth' similar to a `D'
  with a horizontal bar through the stem.  It is unavailable
  in \plain\ \TeX.}
\endsetslot

\setslot{\uctop{Ntilde}{ntilde}}
\comment{The letter `\~ N'.}
\endsetslot

\setslot{\uctop{Ograve}{ograve}}
\comment{The letter `\` O'.}
\endsetslot

\setslot{\uctop{Oacute}{oacute}}
\comment{The letter `\' O'.}
\endsetslot

\setslot{\uctop{Ocircumflex}{ocircumflex}}
\comment{The letter `\^ O'.}
\endsetslot

\setslot{\uctop{Otilde}{otilde}}
\comment{The letter `\~ O'.}
\endsetslot

\setslot{\uctop{Odieresis}{odieresis}}
\comment{The letter `\" O'.}
\endsetslot

\setslot{\uc{OE}{oe}}
\comment{The letter `\OE'.  This is a single letter, and should not be
  faked with `OE'.}
\endsetslot

\setslot{\uc{Oslash}{oslash}}
\comment{The letter `\O'.}
\endsetslot

\setslot{\uctop{Ugrave}{ugrave}}
\comment{The letter `\` U'.}
\endsetslot

\setslot{\uctop{Uacute}{uacute}}
\comment{The letter `\' U'.}
\endsetslot

\setslot{\uctop{Ucircumflex}{ucircumflex}}
\comment{The letter `\^ U'.}
\endsetslot

\setslot{\uctop{Udieresis}{udieresis}}
\comment{The letter `\" U'.}
\endsetslot

\setslot{\uctop{Yacute}{yacute}}
\comment{The letter `\' Y'.}
\endsetslot

\setslot{\uc{Thorn}{thorn}}
\comment{The Icelandic capital letter Thorn, similar to a `P'
  with the bowl moved down.  It is unavailable in \plain\ \TeX.}
\endsetslot

\setslot{\uclig{SS}{germandbls}}
\comment{The ligature `SS', used to give an upper case `\ss'.
  In a monowidth font it should be two letters wide.}
\endsetslot

\setslot{\lctop{Agrave}{agrave}}
\comment{The letter `\` a'.}
\endsetslot

\setslot{\lctop{Aacute}{aacute}}
\comment{The letter `\' a'.}
\endsetslot

\setslot{\lctop{Acircumflex}{acircumflex}}
\comment{The letter `\^ a'.}
\endsetslot

\setslot{\lctop{Atilde}{atilde}}
\comment{The letter `\~ a'.}
\endsetslot

\setslot{\lctop{Adieresis}{adieresis}}
\comment{The letter `\" a'.}
\endsetslot

\setslot{\lctop{Aring}{aring}}
\comment{The letter `\r a'.}
\endsetslot

\setslot{\lc{AE}{ae}}
\comment{The letter `\ae'.  This is a single letter, and should not be
  faked with `ae'.}
\endsetslot

\setslot{\lc{Ccedilla}{ccedilla}}
\comment{The letter `\c c'.}
\endsetslot

\setslot{\lctop{Egrave}{egrave}}
\comment{The letter `\` e'.}
\endsetslot

\setslot{\lctop{Eacute}{eacute}}
\comment{The letter `\' e'.}
\endsetslot

\setslot{\lctop{Ecircumflex}{ecircumflex}}
\comment{The letter `\^ e'.}
\endsetslot

\setslot{\lctop{Edieresis}{edieresis}}
\comment{The letter `\" e'.}
\endsetslot

\setslot{\lctop{Igrave}{igrave}}
\comment{The letter `\`\i'.}
\endsetslot

\setslot{\lctop{Iacute}{iacute}}
\comment{The letter `\'\i'.}
\endsetslot

\setslot{\lctop{Icircumflex}{icircumflex}}
\comment{The letter `\^\i'.}
\endsetslot

\setslot{\lctop{Idieresis}{idieresis}}
\comment{The letter `\"\i'.}
\endsetslot

\setslot{\lc{Eth}{eth}}
\comment{The Icelandic lowercase letter `eth' similar to
  a `$\partial$' with an oblique bar through the stem.
  It is unavailable in \plain\ \TeX.}
\endsetslot

\setslot{\lctop{Ntilde}{ntilde}}
\comment{The letter `\~ n'.}
\endsetslot

\setslot{\lctop{Ograve}{ograve}}
\comment{The letter `\` o'.}
\endsetslot

\setslot{\lctop{Oacute}{oacute}}
\comment{The letter `\' o'.}
\endsetslot

\setslot{\lctop{Ocircumflex}{ocircumflex}}
\comment{The letter `\^ o'.}
\endsetslot

\setslot{\lctop{Otilde}{otilde}}
\comment{The letter `\~ o'.}
\endsetslot

\setslot{\lctop{Odieresis}{odieresis}}
\comment{The letter `\" o'.}
\endsetslot

\setslot{\lc{OE}{oe}}
\comment{The letter `\oe'.  This is a single letter, and should not be
  faked with `oe'.}
\endsetslot

\setslot{\lc{Oslash}{oslash}}
\comment{The letter `\o'.}
\endsetslot

\setslot{\lctop{Ugrave}{ugrave}}
\comment{The letter `\` u'.}
\endsetslot

\setslot{\lctop{Uacute}{uacute}}
\comment{The letter `\' u'.}
\endsetslot

\setslot{\lctop{Ucircumflex}{ucircumflex}}
\comment{The letter `\^ u'.}
\endsetslot

\setslot{\lctop{Udieresis}{udieresis}}
\comment{The letter `\" u'.}
\endsetslot

\setslot{\lctop{Yacute}{yacute}}
\comment{The letter `\' y'.}
\endsetslot

\setslot{\lc{Thorn}{thorn}}
\comment{The Icelandic lowercase letter `thorn', similar to a `p'
  with an ascender rising from the stem.  It is unavailable
  in \plain\ \TeX.}
\endsetslot

\setslot{\lc{SS}{germandbls}}
\comment{The letter `\ss'.}
\endsetslot

\endencoding
%    \end{macrocode}
% \iffalse
%</t1j-yes>
% \fi
% \iffalse
%<*t1j-yesw>
% \fi
%    \begin{macrocode}
\relax
\encoding

\needsfontinstversion{1.910}

\setcommand\lc#1#2{#2}
\setcommand\uc#1#2{#1}
\setcommand\lctop#1#2{#2}
\setcommand\uctop#1#2{#1}
\setcommand\lclig#1#2{#2}
\ifisint{letterspacing}\then
\ifnumber{\int{letterspacing}}={0}\then \Else
\setcommand\uclig#1#2{#1spaced}
\comment{Here we set \verb|\uclig#1#2| to \verb|#1spaced|, but 
  you can't see it as \verb|\setcommand| commands are invisible in 
  the typeset output.}
\Fi
\Fi
\setcommand\uclig#1#2{#1}
\setcommand\digit#1{#1.oldstyle}

\ifisint{monowidth}\then
\setint{ligaturing}{0}
\Else
% The following empty line is *important* to get the formatting
% right here (sigh)! (Remember that it is a \par token.)

\ifisint{letterspacing}\then
\ifnumber{\int{letterspacing}}={0}\then \Else
\setint{ligaturing}{0}
\Fi
\Fi
\setint{ligaturing}{1}
\Fi

\setint{italicslant}{0}
\setint{quad}{1000}
\setint{baselineskip}{1200}

\ifisglyph{x}\then
\setint{xheight}{\height{x}}
\Else
\setint{xheight}{500}
\Fi

\ifisglyph{space}\then
\setint{interword}{\width{space}}
\Else\ifisglyph{i}\then
\setint{interword}{\width{i}}
\Else
\setint{interword}{333}
\Fi\Fi

\ifisint{monowidth}\then
\setint{stretchword}{0}
\setint{shrinkword}{0}
\setint{extraspace}{\int{interword}}
\Else
\setint{stretchword}{\scale{\int{interword}}{600}}
\setint{shrinkword}{\scale{\int{interword}}{240}}
\setint{extraspace}{\scale{\int{interword}}{240}}
\Fi

\ifisglyph{X}\then
\setint{capheight}{\height{X}}
\Else
\setint{capheight}{750}
\Fi

\ifisglyph{d}\then
\setint{ascender}{\height{d}}
\Else\ifisint{capheight}\then
\setint{ascender}{\int{capheight}}
\Else
\setint{ascender}{750}
\Fi\Fi

\ifisglyph{Aring}\then
\setint{acccapheight}{\height{Aring}}
\Else
\setint{acccapheight}{999}
\Fi

\ifisint{descender_neg}\then
\setint{descender}{\neg{\int{descender_neg}}}
\Else\ifisglyph{p}\then
\setint{descender}{\depth{p}}
\Else
\setint{descender}{250}
\Fi\Fi

\ifisglyph{Aring}\then
\setint{maxheight}{\height{Aring}}
\Else
\setint{maxheight}{1000}
\Fi

\ifisint{maxdepth_neg}\then
\setint{maxdepth}{\neg{\int{maxdepth_neg}}}
\Else\ifisglyph{j}\then
\setint{maxdepth}{\depth{j}}
\Else
\setint{maxdepth}{250}
\Fi\Fi

\ifisglyph{six}\then
\setint{digitwidth}{\width{six}}
\Else
\setint{digitwidth}{500}
\Fi

\setint{capstem}{0} % not in AFM files

\setfontdimen{1}{italicslant}    % italic slant
\setfontdimen{2}{interword}      % interword space
\setfontdimen{3}{stretchword}    % interword stretch
\setfontdimen{4}{shrinkword}     % interword shrink
\setfontdimen{5}{xheight}        % x-height
\setfontdimen{6}{quad}           % quad
\setfontdimen{7}{extraspace}     % extra space after .
\setfontdimen{8}{capheight}      % cap height
\setfontdimen{9}{ascender}       % ascender
\setfontdimen{10}{acccapheight}  % accented cap height
\setfontdimen{11}{descender}     % descender's depth
\setfontdimen{12}{maxheight}     % max height
\setfontdimen{13}{maxdepth}      % max depth
\setfontdimen{14}{digitwidth}    % digit width
\setfontdimen{15}{verticalstem}  % dominant width of verical stems
\setfontdimen{16}{baselineskip}  % baselineskip

\ifnumber{\int{ligaturing}}<{0}\then 
\comment{In this case, the codingscheme can be different from the 
  default, and therefore we refrain from setting it.}
\Else
\setstr{codingscheme}{EXTENDED TEX ENC - ELECTRUM OSF LIG}
\Fi

\setslot{\lc{Grave}{grave}}
\comment{The grave accent `\`{}'.}
\endsetslot

\setslot{\lc{Acute}{acute}}
\comment{The acute accent `\'{}'.}
\endsetslot

\setslot{\lc{Circumflex}{circumflex}}
\comment{The circumflex accent `\^{}'.}
\endsetslot

\setslot{\lc{Tilde}{tilde}}
\comment{The tilde accent `\~{}'.}
\endsetslot

\setslot{\lc{Dieresis}{dieresis}}
\comment{The umlaut or dieresis accent `\"{}'.}
\endsetslot

\setslot{\lc{Hungarumlaut}{hungarumlaut}}
\comment{The long Hungarian umlaut `\H{}'.}
\endsetslot

\setslot{\lc{Ring}{ring}}
\comment{The ring accent `\r{}'.}
\endsetslot

\setslot{\lc{Caron}{caron}}
\comment{The caron or h\'a\v cek accent `\v{}'.}
\endsetslot

\setslot{\lc{Breve}{breve}}
\comment{The breve accent `\u{}'.}
\endsetslot

\setslot{\lc{Macron}{macron}}
\comment{The macron accent `\={}'.}
\endsetslot

\setslot{\lc{Dotaccent}{dotaccent}}
\comment{The dot accent `\.{}'.}
\endsetslot

\setslot{\lc{Cedilla}{cedilla}}
\comment{The cedilla accent `\c {}'.}
\endsetslot

\setslot{\lc{Ogonek}{ogonek}}
\comment{The ogonek accent `\k {}'.}
\endsetslot

\setslot{quotesinglbase}
\comment{A German single quote mark `\quotesinglbase' similar to a comma,
  but with different sidebearings.}
\endsetslot

\setslot{guilsinglleft}
\comment{A French single opening quote mark `\guilsinglleft',
  unavailable in \plain\ \TeX.}
\endsetslot

\setslot{guilsinglright}
\comment{A French single closing quote mark `\guilsinglright',
  unavailable in \plain\ \TeX.}
\endsetslot

\setslot{quotedblleft}
\comment{The English opening quote mark `\,\textquotedblleft\,'.}
\endsetslot

\setslot{quotedblright}
\comment{The English closing quote mark `\,\textquotedblright\,'.}
\endsetslot

\setslot{quotedblbase}
\comment{A German double quote mark `\quotedblbase' similar to two commas,
  but with tighter letterspacing and different sidebearings.}
\endsetslot

\setslot{guillemotleft}
\comment{A French double opening quote mark `\guillemotleft',
  unavailable in \plain\ \TeX.}
\endsetslot

\setslot{guillemotright}
\comment{A French closing opening quote mark `\guillemotright',
  unavailable in \plain\ \TeX.}
\endsetslot

\setslot{endash}
\ligature{LIG}{hyphen}{emdash}
\comment{The number range dash `1--9'. This is called `rangedash' by fontinst's t1.etx, but it needs to be called `endash' to work right. The `\textendash'.  In a monowidth font, this
  might be set as `\texttt{1{-}9}'.}
\endsetslot

\setslot{emdash}
\comment{The punctuation dash `Oh---boy.' This is calle `punctdash' by fontinst's t1.etx, but needs to be called `emdash' to work right. The `\textemdash'.  In a monowidth font, this
  might be set as `\texttt{Oh{-}{-}boy.}'}
\endsetslot

\setslot{compwordmark}
\comment{An invisible glyph, with zero width and depth, but the
  height of lowercase letters without ascenders.
  It is used to stop ligaturing in words like `shelf{}ful'.}
\endsetslot

\setslot{zero.denominator}
\comment{A glyph which is placed after `\%' to produce a
  `per-thousand', or twice to produce `per-ten-thousand'.
  Your guess is as good as mine as to what this glyph should look
  like in a monowidth font.}
\endsetslot

\setslot{\lc{dotlessI}{dotlessi}}
\comment{A dotless i `\i', used to produce accented letters such as
  `\=\i'.}
\endsetslot

\setslot{\lc{dotlessJ}{dotlessj}}
\comment{A dotless j `\j', used to produce accented letters such as
  `\=\j'.  Most non-\TeX\ fonts do not have this glyph.}
\endsetslot

\ifnumber{\int{ligaturing}}<{0}\then \skipslots{5}\Else

\setslot{\lclig{FF}{f_f}}
\ifnumber{\int{ligaturing}}>{0}\then
\ligature{LIG}{\lc{I}{i}}{\lclig{FFI}{f_f_i}}
\ligature{LIG}{\lc{L}{l}}{\lclig{FFL}{f_f_l}}
\Fi
\comment{The `ff' ligature.  It should be two characters wide in a
  monowidth font.}
\endsetslot

\setslot{\lclig{FI}{fi}}
\comment{The `fi' ligature.  It should be two characters wide in a
  monowidth font.}
\endsetslot

\setslot{\lclig{FL}{fl}}
\comment{The `fl' ligature.  It should be two characters wide in a
  monowidth font.}
\endsetslot

\setslot{\lclig{FFI}{f_f_i}}
\comment{The `ffi' ligature.  It should be three characters wide in a
  monowidth font.}
\endsetslot

\setslot{\lclig{FFL}{f_f_l}}
\comment{The `ffl' ligature.  It should be three characters wide in a
  monowidth font.}
\endsetslot

\Fi

\setslot{visiblespace}
\comment{A visible space glyph `\textvisiblespace'.}
\endsetslot

\setslot{exclam}
\ligature{LIG}{quoteleft}{exclamdown}
\comment{The exclamation mark `!'.}
\endsetslot

\ifnumber{\int{ligaturing}}<{0}\then \skipslots{1}\Else

\setslot{\lclig{CT}{c_t}}
\endsetslot

\Fi

\setslot{numbersign}
\comment{The hash sign `\#'.}
\endsetslot

\setslot{dollar.oldstyle}
\comment{The dollar sign `\$'.}
\endsetslot

\setslot{percent.oldstyle}
\comment{The percent sign `\%'.}
\endsetslot

\setslot{ampersand.oldstyle}
\comment{The ampersand sign `\&'.}
\endsetslot

\setslot{quoteright}
\ligature{LIG}{quoteright}{quotedblright}
\comment{The English closing single quote mark `\,\textquoteright\,'.}
\endsetslot

\setslot{parenleft}
\comment{The opening parenthesis `('.}
\endsetslot

\setslot{parenright}
\comment{The closing parenthesis `)'.}
\endsetslot

\setslot{asterisk}
\comment{The raised asterisk `*'.}
\endsetslot

\setslot{plus}
\comment{The addition sign `+'.}
\endsetslot

\setslot{comma}
\ligature{LIG}{comma}{quotedblbase}
\comment{The comma `,'.}
\endsetslot

\setslot{hyphen}
\ligature{LIG}{hyphen}{endash}
\ligature{LIG}{hyphenchar}{hyphenchar}
\comment{The hyphen `-'.}
\endsetslot

\setslot{period}
\comment{The period `.'.}
\endsetslot

\setslot{slash}
\comment{The forward oblique `/'.}
\endsetslot

\setslot{\digit{zero}}
\comment{The number `0'.  This (and all the other numerals) may be
  old style or ranging digits.}
\endsetslot

\setslot{\digit{one}}
\comment{The number `1'.}
\endsetslot

\setslot{\digit{two}}
\comment{The number `2'.}
\endsetslot

\setslot{\digit{three}}
\comment{The number `3'.}
\endsetslot

\setslot{\digit{four}}
\comment{The number `4'.}
\endsetslot

\setslot{\digit{five}}
\comment{The number `5'.}
\endsetslot

\setslot{\digit{six}}
\comment{The number `6'.}
\endsetslot

\setslot{\digit{seven}}
\comment{The number `7'.}
\endsetslot

\setslot{\digit{eight}}
\comment{The number `8'.}
\endsetslot

\setslot{\digit{nine}}
\comment{The number `9'.}
\endsetslot

\setslot{colon}
\comment{The colon punctuation mark `:'.}
\endsetslot

\setslot{semicolon}
\comment{The semi-colon punctuation mark `;'.}
\endsetslot

\setslot{less}
\ligature{LIG}{less}{guillemotleft}
\comment{The less-than sign `\textless'.}
\endsetslot

\setslot{equal}
\comment{The equals sign `='.}
\endsetslot

\setslot{greater}
\ligature{LIG}{greater}{guillemotright}
\comment{The greater-than sign `\textgreater'.}
\endsetslot

\setslot{question}
\ligature{LIG}{quoteleft}{questiondown}
\comment{The question mark `?'.}
\endsetslot

\setslot{at}
\comment{The at sign `@'.}
\endsetslot

\setslot{\uc{A}{a}}
\comment{The letter `{A}'.}
\endsetslot

\setslot{\uc{B}{b}}
\comment{The letter `{B}'.}
\endsetslot

\setslot{\uc{C}{c}}
\comment{The letter `{C}'.}
\endsetslot

\setslot{\uc{D}{d}}
\comment{The letter `{D}'.}
\endsetslot

\setslot{\uc{E}{e}}
\comment{The letter `{E}'.}
\endsetslot

\setslot{\uc{F}{f}}
\comment{The letter `{F}'.}
\endsetslot

\setslot{\uc{G}{g}}
\comment{The letter `{G}'.}
\endsetslot

\setslot{\uc{H}{h}}
\comment{The letter `{H}'.}
\endsetslot

\ifnumber{\int{ligaturing}}<{-1}\then \skipslots{1}\Else

\setslot{\uc{I}{i}}
\comment{The letter `{I}'.}
\endsetslot

\Fi

\setslot{\uc{J}{j}}
\comment{The letter `{J}'.}
\endsetslot

\setslot{\uc{K}{k}}
\comment{The letter `{K}'.}
\endsetslot

\setslot{\uc{L}{l}}
\comment{The letter `{L}'.}
\endsetslot

\setslot{\uc{M}{m}}
\comment{The letter `{M}'.}
\endsetslot

\setslot{\uc{N}{n}}
\comment{The letter `{N}'.}
\endsetslot

\setslot{\uc{O}{o}}
\comment{The letter `{O}'.}
\endsetslot

\setslot{\uc{P}{p}}
\comment{The letter `{P}'.}
\endsetslot

\setslot{\uc{Qalt}{q}}
\ifnumber{\int{ligaturing}}>{0}\then
\ligature{LIG}{asterisk}{\uc{Q}{q}}
\Fi
\comment{The letter `{Q}'.}
\endsetslot

\setslot{\uc{R}{r}}
\comment{The letter `{R}'.}
\endsetslot

\setslot{\uc{S}{s}}
\comment{The letter `{S}'.}
\endsetslot

\setslot{\uc{T}{t}}
\comment{The letter `{T}'.}
\endsetslot

\setslot{\uc{U}{u}}
\comment{The letter `{U}'.}
\endsetslot

\setslot{\uc{V}{v}}
\comment{The letter `{V}'.}
\endsetslot

\setslot{\uc{W}{w}}
\comment{The letter `{W}'.}
\endsetslot

\setslot{\uc{X}{x}}
\comment{The letter `{X}'.}
\endsetslot

\setslot{\uc{Y}{y}}
\comment{The letter `{Y}'.}
\endsetslot

\setslot{\uc{Z}{z}}
\comment{The letter `{Z}'.}
\endsetslot

\setslot{bracketleft}
\comment{The opening square bracket `['.}
\endsetslot

\setslot{backslash}
\comment{The backwards oblique `\textbackslash'.}
\endsetslot

\setslot{bracketright}
\comment{The closing square bracket `]'.}
\endsetslot

\ifnumber{\int{ligaturing}}<{0}\then \skipslots{1}\Else

\setslot{\lclig{SP}{s_p}}
\endsetslot

\Fi

\setslot{underscore}
\comment{The ASCII underline character `\textunderscore', usually
  set on the baseline.
  This is included for compatibility with typewriter fonts used
  for computer listings.}
\endsetslot

\setslot{quoteleft}
\ligature{LIG}{quoteleft}{quotedblleft}
\comment{The English opening single quote mark `\,\textquoteleft\,'.}
\endsetslot

\setslot{\lc{A}{a}}
\comment{The letter `{a}'.}
\endsetslot

\setslot{\lc{B}{b}}
\comment{The letter `{b}'.}
\endsetslot

\ifnumber{\int{ligaturing}}<{-1}\then \skipslots{1}\Else

\setslot{\lc{C}{c}}
\ifnumber{\int{ligaturing}}>{0}\then
\ligature{LIG}{\lc{T}{t}}{\lclig{CT}{c_t}}
\Fi
\comment{The letter `{c}'.}
\endsetslot

\Fi

\setslot{\lc{D}{d}}
\comment{The letter `{d}'.}
\endsetslot

\setslot{\lc{E}{e}}
\comment{The letter `{e}'.}
\endsetslot

\ifnumber{\int{ligaturing}}<{-1}\then \skipslots{1}\Else

\setslot{\lc{F}{f}}
\ifnumber{\int{ligaturing}}>{0}\then
\ligature{LIG}{\lc{I}{i}}{\lclig{FI}{fi}}
\ligature{LIG}{\lc{F}{f}}{\lclig{FF}{f_f}}
\ligature{LIG}{\lc{L}{l}}{\lclig{FL}{fl}}
\Fi
\comment{The letter `{f}'.}
\endsetslot

\Fi

\setslot{\lc{G}{g}}
\comment{The letter `{g}'.}
\endsetslot

\setslot{\lc{H}{h}}
\comment{The letter `{h}'.}
\endsetslot

\ifnumber{\int{ligaturing}}<{-1}\then \skipslots{1}\Else

\setslot{\lc{I}{i}}
\ifnumber{\int{ligaturing}}>{0}\then
\ligature{LIG}{\lc{T}{t}}{\lclig{IT}{i_t}}
\Fi
\comment{The letter `{i}'.}
\endsetslot

\Fi

\setslot{\lc{J}{j}}
\comment{The letter `{j}'.}
\endsetslot

\setslot{\lc{K}{k}}
\comment{The letter `{k}'.}
\endsetslot

\setslot{\lc{L}{l}}
\comment{The letter `{l}'.}
\endsetslot

\setslot{\lc{M}{m}}
\comment{The letter `{m}'.}
\endsetslot

\setslot{\lc{N}{n}}
\comment{The letter `{n}'.}
\endsetslot

\setslot{\lc{O}{o}}
\comment{The letter `{o}'.}
\endsetslot

\setslot{\lc{P}{p}}
\comment{The letter `{p}'.}
\endsetslot

\setslot{\lc{Qalt}{q}}
\comment{The letter `{q}'.}
\endsetslot

\setslot{\lc{R}{r}}
\comment{The letter `{r}'.}
\endsetslot

\ifnumber{\int{ligaturing}}<{-1}\then \skipslots{1}\Else

\setslot{\lc{S}{s}}
\ifnumber{\int{ligaturing}}>{0}\then
\ligature{LIG}{\lc{P}{p}}{\lclig{SP}{s_p}}
\ligature{LIG}{\lc{T}{t}}{\lclig{ST}{s_t}}
\Fi
\comment{The letter `{s}'.}
\endsetslot

\Fi

\setslot{\lc{T}{t}}
\ifnumber{\int{ligaturing}}>{0}\then
\ligature{LIG}{\lc{T}{t}}{\lclig{TT}{t_t}}
\Fi
\comment{The letter `{t}'.}
\endsetslot

\setslot{\lc{U}{u}}
\comment{The letter `{u}'.}
\endsetslot

\setslot{\lc{V}{v}}
\comment{The letter `{v}'.}
\endsetslot

\setslot{\lc{W}{w}}
\comment{The letter `{w}'.}
\endsetslot

\setslot{\lc{X}{x}}
\comment{The letter `{x}'.}
\endsetslot

\setslot{\lc{Y}{y}}
\comment{The letter `{y}'.}
\endsetslot

\setslot{\lc{Z}{z}}
\comment{The letter `{z}'.}
\endsetslot

\setslot{braceleft}
\comment{The opening curly brace `\textbraceleft'.}
\endsetslot

\setslot{bar}
\comment{The ASCII vertical bar `\textbar'.
  This is included for compatibility with typewriter fonts used
  for computer listings.}
\endsetslot

\setslot{braceright}
\comment{The closing curly brace `\textbraceright'.}
\endsetslot

\ifnumber{\int{ligaturing}}<{0}\then \skipslots{1}\Else

\setslot{\lclig{IT}{i_t}}
\endsetslot

\Fi

\setslot{hyphenchar}
\comment{The glyph used for hyphenation in this font, which will
  almost always be the same as `hyphen'.}
\endsetslot

\setslot{\uctop{Abreve}{abreve}}
\comment{The letter `\u A'.}
\endsetslot

\setslot{\uc{Aogonek}{aogonek}}
\comment{The letter `\k A'.}
\endsetslot

\setslot{\uctop{Cacute}{cacute}}
\comment{The letter `\' C'.}
\endsetslot

\setslot{\uctop{Ccaron}{ccaron}}
\comment{The letter `\v C'.}
\endsetslot

\setslot{\uctop{Dcaron}{dcaron}}
\comment{The letter `\v D'.}
\endsetslot

\setslot{\uctop{Ecaron}{ecaron}}
\comment{The letter `\v E'.}
\endsetslot

\setslot{\uc{Eogonek}{eogonek}}
\comment{The letter `\k E'.}
\endsetslot

\setslot{\uctop{Gbreve}{gbreve}}
\comment{The letter `\u G'.}
\endsetslot

\setslot{\uctop{Lacute}{lacute}}
\comment{The letter `\' L'.}
\endsetslot

\setslot{\uc{Lcaron}{lcaron}}
\comment{The letter `\v L'.}
\endsetslot

\setslot{\uc{Lslash}{lslash}}
\comment{The letter `\L'.}
\endsetslot

\setslot{\uctop{Nacute}{nacute}}
\comment{The letter `\' N'.}
\endsetslot

\setslot{\uctop{Ncaron}{ncaron}}
\comment{The letter `\v N'.}
\endsetslot

\ifnumber{\int{ligaturing}}<{0}\then \skipslots{1}\Else

\setslot{\lclig{TT}{t_t}}
\endsetslot

\Fi

\setslot{\uctop{Ohungarumlaut}{ohungarumlaut}}
\comment{The letter `\H O'.}
\endsetslot

\setslot{\uctop{Racute}{racute}}
\comment{The letter `\' R'.}
\endsetslot

\setslot{\uctop{Rcaron}{rcaron}}
\comment{The letter `\v R'.}
\endsetslot

\setslot{\uctop{Sacute}{sacute}}
\comment{The letter `\' S'.}
\endsetslot

\setslot{\uctop{Scaron}{scaron}}
\comment{The letter `\v S'.}
\endsetslot

\setslot{\uc{Scedilla}{scedilla}}
\comment{The letter `\c S'.}
\endsetslot

\setslot{\uctop{Tcaron}{tcaron}}
\comment{The letter `\v T'.}
\endsetslot

\setslot{\uc{Tcedilla}{tcedilla}}
\comment{The letter `\c T'.}
\endsetslot

\setslot{\uctop{Uhungarumlaut}{uhungarumlaut}}
\comment{The letter `\H U'.}
\endsetslot

\setslot{\uctop{Uring}{uring}}
\comment{The letter `\r U'.}
\endsetslot

\setslot{\uctop{Ydieresis}{ydieresis}}
\comment{The letter `\" Y'.}
\endsetslot

\setslot{\uctop{Zacute}{zacute}}
\comment{The letter `\' Z'.}
\endsetslot

\setslot{\uctop{Zcaron}{zcaron}}
\comment{The letter `\v Z'.}
\endsetslot

\setslot{\uctop{Zdotaccent}{zdotaccent}}
\comment{The letter `\. Z'.}
\endsetslot

\ifnumber{\int{ligaturing}}<{0}\then \skipslots{1}\Else

\setslot{\uclig{IJ}{ij}}
\comment{The letter `IJ'.  This is a single letter, and in a 
  monowidth font should ideally be one letter wide.}
\endsetslot

\Fi

\setslot{\uctop{Idotaccent}{idotaccent}}
\comment{The letter `\. I'.}
\endsetslot

\ifnumber{\int{ligaturing}}<{0}\then \skipslots{1}\Else

\setslot{\lclig{ST}{s_t}}
\endsetslot

\Fi

\setslot{section}
\comment{The section mark `\textsection'.}
\endsetslot

\setslot{\lctop{Abreve}{abreve}}
\comment{The letter `\u a'.}
\endsetslot

\setslot{\lc{Aogonek}{aogonek}}
\comment{The letter `\k a'.}
\endsetslot

\setslot{\lctop{Cacute}{cacute}}
\comment{The letter `\' c'.}
\endsetslot

\setslot{\lctop{Ccaron}{ccaron}}
\comment{The letter `\v c'.}
\endsetslot

\setslot{\lctop{Dcaron}{dcaron}}
\comment{The letter `\v d'.}
\endsetslot

\setslot{\lctop{Ecaron}{ecaron}}
\comment{The letter `\v e'.}
\endsetslot

\setslot{\lc{Eogonek}{eogonek}}
\comment{The letter `\k e'.}
\endsetslot

\setslot{\lctop{Gbreve}{gbreve}}
\comment{The letter `\u g'.}
\endsetslot

\setslot{\lctop{Lacute}{lacute}}
\comment{The letter `\' l'.}
\endsetslot

\setslot{\lc{Lcaron}{lcaron}}
\comment{The letter `\v l'.}
\endsetslot

\setslot{\lc{Lslash}{lslash}}
\comment{The letter `\l'.}
\endsetslot

\setslot{\lctop{Nacute}{nacute}}
\comment{The letter `\' n'.}
\endsetslot

\setslot{\lctop{Ncaron}{ncaron}}
\comment{The letter `\v n'.}
\endsetslot

\setslot{\uc{Q}{q}}
\endsetslot

\setslot{\lctop{Ohungarumlaut}{ohungarumlaut}}
\comment{The letter `\H o'.}
\endsetslot

\setslot{\lctop{Racute}{racute}}
\comment{The letter `\' r'.}
\endsetslot

\setslot{\lctop{Rcaron}{rcaron}}
\comment{The letter `\v r'.}
\endsetslot

\setslot{\lctop{Sacute}{sacute}}
\comment{The letter `\' s'.}
\endsetslot

\setslot{\lctop{Scaron}{scaron}}
\comment{The letter `\v s'.}
\endsetslot

\setslot{\lc{Scedilla}{scedilla}}
\comment{The letter `\c s'.}
\endsetslot

\setslot{\lctop{Tcaron}{tcaron}}
\comment{The letter `\v t'.}
\endsetslot

\setslot{\lc{Tcedilla}{tcedilla}}
\comment{The letter `\c t'.}
\endsetslot

\setslot{\lctop{Uhungarumlaut}{uhungarumlaut}}
\comment{The letter `\H u'.}
\endsetslot

\setslot{\lctop{Uring}{uring}}
\comment{The letter `\r u'.}
\endsetslot

\setslot{\lctop{Ydieresis}{ydieresis}}
\comment{The letter `\" y'.}
\endsetslot

\setslot{\lctop{Zacute}{zacute}}
\comment{The letter `\' z'.}
\endsetslot

\setslot{\lctop{Zcaron}{zcaron}}
\comment{The letter `\v z'.}
\endsetslot

\setslot{\lctop{Zdotaccent}{zdotaccent}}
\comment{The letter `\. z'.}
\endsetslot

\ifnumber{\int{ligaturing}}<{0}\then \skipslots{1}\Else

\setslot{\lclig{IJ}{ij}}
\comment{The letter `ij'.  This is a single letter, and in a 
  monowidth font should ideally be one letter wide.}
\endsetslot

\Fi

\setslot{exclamdown}
\comment{The Spanish punctuation mark `!`'.}
\endsetslot

\setslot{questiondown}
\comment{The Spanish punctuation mark `?`'.}
\endsetslot

\setslot{sterling.oldstyle}
\comment{The British currency mark `\textsterling'.}
\endsetslot

\setslot{\uctop{Agrave}{agrave}}
\comment{The letter `\` A'.}
\endsetslot

\setslot{\uctop{Aacute}{aacute}}
\comment{The letter `\' A'.}
\endsetslot

\setslot{\uctop{Acircumflex}{acircumflex}}
\comment{The letter `\^ A'.}
\endsetslot

\setslot{\uctop{Atilde}{atilde}}
\comment{The letter `\~ A'.}
\endsetslot

\setslot{\uctop{Adieresis}{adieresis}}
\comment{The letter `\" A'.}
\endsetslot

\setslot{\uctop{Aring}{aring}}
\comment{The letter `\r A'.}
\endsetslot

\setslot{\uc{AE}{ae}}
\comment{The letter `\AE'.  This is a single letter, and should not be
  faked with `AE'.}
\endsetslot

\setslot{\uc{Ccedilla}{ccedilla}}
\comment{The letter `\c C'.}
\endsetslot

\setslot{\uctop{Egrave}{egrave}}
\comment{The letter `\` E'.}
\endsetslot

\setslot{\uctop{Eacute}{eacute}}
\comment{The letter `\' E'.}
\endsetslot

\setslot{\uctop{Ecircumflex}{ecircumflex}}
\comment{The letter `\^ E'.}
\endsetslot

\setslot{\uctop{Edieresis}{edieresis}}
\comment{The letter `\" E'.}
\endsetslot

\setslot{\uctop{Igrave}{igrave}}
\comment{The letter `\` I'.}
\endsetslot

\setslot{\uctop{Iacute}{iacute}}
\comment{The letter `\' I'.}
\endsetslot

\setslot{\uctop{Icircumflex}{icircumflex}}
\comment{The letter `\^ I'.}
\endsetslot

\setslot{\uctop{Idieresis}{idieresis}}
\comment{The letter `\" I'.}
\endsetslot

\setslot{\uc{Eth}{eth}}
\comment{The uppercase Icelandic letter `Eth' similar to a `D'
  with a horizontal bar through the stem.  It is unavailable
  in \plain\ \TeX.}
\endsetslot

\setslot{\uctop{Ntilde}{ntilde}}
\comment{The letter `\~ N'.}
\endsetslot

\setslot{\uctop{Ograve}{ograve}}
\comment{The letter `\` O'.}
\endsetslot

\setslot{\uctop{Oacute}{oacute}}
\comment{The letter `\' O'.}
\endsetslot

\setslot{\uctop{Ocircumflex}{ocircumflex}}
\comment{The letter `\^ O'.}
\endsetslot

\setslot{\uctop{Otilde}{otilde}}
\comment{The letter `\~ O'.}
\endsetslot

\setslot{\uctop{Odieresis}{odieresis}}
\comment{The letter `\" O'.}
\endsetslot

\setslot{\uc{OE}{oe}}
\comment{The letter `\OE'.  This is a single letter, and should not be
  faked with `OE'.}
\endsetslot

\setslot{\uc{Oslash}{oslash}}
\comment{The letter `\O'.}
\endsetslot

\setslot{\uctop{Ugrave}{ugrave}}
\comment{The letter `\` U'.}
\endsetslot

\setslot{\uctop{Uacute}{uacute}}
\comment{The letter `\' U'.}
\endsetslot

\setslot{\uctop{Ucircumflex}{ucircumflex}}
\comment{The letter `\^ U'.}
\endsetslot

\setslot{\uctop{Udieresis}{udieresis}}
\comment{The letter `\" U'.}
\endsetslot

\setslot{\uctop{Yacute}{yacute}}
\comment{The letter `\' Y'.}
\endsetslot

\setslot{\uc{Thorn}{thorn}}
\comment{The Icelandic capital letter Thorn, similar to a `P'
  with the bowl moved down.  It is unavailable in \plain\ \TeX.}
\endsetslot

\setslot{\uclig{SS}{germandbls}}
\comment{The ligature `SS', used to give an upper case `\ss'.
  In a monowidth font it should be two letters wide.}
\endsetslot

\setslot{\lctop{Agrave}{agrave}}
\comment{The letter `\` a'.}
\endsetslot

\setslot{\lctop{Aacute}{aacute}}
\comment{The letter `\' a'.}
\endsetslot

\setslot{\lctop{Acircumflex}{acircumflex}}
\comment{The letter `\^ a'.}
\endsetslot

\setslot{\lctop{Atilde}{atilde}}
\comment{The letter `\~ a'.}
\endsetslot

\setslot{\lctop{Adieresis}{adieresis}}
\comment{The letter `\" a'.}
\endsetslot

\setslot{\lctop{Aring}{aring}}
\comment{The letter `\r a'.}
\endsetslot

\setslot{\lc{AE}{ae}}
\comment{The letter `\ae'.  This is a single letter, and should not be
  faked with `ae'.}
\endsetslot

\setslot{\lc{Ccedilla}{ccedilla}}
\comment{The letter `\c c'.}
\endsetslot

\setslot{\lctop{Egrave}{egrave}}
\comment{The letter `\` e'.}
\endsetslot

\setslot{\lctop{Eacute}{eacute}}
\comment{The letter `\' e'.}
\endsetslot

\setslot{\lctop{Ecircumflex}{ecircumflex}}
\comment{The letter `\^ e'.}
\endsetslot

\setslot{\lctop{Edieresis}{edieresis}}
\comment{The letter `\" e'.}
\endsetslot

\setslot{\lctop{Igrave}{igrave}}
\comment{The letter `\`\i'.}
\endsetslot

\setslot{\lctop{Iacute}{iacute}}
\comment{The letter `\'\i'.}
\endsetslot

\setslot{\lctop{Icircumflex}{icircumflex}}
\comment{The letter `\^\i'.}
\endsetslot

\setslot{\lctop{Idieresis}{idieresis}}
\comment{The letter `\"\i'.}
\endsetslot

\setslot{\lc{Eth}{eth}}
\comment{The Icelandic lowercase letter `eth' similar to
  a `$\partial$' with an oblique bar through the stem.
  It is unavailable in \plain\ \TeX.}
\endsetslot

\setslot{\lctop{Ntilde}{ntilde}}
\comment{The letter `\~ n'.}
\endsetslot

\setslot{\lctop{Ograve}{ograve}}
\comment{The letter `\` o'.}
\endsetslot

\setslot{\lctop{Oacute}{oacute}}
\comment{The letter `\' o'.}
\endsetslot

\setslot{\lctop{Ocircumflex}{ocircumflex}}
\comment{The letter `\^ o'.}
\endsetslot

\setslot{\lctop{Otilde}{otilde}}
\comment{The letter `\~ o'.}
\endsetslot

\setslot{\lctop{Odieresis}{odieresis}}
\comment{The letter `\" o'.}
\endsetslot

\setslot{\lc{OE}{oe}}
\comment{The letter `\oe'.  This is a single letter, and should not be
  faked with `oe'.}
\endsetslot

\setslot{\lc{Oslash}{oslash}}
\comment{The letter `\o'.}
\endsetslot

\setslot{\lctop{Ugrave}{ugrave}}
\comment{The letter `\` u'.}
\endsetslot

\setslot{\lctop{Uacute}{uacute}}
\comment{The letter `\' u'.}
\endsetslot

\setslot{\lctop{Ucircumflex}{ucircumflex}}
\comment{The letter `\^ u'.}
\endsetslot

\setslot{\lctop{Udieresis}{udieresis}}
\comment{The letter `\" u'.}
\endsetslot

\setslot{\lctop{Yacute}{yacute}}
\comment{The letter `\' y'.}
\endsetslot

\setslot{\lc{Thorn}{thorn}}
\comment{The Icelandic lowercase letter `thorn', similar to a `p'
  with an ascender rising from the stem.  It is unavailable
  in \plain\ \TeX.}
\endsetslot

\setslot{\lc{SS}{germandbls}}
\comment{The letter `\ss'.}
\endsetslot

\endencoding
%    \end{macrocode}
% \iffalse
%</t1j-yesw>
% \fi
% \iffalse
%<*t1-yesw-sc>
% \fi
%    \begin{macrocode}
\relax
\encoding

\needsfontinstversion{1.910}

\setcommand\lc#1#2{#2}
\setcommand\uc#1#2{#1}
\setcommand\lctop#1#2{#2}
\setcommand\uctop#1#2{#1}
\setcommand\lclig#1#2{#2}
\ifisint{letterspacing}\then
\ifnumber{\int{letterspacing}}={0}\then \Else
\setcommand\uclig#1#2{#1spaced}
\comment{Here we set \verb|\uclig#1#2| to \verb|#1spaced|, but 
  you can't see it as \verb|\setcommand| commands are invisible in 
  the typeset output.}
\Fi
\Fi
\setcommand\uclig#1#2{#1}
\setcommand\digit#1{#1}

\ifisint{monowidth}\then
\setint{ligaturing}{0}
\Else
% The following empty line is *important* to get the formatting
% right here (sigh)! (Remember that it is a \par token.)

\ifisint{letterspacing}\then
\ifnumber{\int{letterspacing}}={0}\then \Else
\setint{ligaturing}{0}
\Fi
\Fi
\setint{ligaturing}{1}
\Fi

\setint{italicslant}{0}
\setint{quad}{1000}
\setint{baselineskip}{1200}

\ifisglyph{x}\then
\setint{xheight}{\height{x}}
\Else
\setint{xheight}{500}
\Fi

\ifisglyph{space}\then
\setint{interword}{\width{space}}
\Else\ifisglyph{i}\then
\setint{interword}{\width{i}}
\Else
\setint{interword}{333}
\Fi\Fi

\ifisint{monowidth}\then
\setint{stretchword}{0}
\setint{shrinkword}{0}
\setint{extraspace}{\int{interword}}
\Else
\setint{stretchword}{\scale{\int{interword}}{600}}
\setint{shrinkword}{\scale{\int{interword}}{240}}
\setint{extraspace}{\scale{\int{interword}}{240}}
\Fi

\ifisglyph{X}\then
\setint{capheight}{\height{X}}
\Else
\setint{capheight}{750}
\Fi

\ifisglyph{d}\then
\setint{ascender}{\height{d}}
\Else\ifisint{capheight}\then
\setint{ascender}{\int{capheight}}
\Else
\setint{ascender}{750}
\Fi\Fi

\ifisglyph{Aring}\then
\setint{acccapheight}{\height{Aring}}
\Else
\setint{acccapheight}{999}
\Fi

\ifisint{descender_neg}\then
\setint{descender}{\neg{\int{descender_neg}}}
\Else\ifisglyph{p}\then
\setint{descender}{\depth{p}}
\Else
\setint{descender}{250}
\Fi\Fi

\ifisglyph{Aring}\then
\setint{maxheight}{\height{Aring}}
\Else
\setint{maxheight}{1000}
\Fi

\ifisint{maxdepth_neg}\then
\setint{maxdepth}{\neg{\int{maxdepth_neg}}}
\Else\ifisglyph{j}\then
\setint{maxdepth}{\depth{j}}
\Else
\setint{maxdepth}{250}
\Fi\Fi

\ifisglyph{six}\then
\setint{digitwidth}{\width{six}}
\Else
\setint{digitwidth}{500}
\Fi

\setint{capstem}{0} % not in AFM files

\setfontdimen{1}{italicslant}    % italic slant
\setfontdimen{2}{interword}      % interword space
\setfontdimen{3}{stretchword}    % interword stretch
\setfontdimen{4}{shrinkword}     % interword shrink
\setfontdimen{5}{xheight}        % x-height
\setfontdimen{6}{quad}           % quad
\setfontdimen{7}{extraspace}     % extra space after .
\setfontdimen{8}{capheight}      % cap height
\setfontdimen{9}{ascender}       % ascender
\setfontdimen{10}{acccapheight}  % accented cap height
\setfontdimen{11}{descender}     % descender's depth
\setfontdimen{12}{maxheight}     % max height
\setfontdimen{13}{maxdepth}      % max depth
\setfontdimen{14}{digitwidth}    % digit width
\setfontdimen{15}{verticalstem}  % dominant width of verical stems
\setfontdimen{16}{baselineskip}  % baselineskip

\ifnumber{\int{ligaturing}}<{0}\then 
\comment{In this case, the codingscheme can be different from the 
  default, and therefore we refrain from setting it.}
\Else
\setstr{codingscheme}{EXTENDED TEX ENC - ELECTRUM LIG SC}
\Fi

\setslot{\lc{Grave}{grave}}
\comment{The grave accent `\`{}'.}
\endsetslot

\setslot{\lc{Acute}{acute}}
\comment{The acute accent `\'{}'.}
\endsetslot

\setslot{\lc{Circumflex}{circumflex}}
\comment{The circumflex accent `\^{}'.}
\endsetslot

\setslot{\lc{Tilde}{tilde}}
\comment{The tilde accent `\~{}'.}
\endsetslot

\setslot{\lc{Dieresis}{dieresis}}
\comment{The umlaut or dieresis accent `\"{}'.}
\endsetslot

\setslot{\lc{Hungarumlaut}{hungarumlaut}}
\comment{The long Hungarian umlaut `\H{}'.}
\endsetslot

\setslot{\lc{Ring}{ring}}
\comment{The ring accent `\r{}'.}
\endsetslot

\setslot{\lc{Caron}{caron}}
\comment{The caron or h\'a\v cek accent `\v{}'.}
\endsetslot

\setslot{\lc{Breve}{breve}}
\comment{The breve accent `\u{}'.}
\endsetslot

\setslot{\lc{Macron}{macron}}
\comment{The macron accent `\={}'.}
\endsetslot

\setslot{\lc{Dotaccent}{dotaccent}}
\comment{The dot accent `\.{}'.}
\endsetslot

\setslot{\lc{Cedilla}{cedilla}}
\comment{The cedilla accent `\c {}'.}
\endsetslot

\setslot{\lc{Ogonek}{ogonek}}
\comment{The ogonek accent `\k {}'.}
\endsetslot

\setslot{quotesinglbase}
\comment{A German single quote mark `\quotesinglbase' similar to a comma,
  but with different sidebearings.}
\endsetslot

\setslot{guilsinglleft}
\comment{A French single opening quote mark `\guilsinglleft',
  unavailable in \plain\ \TeX.}
\endsetslot

\setslot{guilsinglright}
\comment{A French single closing quote mark `\guilsinglright',
  unavailable in \plain\ \TeX.}
\endsetslot

\setslot{quotedblleft}
\comment{The English opening quote mark `\,\textquotedblleft\,'.}
\endsetslot

\setslot{quotedblright}
\comment{The English closing quote mark `\,\textquotedblright\,'.}
\endsetslot

\setslot{quotedblbase}
\comment{A German double quote mark `\quotedblbase' similar to two commas,
  but with tighter letterspacing and different sidebearings.}
\endsetslot

\setslot{guillemotleft}
\comment{A French double opening quote mark `\guillemotleft',
  unavailable in \plain\ \TeX.}
\endsetslot

\setslot{guillemotright}
\comment{A French closing opening quote mark `\guillemotright',
  unavailable in \plain\ \TeX.}
\endsetslot

\setslot{endash}
\ligature{LIG}{hyphen}{emdash}
\comment{The number range dash `1--9'. This is called `rangedash' by fontinst's t1.etx, but it needs to be called `endash' to work right. The `\textendash'.  In a monowidth font, this
  might be set as `\texttt{1{-}9}'.}
\endsetslot

\setslot{emdash}
\comment{The punctuation dash `Oh---boy.' This is calle `punctdash' by fontinst's t1.etx, but needs to be called `emdash' to work right. The `\textemdash'.  In a monowidth font, this
  might be set as `\texttt{Oh{-}{-}boy.}'}
\endsetslot

\setslot{compwordmark}
\comment{An invisible glyph, with zero width and depth, but the
  height of lowercase letters without ascenders.
  It is used to stop ligaturing in words like `shelf{}ful'.}
\endsetslot

\setslot{zero.denominator}
\comment{A glyph which is placed after `\%' to produce a
  `per-thousand', or twice to produce `per-ten-thousand'.
  Your guess is as good as mine as to what this glyph should look
  like in a monowidth font.}
\endsetslot

\setslot{\lc{dotlessI}{dotlessi}}
\comment{A dotless i `\i', used to produce accented letters such as
  `\=\i'.}
\endsetslot

\setslot{\lc{dotlessJ}{dotlessj}}
\comment{A dotless j `\j', used to produce accented letters such as
  `\=\j'.  Most non-\TeX\ fonts do not have this glyph.}
\endsetslot

\ifnumber{\int{ligaturing}}<{0}\then \skipslots{5}\Else

\setslot{\lclig{FF}{f_f}}
\ifnumber{\int{ligaturing}}>{0}\then
\ligature{LIG}{\lc{I}{i}}{\lclig{FFI}{f_f_i}}
\ligature{LIG}{\lc{L}{l}}{\lclig{FFL}{f_f_l}}
\Fi
\comment{The `ff' ligature.  It should be two characters wide in a
  monowidth font.}
\endsetslot

\setslot{\lclig{FI}{fi}}
\comment{The `fi' ligature.  It should be two characters wide in a
  monowidth font.}
\endsetslot

\setslot{\lclig{FL}{fl}}
\comment{The `fl' ligature.  It should be two characters wide in a
  monowidth font.}
\endsetslot

\setslot{\lclig{FFI}{f_f_i}}
\comment{The `ffi' ligature.  It should be three characters wide in a
  monowidth font.}
\endsetslot

\setslot{\lclig{FFL}{f_f_l}}
\comment{The `ffl' ligature.  It should be three characters wide in a
  monowidth font.}
\endsetslot

\Fi

\setslot{visiblespace}
\comment{A visible space glyph `\textvisiblespace'.}
\endsetslot

\setslot{exclam}
\ligature{LIG}{quoteleft}{exclamdown}
\comment{The exclamation mark `!'.}
\endsetslot

\setslot{quotedbl}
\comment{The `neutral' double quotation mark `\,\textquotedbl\,',
  included for use in monowidth fonts, or for setting computer
  programs.  Note that the inclusion of this glyph in this slot
  means that \TeX\ documents which used `{\tt\char`\"}' as an
  input character will no longer work.}
\endsetslot

\setslot{numbersign}
\comment{The hash sign `\#'.}
\endsetslot

\setslot{dollar}
\comment{The dollar sign `\$'.}
\endsetslot

\setslot{percent}
\comment{The percent sign `\%'.}
\endsetslot

\setslot{ampersand.sc}
\comment{The ampersand sign `\&'.}
\endsetslot

\setslot{quoteright}
\ligature{LIG}{quoteright}{quotedblright}
\comment{The English closing single quote mark `\,\textquoteright\,'.}
\endsetslot

\setslot{parenleft}
\comment{The opening parenthesis `('.}
\endsetslot

\setslot{parenright}
\comment{The closing parenthesis `)'.}
\endsetslot

\setslot{asterisk}
\comment{The raised asterisk `*'.}
\endsetslot

\setslot{plus}
\comment{The addition sign `+'.}
\endsetslot

\setslot{comma}
\ligature{LIG}{comma}{quotedblbase}
\comment{The comma `,'.}
\endsetslot

\setslot{hyphen}
\ligature{LIG}{hyphen}{endash}
\ligature{LIG}{hyphenchar}{hyphenchar}
\comment{The hyphen `-'.}
\endsetslot

\setslot{period}
\comment{The period `.'.}
\endsetslot

\setslot{slash}
\comment{The forward oblique `/'.}
\endsetslot

\setslot{\digit{zero}}
\comment{The number `0'.  This (and all the other numerals) may be
  old style or ranging digits.}
\endsetslot

\setslot{\digit{one}}
\comment{The number `1'.}
\endsetslot

\setslot{\digit{two}}
\comment{The number `2'.}
\endsetslot

\setslot{\digit{three}}
\comment{The number `3'.}
\endsetslot

\setslot{\digit{four}}
\comment{The number `4'.}
\endsetslot

\setslot{\digit{five}}
\comment{The number `5'.}
\endsetslot

\setslot{\digit{six}}
\comment{The number `6'.}
\endsetslot

\setslot{\digit{seven}}
\comment{The number `7'.}
\endsetslot

\setslot{\digit{eight}}
\comment{The number `8'.}
\endsetslot

\setslot{\digit{nine}}
\comment{The number `9'.}
\endsetslot

\setslot{colon}
\comment{The colon punctuation mark `:'.}
\endsetslot

\setslot{semicolon}
\comment{The semi-colon punctuation mark `;'.}
\endsetslot

\setslot{less}
\ligature{LIG}{less}{guillemotleft}
\comment{The less-than sign `\textless'.}
\endsetslot

\setslot{equal}
\comment{The equals sign `='.}
\endsetslot

\setslot{greater}
\ligature{LIG}{greater}{guillemotright}
\comment{The greater-than sign `\textgreater'.}
\endsetslot

\setslot{question}
\ligature{LIG}{quoteleft}{questiondown}
\comment{The question mark `?'.}
\endsetslot

\setslot{at}
\comment{The at sign `@'.}
\endsetslot

\setslot{\uc{A}{a}}
\comment{The letter `{A}'.}
\endsetslot

\setslot{\uc{B}{b}}
\comment{The letter `{B}'.}
\endsetslot

\setslot{\uc{C}{c}}
\comment{The letter `{C}'.}
\endsetslot

\setslot{\uc{D}{d}}
\comment{The letter `{D}'.}
\endsetslot

\setslot{\uc{E}{e}}
\comment{The letter `{E}'.}
\endsetslot

\setslot{\uc{F}{f}}
\comment{The letter `{F}'.}
\endsetslot

\setslot{\uc{G}{g}}
\comment{The letter `{G}'.}
\endsetslot

\setslot{\uc{H}{h}}
\comment{The letter `{H}'.}
\endsetslot

\ifnumber{\int{ligaturing}}<{-1}\then \skipslots{1}\Else

\setslot{\uc{I}{i}}
\comment{The letter `{I}'.}
\endsetslot

\Fi

\setslot{\uc{J}{j}}
\comment{The letter `{J}'.}
\endsetslot

\setslot{\uc{K}{k}}
\comment{The letter `{K}'.}
\endsetslot

\setslot{\uc{L}{l}}
\comment{The letter `{L}'.}
\endsetslot

\setslot{\uc{M}{m}}
\comment{The letter `{M}'.}
\endsetslot

\setslot{\uc{N}{n}}
\comment{The letter `{N}'.}
\endsetslot

\setslot{\uc{O}{o}}
\comment{The letter `{O}'.}
\endsetslot

\setslot{\uc{P}{p}}
\comment{The letter `{P}'.}
\endsetslot

\setslot{\uc{Qalt}{q}}
\ifnumber{\int{ligaturing}}>{0}\then
\ligature{LIG}{asterisk}{\uc{Q}{q}}
\Fi
\comment{The letter `{Qalt}'.}
\endsetslot

\setslot{\uc{R}{r}}
\comment{The letter `{R}'.}
\endsetslot

\setslot{\uc{S}{s}}
\comment{The letter `{S}'.}
\endsetslot

\setslot{\uc{T}{t}}
\comment{The letter `{T}'.}
\endsetslot

\setslot{\uc{U}{u}}
\comment{The letter `{U}'.}
\endsetslot

\setslot{\uc{V}{v}}
\comment{The letter `{V}'.}
\endsetslot

\setslot{\uc{W}{w}}
\comment{The letter `{W}'.}
\endsetslot

\setslot{\uc{X}{x}}
\comment{The letter `{X}'.}
\endsetslot

\setslot{\uc{Y}{y}}
\comment{The letter `{Y}'.}
\endsetslot

\setslot{\uc{Z}{z}}
\comment{The letter `{Z}'.}
\endsetslot

\setslot{bracketleft}
\comment{The opening square bracket `['.}
\endsetslot

\setslot{backslash}
\comment{The backwards oblique `\textbackslash'.}
\endsetslot

\setslot{bracketright}
\comment{The closing square bracket `]'.}
\endsetslot

\setslot{asciicircum}
\comment{The ASCII upward-pointing arrow head `\textasciicircum'.
  This is included for compatibility with typewriter fonts used
  for computer listings.}
\endsetslot

\setslot{underscore}
\comment{The ASCII underline character `\textunderscore', usually
  set on the baseline.
  This is included for compatibility with typewriter fonts used
  for computer listings.}
\endsetslot

\setslot{quoteleft}
\ligature{LIG}{quoteleft}{quotedblleft}
\comment{The English opening single quote mark `\,\textquoteleft\,'.}
\endsetslot

\setslot{\lc{A}{a}}
\comment{The letter `{a}'.}
\endsetslot

\setslot{\lc{B}{b}}
\comment{The letter `{b}'.}
\endsetslot

\ifnumber{\int{ligaturing}}<{-1}\then \skipslots{1}\Else

\setslot{\lc{C}{c}}
\comment{The letter `{c}'.}
\endsetslot

\Fi

\setslot{\lc{D}{d}}
\comment{The letter `{d}'.}
\endsetslot

\setslot{\lc{E}{e}}
\comment{The letter `{e}'.}
\endsetslot

\ifnumber{\int{ligaturing}}<{-1}\then \skipslots{1}\Else

\setslot{\lc{F}{f}}
\ifnumber{\int{ligaturing}}>{0}\then
\ligature{LIG}{\lc{I}{i}}{\lclig{FI}{fi}}
\ligature{LIG}{\lc{F}{f}}{\lclig{FF}{f_f}}
\ligature{LIG}{\lc{L}{l}}{\lclig{FL}{fl}}
\Fi
\comment{The letter `{f}'.}
\endsetslot

\Fi

\setslot{\lc{G}{g}}
\comment{The letter `{g}'.}
\endsetslot

\setslot{\lc{H}{h}}
\comment{The letter `{h}'.}
\endsetslot

\ifnumber{\int{ligaturing}}<{-1}\then \skipslots{1}\Else

\setslot{\lc{I}{i}}
\comment{The letter `{i}'.}
\endsetslot

\Fi

\setslot{\lc{J}{j}}
\comment{The letter `{j}'.}
\endsetslot

\setslot{\lc{K}{k}}
\comment{The letter `{k}'.}
\endsetslot

\setslot{\lc{L}{l}}
\comment{The letter `{l}'.}
\endsetslot

\setslot{\lc{M}{m}}
\comment{The letter `{m}'.}
\endsetslot

\setslot{\lc{N}{n}}
\comment{The letter `{n}'.}
\endsetslot

\setslot{\lc{O}{o}}
\comment{The letter `{o}'.}
\endsetslot

\setslot{\lc{P}{p}}
\comment{The letter `{p}'.}
\endsetslot

\setslot{\lc{Qalt}{q}}
\comment{The letter `{q}'.}
\endsetslot

\setslot{\lc{R}{r}}
\comment{The letter `{r}'.}
\endsetslot

\ifnumber{\int{ligaturing}}<{-1}\then \skipslots{1}\Else

\setslot{\lc{S}{s}}
\comment{The letter `{s}'.}
\endsetslot

\Fi

\setslot{\lc{T}{t}}
\ifnumber{\int{ligaturing}}>{0}\then
\ligature{LIG}{\lc{T}{t}}{\lclig{TT}{t_t}}
\Fi
\comment{The letter `{t}'.}
\endsetslot

\setslot{\lc{U}{u}}
\comment{The letter `{u}'.}
\endsetslot

\setslot{\lc{V}{v}}
\comment{The letter `{v}'.}
\endsetslot

\setslot{\lc{W}{w}}
\comment{The letter `{w}'.}
\endsetslot

\setslot{\lc{X}{x}}
\comment{The letter `{x}'.}
\endsetslot

\setslot{\lc{Y}{y}}
\comment{The letter `{y}'.}
\endsetslot

\setslot{\lc{Z}{z}}
\comment{The letter `{z}'.}
\endsetslot

\setslot{braceleft}
\comment{The opening curly brace `\textbraceleft'.}
\endsetslot

\setslot{bar}
\comment{The ASCII vertical bar `\textbar'.
  This is included for compatibility with typewriter fonts used
  for computer listings.}
\endsetslot

\setslot{braceright}
\comment{The closing curly brace `\textbraceright'.}
\endsetslot

\setslot{asciitilde}
\comment{The ASCII tilde `\textasciitilde'.
  This is included for compatibility with typewriter fonts used
  for computer listings.}
\endsetslot

\setslot{hyphenchar}
\comment{The glyph used for hyphenation in this font, which will
  almost always be the same as `hyphen'.}
\endsetslot

\setslot{\uctop{Abreve}{abreve}}
\comment{The letter `\u A'.}
\endsetslot

\setslot{\uc{Aogonek}{aogonek}}
\comment{The letter `\k A'.}
\endsetslot

\setslot{\uctop{Cacute}{cacute}}
\comment{The letter `\' C'.}
\endsetslot

\setslot{\uctop{Ccaron}{ccaron}}
\comment{The letter `\v C'.}
\endsetslot

\setslot{\uctop{Dcaron}{dcaron}}
\comment{The letter `\v D'.}
\endsetslot

\setslot{\uctop{Ecaron}{ecaron}}
\comment{The letter `\v E'.}
\endsetslot

\setslot{\uc{Eogonek}{eogonek}}
\comment{The letter `\k E'.}
\endsetslot

\setslot{\uctop{Gbreve}{gbreve}}
\comment{The letter `\u G'.}
\endsetslot

\setslot{\uctop{Lacute}{lacute}}
\comment{The letter `\' L'.}
\endsetslot

\setslot{\uc{Lcaron}{lcaron}}
\comment{The letter `\v L'.}
\endsetslot

\setslot{\uc{Lslash}{lslash}}
\comment{The letter `\L'.}
\endsetslot

\setslot{\uctop{Nacute}{nacute}}
\comment{The letter `\' N'.}
\endsetslot

\setslot{\uctop{Ncaron}{ncaron}}
\comment{The letter `\v N'.}
\endsetslot

\ifnumber{\int{ligaturing}}<{0}\then \skipslots{1}\Else

\setslot{\lclig{TT}{t_t}}
\endsetslot

\Fi

\setslot{\uctop{Ohungarumlaut}{ohungarumlaut}}
\comment{The letter `\H O'.}
\endsetslot

\setslot{\uctop{Racute}{racute}}
\comment{The letter `\' R'.}
\endsetslot

\setslot{\uctop{Rcaron}{rcaron}}
\comment{The letter `\v R'.}
\endsetslot

\setslot{\uctop{Sacute}{sacute}}
\comment{The letter `\' S'.}
\endsetslot

\setslot{\uctop{Scaron}{scaron}}
\comment{The letter `\v S'.}
\endsetslot

\setslot{\uc{Scedilla}{scedilla}}
\comment{The letter `\c S'.}
\endsetslot

\setslot{\uctop{Tcaron}{tcaron}}
\comment{The letter `\v T'.}
\endsetslot

\setslot{\uc{Tcedilla}{tcedilla}}
\comment{The letter `\c T'.}
\endsetslot

\setslot{\uctop{Uhungarumlaut}{uhungarumlaut}}
\comment{The letter `\H U'.}
\endsetslot

\setslot{\uctop{Uring}{uring}}
\comment{The letter `\r U'.}
\endsetslot

\setslot{\uctop{Ydieresis}{ydieresis}}
\comment{The letter `\" Y'.}
\endsetslot

\setslot{\uctop{Zacute}{zacute}}
\comment{The letter `\' Z'.}
\endsetslot

\setslot{\uctop{Zcaron}{zcaron}}
\comment{The letter `\v Z'.}
\endsetslot

\setslot{\uctop{Zdotaccent}{zdotaccent}}
\comment{The letter `\. Z'.}
\endsetslot

\ifnumber{\int{ligaturing}}<{0}\then \skipslots{1}\Else

\setslot{\uclig{IJ}{ij}}
\comment{The letter `IJ'.  This is a single letter, and in a 
  monowidth font should ideally be one letter wide.}
\endsetslot

\Fi

\setslot{\uctop{Idotaccent}{idotaccent}}
\comment{The letter `\. I'.}
\endsetslot

\setslot{\lc{Dbar}{dbar}}
\comment{The letter `\dj'.}
\endsetslot

\setslot{section}
\comment{The section mark `\textsection'.}
\endsetslot

\setslot{\lctop{Abreve}{abreve}}
\comment{The letter `\u a'.}
\endsetslot

\setslot{\lc{Aogonek}{aogonek}}
\comment{The letter `\k a'.}
\endsetslot

\setslot{\lctop{Cacute}{cacute}}
\comment{The letter `\' c'.}
\endsetslot

\setslot{\lctop{Ccaron}{ccaron}}
\comment{The letter `\v c'.}
\endsetslot

\setslot{\lctop{Dcaron}{dcaron}}
\comment{The letter `\v d'.}
\endsetslot

\setslot{\lctop{Ecaron}{ecaron}}
\comment{The letter `\v e'.}
\endsetslot

\setslot{\lc{Eogonek}{eogonek}}
\comment{The letter `\k e'.}
\endsetslot

\setslot{\lctop{Gbreve}{gbreve}}
\comment{The letter `\u g'.}
\endsetslot

\setslot{\lctop{Lacute}{lacute}}
\comment{The letter `\' l'.}
\endsetslot

\setslot{\lc{Lcaron}{lcaron}}
\comment{The letter `\v l'.}
\endsetslot

\setslot{\lc{Lslash}{lslash}}
\comment{The letter `\l'.}
\endsetslot

\setslot{\lctop{Nacute}{nacute}}
\comment{The letter `\' n'.}
\endsetslot

\setslot{\lctop{Ncaron}{ncaron}}
\comment{The letter `\v n'.}
\endsetslot

\setslot{\uc{Q}{q}}
\endsetslot

\setslot{\lctop{Ohungarumlaut}{ohungarumlaut}}
\comment{The letter `\H o'.}
\endsetslot

\setslot{\lctop{Racute}{racute}}
\comment{The letter `\' r'.}
\endsetslot

\setslot{\lctop{Rcaron}{rcaron}}
\comment{The letter `\v r'.}
\endsetslot

\setslot{\lctop{Sacute}{sacute}}
\comment{The letter `\' s'.}
\endsetslot

\setslot{\lctop{Scaron}{scaron}}
\comment{The letter `\v s'.}
\endsetslot

\setslot{\lc{Scedilla}{scedilla}}
\comment{The letter `\c s'.}
\endsetslot

\setslot{\lctop{Tcaron}{tcaron}}
\comment{The letter `\v t'.}
\endsetslot

\setslot{\lc{Tcedilla}{tcedilla}}
\comment{The letter `\c t'.}
\endsetslot

\setslot{\lctop{Uhungarumlaut}{uhungarumlaut}}
\comment{The letter `\H u'.}
\endsetslot

\setslot{\lctop{Uring}{uring}}
\comment{The letter `\r u'.}
\endsetslot

\setslot{\lctop{Ydieresis}{ydieresis}}
\comment{The letter `\" y'.}
\endsetslot

\setslot{\lctop{Zacute}{zacute}}
\comment{The letter `\' z'.}
\endsetslot

\setslot{\lctop{Zcaron}{zcaron}}
\comment{The letter `\v z'.}
\endsetslot

\setslot{\lctop{Zdotaccent}{zdotaccent}}
\comment{The letter `\. z'.}
\endsetslot

\ifnumber{\int{ligaturing}}<{0}\then \skipslots{1}\Else

\setslot{\lclig{IJ}{ij}}
\comment{The letter `ij'.  This is a single letter, and in a 
  monowidth font should ideally be one letter wide.}
\endsetslot

\Fi

\setslot{exclamdown}
\comment{The Spanish punctuation mark `!`'.}
\endsetslot

\setslot{questiondown}
\comment{The Spanish punctuation mark `?`'.}
\endsetslot

\setslot{sterling}
\comment{The British currency mark `\textsterling'.}
\endsetslot

\setslot{\uctop{Agrave}{agrave}}
\comment{The letter `\` A'.}
\endsetslot

\setslot{\uctop{Aacute}{aacute}}
\comment{The letter `\' A'.}
\endsetslot

\setslot{\uctop{Acircumflex}{acircumflex}}
\comment{The letter `\^ A'.}
\endsetslot

\setslot{\uctop{Atilde}{atilde}}
\comment{The letter `\~ A'.}
\endsetslot

\setslot{\uctop{Adieresis}{adieresis}}
\comment{The letter `\" A'.}
\endsetslot

\setslot{\uctop{Aring}{aring}}
\comment{The letter `\r A'.}
\endsetslot

\setslot{\uc{AE}{ae}}
\comment{The letter `\AE'.  This is a single letter, and should not be
  faked with `AE'.}
\endsetslot

\setslot{\uc{Ccedilla}{ccedilla}}
\comment{The letter `\c C'.}
\endsetslot

\setslot{\uctop{Egrave}{egrave}}
\comment{The letter `\` E'.}
\endsetslot

\setslot{\uctop{Eacute}{eacute}}
\comment{The letter `\' E'.}
\endsetslot

\setslot{\uctop{Ecircumflex}{ecircumflex}}
\comment{The letter `\^ E'.}
\endsetslot

\setslot{\uctop{Edieresis}{edieresis}}
\comment{The letter `\" E'.}
\endsetslot

\setslot{\uctop{Igrave}{igrave}}
\comment{The letter `\` I'.}
\endsetslot

\setslot{\uctop{Iacute}{iacute}}
\comment{The letter `\' I'.}
\endsetslot

\setslot{\uctop{Icircumflex}{icircumflex}}
\comment{The letter `\^ I'.}
\endsetslot

\setslot{\uctop{Idieresis}{idieresis}}
\comment{The letter `\" I'.}
\endsetslot

\setslot{\uc{Eth}{eth}}
\comment{The uppercase Icelandic letter `Eth' similar to a `D'
  with a horizontal bar through the stem.  It is unavailable
  in \plain\ \TeX.}
\endsetslot

\setslot{\uctop{Ntilde}{ntilde}}
\comment{The letter `\~ N'.}
\endsetslot

\setslot{\uctop{Ograve}{ograve}}
\comment{The letter `\` O'.}
\endsetslot

\setslot{\uctop{Oacute}{oacute}}
\comment{The letter `\' O'.}
\endsetslot

\setslot{\uctop{Ocircumflex}{ocircumflex}}
\comment{The letter `\^ O'.}
\endsetslot

\setslot{\uctop{Otilde}{otilde}}
\comment{The letter `\~ O'.}
\endsetslot

\setslot{\uctop{Odieresis}{odieresis}}
\comment{The letter `\" O'.}
\endsetslot

\setslot{\uc{OE}{oe}}
\comment{The letter `\OE'.  This is a single letter, and should not be
  faked with `OE'.}
\endsetslot

\setslot{\uc{Oslash}{oslash}}
\comment{The letter `\O'.}
\endsetslot

\setslot{\uctop{Ugrave}{ugrave}}
\comment{The letter `\` U'.}
\endsetslot

\setslot{\uctop{Uacute}{uacute}}
\comment{The letter `\' U'.}
\endsetslot

\setslot{\uctop{Ucircumflex}{ucircumflex}}
\comment{The letter `\^ U'.}
\endsetslot

\setslot{\uctop{Udieresis}{udieresis}}
\comment{The letter `\" U'.}
\endsetslot

\setslot{\uctop{Yacute}{yacute}}
\comment{The letter `\' Y'.}
\endsetslot

\setslot{\uc{Thorn}{thorn}}
\comment{The Icelandic capital letter Thorn, similar to a `P'
  with the bowl moved down.  It is unavailable in \plain\ \TeX.}
\endsetslot

\setslot{\uclig{SS}{germandbls}}
\comment{The ligature `SS', used to give an upper case `\ss'.
  In a monowidth font it should be two letters wide.}
\endsetslot

\setslot{\lctop{Agrave}{agrave}}
\comment{The letter `\` a'.}
\endsetslot

\setslot{\lctop{Aacute}{aacute}}
\comment{The letter `\' a'.}
\endsetslot

\setslot{\lctop{Acircumflex}{acircumflex}}
\comment{The letter `\^ a'.}
\endsetslot

\setslot{\lctop{Atilde}{atilde}}
\comment{The letter `\~ a'.}
\endsetslot

\setslot{\lctop{Adieresis}{adieresis}}
\comment{The letter `\" a'.}
\endsetslot

\setslot{\lctop{Aring}{aring}}
\comment{The letter `\r a'.}
\endsetslot

\setslot{\lc{AE}{ae}}
\comment{The letter `\ae'.  This is a single letter, and should not be
  faked with `ae'.}
\endsetslot

\setslot{\lc{Ccedilla}{ccedilla}}
\comment{The letter `\c c'.}
\endsetslot

\setslot{\lctop{Egrave}{egrave}}
\comment{The letter `\` e'.}
\endsetslot

\setslot{\lctop{Eacute}{eacute}}
\comment{The letter `\' e'.}
\endsetslot

\setslot{\lctop{Ecircumflex}{ecircumflex}}
\comment{The letter `\^ e'.}
\endsetslot

\setslot{\lctop{Edieresis}{edieresis}}
\comment{The letter `\" e'.}
\endsetslot

\setslot{\lctop{Igrave}{igrave}}
\comment{The letter `\`\i'.}
\endsetslot

\setslot{\lctop{Iacute}{iacute}}
\comment{The letter `\'\i'.}
\endsetslot

\setslot{\lctop{Icircumflex}{icircumflex}}
\comment{The letter `\^\i'.}
\endsetslot

\setslot{\lctop{Idieresis}{idieresis}}
\comment{The letter `\"\i'.}
\endsetslot

\setslot{\lc{Eth}{eth}}
\comment{The Icelandic lowercase letter `eth' similar to
  a `$\partial$' with an oblique bar through the stem.
  It is unavailable in \plain\ \TeX.}
\endsetslot

\setslot{\lctop{Ntilde}{ntilde}}
\comment{The letter `\~ n'.}
\endsetslot

\setslot{\lctop{Ograve}{ograve}}
\comment{The letter `\` o'.}
\endsetslot

\setslot{\lctop{Oacute}{oacute}}
\comment{The letter `\' o'.}
\endsetslot

\setslot{\lctop{Ocircumflex}{ocircumflex}}
\comment{The letter `\^ o'.}
\endsetslot

\setslot{\lctop{Otilde}{otilde}}
\comment{The letter `\~ o'.}
\endsetslot

\setslot{\lctop{Odieresis}{odieresis}}
\comment{The letter `\" o'.}
\endsetslot

\setslot{\lc{OE}{oe}}
\comment{The letter `\oe'.  This is a single letter, and should not be
  faked with `oe'.}
\endsetslot

\setslot{\lc{Oslash}{oslash}}
\comment{The letter `\o'.}
\endsetslot

\setslot{\lctop{Ugrave}{ugrave}}
\comment{The letter `\` u'.}
\endsetslot

\setslot{\lctop{Uacute}{uacute}}
\comment{The letter `\' u'.}
\endsetslot

\setslot{\lctop{Ucircumflex}{ucircumflex}}
\comment{The letter `\^ u'.}
\endsetslot

\setslot{\lctop{Udieresis}{udieresis}}
\comment{The letter `\" u'.}
\endsetslot

\setslot{\lctop{Yacute}{yacute}}
\comment{The letter `\' y'.}
\endsetslot

\setslot{\lc{Thorn}{thorn}}
\comment{The Icelandic lowercase letter `thorn', similar to a `p'
  with an ascender rising from the stem.  It is unavailable
  in \plain\ \TeX.}
\endsetslot

\setslot{\lc{SS}{germandbls}}
\comment{The letter `\ss'.}
\endsetslot

\endencoding
%    \end{macrocode}
% \iffalse
%</t1-yesw-sc>
% \fi
% \iffalse
%<*t1j-yesw-sc>
% \fi
%    \begin{macrocode}
\relax
\encoding

\needsfontinstversion{1.910}

\setcommand\lc#1#2{#2}
\setcommand\uc#1#2{#1}
\setcommand\lctop#1#2{#2}
\setcommand\uctop#1#2{#1}
\setcommand\lclig#1#2{#2}
\ifisint{letterspacing}\then
\ifnumber{\int{letterspacing}}={0}\then \Else
\setcommand\uclig#1#2{#1spaced}
\comment{Here we set \verb|\uclig#1#2| to \verb|#1spaced|, but 
  you can't see it as \verb|\setcommand| commands are invisible in 
  the typeset output.}
\Fi
\Fi
\setcommand\uclig#1#2{#1}
\setcommand\digit#1{#1.oldstyle}

\ifisint{monowidth}\then
\setint{ligaturing}{0}
\Else
% The following empty line is *important* to get the formatting
% right here (sigh)! (Remember that it is a \par token.)

\ifisint{letterspacing}\then
\ifnumber{\int{letterspacing}}={0}\then \Else
\setint{ligaturing}{0}
\Fi
\Fi
\setint{ligaturing}{1}
\Fi

\setint{italicslant}{0}
\setint{quad}{1000}
\setint{baselineskip}{1200}

\ifisglyph{x}\then
\setint{xheight}{\height{x}}
\Else
\setint{xheight}{500}
\Fi

\ifisglyph{space}\then
\setint{interword}{\width{space}}
\Else\ifisglyph{i}\then
\setint{interword}{\width{i}}
\Else
\setint{interword}{333}
\Fi\Fi

\ifisint{monowidth}\then
\setint{stretchword}{0}
\setint{shrinkword}{0}
\setint{extraspace}{\int{interword}}
\Else
\setint{stretchword}{\scale{\int{interword}}{600}}
\setint{shrinkword}{\scale{\int{interword}}{240}}
\setint{extraspace}{\scale{\int{interword}}{240}}
\Fi

\ifisglyph{X}\then
\setint{capheight}{\height{X}}
\Else
\setint{capheight}{750}
\Fi

\ifisglyph{d}\then
\setint{ascender}{\height{d}}
\Else\ifisint{capheight}\then
\setint{ascender}{\int{capheight}}
\Else
\setint{ascender}{750}
\Fi\Fi

\ifisglyph{Aring}\then
\setint{acccapheight}{\height{Aring}}
\Else
\setint{acccapheight}{999}
\Fi

\ifisint{descender_neg}\then
\setint{descender}{\neg{\int{descender_neg}}}
\Else\ifisglyph{p}\then
\setint{descender}{\depth{p}}
\Else
\setint{descender}{250}
\Fi\Fi

\ifisglyph{Aring}\then
\setint{maxheight}{\height{Aring}}
\Else
\setint{maxheight}{1000}
\Fi

\ifisint{maxdepth_neg}\then
\setint{maxdepth}{\neg{\int{maxdepth_neg}}}
\Else\ifisglyph{j}\then
\setint{maxdepth}{\depth{j}}
\Else
\setint{maxdepth}{250}
\Fi\Fi

\ifisglyph{six}\then
\setint{digitwidth}{\width{six}}
\Else
\setint{digitwidth}{500}
\Fi

\setint{capstem}{0} % not in AFM files

\setfontdimen{1}{italicslant}    % italic slant
\setfontdimen{2}{interword}      % interword space
\setfontdimen{3}{stretchword}    % interword stretch
\setfontdimen{4}{shrinkword}     % interword shrink
\setfontdimen{5}{xheight}        % x-height
\setfontdimen{6}{quad}           % quad
\setfontdimen{7}{extraspace}     % extra space after .
\setfontdimen{8}{capheight}      % cap height
\setfontdimen{9}{ascender}       % ascender
\setfontdimen{10}{acccapheight}  % accented cap height
\setfontdimen{11}{descender}     % descender's depth
\setfontdimen{12}{maxheight}     % max height
\setfontdimen{13}{maxdepth}      % max depth
\setfontdimen{14}{digitwidth}    % digit width
\setfontdimen{15}{verticalstem}  % dominant width of verical stems
\setfontdimen{16}{baselineskip}  % baselineskip

\ifnumber{\int{ligaturing}}<{0}\then 
\comment{In this case, the codingscheme can be different from the 
  default, and therefore we refrain from setting it.}
\Else
\setstr{codingscheme}{EXTENDED TEX ENC - ELECTRUM OSF LIG SC}
\Fi

\setslot{\lc{Grave}{grave}}
\comment{The grave accent `\`{}'.}
\endsetslot

\setslot{\lc{Acute}{acute}}
\comment{The acute accent `\'{}'.}
\endsetslot

\setslot{\lc{Circumflex}{circumflex}}
\comment{The circumflex accent `\^{}'.}
\endsetslot

\setslot{\lc{Tilde}{tilde}}
\comment{The tilde accent `\~{}'.}
\endsetslot

\setslot{\lc{Dieresis}{dieresis}}
\comment{The umlaut or dieresis accent `\"{}'.}
\endsetslot

\setslot{\lc{Hungarumlaut}{hungarumlaut}}
\comment{The long Hungarian umlaut `\H{}'.}
\endsetslot

\setslot{\lc{Ring}{ring}}
\comment{The ring accent `\r{}'.}
\endsetslot

\setslot{\lc{Caron}{caron}}
\comment{The caron or h\'a\v cek accent `\v{}'.}
\endsetslot

\setslot{\lc{Breve}{breve}}
\comment{The breve accent `\u{}'.}
\endsetslot

\setslot{\lc{Macron}{macron}}
\comment{The macron accent `\={}'.}
\endsetslot

\setslot{\lc{Dotaccent}{dotaccent}}
\comment{The dot accent `\.{}'.}
\endsetslot

\setslot{\lc{Cedilla}{cedilla}}
\comment{The cedilla accent `\c {}'.}
\endsetslot

\setslot{\lc{Ogonek}{ogonek}}
\comment{The ogonek accent `\k {}'.}
\endsetslot

\setslot{quotesinglbase}
\comment{A German single quote mark `\quotesinglbase' similar to a comma,
  but with different sidebearings.}
\endsetslot

\setslot{guilsinglleft}
\comment{A French single opening quote mark `\guilsinglleft',
  unavailable in \plain\ \TeX.}
\endsetslot

\setslot{guilsinglright}
\comment{A French single closing quote mark `\guilsinglright',
  unavailable in \plain\ \TeX.}
\endsetslot

\setslot{quotedblleft}
\comment{The English opening quote mark `\,\textquotedblleft\,'.}
\endsetslot

\setslot{quotedblright}
\comment{The English closing quote mark `\,\textquotedblright\,'.}
\endsetslot

\setslot{quotedblbase}
\comment{A German double quote mark `\quotedblbase' similar to two commas,
  but with tighter letterspacing and different sidebearings.}
\endsetslot

\setslot{guillemotleft}
\comment{A French double opening quote mark `\guillemotleft',
  unavailable in \plain\ \TeX.}
\endsetslot

\setslot{guillemotright}
\comment{A French closing opening quote mark `\guillemotright',
  unavailable in \plain\ \TeX.}
\endsetslot

\setslot{endash}
\ligature{LIG}{hyphen}{emdash}
\comment{The number range dash `1--9'. This is called `rangedash' by fontinst's t1.etx, but it needs to be called `endash' to work right. The `\textendash'.  In a monowidth font, this
  might be set as `\texttt{1{-}9}'.}
\endsetslot

\setslot{emdash}
\comment{The punctuation dash `Oh---boy.' This is calle `punctdash' by fontinst's t1.etx, but needs to be called `emdash' to work right. The `\textemdash'.  In a monowidth font, this
  might be set as `\texttt{Oh{-}{-}boy.}'}
\endsetslot

\setslot{compwordmark}
\comment{An invisible glyph, with zero width and depth, but the
  height of lowercase letters without ascenders.
  It is used to stop ligaturing in words like `shelf{}ful'.}
\endsetslot

\setslot{zero.denominator}
\comment{A glyph which is placed after `\%' to produce a
  `per-thousand', or twice to produce `per-ten-thousand'.
  Your guess is as good as mine as to what this glyph should look
  like in a monowidth font.}
\endsetslot

\setslot{\lc{dotlessI}{dotlessi}}
\comment{A dotless i `\i', used to produce accented letters such as
  `\=\i'.}
\endsetslot

\setslot{\lc{dotlessJ}{dotlessj}}
\comment{A dotless j `\j', used to produce accented letters such as
  `\=\j'.  Most non-\TeX\ fonts do not have this glyph.}
\endsetslot

\ifnumber{\int{ligaturing}}<{0}\then \skipslots{5}\Else

\setslot{\lclig{FF}{f_f}}
\ifnumber{\int{ligaturing}}>{0}\then
\ligature{LIG}{\lc{I}{i}}{\lclig{FFI}{f_f_i}}
\ligature{LIG}{\lc{L}{l}}{\lclig{FFL}{f_f_l}}
\Fi
\comment{The `ff' ligature.  It should be two characters wide in a
  monowidth font.}
\endsetslot

\setslot{\lclig{FI}{fi}}
\comment{The `fi' ligature.  It should be two characters wide in a
  monowidth font.}
\endsetslot

\setslot{\lclig{FL}{fl}}
\comment{The `fl' ligature.  It should be two characters wide in a
  monowidth font.}
\endsetslot

\setslot{\lclig{FFI}{f_f_i}}
\comment{The `ffi' ligature.  It should be three characters wide in a
  monowidth font.}
\endsetslot

\setslot{\lclig{FFL}{f_f_l}}
\comment{The `ffl' ligature.  It should be three characters wide in a
  monowidth font.}
\endsetslot

\Fi

\setslot{visiblespace}
\comment{A visible space glyph `\textvisiblespace'.}
\endsetslot

\setslot{exclam}
\ligature{LIG}{quoteleft}{exclamdown}
\comment{The exclamation mark `!'.}
\endsetslot

\setslot{quotedbl}
\comment{The `neutral' double quotation mark `\,\textquotedbl\,',
  included for use in monowidth fonts, or for setting computer
  programs.  Note that the inclusion of this glyph in this slot
  means that \TeX\ documents which used `{\tt\char`\"}' as an
  input character will no longer work.}
\endsetslot

\setslot{numbersign}
\comment{The hash sign `\#'.}
\endsetslot

\setslot{dollar.oldstyle}
\comment{The dollar sign `\$'.}
\endsetslot

\setslot{percent.oldstyle}
\comment{The percent sign `\%'.}
\endsetslot

\setslot{ampersand.sc}
\comment{The ampersand sign `\&'.}
\endsetslot

\setslot{quoteright}
\ligature{LIG}{quoteright}{quotedblright}
\comment{The English closing single quote mark `\,\textquoteright\,'.}
\endsetslot

\setslot{parenleft}
\comment{The opening parenthesis `('.}
\endsetslot

\setslot{parenright}
\comment{The closing parenthesis `)'.}
\endsetslot

\setslot{asterisk}
\comment{The raised asterisk `*'.}
\endsetslot

\setslot{plus}
\comment{The addition sign `+'.}
\endsetslot

\setslot{comma}
\ligature{LIG}{comma}{quotedblbase}
\comment{The comma `,'.}
\endsetslot

\setslot{hyphen}
\ligature{LIG}{hyphen}{endash}
\ligature{LIG}{hyphenchar}{hyphenchar}
\comment{The hyphen `-'.}
\endsetslot

\setslot{period}
\comment{The period `.'.}
\endsetslot

\setslot{slash}
\comment{The forward oblique `/'.}
\endsetslot

\setslot{\digit{zero}}
\comment{The number `0'.  This (and all the other numerals) may be
  old style or ranging digits.}
\endsetslot

\setslot{\digit{one}}
\comment{The number `1'.}
\endsetslot

\setslot{\digit{two}}
\comment{The number `2'.}
\endsetslot

\setslot{\digit{three}}
\comment{The number `3'.}
\endsetslot

\setslot{\digit{four}}
\comment{The number `4'.}
\endsetslot

\setslot{\digit{five}}
\comment{The number `5'.}
\endsetslot

\setslot{\digit{six}}
\comment{The number `6'.}
\endsetslot

\setslot{\digit{seven}}
\comment{The number `7'.}
\endsetslot

\setslot{\digit{eight}}
\comment{The number `8'.}
\endsetslot

\setslot{\digit{nine}}
\comment{The number `9'.}
\endsetslot

\setslot{colon}
\comment{The colon punctuation mark `:'.}
\endsetslot

\setslot{semicolon}
\comment{The semi-colon punctuation mark `;'.}
\endsetslot

\setslot{less}
\ligature{LIG}{less}{guillemotleft}
\comment{The less-than sign `\textless'.}
\endsetslot

\setslot{equal}
\comment{The equals sign `='.}
\endsetslot

\setslot{greater}
\ligature{LIG}{greater}{guillemotright}
\comment{The greater-than sign `\textgreater'.}
\endsetslot

\setslot{question}
\ligature{LIG}{quoteleft}{questiondown}
\comment{The question mark `?'.}
\endsetslot

\setslot{at}
\comment{The at sign `@'.}
\endsetslot

\setslot{\uc{A}{a}}
\comment{The letter `{A}'.}
\endsetslot

\setslot{\uc{B}{b}}
\comment{The letter `{B}'.}
\endsetslot

\setslot{\uc{C}{c}}
\comment{The letter `{C}'.}
\endsetslot

\setslot{\uc{D}{d}}
\comment{The letter `{D}'.}
\endsetslot

\setslot{\uc{E}{e}}
\comment{The letter `{E}'.}
\endsetslot

\setslot{\uc{F}{f}}
\comment{The letter `{F}'.}
\endsetslot

\setslot{\uc{G}{g}}
\comment{The letter `{G}'.}
\endsetslot

\setslot{\uc{H}{h}}
\comment{The letter `{H}'.}
\endsetslot

\ifnumber{\int{ligaturing}}<{-1}\then \skipslots{1}\Else

\setslot{\uc{I}{i}}
\comment{The letter `{I}'.}
\endsetslot

\Fi

\setslot{\uc{J}{j}}
\comment{The letter `{J}'.}
\endsetslot

\setslot{\uc{K}{k}}
\comment{The letter `{K}'.}
\endsetslot

\setslot{\uc{L}{l}}
\comment{The letter `{L}'.}
\endsetslot

\setslot{\uc{M}{m}}
\comment{The letter `{M}'.}
\endsetslot

\setslot{\uc{N}{n}}
\comment{The letter `{N}'.}
\endsetslot

\setslot{\uc{O}{o}}
\comment{The letter `{O}'.}
\endsetslot

\setslot{\uc{P}{p}}
\comment{The letter `{P}'.}
\endsetslot

\setslot{\uc{Qalt}{q}}
\ifnumber{\int{ligaturing}}>{0}\then
\ligature{LIG}{asterisk}{\uc{Q}{q}}
\Fi
\comment{The letter `{Qalt}'.}
\endsetslot

\setslot{\uc{R}{r}}
\comment{The letter `{R}'.}
\endsetslot

\setslot{\uc{S}{s}}
\comment{The letter `{S}'.}
\endsetslot

\setslot{\uc{T}{t}}
\comment{The letter `{T}'.}
\endsetslot

\setslot{\uc{U}{u}}
\comment{The letter `{U}'.}
\endsetslot

\setslot{\uc{V}{v}}
\comment{The letter `{V}'.}
\endsetslot

\setslot{\uc{W}{w}}
\comment{The letter `{W}'.}
\endsetslot

\setslot{\uc{X}{x}}
\comment{The letter `{X}'.}
\endsetslot

\setslot{\uc{Y}{y}}
\comment{The letter `{Y}'.}
\endsetslot

\setslot{\uc{Z}{z}}
\comment{The letter `{Z}'.}
\endsetslot

\setslot{bracketleft}
\comment{The opening square bracket `['.}
\endsetslot

\setslot{backslash}
\comment{The backwards oblique `\textbackslash'.}
\endsetslot

\setslot{bracketright}
\comment{The closing square bracket `]'.}
\endsetslot

\setslot{asciicircum}
\comment{The ASCII upward-pointing arrow head `\textasciicircum'.
  This is included for compatibility with typewriter fonts used
  for computer listings.}
\endsetslot

\setslot{underscore}
\comment{The ASCII underline character `\textunderscore', usually
  set on the baseline.
  This is included for compatibility with typewriter fonts used
  for computer listings.}
\endsetslot

\setslot{quoteleft}
\ligature{LIG}{quoteleft}{quotedblleft}
\comment{The English opening single quote mark `\,\textquoteleft\,'.}
\endsetslot

\setslot{\lc{A}{a}}
\comment{The letter `{a}'.}
\endsetslot

\setslot{\lc{B}{b}}
\comment{The letter `{b}'.}
\endsetslot

\ifnumber{\int{ligaturing}}<{-1}\then \skipslots{1}\Else

\setslot{\lc{C}{c}}
\comment{The letter `{c}'.}
\endsetslot

\Fi

\setslot{\lc{D}{d}}
\comment{The letter `{d}'.}
\endsetslot

\setslot{\lc{E}{e}}
\comment{The letter `{e}'.}
\endsetslot

\ifnumber{\int{ligaturing}}<{-1}\then \skipslots{1}\Else

\setslot{\lc{F}{f}}
\ifnumber{\int{ligaturing}}>{0}\then
\ligature{LIG}{\lc{I}{i}}{\lclig{FI}{fi}}
\ligature{LIG}{\lc{F}{f}}{\lclig{FF}{f_f}}
\ligature{LIG}{\lc{L}{l}}{\lclig{FL}{fl}}
\Fi
\comment{The letter `{f}'.}
\endsetslot

\Fi

\setslot{\lc{G}{g}}
\comment{The letter `{g}'.}
\endsetslot

\setslot{\lc{H}{h}}
\comment{The letter `{h}'.}
\endsetslot

\ifnumber{\int{ligaturing}}<{-1}\then \skipslots{1}\Else

\setslot{\lc{I}{i}}
\comment{The letter `{i}'.}
\endsetslot

\Fi

\setslot{\lc{J}{j}}
\comment{The letter `{j}'.}
\endsetslot

\setslot{\lc{K}{k}}
\comment{The letter `{k}'.}
\endsetslot

\setslot{\lc{L}{l}}
\comment{The letter `{l}'.}
\endsetslot

\setslot{\lc{M}{m}}
\comment{The letter `{m}'.}
\endsetslot

\setslot{\lc{N}{n}}
\comment{The letter `{n}'.}
\endsetslot

\setslot{\lc{O}{o}}
\comment{The letter `{o}'.}
\endsetslot

\setslot{\lc{P}{p}}
\comment{The letter `{p}'.}
\endsetslot

\setslot{\lc{Qalt}{q}}
\comment{The letter `{q}'.}
\endsetslot

\setslot{\lc{R}{r}}
\comment{The letter `{r}'.}
\endsetslot

\ifnumber{\int{ligaturing}}<{-1}\then \skipslots{1}\Else

\setslot{\lc{S}{s}}
\comment{The letter `{s}'.}
\endsetslot

\Fi

\setslot{\lc{T}{t}}
\ifnumber{\int{ligaturing}}>{0}\then
\ligature{LIG}{\lc{T}{t}}{\lclig{TT}{t_t}}
\Fi
\comment{The letter `{t}'.}
\endsetslot

\setslot{\lc{U}{u}}
\comment{The letter `{u}'.}
\endsetslot

\setslot{\lc{V}{v}}
\comment{The letter `{v}'.}
\endsetslot

\setslot{\lc{W}{w}}
\comment{The letter `{w}'.}
\endsetslot

\setslot{\lc{X}{x}}
\comment{The letter `{x}'.}
\endsetslot

\setslot{\lc{Y}{y}}
\comment{The letter `{y}'.}
\endsetslot

\setslot{\lc{Z}{z}}
\comment{The letter `{z}'.}
\endsetslot

\setslot{braceleft}
\comment{The opening curly brace `\textbraceleft'.}
\endsetslot

\setslot{bar}
\comment{The ASCII vertical bar `\textbar'.
  This is included for compatibility with typewriter fonts used
  for computer listings.}
\endsetslot

\setslot{braceright}
\comment{The closing curly brace `\textbraceright'.}
\endsetslot

\setslot{asciitilde}
\comment{The ASCII tilde `\textasciitilde'.
  This is included for compatibility with typewriter fonts used
  for computer listings.}
\endsetslot

\setslot{hyphenchar}
\comment{The glyph used for hyphenation in this font, which will
  almost always be the same as `hyphen'.}
\endsetslot

\setslot{\uctop{Abreve}{abreve}}
\comment{The letter `\u A'.}
\endsetslot

\setslot{\uc{Aogonek}{aogonek}}
\comment{The letter `\k A'.}
\endsetslot

\setslot{\uctop{Cacute}{cacute}}
\comment{The letter `\' C'.}
\endsetslot

\setslot{\uctop{Ccaron}{ccaron}}
\comment{The letter `\v C'.}
\endsetslot

\setslot{\uctop{Dcaron}{dcaron}}
\comment{The letter `\v D'.}
\endsetslot

\setslot{\uctop{Ecaron}{ecaron}}
\comment{The letter `\v E'.}
\endsetslot

\setslot{\uc{Eogonek}{eogonek}}
\comment{The letter `\k E'.}
\endsetslot

\setslot{\uctop{Gbreve}{gbreve}}
\comment{The letter `\u G'.}
\endsetslot

\setslot{\uctop{Lacute}{lacute}}
\comment{The letter `\' L'.}
\endsetslot

\setslot{\uc{Lcaron}{lcaron}}
\comment{The letter `\v L'.}
\endsetslot

\setslot{\uc{Lslash}{lslash}}
\comment{The letter `\L'.}
\endsetslot

\setslot{\uctop{Nacute}{nacute}}
\comment{The letter `\' N'.}
\endsetslot

\setslot{\uctop{Ncaron}{ncaron}}
\comment{The letter `\v N'.}
\endsetslot

\ifnumber{\int{ligaturing}}<{0}\then \skipslots{1}\Else

\setslot{\lclig{TT}{t_t}}
\endsetslot

\Fi

\setslot{\uctop{Ohungarumlaut}{ohungarumlaut}}
\comment{The letter `\H O'.}
\endsetslot

\setslot{\uctop{Racute}{racute}}
\comment{The letter `\' R'.}
\endsetslot

\setslot{\uctop{Rcaron}{rcaron}}
\comment{The letter `\v R'.}
\endsetslot

\setslot{\uctop{Sacute}{sacute}}
\comment{The letter `\' S'.}
\endsetslot

\setslot{\uctop{Scaron}{scaron}}
\comment{The letter `\v S'.}
\endsetslot

\setslot{\uc{Scedilla}{scedilla}}
\comment{The letter `\c S'.}
\endsetslot

\setslot{\uctop{Tcaron}{tcaron}}
\comment{The letter `\v T'.}
\endsetslot

\setslot{\uc{Tcedilla}{tcedilla}}
\comment{The letter `\c T'.}
\endsetslot

\setslot{\uctop{Uhungarumlaut}{uhungarumlaut}}
\comment{The letter `\H U'.}
\endsetslot

\setslot{\uctop{Uring}{uring}}
\comment{The letter `\r U'.}
\endsetslot

\setslot{\uctop{Ydieresis}{ydieresis}}
\comment{The letter `\" Y'.}
\endsetslot

\setslot{\uctop{Zacute}{zacute}}
\comment{The letter `\' Z'.}
\endsetslot

\setslot{\uctop{Zcaron}{zcaron}}
\comment{The letter `\v Z'.}
\endsetslot

\setslot{\uctop{Zdotaccent}{zdotaccent}}
\comment{The letter `\. Z'.}
\endsetslot

\ifnumber{\int{ligaturing}}<{0}\then \skipslots{1}\Else

\setslot{\uclig{IJ}{ij}}
\comment{The letter `IJ'.  This is a single letter, and in a 
  monowidth font should ideally be one letter wide.}
\endsetslot

\Fi

\setslot{\uctop{Idotaccent}{idotaccent}}
\comment{The letter `\. I'.}
\endsetslot

\setslot{\lc{Dbar}{dbar}}
\comment{The letter `\dj'.}
\endsetslot

\setslot{section}
\comment{The section mark `\textsection'.}
\endsetslot

\setslot{\lctop{Abreve}{abreve}}
\comment{The letter `\u a'.}
\endsetslot

\setslot{\lc{Aogonek}{aogonek}}
\comment{The letter `\k a'.}
\endsetslot

\setslot{\lctop{Cacute}{cacute}}
\comment{The letter `\' c'.}
\endsetslot

\setslot{\lctop{Ccaron}{ccaron}}
\comment{The letter `\v c'.}
\endsetslot

\setslot{\lctop{Dcaron}{dcaron}}
\comment{The letter `\v d'.}
\endsetslot

\setslot{\lctop{Ecaron}{ecaron}}
\comment{The letter `\v e'.}
\endsetslot

\setslot{\lc{Eogonek}{eogonek}}
\comment{The letter `\k e'.}
\endsetslot

\setslot{\lctop{Gbreve}{gbreve}}
\comment{The letter `\u g'.}
\endsetslot

\setslot{\lctop{Lacute}{lacute}}
\comment{The letter `\' l'.}
\endsetslot

\setslot{\lc{Lcaron}{lcaron}}
\comment{The letter `\v l'.}
\endsetslot

\setslot{\lc{Lslash}{lslash}}
\comment{The letter `\l'.}
\endsetslot

\setslot{\lctop{Nacute}{nacute}}
\comment{The letter `\' n'.}
\endsetslot

\setslot{\lctop{Ncaron}{ncaron}}
\comment{The letter `\v n'.}
\endsetslot

\setslot{\uc{Q}{q}}
\endsetslot

\setslot{\lctop{Ohungarumlaut}{ohungarumlaut}}
\comment{The letter `\H o'.}
\endsetslot

\setslot{\lctop{Racute}{racute}}
\comment{The letter `\' r'.}
\endsetslot

\setslot{\lctop{Rcaron}{rcaron}}
\comment{The letter `\v r'.}
\endsetslot

\setslot{\lctop{Sacute}{sacute}}
\comment{The letter `\' s'.}
\endsetslot

\setslot{\lctop{Scaron}{scaron}}
\comment{The letter `\v s'.}
\endsetslot

\setslot{\lc{Scedilla}{scedilla}}
\comment{The letter `\c s'.}
\endsetslot

\setslot{\lctop{Tcaron}{tcaron}}
\comment{The letter `\v t'.}
\endsetslot

\setslot{\lc{Tcedilla}{tcedilla}}
\comment{The letter `\c t'.}
\endsetslot

\setslot{\lctop{Uhungarumlaut}{uhungarumlaut}}
\comment{The letter `\H u'.}
\endsetslot

\setslot{\lctop{Uring}{uring}}
\comment{The letter `\r u'.}
\endsetslot

\setslot{\lctop{Ydieresis}{ydieresis}}
\comment{The letter `\" y'.}
\endsetslot

\setslot{\lctop{Zacute}{zacute}}
\comment{The letter `\' z'.}
\endsetslot

\setslot{\lctop{Zcaron}{zcaron}}
\comment{The letter `\v z'.}
\endsetslot

\setslot{\lctop{Zdotaccent}{zdotaccent}}
\comment{The letter `\. z'.}
\endsetslot

\ifnumber{\int{ligaturing}}<{0}\then \skipslots{1}\Else

\setslot{\lclig{IJ}{ij}}
\comment{The letter `ij'.  This is a single letter, and in a 
  monowidth font should ideally be one letter wide.}
\endsetslot

\Fi

\setslot{exclamdown}
\comment{The Spanish punctuation mark `!`'.}
\endsetslot

\setslot{questiondown}
\comment{The Spanish punctuation mark `?`'.}
\endsetslot

\setslot{sterling.oldstyle}
\comment{The British currency mark `\textsterling'.}
\endsetslot

\setslot{\uctop{Agrave}{agrave}}
\comment{The letter `\` A'.}
\endsetslot

\setslot{\uctop{Aacute}{aacute}}
\comment{The letter `\' A'.}
\endsetslot

\setslot{\uctop{Acircumflex}{acircumflex}}
\comment{The letter `\^ A'.}
\endsetslot

\setslot{\uctop{Atilde}{atilde}}
\comment{The letter `\~ A'.}
\endsetslot

\setslot{\uctop{Adieresis}{adieresis}}
\comment{The letter `\" A'.}
\endsetslot

\setslot{\uctop{Aring}{aring}}
\comment{The letter `\r A'.}
\endsetslot

\setslot{\uc{AE}{ae}}
\comment{The letter `\AE'.  This is a single letter, and should not be
  faked with `AE'.}
\endsetslot

\setslot{\uc{Ccedilla}{ccedilla}}
\comment{The letter `\c C'.}
\endsetslot

\setslot{\uctop{Egrave}{egrave}}
\comment{The letter `\` E'.}
\endsetslot

\setslot{\uctop{Eacute}{eacute}}
\comment{The letter `\' E'.}
\endsetslot

\setslot{\uctop{Ecircumflex}{ecircumflex}}
\comment{The letter `\^ E'.}
\endsetslot

\setslot{\uctop{Edieresis}{edieresis}}
\comment{The letter `\" E'.}
\endsetslot

\setslot{\uctop{Igrave}{igrave}}
\comment{The letter `\` I'.}
\endsetslot

\setslot{\uctop{Iacute}{iacute}}
\comment{The letter `\' I'.}
\endsetslot

\setslot{\uctop{Icircumflex}{icircumflex}}
\comment{The letter `\^ I'.}
\endsetslot

\setslot{\uctop{Idieresis}{idieresis}}
\comment{The letter `\" I'.}
\endsetslot

\setslot{\uc{Eth}{eth}}
\comment{The uppercase Icelandic letter `Eth' similar to a `D'
  with a horizontal bar through the stem.  It is unavailable
  in \plain\ \TeX.}
\endsetslot

\setslot{\uctop{Ntilde}{ntilde}}
\comment{The letter `\~ N'.}
\endsetslot

\setslot{\uctop{Ograve}{ograve}}
\comment{The letter `\` O'.}
\endsetslot

\setslot{\uctop{Oacute}{oacute}}
\comment{The letter `\' O'.}
\endsetslot

\setslot{\uctop{Ocircumflex}{ocircumflex}}
\comment{The letter `\^ O'.}
\endsetslot

\setslot{\uctop{Otilde}{otilde}}
\comment{The letter `\~ O'.}
\endsetslot

\setslot{\uctop{Odieresis}{odieresis}}
\comment{The letter `\" O'.}
\endsetslot

\setslot{\uc{OE}{oe}}
\comment{The letter `\OE'.  This is a single letter, and should not be
  faked with `OE'.}
\endsetslot

\setslot{\uc{Oslash}{oslash}}
\comment{The letter `\O'.}
\endsetslot

\setslot{\uctop{Ugrave}{ugrave}}
\comment{The letter `\` U'.}
\endsetslot

\setslot{\uctop{Uacute}{uacute}}
\comment{The letter `\' U'.}
\endsetslot

\setslot{\uctop{Ucircumflex}{ucircumflex}}
\comment{The letter `\^ U'.}
\endsetslot

\setslot{\uctop{Udieresis}{udieresis}}
\comment{The letter `\" U'.}
\endsetslot

\setslot{\uctop{Yacute}{yacute}}
\comment{The letter `\' Y'.}
\endsetslot

\setslot{\uc{Thorn}{thorn}}
\comment{The Icelandic capital letter Thorn, similar to a `P'
  with the bowl moved down.  It is unavailable in \plain\ \TeX.}
\endsetslot

\setslot{\uclig{SS}{germandbls}}
\comment{The ligature `SS', used to give an upper case `\ss'.
  In a monowidth font it should be two letters wide.}
\endsetslot

\setslot{\lctop{Agrave}{agrave}}
\comment{The letter `\` a'.}
\endsetslot

\setslot{\lctop{Aacute}{aacute}}
\comment{The letter `\' a'.}
\endsetslot

\setslot{\lctop{Acircumflex}{acircumflex}}
\comment{The letter `\^ a'.}
\endsetslot

\setslot{\lctop{Atilde}{atilde}}
\comment{The letter `\~ a'.}
\endsetslot

\setslot{\lctop{Adieresis}{adieresis}}
\comment{The letter `\" a'.}
\endsetslot

\setslot{\lctop{Aring}{aring}}
\comment{The letter `\r a'.}
\endsetslot

\setslot{\lc{AE}{ae}}
\comment{The letter `\ae'.  This is a single letter, and should not be
  faked with `ae'.}
\endsetslot

\setslot{\lc{Ccedilla}{ccedilla}}
\comment{The letter `\c c'.}
\endsetslot

\setslot{\lctop{Egrave}{egrave}}
\comment{The letter `\` e'.}
\endsetslot

\setslot{\lctop{Eacute}{eacute}}
\comment{The letter `\' e'.}
\endsetslot

\setslot{\lctop{Ecircumflex}{ecircumflex}}
\comment{The letter `\^ e'.}
\endsetslot

\setslot{\lctop{Edieresis}{edieresis}}
\comment{The letter `\" e'.}
\endsetslot

\setslot{\lctop{Igrave}{igrave}}
\comment{The letter `\`\i'.}
\endsetslot

\setslot{\lctop{Iacute}{iacute}}
\comment{The letter `\'\i'.}
\endsetslot

\setslot{\lctop{Icircumflex}{icircumflex}}
\comment{The letter `\^\i'.}
\endsetslot

\setslot{\lctop{Idieresis}{idieresis}}
\comment{The letter `\"\i'.}
\endsetslot

\setslot{\lc{Eth}{eth}}
\comment{The Icelandic lowercase letter `eth' similar to
  a `$\partial$' with an oblique bar through the stem.
  It is unavailable in \plain\ \TeX.}
\endsetslot

\setslot{\lctop{Ntilde}{ntilde}}
\comment{The letter `\~ n'.}
\endsetslot

\setslot{\lctop{Ograve}{ograve}}
\comment{The letter `\` o'.}
\endsetslot

\setslot{\lctop{Oacute}{oacute}}
\comment{The letter `\' o'.}
\endsetslot

\setslot{\lctop{Ocircumflex}{ocircumflex}}
\comment{The letter `\^ o'.}
\endsetslot

\setslot{\lctop{Otilde}{otilde}}
\comment{The letter `\~ o'.}
\endsetslot

\setslot{\lctop{Odieresis}{odieresis}}
\comment{The letter `\" o'.}
\endsetslot

\setslot{\lc{OE}{oe}}
\comment{The letter `\oe'.  This is a single letter, and should not be
  faked with `oe'.}
\endsetslot

\setslot{\lc{Oslash}{oslash}}
\comment{The letter `\o'.}
\endsetslot

\setslot{\lctop{Ugrave}{ugrave}}
\comment{The letter `\` u'.}
\endsetslot

\setslot{\lctop{Uacute}{uacute}}
\comment{The letter `\' u'.}
\endsetslot

\setslot{\lctop{Ucircumflex}{ucircumflex}}
\comment{The letter `\^ u'.}
\endsetslot

\setslot{\lctop{Udieresis}{udieresis}}
\comment{The letter `\" u'.}
\endsetslot

\setslot{\lctop{Yacute}{yacute}}
\comment{The letter `\' y'.}
\endsetslot

\setslot{\lc{Thorn}{thorn}}
\comment{The Icelandic lowercase letter `thorn', similar to a `p'
  with an ascender rising from the stem.  It is unavailable
  in \plain\ \TeX.}
\endsetslot

\setslot{\lc{SS}{germandbls}}
\comment{The letter `\ss'.}
\endsetslot

\endencoding
%    \end{macrocode}
% \iffalse
%</t1j-yesw-sc>
% \fi
% \iffalse
%<*ts1-yes>
% \fi
%    \begin{macrocode}
\relax
\encoding

\setstr{codingscheme}{TEX TEXT COMPANION - ELECTRUM}

\ifisglyph{x}\then
\setint{xheight}{\height{x}}
\else
\setint{xheight}{500}
\fi

\ifisglyph{space}\then
\setint{interword}{\width{space}}
\else\ifisglyph{i}\then
\setint{interword}{\width{i}}
\else
\setint{interword}{333}
\fi\fi


\setint{italicslant}{0}


\setint{fontdimen(1)}{\int{italicslant}}              % italic slant
\setint{fontdimen(2)}{\int{interword}}                % interword space
\setint{fontdimen(3)}{0}                              % interword stretch
\setint{fontdimen(4)}{0}                              % interword shrink
\setint{fontdimen(5)}{\int{xheight}}                  % x-height
\setint{fontdimen(6)}{1000}                           % quad
\setint{fontdimen(7)}{\int{interword}}                % extra space after .


\nextslot{0}
\setslot{capitalgrave}
\comment{The grave accent `\capitalgrave{}', intended for use with
  capital letters.}
\endsetslot

\setslot{capitalacute}
\comment{The acute accent `\capitalacute{}', intended for use with
  capital letters.}
\endsetslot

\setslot{capitalcircumflex}
\comment{The circumflex accent `\capitalcircumflex{}', intended for
  use with capital letters.}
\endsetslot

\setslot{capitaltilde}
\comment{The tilde accent `\capitaltilde{}', intended for use with
  capital letters.}
\endsetslot

\setslot{capitaldieresis}
\comment{The umlaut or dieresis accent `\capitaldieresis{}',
  intended for use with capital letters.}
\endsetslot

\setslot{capitalhungarumlaut}
\comment{The long Hungarian umlaut `\capitalhungarumlaut{}',
  intended for use with capital letters.}
\endsetslot

\setslot{capitalring}
\comment{The ring accent `\capitalring{}', intended for use with
  capital letters.}
\endsetslot

\setslot{capitalcaron}
\comment{The caron or h\'a\v cek accent `\capitalcaron{}', intended
  for use with capital letters.}
\endsetslot

\setslot{capitalbreve}
\comment{The breve accent `\capitalbreve{}', intended for use with
  capital letters.}
\endsetslot

\setslot{capitalmacron}
\comment{The macron accent `\capitalmacron{}', intended for use with
  capital letters.}
\endsetslot

\setslot{capitaldotaccent}
\comment{The dot accent `\capitaldotaccent{}', intended for use with
  capital letters.}
\endsetslot

\setslot{cedilla}
\comment{The cedilla accent `\capitalcedilla{}', intended for use
  with capital letters.}
\endsetslot

\setslot{ogonek}
\comment{The ogonek accent `\capitalogonek{}', intended for use with
  capital letters.}
\endsetslot

\nextslot{13}
\setslot{quotesinglbase}
\comment{A straight single quote mark on the baseline,
  `\textquotestraightbase'.}
\endsetslot

\nextslot{18}
\setslot{quotedblbase}
\comment{A straight double quote mark on the baseline,
  `\textquotestraightdblbase'.}
\endsetslot

\nextslot{21}
\setslot{twelveudash}
\comment{A 2/3~em dash, `\texttwelveudash'.}
\endsetslot

\setslot{threequartersemdash}
\comment{A 3/4~em dash, `\textthreequartersemdash'.}
\endsetslot

\nextslot{23}
\setslot{capitalcompwordmark}
\comment{An invisible glyph, with zero width and depth, but the
  height of capital letters.
  It is used to stop ligaturing in words like `shelf{}ful'.}
\endsetslot

\nextslot{24}
\setslot{arrowleft}
\comment{A left pointing arrow, `\textleftarrow', unavailable in
  most PostScript fonts.}
\endsetslot

\setslot{arrowright}
\comment{A right pointing arrow, `\textrightarrow', unavailable in
  most PostScript fonts.}
\endsetslot

\nextslot{26}
\setslot{tieaccentlowercase}
\comment{The original tie accent `\t{}', intended for use with
  lowercase letters.}
\endsetslot

\setslot{tieaccentcapital}
\comment{The tie accent `\capitaltie{}', intended for use with
  capital letters.}
\endsetslot

\setslot{newtieaccentlowercase}
\comment{A new tie accent `\newtie{}', intended for use with
  lowercase letters.}
\endsetslot

\setslot{newtieaccentcapital}
\comment{A new tie accent `\capitalnewtie{}', intended for use
  with capital letters.}
\endsetslot

\nextslot{31}
\setslot{ascendercompwordmark}
\comment{An invisible glyph, with zero width and depth, but the
  height of lowercase letters with ascenders.
  It is used to stop ligaturing in words like `shelf{}ful'.}
\endsetslot

\nextslot{32}
\setslot{blank}
\comment{The blank indicator `\textblank', similar to the letter `b'
  with an oblique bar throgh the stem.}
\endsetslot

\nextslot{36}
\setslot{dollar}
\comment{The dollar sign `\textdollar'.}
\endsetslot

\nextslot{39}
\setslot{quotesingle}
\comment{A straight single quote mark, `\textquotesingle'.}
\endsetslot

\nextslot{42}
\setslot{asteriskcentered}
\comment{The centered asterisk, `\textasteriskcentered'.}
\endsetslot

\nextslot{44}
\setslot{comma}
\comment{The decimal comma `,'.}
\endsetslot

\nextslot{45}
\setslot{hyphendbl}
\comment{An alternate double hyphen, `\textdblhyphen'.}
\endsetslot

\nextslot{46}
\setslot{period}
\comment{The decimal point `.'.}
\endsetslot

\nextslot{47}
\setslot{fraction}
\comment{The fraction slash `\textfractionsolidus'.}
\endsetslot

\nextslot{48}
\setslot{zero.oldstyle}
\comment{The oldstyle number `\oldstylenums{0}'.}
\endsetslot

\setslot{one.oldstyle}
\comment{The oldstyle number `\oldstylenums{1}'.}
\endsetslot

\setslot{two.oldstyle}
\comment{The oldstyle number `\oldstylenums{2}'.}
\endsetslot

\setslot{three.oldstyle}
\comment{The oldstyle number `\oldstylenums{3}'.}
\endsetslot

\setslot{four.oldstyle}
\comment{The oldstyle number `\oldstylenums{4}'.}
\endsetslot

\setslot{five.oldstyle}
\comment{The oldstyle number `\oldstylenums{5}'.}
\endsetslot

\setslot{six.oldstyle}
\comment{The oldstyle number `\oldstylenums{6}'.}
\endsetslot

\setslot{seven.oldstyle}
\comment{The oldstyle number `\oldstylenums{7}'.}
\endsetslot

\setslot{eight.oldstyle}
\comment{The oldstyle number `\oldstylenums{8}'.}
\endsetslot

\setslot{nine.oldstyle}
\comment{The oldstyle number `\oldstylenums{9}'.}
\endsetslot

\nextslot{60}
\setslot{angbracketleft}
\comment{The opening angle bracket `\textlangle', unavailable in
  most PostScript fonts.}
\endsetslot

\nextslot{61}
\setslot{minus}
\comment{The subtraction sign `\textminus'.}
\endsetslot

\nextslot{62}
\setslot{angbracketright}
\comment{The closing angle bracket `\textrangle', unavailable in
  most PostScript fonts.}
\endsetslot

\nextslot{77}
\setslot{Omegainv}
\comment{The inverted Ohm sign `\textmho', unavailable in most fonts.}
\endsetslot

\nextslot{79}
\comment{A circle `\textbigcircle', big enough to enclose a letter
  as in `\textcopyright' or `\textregistered'.}
\setslot{bigcircle}
\endsetslot

\nextslot{87}
\setslot{Omega}
\comment{The upright Ohm sign `\textohm', unavailable in most fonts.
  Even if it is available in Mac-encoded fonts, it isn't directly
  accessible in the 8r or 8y encodings.}
\endsetslot

\nextslot{91}
\setslot{openbracketleft}
\comment{The opening double square bracket `\textlbrackdbl',
  unavailable in most PostScript fonts.}
\endsetslot

\nextslot{93}
\setslot{openbracketright}
\comment{The closing double square bracket `\textrbrackdbl',
  unavailable in most PostScript fonts.}
\endsetslot

\nextslot{94}
\setslot{arrowup}
\comment{An upwards pointing arrow `\textuparrow', unavailable in
  most PostScript fonts.}
\endsetslot

\nextslot{95}
\setslot{arrowdown}
\comment{An downwards pointing arrow `\textdownarrow', unavailable
  in most PostScript fonts.}
\endsetslot

\nextslot{96}
\setslot{asciigrave}
\comment{An ASCII-style grave `\textasciigrave'. This is supposed
  to be a character by itself rather than a combining accents.}
\endsetslot

\nextslot{98}
\setslot{born}
\comment{The born symbol `\textborn', unavailable in most PostScript
  fonts.}
\endsetslot

\nextslot{99}
\setslot{divorced}
\comment{The divorced symbol `\textdivorced', unavailable in most
  PostScript fonts.}
\endsetslot

\nextslot{100}
\setslot{died}
\comment{The died symbol `\textdied', unavailable in most PostScript
  fonts.}
\endsetslot

\nextslot{108}
\setslot{leaf}
\comment{The leaf symbol `\textleaf', unavailable in most PostScript
  fonts.}
\endsetslot

\nextslot{109}
\setslot{married}
\comment{The married symbol `\textmarried', unavailable in most
  PostScript  fonts.}
\endsetslot

\nextslot{110}
\setslot{musicalnote}
\comment{A musical note symbol `\textmusicalnote', unavailable in
  most PostScript fonts.}
\endsetslot

\nextslot{126}
\setslot{tildelow}
\comment{A lowered tilde `\texttildelow'.  In most PostScript fonts
  it can be substituted by `asciitilde', while `\textasciitilde'
  is supposed to be a raised `tilde'.}
\endsetslot

\nextslot{127}
\setslot{hyphendblchar}
\comment{The glyph used for hyphenation in this font, which will
  almost always be the same as `hyphendbl'.}
\endsetslot

\nextslot{128}
\setslot{asciibreve}
\comment{An ASCII-style breve `\textasciibreve'. This is supposed
  to be a character by itself rather than a combining accents.}
\endsetslot

\setslot{asciicaron}
\comment{An ASCII-style caron `\textasciicaron'. This is supposed
  to be a character by itself rather than a combining accents.}
\endsetslot

\setslot{asciiacutedbl}
\comment{An ASCII-style double tick mark, `\textacutedbl'.}
\endsetslot

\setslot{asciigravedbl}
\comment{An ASCII-style double backtick mark, `\textgravedbl'.}
\endsetslot

\setslot{dagger}
\comment{The single dagger `\textdagger'.}
\endsetslot

\setslot{daggerdbl}
\comment{The double dagger `\textdaggerdbl'.}
\endsetslot

\setslot{bardbl}
\comment{The double vertical bar `\textbardbl'.}
\endsetslot

\setslot{perthousand}
\comment{The perthousand sign `\textperthousand'.}
\endsetslot

\setslot{bullet}
\comment{The centered bullet `\textbullet'.}
\endsetslot

\setslot{centigrade}
\comment{The degree centigrade symbol `\textcelsius'.}
\endsetslot

\setslot{dollar.oldstyle}
\comment{An oldstyle dollar sign `\textdollaroldstyle'.}
\endsetslot

\setslot{cent.oldstyle}
\comment{An oldstyle cent sign `\textcentoldstyle'.}
\endsetslot

\setslot{florin}
\comment{The florin sign `\textflorin'.}
\endsetslot

\setslot{colonmonetary}
\comment{The Colon currency sign `\textcolonmonetary', similar to
  a capital `C' with a vertical bar through the middle.}
\endsetslot

\setslot{won}
\comment{The Won currency sign `\textwon', similar to a capital `W'
  with two horizontal bars.}
\endsetslot

\setslot{naira}
\comment{The Naira currency sign `\textnaira', similar to a
  capital `N' with two horizontal bars.}
\endsetslot

\setslot{guarani}
\comment{The Guarani currency sign `\textguarani',  similar to
  a capital `G' with a vertical bar through the middle.}
\endsetslot

\setslot{peso}
\comment{The Peso currency sign `\textpeso', similar to a capital `P'
  with a horizontal bar through the bowl or below the bowl.}
\endsetslot

\setslot{lira}
\comment{The Lira currency sign `\textlira', similar to a sterling
  sign `\textsterling' with two horizontal bars.}
\endsetslot

\setslot{recipe}
\comment{The recipe symbol `\textrecipe', similar to a capital `R'
  with an oblique bar through the tail.}
\endsetslot

\setslot{interrobang}
\comment{The interrobang symbol `\textinterrobang', similar to
  a combination of an exclamation mark and a question mark.}
\endsetslot

\setslot{interrobangdown}
\comment{The inverted interrobang symbol `\textinterrobangdown',
  similar to a combination of an inverted exclamation mark
  and an inverted question mark.}
\endsetslot

\setslot{dong}
\comment{The Dong currency sign `\textdong', similar to a lowercase
  `d'  with a horizontal bar through the stem and another bar below
  the letter.}
\endsetslot

\setslot{trademark}
\comment{The trademark sign `\texttrademark', similar to the raised
  letters `TM'.}
\endsetslot

\setslot{pertenthousand}
\comment{The pertenthousand sign `\textpertenthousand', unavailable
  in most PostScript fonts.}
\endsetslot

\setslot{pilcrow}
\comment{The pilcrow mark `\textpilcrow', similar to a paragraph
  mark `\textparagraph' with a single stem.}
\endsetslot

\setslot{baht}
\comment{The Baht currency sign `\textbaht', similar to a capital `B'
  with a vertical bar through the middle.}
\endsetslot

\setslot{numero}
\comment{The numero sign `\textnumero', similar to the letter `N'
  with a raised `o', unavailable in most PostScript fonts.}
\endsetslot

\setslot{discount}
\comment{The discount sign `\textdiscount', similar to a stylized
  percent sign, unavailable in most PostScript fonts.}
\endsetslot

\setslot{estimated}
\comment{The estimated sign `\textestimated', similar to an enlarged
  lowercase `e', unavailable in most PostScript fonts.}
\endsetslot

\setslot{openbullet}
\comment{The centered open bullet `\textopenbullet'', unavailable
  in most PostScript fonts.}
\endsetslot

\setslot{servicemark}
\comment{The service mark sign `\textservicemark', similar to the
  raised letters `SM', unavailable in most PostScript fonts.}
\endsetslot

\nextslot{160}
\setslot{quillbracketleft}
\comment{The opening quill bracket `\textlquill', unavailable in
  most PostScript fonts.}
\endsetslot

\setslot{quillbracketright}
\comment{The closing quill bracket `\textrquill', unavailable in
  most PostScript fonts.}
\endsetslot

\setslot{cent}
\comment{The cent sign `\textcent'.}
\endsetslot

\setslot{sterling}
\comment{The British currency sign, `\textsterling'.}
\endsetslot

\setslot{currency}
\comment{The international currency sign, `\textcurrency'.}
\endsetslot

\setslot{yen}
\comment{The Japanese currency sign, `\textyen'.}
\endsetslot

\setslot{brokenbar}
\comment{A broken vertical bar, `\textbrokenbar', similar to
  `\textbar' with a gap through the middle.}
\endsetslot

\setslot{section}
\comment{The section mark `\textsection'.}
\endsetslot

\setslot{asciidieresis}
\comment{An ASCII-style dieresis `\textasciidieresis'. This is
  supposed to be character by itself  rather than an accents.}
\endsetslot

\setslot{copyright}
\comment{The copyright sign `\textcopyright',  similar to a small
  letter `C' enclosed by a circle.}
\endsetslot

\setslot{ordfeminine}
\comment{The raised letter `\textordfeminine'.}
\endsetslot

\setslot{copyleft}
\comment{The reversed copyright sign `\textcopyleft', similar to
  a small reversed `C' enclosed by a circle.}
\endsetslot

\setslot{logicalnot}
\comment{The logical not sign `\textlnot'.}
\endsetslot

\setslot{circledP}
\comment{A small letter `P' enclosed by a circle, `\textcircledP',
  unavailable in most fonts.}
\endsetslot

\setslot{registered}
\comment{The registered trademark sign `\textregistered', similar to
  a small letter `R' enclosed by a circle.}
\endsetslot

\setslot{asciimacron}
\comment{An ASCII-style macron `\textasciimacron'. This is supposed
  to be a character by itself rather than a combining accents.}
\endsetslot

\setslot{degree}
\comment{The degree sign `\textdegree'.}
\endsetslot

\setslot{plusminus}
\comment{The plus or minus sign `\textpm'.}
\endsetslot

\setslot{two.superior}
\comment{The raised digit `\texttwosuperior'.}
\endsetslot

\setslot{three.superior}
\comment{The raised digit `\textthreesuperior'.}
\endsetslot

\setslot{asciiacute}
\comment{An ASCII-style acute `\textasciiacute'. This is supposed
  to be a character by itself rather than a combining accents.}
\endsetslot

\setslot{mu}
\comment{The lowercase Greek letter `\textmu', intended  for use as
  a prefix `micro' in physical units.}
\endsetslot

\setslot{paragraph}
\comment{The paragraph mark `\textparagraph'.}
\endsetslot

\setslot{periodcentered}
\comment{The centered period `\textperiodcentered'.}
\endsetslot

\setslot{referencemark}
\comment{The reference mark `\textreferencemark', similar to
  a combination of the `multiply' and `divide' symbols.}
\endsetslot

\setslot{one.superior}
\comment{The raised digit `\textonesuperior'.}
\endsetslot

\setslot{ordmasculine}
\comment{The raised letter `\textordmasculine'.}
\endsetslot

\setslot{radical}
\comment{The radical sign `\textsurd', unavailable in most fonts.
  Even if it is available in Mac-encoded fonts, it isn't directly
  accessible in the 8r or 8y encodings.}
\endsetslot

\setslot{onequarter}
\comment{The fraction `\textonequarter'.}
\endsetslot

\setslot{onehalf}
\comment{The fraction `\textonehalf'.}
\endsetslot

\setslot{threequarters}
\comment{The fraction `\textthreequarters'.}
\endsetslot

\ifisglyph{euro}\then
\setslot{euro}
\comment{The European currency sign, similar to `\texteuro'.}
\endsetslot
\Else
\setslot{Euro}
\comment{The European currency sign, similar to `\texteuro'.}
\endsetslot
\Fi

\setslot{Euro}
\comment{This just makes sure that any glyph labelled `Euro' in the font gets encoded. The TS1 encoding will use the previous slot when the font is actually used by tex. At least, I think so. That is, since we've got spare slots in this encoding, we can use them to enable "either... or..." encoding options both for reencoding the fonts for fontinst and for the tex encodings. (?!)}
\endsetslot

\setslot{euro}
\comment{This just makes sure that any glyph labelled `euro' in the font gets encoded. The TS1 encoding will use the previous slot when the font is actually used by tex. At least, I think so. That is, since we've got spare slots in this encoding, we can use them to enable "either... or..." encoding options both for reencoding the fonts for fontinst and for the tex encodings. (?!)}
\endsetslot

\nextslot{214}
\setslot{multiply}
\comment{The multiplication sign `\texttimes'.
  This symbol was originally intended to be put into slot~215,
  but ended up in this slot by mistake, at which time it was
  considered too late to change it.}
\endsetslot

\nextslot{246}
\setslot{divide}
\comment{The divison sign `\textdiv'.
  This symbol was originally intended to be put into slot~247,
  but ended up in this slot by mistake, at which time it was
  onsidered too late to change it.}
\endsetslot

\setslot{percent}
\comment{This doesn't really belong here but seems the most natural place for it. It is not part of the TS1 encoding. Hopefully it will at least do no harm.}
\endsetslot

\endencoding
%    \end{macrocode}
% \iffalse
%</ts1-yes>
% \fi
% \iffalse
%<*ts1-dotoldstyle-yes>
% \fi
%    \begin{macrocode}
\relax
\encoding

\setstr{codingscheme}{TEX TS1 - DOTOLDSTYLE ELECTRUM}

\ifisglyph{x}\then
   \setint{xheight}{\height{x}}
\else
   \setint{xheight}{500}
\fi

\ifisglyph{space}\then
   \setint{interword}{\width{space}}
\else\ifisglyph{i}\then
   \setint{interword}{\width{i}}
\else
   \setint{interword}{333}
\fi\fi


\setint{italicslant}{0}


\setint{fontdimen(1)}{\int{italicslant}}              % italic slant
\setint{fontdimen(2)}{\int{interword}}                % interword space
\setint{fontdimen(3)}{0}                              % interword stretch
\setint{fontdimen(4)}{0}                              % interword shrink
\setint{fontdimen(5)}{\int{xheight}}                  % x-height
\setint{fontdimen(6)}{1000}                           % quad
\setint{fontdimen(7)}{\int{interword}}                % extra space after .


\nextslot{0}
\setslot{capitalgrave}
   \comment{The grave accent `\capitalgrave{}', intended for use with
      capital letters.}
\endsetslot

\setslot{capitalacute}
   \comment{The acute accent `\capitalacute{}', intended for use with
      capital letters.}
\endsetslot

\setslot{capitalcircumflex}
   \comment{The circumflex accent `\capitalcircumflex{}', intended for
      use with capital letters.}
\endsetslot

\setslot{capitaltilde}
   \comment{The tilde accent `\capitaltilde{}', intended for use with
      capital letters.}
\endsetslot

\setslot{capitaldieresis}
   \comment{The umlaut or dieresis accent `\capitaldieresis{}',
      intended for use with capital letters.}
\endsetslot

\setslot{capitalhungarumlaut}
   \comment{The long Hungarian umlaut `\capitalhungarumlaut{}',
      intended for use with capital letters.}
\endsetslot

\setslot{capitalring}
   \comment{The ring accent `\capitalring{}', intended for use with
      capital letters.}
\endsetslot

\setslot{capitalcaron}
   \comment{The caron or h\'a\v cek accent `\capitalcaron{}', intended
      for use with capital letters.}
\endsetslot

\setslot{capitalbreve}
   \comment{The breve accent `\capitalbreve{}', intended for use with
      capital letters.}
\endsetslot

\setslot{capitalmacron}
   \comment{The macron accent `\capitalmacron{}', intended for use with
      capital letters.}
\endsetslot

\setslot{capitaldotaccent}
   \comment{The dot accent `\capitaldotaccent{}', intended for use with
      capital letters.}
\endsetslot

\setslot{cedilla}
   \comment{The cedilla accent `\capitalcedilla{}', intended for use
      with capital letters.}
\endsetslot

\setslot{ogonek}
   \comment{The ogonek accent `\capitalogonek{}', intended for use with
      capital letters.}
\endsetslot

\nextslot{13}
\setslot{quotesinglbase}
   \comment{A straight single quote mark on the baseline,
      `\textquotestraightbase'.}
\endsetslot

\nextslot{18}
\setslot{quotedblbase}
   \comment{A straight double quote mark on the baseline,
      `\textquotestraightdblbase'.}
\endsetslot

\nextslot{21}
\setslot{twelveudash}
   \comment{A 2/3~em dash, `\texttwelveudash'.}
\endsetslot

\setslot{threequartersemdash}
   \comment{A 3/4~em dash, `\textthreequartersemdash'.}
\endsetslot

\nextslot{23}
\setslot{capitalcompwordmark}
    \comment{An invisible glyph, with zero width and depth, but the
      height of capital letters.
      It is used to stop ligaturing in words like `shelf{}ful'.}
\endsetslot

\nextslot{24}
\setslot{arrowleft}
   \comment{A left pointing arrow, `\textleftarrow', unavailable in
      most PostScript fonts.}
\endsetslot

\setslot{arrowright}
   \comment{A right pointing arrow, `\textrightarrow', unavailable in
      most PostScript fonts.}
\endsetslot

\nextslot{26}
\setslot{tieaccentlowercase}
   \comment{The original tie accent `\t{}', intended for use with
      lowercase letters.}
\endsetslot

\setslot{tieaccentcapital}
   \comment{The tie accent `\capitaltie{}', intended for use with
      capital letters.}
\endsetslot

\setslot{newtieaccentlowercase}
   \comment{A new tie accent `\newtie{}', intended for use with
      lowercase letters.}
\endsetslot

\setslot{newtieaccentcapital}
   \comment{A new tie accent `\capitalnewtie{}', intended for use
      with capital letters.}
\endsetslot

\nextslot{31}
\setslot{ascendercompwordmark}
    \comment{An invisible glyph, with zero width and depth, but the
      height of lowercase letters with ascenders.
      It is used to stop ligaturing in words like `shelf{}ful'.}
\endsetslot

\nextslot{32}
\setslot{blank}
   \comment{The blank indicator `\textblank', similar to the letter `b'
      with an oblique bar throgh the stem.}
\endsetslot

\nextslot{36}
\setslot{dollar.oldstyle}
   \comment{The dollar sign `\textdollar'.}
\endsetslot

\nextslot{39}
\setslot{quotesingle}
   \comment{A straight single quote mark, `\textquotesingle'.}
\endsetslot

\nextslot{42}
\setslot{asteriskcentered}
   \comment{The centered asterisk, `\textasteriskcentered'.}
\endsetslot

\nextslot{44}
\setslot{comma}
   \comment{The decimal comma `,'.}
\endsetslot

\nextslot{45}
\setslot{hyphendbl}
   \comment{An alternate double hyphen, `\textdblhyphen'.}
\endsetslot

\nextslot{46}
\setslot{period}
   \comment{The decimal point `.'.}
\endsetslot

\nextslot{47}
\setslot{fraction}
   \comment{The fraction slash `\textfractionsolidus'.}
\endsetslot

\nextslot{48}
\setslot{zero.oldstyle}
   \comment{The oldstyle number `\oldstylenums{0}'.}
\endsetslot

\setslot{one.oldstyle}
   \comment{The oldstyle number `\oldstylenums{1}'.}
\endsetslot

\setslot{two.oldstyle}
   \comment{The oldstyle number `\oldstylenums{2}'.}
\endsetslot

\setslot{three.oldstyle}
   \comment{The oldstyle number `\oldstylenums{3}'.}
\endsetslot

\setslot{four.oldstyle}
   \comment{The oldstyle number `\oldstylenums{4}'.}
\endsetslot

\setslot{five.oldstyle}
   \comment{The oldstyle number `\oldstylenums{5}'.}
\endsetslot

\setslot{six.oldstyle}
   \comment{The oldstyle number `\oldstylenums{6}'.}
\endsetslot

\setslot{seven.oldstyle}
   \comment{The oldstyle number `\oldstylenums{7}'.}
\endsetslot

\setslot{eight.oldstyle}
   \comment{The oldstyle number `\oldstylenums{8}'.}
\endsetslot

\setslot{nine.oldstyle}
   \comment{The oldstyle number `\oldstylenums{9}'.}
\endsetslot

\nextslot{60}
\setslot{angbracketleft}
   \comment{The opening angle bracket `\textlangle', unavailable in
      most PostScript fonts.}
\endsetslot

\nextslot{61}
\setslot{minus}
   \comment{The subtraction sign `\textminus'.}
\endsetslot

\nextslot{62}
\setslot{angbracketright}
   \comment{The closing angle bracket `\textrangle', unavailable in
      most PostScript fonts.}
\endsetslot

\nextslot{77}
\setslot{Omegainv}
   \comment{The inverted Ohm sign `\textmho', unavailable in most fonts.}
\endsetslot

\nextslot{79}
   \comment{A circle `\textbigcircle', big enough to enclose a letter
      as in `\textcopyright' or `\textregistered'.}
\setslot{bigcircle}
\endsetslot

\nextslot{87}
\setslot{Omega}
   \comment{The upright Ohm sign `\textohm', unavailable in most fonts.
      Even if it is available in Mac-encoded fonts, it isn't directly
      accessible in the 8r or 8y encodings.}
\endsetslot

\nextslot{91}
\setslot{openbracketleft}
   \comment{The opening double square bracket `\textlbrackdbl',
      unavailable in most PostScript fonts.}
\endsetslot

\nextslot{93}
\setslot{openbracketright}
   \comment{The closing double square bracket `\textrbrackdbl',
      unavailable in most PostScript fonts.}
\endsetslot

\nextslot{94}
\setslot{arrowup}
   \comment{An upwards pointing arrow `\textuparrow', unavailable in
      most PostScript fonts.}
\endsetslot

\nextslot{95}
\setslot{arrowdown}
   \comment{An downwards pointing arrow `\textdownarrow', unavailable
      in most PostScript fonts.}
\endsetslot

\nextslot{96}
\setslot{asciigrave}
   \comment{An ASCII-style grave `\textasciigrave'. This is supposed
      to be a character by itself rather than a combining accents.}
\endsetslot

\nextslot{98}
\setslot{born}
   \comment{The born symbol `\textborn', unavailable in most PostScript
      fonts.}
\endsetslot

\nextslot{99}
\setslot{divorced}
   \comment{The divorced symbol `\textdivorced', unavailable in most
      PostScript fonts.}
\endsetslot

\nextslot{100}
\setslot{died}
   \comment{The died symbol `\textdied', unavailable in most PostScript
      fonts.}
\endsetslot

\nextslot{108}
\setslot{leaf}
   \comment{The leaf symbol `\textleaf', unavailable in most PostScript
      fonts.}
\endsetslot

\nextslot{109}
\setslot{married}
   \comment{The married symbol `\textmarried', unavailable in most
      PostScript  fonts.}
\endsetslot

\nextslot{110}
\setslot{musicalnote}
   \comment{A musical note symbol `\textmusicalnote', unavailable in
      most PostScript fonts.}
\endsetslot

\nextslot{126}
\setslot{tildelow}
   \comment{A lowered tilde `\texttildelow'.  In most PostScript fonts
      it can be substituted by `asciitilde', while `\textasciitilde'
      is supposed to be a raised `tilde'.}
\endsetslot

\nextslot{127}
\setslot{hyphendblchar}
    \comment{The glyph used for hyphenation in this font, which will
      almost always be the same as `hyphendbl'.}
\endsetslot

\nextslot{128}
\setslot{asciibreve}
   \comment{An ASCII-style breve `\textasciibreve'. This is supposed
      to be a character by itself rather than a combining accents.}
\endsetslot

\setslot{asciicaron}
   \comment{An ASCII-style caron `\textasciicaron'. This is supposed
      to be a character by itself rather than a combining accents.}
\endsetslot

\setslot{asciiacutedbl}
   \comment{An ASCII-style double tick mark, `\textacutedbl'.}
\endsetslot

\setslot{asciigravedbl}
   \comment{An ASCII-style double backtick mark, `\textgravedbl'.}
\endsetslot

\setslot{dagger}
   \comment{The single dagger `\textdagger'.}
\endsetslot

\setslot{daggerdbl}
   \comment{The double dagger `\textdaggerdbl'.}
\endsetslot

\setslot{bardbl}
   \comment{The double vertical bar `\textbardbl'.}
\endsetslot

\setslot{perthousand.oldstyle}
   \comment{The perthousand sign `\textperthousand'.}
\endsetslot

\setslot{bullet}
   \comment{The centered bullet `\textbullet'.}
\endsetslot

\setslot{centigrade}
   \comment{The degree centigrade symbol `\textcelsius'.}
\endsetslot

\setslot{dollar.oldstyle}
   \comment{An oldstyle dollar sign `\textdollaroldstyle'.}
\endsetslot

\setslot{cent.oldstyle}
   \comment{An oldstyle cent sign `\textcentoldstyle'.}
\endsetslot

\setslot{florin}
   \comment{The florin sign `\textflorin'.}
\endsetslot

\setslot{colonmonetary}
   \comment{The Colon currency sign `\textcolonmonetary', similar to
      a capital `C' with a vertical bar through the middle.}
\endsetslot

\setslot{won}
   \comment{The Won currency sign `\textwon', similar to a capital `W'
      with two horizontal bars.}
\endsetslot

\setslot{naira}
   \comment{The Naira currency sign `\textnaira', similar to a
      capital `N' with two horizontal bars.}
\endsetslot

\setslot{guarani}
   \comment{The Guarani currency sign `\textguarani',  similar to
      a capital `G' with a vertical bar through the middle.}
\endsetslot

\setslot{peso}
   \comment{The Peso currency sign `\textpeso', similar to a capital `P'
      with a horizontal bar through the bowl or below the bowl.}
\endsetslot

\setslot{lira}
   \comment{The Lira currency sign `\textlira', similar to a sterling
      sign `\textsterling' with two horizontal bars.}
\endsetslot

\setslot{recipe}
   \comment{The recipe symbol `\textrecipe', similar to a capital `R'
      with an oblique bar through the tail.}
\endsetslot

\setslot{interrobang}
   \comment{The interrobang symbol `\textinterrobang', similar to
      a combination of an exclamation mark and a question mark.}
\endsetslot

\setslot{interrobangdown}
   \comment{The inverted interrobang symbol `\textinterrobangdown',
      similar to a combination of an inverted exclamation mark
      and an inverted question mark.}
\endsetslot

\setslot{dong}
   \comment{The Dong currency sign `\textdong', similar to a lowercase
      `d'  with a horizontal bar through the stem and another bar below
      the letter.}
\endsetslot

\setslot{trademark}
   \comment{The trademark sign `\texttrademark', similar to the raised
     letters `TM'.}
\endsetslot

\setslot{pertenthousand}
   \comment{The pertenthousand sign `\textpertenthousand', unavailable
     in most PostScript fonts.}
\endsetslot

\setslot{pilcrow}
   \comment{The pilcrow mark `\textpilcrow', similar to a paragraph
      mark `\textparagraph' with a single stem.}
\endsetslot

\setslot{baht}
   \comment{The Baht currency sign `\textbaht', similar to a capital `B'
      with a vertical bar through the middle.}
\endsetslot

\setslot{numero}
   \comment{The numero sign `\textnumero', similar to the letter `N'
      with a raised `o', unavailable in most PostScript fonts.}
\endsetslot

\setslot{discount}
   \comment{The discount sign `\textdiscount', similar to a stylized
      percent sign, unavailable in most PostScript fonts.}
\endsetslot

\setslot{estimated}
   \comment{The estimated sign `\textestimated', similar to an enlarged
      lowercase `e', unavailable in most PostScript fonts.}
\endsetslot

\setslot{openbullet}
   \comment{The centered open bullet `\textopenbullet'', unavailable
      in most PostScript fonts.}
\endsetslot

\setslot{servicemark}
   \comment{The service mark sign `\textservicemark', similar to the
      raised letters `SM', unavailable in most PostScript fonts.}
\endsetslot

\nextslot{160}
\setslot{quillbracketleft}
   \comment{The opening quill bracket `\textlquill', unavailable in
      most PostScript fonts.}
\endsetslot

\setslot{quillbracketright}
   \comment{The closing quill bracket `\textrquill', unavailable in
      most PostScript fonts.}
\endsetslot

\setslot{cent.oldstyle}
   \comment{The cent sign `\textcent'.}
\endsetslot

\setslot{sterling.oldstyle}
   \comment{The British currency sign, `\textsterling'.}
\endsetslot

\setslot{currency}
   \comment{The international currency sign, `\textcurrency'.}
\endsetslot

\setslot{yen.oldstyle}
   \comment{The Japanese currency sign, `\textyen'.}
\endsetslot

\setslot{brokenbar}
   \comment{A broken vertical bar, `\textbrokenbar', similar to
      `\textbar' with a gap through the middle.}
\endsetslot

\setslot{section}
   \comment{The section mark `\textsection'.}
\endsetslot

\setslot{asciidieresis}
   \comment{An ASCII-style dieresis `\textasciidieresis'. This is
       supposed to be character by itself  rather than an accents.}
\endsetslot

\setslot{copyright}
   \comment{The copyright sign `\textcopyright',  similar to a small
       letter `C' enclosed by a circle.}
\endsetslot

\setslot{ordfeminine}
   \comment{The raised letter `\textordfeminine'.}
\endsetslot

\setslot{copyleft}
   \comment{The reversed copyright sign `\textcopyleft', similar to
      a small reversed `C' enclosed by a circle.}
\endsetslot

\setslot{logicalnot}
   \comment{The logical not sign `\textlnot'.}
\endsetslot

\setslot{circledP}
   \comment{A small letter `P' enclosed by a circle, `\textcircledP',
      unavailable in most fonts.}
\endsetslot

\setslot{registered}
   \comment{The registered trademark sign `\textregistered', similar to
      a small letter `R' enclosed by a circle.}
\endsetslot

\setslot{asciimacron}
   \comment{An ASCII-style macron `\textasciimacron'. This is supposed
      to be a character by itself rather than a combining accents.}
\endsetslot

\setslot{degree}
   \comment{The degree sign `\textdegree'.}
\endsetslot

\setslot{plusminus}
   \comment{The plus or minus sign `\textpm'.}
\endsetslot

\setslot{two.superior}
   \comment{The raised digit `\texttwosuperior'.}
\endsetslot

\setslot{three.superior}
   \comment{The raised digit `\textthreesuperior'.}
\endsetslot

\setslot{asciiacute}
   \comment{An ASCII-style acute `\textasciiacute'. This is supposed
      to be a character by itself rather than a combining accents.}
\endsetslot

\setslot{mu}
   \comment{The lowercase Greek letter `\textmu', intended  for use as
      a prefix `micro' in physical units.}
\endsetslot

\setslot{paragraph}
   \comment{The paragraph mark `\textparagraph'.}
\endsetslot

\setslot{periodcentered}
   \comment{The centered period `\textperiodcentered'.}
\endsetslot

\setslot{referencemark}
   \comment{The reference mark `\textreferencemark', similar to
      a combination of the `multiply' and `divide' symbols.}
\endsetslot

\setslot{one.superior}
   \comment{The raised digit `\textonesuperior'.}
\endsetslot

\setslot{ordmasculine}
   \comment{The raised letter `\textordmasculine'.}
\endsetslot

\setslot{radical}
   \comment{The radical sign `\textsurd', unavailable in most fonts.
      Even if it is available in Mac-encoded fonts, it isn't directly
      accessible in the 8r or 8y encodings.}
\endsetslot

\setslot{onequarter}
   \comment{The fraction `\textonequarter'.}
\endsetslot

\setslot{onehalf}
   \comment{The fraction `\textonehalf'.}
\endsetslot

\setslot{threequarters}
   \comment{The fraction `\textthreequarters'.}
\endsetslot

\ifisglyph{euro.oldstyle}\then
  \setslot{euro.oldstyle}
    \comment{The European currency sign, similar to `\texteuro'.}
  \endsetslot
\Else
  \setslot{Euro.oldstyle}
    \comment{The European currency sign, similar to `\texteuro'.}
  \endsetslot
\Fi

\setslot{Euro.oldstyle}
  \comment{This just makes sure that any glyph labelled `Euro' in the font gets encoded. The TS1 encoding will use the previous slot when the font is actually used by tex. At least, I think so. That is, since we've got spare slots in this encoding, we can use them to enable "either... or..." encoding options both for reencoding the fonts for fontinst and for the tex encodings. (?!)}
\endsetslot
  
\setslot{euro.oldstyle}
  \comment{This just makes sure that any glyph labelled `euro' in the font gets encoded. The TS1 encoding will use the previous slot when the font is actually used by tex. At least, I think so. That is, since we've got spare slots in this encoding, we can use them to enable "either... or..." encoding options both for reencoding the fonts for fontinst and for the tex encodings. (?!)}
\endsetslot

\nextslot{214}
\setslot{multiply}
   \comment{The multiplication sign `\texttimes'.
      This symbol was originally intended to be put into slot~215,
      but ended up in this slot by mistake, at which time it was
      considered too late to change it.}
\endsetslot

\nextslot{246}
\setslot{divide}
   \comment{The divison sign `\textdiv'.
      This symbol was originally intended to be put into slot~247,
      but ended up in this slot by mistake, at which time it was
      onsidered too late to change it.}
\endsetslot

\setslot{percent.oldstyle}
  \comment{This doesn't really belong here but seems the most natural place for it. It is not part of the TS1 encoding. Hopefully it will at least do no harm.}
\endsetslot

\endencoding
%    \end{macrocode}
% \iffalse
%</ts1-dotoldstyle-yes>
% \fi
% \subsection{MEX}
% \verb|mtx| files are used to build ‘fake’ glyphs where these are missing
% from the original fonts.
% We do not fake small-caps or bold, but only glyphs which can be constructed
% without altering the original design.
% This file constructs a fake ‘tt’ ligature for small-caps, since the encoding
% needs this if we are to support the fancy ligature for lowercase.
%
% In addition to this file, we use \verb|mtx| files supplied by \verb|fontinst|
% and some custom files not included in this package's \verb|dtx| as they are not
% specific to Electrum.
% The latter are, however, part of the package:
% \begin{itemize}
%   \item dotscbuild.mtx
%   \item dotscmisc.mtx
%   \item newlatin-dotsc.mtx
% \end{itemize}
% \iffalse
%<*mtx>
% \fi
%    \begin{macrocode}
\relax

\metrics

\needsfontinstversion{1.917}

\ProvidesMtxPackage{build-ttsc}

\setint{smallcapsspacing}{0}

\ifisglyph{t.sc}\then
  \setglyph{tt.sc}
    \glyph{t.sc}{1000}
    \movert{\add{\kerning{t.sc}{t.sc}}{\int{smallcapsspacing}}}
    \glyph{t.sc}{1000}
  \endsetglyph
  \setrightkerning{tt.sc}{t.sc}{1000}
  \setglyph{t_t.sc}
    \glyph{tt.sc}{1000}
  \endsetglyph
  \setrightkerning{t_t.sc}{tt.sc}{1000}
\Fi

\endmetrics
%    \end{macrocode}
% \iffalse
%</mtx>
%<*mtx-per>
% \fi
%    \begin{macrocode}
%% ?? beth?
% based on newlatin.mtx - because I can't figure out how to pass options...
\relax

\documentclass[twocolumn]{article}

\metrics

\needsfontinstversion{1.924}

\usemtxpackage{llbuild}

\usemtxpackage{lubuild}

\usemtxpackage{ltpunct}


\usemtxpackage{ltcmds}

\ifisglyph{zero.denominator}\then
  \ifisglyph{perthousand}\then
  \Else
    \setglyph{perthousand}
      \glyph{percent}{1000}
      \glyph{zero.denominator}{1000}
    \endsetglyph
    \setleftkerning{perthousand}{percent}{1000}
    \setrightkerning{perthousand}{zero.denominator}{1000}
  \Fi
  \ifisglyph{pertenthousand}\then
  \Else
    \setglyph{pertenthousand}
      \glyph{perthousand}{1000}
      \glyph{zero.denominator}{1000}
    \endsetglyph
    \setleftkerning{pertenthousand}{perthousand}{1000}
    \setrightkerning{pertenthousand}{zero.denominator}{1000}
  \Fi
\Fi

\endmetrics
%    \end{macrocode}
% \iffalse
%</mtx-per>
% \fi
%\Finale
% vim: nospell:
