% \iffalse meta-comment
%%%%%%%%%%%%%%%%%%%%%%%%%%%%%%%%%%%%%%%%%%%%%%%%%
% venturisold.dtx
% Additions and changes Copyright (C) 2008-2024 Clea F. Rees.
% Code from skeleton.dtx Copyright (C) 2015-2024 Scott Pakin (see below).
%
% This work may be distributed and/or modified under the
% conditions of the LaTeX Project Public License, either version 1.3
% of this license or (at your option) any later version.
% The latest version of this license is in
%   https://www.latex-project.org/lppl.txt
% and version 1.3 or later is part of all distributions of LaTeX
% version 2005/12/01 or later.
%
% This work has the LPPL maintenance status `maintained'.
%
% The Current Maintainer of this work is Clea F. Rees.
%
% This work consists of all files listed in manifest.txt.
%
% The file venturisold.dtx is a derived work under the terms of the
% LPPL. It is based on version 2.4 of skeleton.dtx which is part of 
% dtxtut by Scott Pakin. A copy of dtxtut, including the 
% unmodified version of skeleton.dtx is available from
% https://www.ctan.org/pkg/dtxtut and released under the LPPL.
%%%%%%%%%%%%%%%%%%%%%%%%%%%%%%%%%%%%%%%%%%%%%%%%%
% \fi
%
% \iffalse
%<*driver>
\RequirePackage{svn-prov}
% ref. ateb Max Chernoff: https://tex.stackexchange.com/a/723294/
\def\MakePrivateLetters{\makeatletter\ExplSyntaxOn\endlinechar13}
\ProvidesFileSVN{$Id: venturisold.dtx 10255 2024-08-19 15:58:36Z cfrees $}[v0.0 \revinfo][\filebase DTX: Venturis ADF for 8-bit engines]
\DefineFileInfoSVN[venturisold]
\documentclass[11pt,british]{ltxdoc}
% l3doc loads fancyvrb
% fancyvrb overwrites svn-prov's macros without warning
% restore \fileversion \filerev in case we're using l3doc
\GetFileInfoSVN{venturisold}
\newcommand* \pkgname{venturisadf}
\EnableCrossrefs
\CodelineIndex
\RecordChanges
\DoNotIndex{\verb,\ProvidesPackageSVN,\NeedsTeXFormat,\ProcessKeyOptions,\revinfo,\filebase,\filename,\filedate,\RequirePackage,\usepackage,\DefineFileInfoSVN,\GetFileInfoSVN,\ProvidesPackageSVN,\documentclass,\MakeAutoQuote,\parindent,\par,\smallskip,\setlength,\bigskip,\maketitle,\title,\author,\date,\ExplSyntaxOn,\ExplSyntaxOff,\bye,\relax}
\usepackage{babel}
\pdfmapfile{-yvt.map}
\pdfmapfile{-yv1.map}
\pdfmapfile{-yv2.map}
\pdfmapfile{-yv3.map}
\pdfmapfile{-yvo.map}
\pdfmapfile{+yvt.map}
\pdfmapfile{+yv1.map}
\pdfmapfile{+yv2.map}
\pdfmapfile{+yv3.map}
\pdfmapfile{+yvo.map}
\usepackage{venturis}
% The following commands are only necessary to enable access to all the fonts in the collection in the same document. 
% Normally, one would load venturis2 or venturisold rather than defining things explicitly.
% But those packages will overwrite the declarations in venturis so aren't an option in this peculiar context.
% (It is assumed that in ordinary usage, only one of the serif/sans-serif combinations provided by venturis and venturis2 or the serif offered by venturisold will be wanted in a single document.)
\DeclareRobustCommand{\venturistworm}{%
  \fontencoding{T1}%
  \fontfamily{yv2}%
  \selectfont}
\DeclareRobustCommand{\venturistwosf}{%
  \fontencoding{T1}%
  \fontfamily{yv3}%
  \selectfont}
\DeclareRobustCommand{\venturisold}{%
  \fontencoding{T1}%
  \fontfamily{yvo}%
  \selectfont}
\DeclareTextFontCommand{\vtworm}{\venturistworm}
\DeclareTextFontCommand{\vtwosf}{\venturistwosf}
\DeclareTextFontCommand{\vo}{\venturisold}
\makeatletter
\DeclareRobustCommand{\vostyle}[1][]{%
  \not@math@alphabet\vostyle\relax
  \fontfamily{yvod}\selectfont}
\makeatother
\DeclareTextFontCommand{\textvo}{\vostyle}
\DeclareTextFontCommand{\textvol}{\vostyle}
\DeclareRobustCommand{\lmrmfamily}{%
  \fontencoding{T1}%
  \fontfamily{lmr}%
  \selectfont}
\DeclareTextFontCommand{\lmrm}{\lmrmfamily}
\renewcommand{\ttdefault}{lmvtt}
\usepackage{fancyhdr}
\usepackage{fancyref}
\usepackage{array}
\usepackage{longtable}
\usepackage{metalogo}
	\setlogokern{Te}{-0.065em}% default: -0.1667em
	\setlogokern{eX}{-0.06em}% default: -0.125em
	\setlogokern{La}{-0.265em}% default: -0.36em
	\setlogokern{aT}{-.055em}% default: -0.15em
%	\setlogokern{X2}{}% default: 0.15em
	\setlogodrop[TeX]{0.355ex}% default: 0.5ex
	\setLaTeXa{\scshape a}
%	\setLaTeXee{<arg>}	
\usepackage{fixfoot}
\usepackage{enumitem}
\usepackage[referable]{threeparttablex}
\usepackage{enumitem}
\makeatletter
\def\TPT@measurement{% ateb David Carlisle: https://tex.stackexchange.com/a/370691/
  \ifdim\wd\@tempboxb<\TPTminimum
    \hsize \TPTminimum
  \else
    \hsize\wd\@tempboxb
  \fi
  \xdef\TPT@hsize{\hsize\the\hsize \noexpand\@parboxrestore}\TPT@hsize
  \ifx\TPT@docapt\@undefined\else
    \TPT@docapt \vskip.2\baselineskip
  \fi \par
  \dimen@\dp\@tempboxb % new
  \box\@tempboxb
  \ifvmode \prevdepth\dimen@ \fi% was \z@ not \dimen@
}
\renewlist{tablenotes}{enumerate}{1}
\setlist[tablenotes]{label=\tnote{\alph*},ref=\alph*,itemsep=\z@,topsep=\z@skip,partopsep=\z@skip,parsep=\z@,itemindent=\z@,labelindent=\tabcolsep,labelsep=.2em,leftmargin=*,align=left,before={\unskip\medskip\footnotesize}}
\makeatother
\usepackage{booktabs}
\usepackage{multirow}
\usepackage{xcolor}
\usepackage{xurl}
\urlstyle{sf}
\usepackage{microtype}
\usepackage[a4paper,headheight=14pt]{geometry}	% use 14pt for 11pt text, 15pt for 12pt text
\usepackage{csquotes}
\MakeAutoQuote{‘}{’}
\MakeAutoQuote*{“}{”}
\usepackage{caption}
\DeclareCaptionFont{lf}{\lstyle}
\captionsetup[table]{labelfont=lf}
% sicrhau hyperindex=false: llwytho CYN bookmark
\usepackage{hypdoc}% ateb Ulrike Fischer: https://tex.stackexchange.com/a/695555/
\usepackage{bookmark}
\hypersetup{%
  colorlinks=true,
  citecolor={moss},
  extension=pdf,
  linkcolor={strawberry},
  linktocpage=true,
  pdfcreator={TeX},
  pdfproducer={pdfeTeX},
  urlcolor={blueberry}%
}
\NewDocElement[%
  idxtype=opt.,
  idxgroup=options,
  printtype=\textit{opt.},
]{Opt}{option}
\NewDocumentCommand \val { m }
{%
  {\ttfamily =\,\meta{#1}}%
}
\ExplSyntaxOn
\NewDocumentCommand \vals { m }
{
  {
    \ttfamily = \, 
    \clist_use:nn { #1 } { \textbar }
  }
} 
\ExplSyntaxOff
\pagestyle{fancy}
\fancyhf[lh]{\itshape\filebase~\fileversion}
%^^A \fancyhf[rh]{\itshape\filetoday}
\fancyhf[rh]{\itshape\filedate}
\fancyhf[ch]{}
\fancyhf[lf]{}
\fancyhf[rf]{}
\fancyhf[cf]{\itshape--- \thepage~/~\lastpage{} ---}
\ExplSyntaxOn
\hook_gput_code:nnn {shipout/lastpage} {.}
{
  \property_record:nn {t:lastpage}{abspage,page,pagenum}
}
\cs_new_protected_nopar:Npn \lastpage 
{
  \property_ref:nn {t:lastpage}{page}
}
\ExplSyntaxOff
\definecolor{strawberry}{rgb}{1.000,0.000,0.502}
\definecolor{blueberry}{rgb}{0.000,0.000,1.000}
\definecolor{moss}{rgb}{0.000,0.502,0.251}
\begin{document}
  \DocInput{\filename}
\end{document}
%</driver>
% \fi
%
%
% \title{\pkgname: venturisold}
% \author{Clea F. Rees\thanks{%
%     Bug tracker:
%   \href{https://codeberg.org/cfr/nfssext/issues}{\url{codeberg.org/cfr/nfssext/issues}}
%   \textbar{} Code:
%   \href{https://codeberg.org/cfr/nfssext}{\url{codeberg.org/cfr/nfssext}}
%   \textbar{} Mirror:
%   \href{https://github.com/cfr42/nfssext}{\url{github.com/cfr42/nfssext}}% 
% }}
%^^A \date{\fileversion~\filetoday}
% \date{\fileversion~\filedate}
%
% \providecommand*{\adf}{\textsc{adf}}
% \providecommand*{\lpack}[1]{\textsf{#1}}
% \providecommand*{\fgroup}[1]{\textsf{#1}}
% \providecommand*{\fname}[1]{\textsf{#1}}
% \providecommand*{\file}[1]{\texttt{#1}}
%
% 
% \maketitle\thispagestyle{empty}
% \pdfinfo{%
% 	/Creator		(TeX)
% 	/Producer		(pdfTeX)
% 	/Author			(Clea F. Rees)
% 	/Title			(venturisold)
% 	/Subject		(TeX)
% 	/Keywords		(TeX,LaTeX,font,fonts,tex,latex,Venturis,venturis,venturisadf,VenturisADF,ADF,adf,Arkandis,Digital,Foundry,arkandis,digital,foundry,Hirwen,Harendal,Clea,Rees)}
% \pdfcatalog{%
% 	/URL				()
% 	/PageMode	/UseOutlines}	
% \setlength{\parindent}{0pt}
% \setlength{\parskip}{0.5em}
% 
% 
% 
% \begin{abstract}
%   \noindent This file is part of \lpack{\pkgname}.
%   \file{\pkgname.pdf} constitutes the main documentation for the package.
% \end{abstract}
% 
% \tableofcontents
% 
% \section{Example}
%
% \iffalse
%<*ee>
% \fi
%    \begin{macrocode}
\documentclass[12pt]{article}
\title{Venturis Old Example}
\author{Postman Pat (\textit{sans} black-and-white cat)}

\newcommand{\alphaline}{ABCDEFGHIJKLMNOPQRSTUVWXYZ\\abcdefghijklmnopqrs{}tuvwxyz\\0123456789\\
	f{}f ff f{}i fi f{}l fl f{}f{}i ffi f{}f{}l ffl s{}t st \& \texteuro \texttrademark \textcopyright \textmu\\
	d d* s s* s+ s+{}t s+t\\
	sphinx of black quartz, judge my vow}% \textcelsius \textnumero \textservicemark}
\newcommand{\abc}{ABCDEFGHIJKLMNOPQRSTUVWXYZ abcdefghijklmnopqrstuvwxyz}
\newcommand{\digits}{0123456789}
\newcommand{\alphatest}{%
  \begin{flushleft}
    \textup{upright shape:\\\alphaline}\\[\smallskipamount]
    \textit{italics:\\\alphaline}\\
  \end{flushleft}%
}

\pdfmapfile{-yvo.map}
\pdfmapfile{+yvo.map}
\usepackage{venturisold}

\begin{document}
\maketitle
\setlength{\parindent}{0pt}

\section*{family: yvo}

A serif family with oldstyle figures.
\medskip

\alphatest

\textbf{\alphatest}

\section*{family: yvod}

\textvt{\alphaline}

\section*{family: yvoa}

\textalt{\alphatest}

\textalt{\textbf{\alphatest}}

\section*{family: yvoad}

\textalt{\textvt{\alphaline}}

\end{document} 
%    \end{macrocode}
% \iffalse
%</ee>
% \fi
%
% \MaybeStop{%
% \PrintChanges
% \PrintIndex
% }
% 
% \section{Implementation}
%
%
% \subsection{venturisold.sty}
%
% \iffalse
%<*sty>
% \fi
% \begin{package}{venturisold.sty}
% This package uses font group \fgroup{venturisold}.
%    \begin{macrocode}
\NeedsTeXFormat{LaTeX2e}
\RequirePackage{svn-prov}
\ProvidesPackageSVN[\filebase.sty]{$Id: venturisold.dtx 10255 2024-08-19 15:58:36Z cfrees $}[\filebase v1.1 \revinfo (for VenturisOldADF PS 1.005)]
\DefineFileInfoSVN[venturisold]
\RequirePackage[T1]{fontenc}
\RequirePackage{nfssext-cfr}
%    \end{macrocode}
% \lpack{nfssext-cfr} provides \cs{ProcessKeyOptions}, \cs{IfFormatAtLeastTF} on older kernels.
%    \begin{macrocode}
\IfFormatAtLeastTF {2020-02-02}{%
%    \end{macrocode}
% To get the oldstyle numbers etc.\ used from TS1, we need to set the subset to 0 or 1.
% Unfortunately, this means characters missing from the fonts will not use default symbols as fallback, but this seems to be unavoidable.
%    \begin{macrocode}
  \DeclareEncodingSubset{TS1}{yvo}{1}%
  \DeclareEncodingSubset{TS1}{yvoa}{1}%
  \DeclareEncodingSubset{TS1}{yvod}{1}%
  \DeclareEncodingSubset{TS1}{yvoad}{1}%
}{%
  \RequirePackage{textcomp}}
\UndeclareTextCommand{\textperthousand}{T1}
\ExplSyntaxOn
%    \end{macrocode}
% The actual \verb|sty| is ultra simple.
% \texttt{scale} takes a factor by which to scale the fonts.
% This is empty by default, which is equivalent to \texttt{1}, but more efficient.
%    \begin{macrocode}
\keys_define:nn { venturisold }
{
%    \end{macrocode}
% \begin{option}{scale}
% \begin{macro}{\yvo@scale}
% Option for scaling 
%    \begin{macrocode}
  scale .tl_set:N = \yvo@scale,
  scale .initial:V = \@empty,
%    \end{macrocode}
% \end{macro}
% \end{option}
%    \begin{macrocode}
}
\ProcessKeyOptions[venturisold]
\ExplSyntaxOff
%    \end{macrocode}
% Make Venturis Old default for roman text
%    \begin{macrocode}
\renewcommand{\rmdefault}{yvo}
%    \end{macrocode}
% \begin{macro}{\vtstyle}
% Introduce special titling commands
%    \begin{macrocode}
\DeclareRobustCommand{\vtstyle}[1][]{% 
%    \end{macrocode}
% Allow an optional argument for consistency with venturis.sty
%    \begin{macrocode}
	\not@math@alphabet\vtstyle\relax
	\fontfamily{yvod}\selectfont}
%    \end{macrocode}
% \end{macro}
% \begin{macro}{\textvt,\textvtl}
% Text font commands
%    \begin{macrocode}
\DeclareTextFontCommand{\textvt}{\vtstyle}
\DeclareTextFontCommand{\textvtl}{\vtstyle}
%    \end{macrocode}
% \end{macro}
%^^A paid â chynnwys \endinput - docstrip yn chwilio amddo fe yn arbennigol
%^^A & bydd doctrip yn ei ychwanegu fe beth bynnag
%^^A (Martin Scharrer: https://tex.stackexchange.com/a/28997/)
%    \begin{macrocode}
%% end venturis.sty
%    \end{macrocode}
% \end{package}
% \iffalse
%</sty>
% \fi
% 
% The remaining files are not used directly, but are required to generate the files which allow \TeX{} and \LaTeX{} to use the fonts.
% The sources use \verb|fontinst| as explained in the (sparse) comments.
% While you can install these files into a \TeX{} tree, they are not required for typesetting.
% 
% \subsection{Driver}
%
% The file does all the initial setup of the fonts.
% It organises the fonts into families, defines shapes and reencodes as required.
%
% \iffalse
%<*drv>
% \fi
%    \begin{macrocode}
\input fontinst.sty
\needsfontinstversion{1.926}
%    \end{macrocode}
% Substitutions
% Bold for bold extended
%    \begin{macrocode}
\substitutesilent{b}{db}
\substitutesilent{bx}{b}
\substitutesilent{m}{bx}
\substitutesilent{sl}{it}
\substitutesilent{scit}{sc}
\substitutesilent{scsl}{scit}
\substitutesilent{si}{scsl}
\substitutenoisy{sc}{n}
%    \end{macrocode}
% Record transformations for later map file creation
%    \begin{macrocode}
\recordtransforms{yvo-rec.tex}
%    \end{macrocode}
% Allow fonts to be scaled via variable in fd files
% Also requires fontinst.lua fnttarg as no means to define variable in fontinst 
%    \begin{macrocode}
\declaresize{}{<-> \string\yvo@@scale}
%    \end{macrocode}
% Transformations : reencode fonts
%    \begin{macrocode}
	\transformfont{t1-yvor}{\reencodefont{t1-venturisold-longs}{\fromafm{yvor8a}}}
%    \end{macrocode}
% Transformations : reencode osf
%    \begin{macrocode}
	\transformfont{t1-yvori}{\reencodefont{t1-venturisold-longs}{\fromafm{yvori8a}}}
%    \end{macrocode}
% Transformations : reencode bold
%    \begin{macrocode}
	\transformfont{t1-yvob}{\reencodefont{t1-venturisold-longs}{\fromafm{yvob8a}}}
%    \end{macrocode}
% Transformations : reencode osf
%    \begin{macrocode}
	\transformfont{t1-yvobi}{\reencodefont{t1-venturisold-longs}{\fromafm{yvobi8a}}}
%    \end{macrocode}
% Transformations : reencode titling
%    \begin{macrocode}
	\transformfont{t1-yvodd}{\reencodefont{t1-f_f}{\fromafm{yvodd8a}}}
%    \end{macrocode}
% Transformations : reencode textcomp
%    \begin{macrocode}
	\transformfont{ts1-yvor}{\reencodefont{ts1-euro}{\fromafm{yvor8a}}}
	\transformfont{ts1-yvori}{\reencodefont{ts1-euro}{\fromafm{yvori8a}}}
	\transformfont{ts1-yvob}{\reencodefont{ts1-euro}{\fromafm{yvob8a}}}
	\transformfont{ts1-yvobi}{\reencodefont{ts1-euro}{\fromafm{yvobi8a}}}
	\transformfont{ts1-yvodd}{\reencodefont{ts1-euro}{\fromafm{yvodd8a}}}
%    \end{macrocode}
% Installation: creation of virtual fonts
%    \begin{macrocode}
	\installfonts
%    \end{macrocode}
% \begin{family}{T1/yvo}
% venturisold
%    \begin{macrocode}
		\installfamily{T1}{yvo}{}
		\installfont{yvor8t}{t1-yvor,newlatin}{t1-venturisold}{T1}{yvo}{m}{n}{}
		\installfont{yvori8t}{t1-yvori,newlatin}{t1-venturisold}{T1}{yvo}{m}{it}{}
		\installfont{yvob8t}{t1-yvob,newlatin}{t1-venturisold}{T1}{yvo}{b}{n}{}
		\installfont{yvobi8t}{t1-yvobi,newlatin}{t1-venturisold}{T1}{yvo}{b}{it}{}
%    \end{macrocode}
% \end{family}
% \begin{family}{T1/yvoa}
% Installation: alternates (T1)
%    \begin{macrocode}
		\installfamily{T1}{yvoa}{}
		\installfont{yvoar8t}{t1-yvor,newlatin}{t1-venturisold-longs}{T1}{yvoa}{m}{n}{}
		\installfont{yvoari8t}{t1-yvori,newlatin}{t1-venturisold-longs}{T1}{yvoa}{m}{it}{}
		\installfont{yvoab8t}{t1-yvob,newlatin}{t1-venturisold-longs}{T1}{yvoa}{b}{n}{}
		\installfont{yvoabi8t}{t1-yvobi,newlatin}{t1-venturisold-longs}{T1}{yvoa}{b}{it}{}
%    \end{macrocode}
% \end{family}
% \begin{family}{TS1/yvo}
% Installation: install with TS1 encoding for extra glyphs through textcomp
%    \begin{macrocode}
		\installfamily{TS1}{yvo}{}
		\installfont{yvor8c}{t1-yvor,ts1-yvor,t1-yvor suffix oldstyle,textcomp}{ts1-euro}{TS1}{yvo}{m}{n}{}
		\installfont{yvori8c}{t1-yvori,ts1-yvori,t1-yvori suffix oldstyle,textcomp}{ts1-euro}{TS1}{yvo}{m}{it}{}
		\installfont{yvob8c}{t1-yvob,ts1-yvob,t1-yvob suffix oldstyle,textcomp}{ts1-euro}{TS1}{yvo}{b}{n}{}
		\installfont{yvobi8c}{t1-yvobi,ts1-yvobi,t1-yvobi suffix oldstyle,textcomp}{ts1-euro}{TS1}{yvo}{b}{it}{}
%    \end{macrocode}
% \end{family}
% \begin{family}{TS1/yvoa}
% Installation: alternates (TS1)
%    \begin{macrocode}
		\installfamily{TS1}{yvoa}{}
		\installfontas{yvor8c}{TS1}{yvoa}{m}{n}{}
		\installfontas{yvori8c}{TS1}{yvoa}{m}{it}{}
		\installfontas{yvob8c}{TS1}{yvoa}{b}{n}{}
		\installfontas{yvobi8c}{TS1}{yvoa}{b}{it}{}
%    \end{macrocode}
% \end{family}
% \begin{family}{T1/yvod}
% Installation: titling 
%    \begin{macrocode}
		\installfamily{T1}{yvod}{}
		\installfont{yvodd8t}{t1-yvodd,newlatin}{t1-f_f}{T1}{yvod}{db}{sc}{}
		\installfontas{yvodd8t}{T1}{yvod}{db}{n}{}
%    \end{macrocode}
% \end{family}
% \begin{family}{T1/yvoad}
% Installation: titling for alternate case
%    \begin{macrocode}
		\installfamily{T1}{yvoad}{}
		\installfontas{yvodd8t}{T1}{yvoad}{db}{sc}{}
		\installfontas{yvodd8t}{T1}{yvoad}{db}{n}{}
%    \end{macrocode}
% \end{family}
% \begin{family}{TS1/yvod}
% Installation: install with TS1 encoding for extra glyphs through textcomp
%    \begin{macrocode}
		\installfamily{TS1}{yvod}{}
		\installfont{yvodd8c}{t1-yvodd,ts1-yvodd,textcomp}{ts1-euro}{TS1}{yvod}{db}{sc}{}
		\installfontas{yvodd8c}{TS1}{yvod}{db}{n}{}
%    \end{macrocode}
% \end{family}
% \begin{family}{TS1/yvoad}
% Installation: for alternate case
%    \begin{macrocode}
		\installfamily{TS1}{yvoad}{}
		\installfontas{yvodd8c}{TS1}{yvoad}{db}{sc}{}
		\installfontas{yvodd8c}{TS1}{yvoad}{db}{n}{}
%    \end{macrocode}
% \end{family}
% Finish up 
%    \begin{macrocode}
	\endinstallfonts
\endrecordtransforms
\bye
%    \end{macrocode}
% \iffalse
%</drv>
% \fi
% \subsection{Map}
% 
% This file is compiled to produce the map file fragment \verb|updmap| needs to install the fonts.
% It uses files recorded during compilation of the driver.
% \iffalse
%<*map>
% \fi
%    \begin{macrocode}
\input finstmsc.sty
\resetstr{PSfontsuffix}{.pfb}
\adddriver{dvips}{yvo.map}
\adddriver{pltotf}{yvo-pltotf.sh}
\input yvo-rec.tex
\donedrivers
\bye
%    \end{macrocode}
% \iffalse
%</map>
% \fi
% 
%\Finale
%^^A vim: tw=0:
